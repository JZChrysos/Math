\documentclass{amsart}
%\documentclass{amsart}
\usepackage[utf8]{inputenc}
\usepackage{amsfonts}
\usepackage{amsmath}
\usepackage{amssymb}
\usepackage{amsthm}
\usepackage{asymptote}
\usepackage{mathtools}
\usepackage{hhline}
\usepackage{graphicx,enumerate}
\usepackage{hyperref}
\usepackage[a4paper, margin=1.2in]{geometry}
%\usepackage{tcolorbox}
\usepackage{tikz-cd}
\usepackage{ytableau}
%\tcbuselibrary{skins,breakable,xparse}
\allowdisplaybreaks
\newcounter{count}
\hypersetup{
	colorlinks=true,
	linkcolor=teal,
	filecolor=magenta,      
	urlcolor=olive,
	citecolor=teal,
	pdfpagemode=FullScreen,
}

%\definecolor{defcolor}{HTML}{478EFF}
%\definecolor{thmcolor}{HTML}{CC0058}
%\definecolor{excolor}{HTML}{F5B400}
%\definecolor{probcolor}{HTML}{DD4803}
%\definecolor{lemcolor}{HTML}{741FEA}
%\definecolor{scarlet}{HTML}{A81111}
%
%\newtheoremstyle{definitionStyle}% Custom style for definitions
%{0.5em}% Space above
%{0.5em}% Space below
%{}% Body font
%{}% Indent amount
%{\bfseries\color{defcolor}}% Theorem head font: bold and red
%{.\\}% Punctuation after theorem head
%{0.5em}% Space after theorem head
%{\thmname{#1}\thmnumber{ #2 (#3)}}% Theorem head spec
%
%\theoremstyle{definitionStyle}
%\newtheorem{df}{Definition}[section]
%
%\newtheoremstyle{theoremStyle}% Custom style for definitions
%{0.5em}% Space above
%{0.5em}% Space below
%{}% Body font
%{}% Indent amount
%{\bfseries\color{thmcolor}}% Theorem head font: bold and red
%{.\\}% Punctuation after theorem head
%{0.5em}% Space after theorem head
%{\thmname{#1}\thmnumber{ #2 (#3)}}% Theorem head spec
%
%\theoremstyle{theoremStyle}
%\newtheorem{thm}{Theorem}[section]
%
%\newtheoremstyle{lemmaStyle}% Custom style for definitions
%{0.5em}% Space above
%{0.5em}% Space below
%{}% Body font
%{}% Indent amount
%{\bfseries\color{lemcolor}}% Theorem head font: bold and red
%{.\\}% Punctuation after theorem head
%{0.5em}% Space after theorem head
%{\thmname{#1}\thmnumber{ #2 (#3)}}% Theorem head spec
%
%\theoremstyle{lemmaStyle}
%\newtheorem{lem}{Lemma}[section]
%\newtheorem{cor}{Corollary}[section]
%
%\newtheoremstyle{exampleStyle}% Custom style for definitions
%{0.5em}% Space above
%{0.5em}% Space below
%{}% Body font
%{}% Indent amount
%{\bfseries\color{excolor}}% Theorem head font: bold and red
%{.\\}% Punctuation after theorem head
%{0.5em}% Space after theorem head
%{\thmname{#1}\thmnumber{ #2 (#3)}}% Theorem head spec
%
%\theoremstyle{exampleStyle}
%\newtheorem{ex}{Example}[section]
%
%\newtheoremstyle{problemStyle}% Custom style for definitions
%{0.5em}% Space above
%{0.5em}% Space below
%{}% Body font
%{}% Indent amount
%{\bfseries\color{probcolor}}% Theorem head font: bold and red
%{.\\}% Punctuation after theorem head
%{0.5em}% Space after theorem head
%{\thmname{#1}\thmnumber{ #2#3}}% Theorem head spec
%
%\theoremstyle{problemStyle}
%\newtheorem{prob}{Problem}[section]

% For Fun
\newcommand{\club}{\color{teal} \clubsuit}
\newcommand{\heart}{\color{red} \heartsuit}
\renewcommand{\star}{\color{scarlet} \bigstar}
\newcommand{\spade}{\color{violet} \spadesuit}

% Symbols
\newcommand{\A}{\mathcal{A}}
\newcommand{\B}{\mathcal{B}}
\newcommand{\C}{\mathbb{C}}
\newcommand{\D}{\mathcal{D}}
\newcommand{\E}{\mathbb{E}}
\newcommand{\F}{\mathbb{F}}
\newcommand{\G}{\mathcal{G}}
% \renewcommand{\H}{\mathcal{H}} Erdos o
\newcommand{\I}{\mathcal{I}}
\newcommand{\J}{\mathcal{J}}
\newcommand{\K}{\mathcal{K}}
% \renewcommand{\L}{\mathcal{L}}
\newcommand{\M}{\mathcal{M}}
\newcommand{\N}{\mathbb{N}}
\renewcommand{\O}{\mathcal{O}}
\renewcommand{\P}{\mathbb{P}}
\newcommand{\Q}{\mathbb{Q}}
\newcommand{\R}{\mathbb{R}}
\renewcommand{\S}{\mathbb{S}}
\newcommand{\T}{\mathbb{T}}
\newcommand{\U}{\mathcal{U}}
\newcommand{\V}{\mathcal{V}}
\newcommand{\W}{\mathcal{W}}
\newcommand{\X}{\mathcal{X}}
\newcommand{\Y}{\mathcal{Y}}
\newcommand{\Z}{\mathbb{Z}}

\renewcommand{\AA}{\mathcal{A}}
\newcommand{\BB}{\mathcal{B}}
\newcommand{\CC}{\mathcal{C}}
\newcommand{\DD}{\mathcal{D}}
\newcommand{\EE}{\mathcal{E}}
\newcommand{\FF}{\mathcal{F}}
\newcommand{\GG}{\mathbb{G}}
\newcommand{\HH}{\mathbb{H}}
\newcommand{\calH}{\mathcal{H}}
\newcommand{\II}{\mathcal{I}}
\newcommand{\JJ}{\mathcal{J}}
\newcommand{\KK}{\mathcal{K}}
\newcommand{\LL}{\mathcal{L}}
\newcommand{\MM}{\mathcal{M}}
\newcommand{\NN}{\mathcal{N}}
\newcommand{\OO}{\mathrm{O}}
\newcommand{\PP}{\mathcal{P}}
\newcommand{\QQ}{\mathcal{Q}}
\newcommand{\RR}{\mathcal{R}}
\renewcommand{\SS}{\mathcal{S}}
\newcommand{\TT}{\mathcal{T}}
\newcommand{\UU}{\mathcal{U}}
\newcommand{\VV}{\mathcal{V}}
\newcommand{\WW}{\mathcal{W}}
\newcommand{\XX}{\mathcal{X}}
\newcommand{\YY}{\mathcal{Y}}
\newcommand{\ZZ}{\mathcal{Z}}
\renewcommand{\d}{\textrm{d}}
% Greek letters
\newcommand{\ep}{\varepsilon}
\newcommand{\ph}{\varphi}
\newcommand{\de}{\delta}
\renewcommand{\a}{\alpha}
\renewcommand{\b}{\beta}
% Fraktur
\newcommand{\mm}{\mathfrak{m}}
\renewcommand{\aa}{\mathfrak{a}}
\newcommand{\bb}{\mathfrak{b}}
\newcommand{\pp}{\mathfrak{p}}
\newcommand{\qq}{\mathfrak{q}}
% Operators
\DeclareMathOperator{\Div}{div}
\DeclareMathOperator{\Gal}{Gal}
\DeclareMathOperator{\vol}{Vol}
\DeclareMathOperator{\Hom}{Hom}
\DeclareMathOperator{\End}{End}
\DeclareMathOperator{\Ext}{Ext}
\DeclareMathOperator{\Tor}{Tor}
\DeclareMathOperator{\tr}{tr}
\DeclareMathOperator{\rk}{rk}
\DeclareMathOperator{\curl}{curl}
\DeclareMathOperator{\mesh}{mesh}
\DeclareMathOperator{\im}{im}
\DeclareMathOperator{\coker}{coker}
\DeclareMathOperator{\width}{width}
\DeclareMathOperator{\diam}{diam}
\DeclareMathOperator{\maps}{Maps}
\DeclareMathOperator{\Frac}{Frac}
\DeclareMathOperator{\Sym}{Sym}
\DeclareMathOperator{\sgn}{sgn}
\DeclareMathOperator{\alt}{Alt}
\DeclareMathOperator{\supp}{supp}
\DeclareMathOperator{\Span}{span}
\DeclareMathOperator{\Var}{Var}
\DeclareMathOperator{\Spec}{Spec}

\newcommand{\nor}{\unlhd}
\DeclareMathOperator{\aut}{Aut}
\DeclareMathOperator{\orb}{Orb}
\DeclareMathOperator{\GL}{GL}
\DeclareMathOperator{\SL}{SL}
\DeclareMathOperator{\SO}{SO}
\DeclareMathOperator{\PGL}{PGL}
\DeclareMathOperator{\PSL}{PSL}
\DeclareMathOperator{\stab}{Stab}
\DeclareMathOperator{\fix}{Fix}
\DeclareMathOperator{\Th}{Th}
\DeclareMathOperator{\Ind}{Ind}
\DeclareMathOperator{\Res}{Res}
\DeclareMathOperator{\Ann}{Ann}
\DeclareMathOperator{\rad}{rad}
\DeclareMathOperator{\len}{len}
\DeclareMathOperator{\ord}{ord}

% \DeclareMathOperator{\arg}{arg}

%% misc
\newcommand{\<}{\langle}
\renewcommand{\>}{\rangle}
\renewcommand{\^}{\wedge}
\renewcommand{\v}{\vee}
\def\Xint#1{\mathchoice
	{\XXint\displaystyle\textstyle{#1}}%
	{\XXint\textstyle\scriptstyle{#1}}%
	{\XXint\scriptstyle\scriptscriptstyle{#1}}%
	{\XXint\scriptscriptstyle\scriptscriptstyle{#1}}%
	\!\int}
\def\XXint#1#2#3{{\setbox0=\hbox{$#1{#2#3}{\int}$ }
		\vcenter{\hbox{$#2#3$ }}\kern-.6\wd0}}
\def\ddashint{\Xint=}
\def\dashint{\Xint-}
%% arrows
\newcommand{\xhra}{\xhookrightarrow}
\newcommand{\xra}{\xrightarrow}
\newcommand{\ra}{\rightarrow}
\newcommand{\rra}{\rightrightarrows}
\newcommand{\lra}{\longrightarrow}
\newcommand{\Ra}{\Rightarrow}
\newcommand{\lRa}{\Longrightarrow}
\newcommand{\lrsa}{\leftrightsquiqarrow}
\newcommand{\ba}{\leftrightarrow}
%% lists
\newcommand{\be}{\begin{enumerate}[(i)]}
	\newcommand{\ee}{\end{enumerate}}
%% integration stuff
\newcommand{\calR}{\mathcal{R}}
\newcommand{\rint}{\calR\!\int}
\newcommand{\calL}{\mathcal{L}}
\newcommand{\lowerint}{\mbox{\b{$\int$}}}
\newcommand{\upperint}{{\textstyle\bar{\int}}}
%% end of proof
\def\endproof{{\hfill $\Box$}}
%% matrix shorthand

\title{PHYS 132 HW \#1}
\author{Jalen Chrysos}

\begin{document}
	
	\maketitle

\textbf{Problem 11}: \\

(a): The initial force experienced by the proton is $F = k q_1 q_2 / r^2 = k(q^2)/(0.0025^2)$ where $q$ is the charge of a proton. By $F=ma$, we have
$$
a = F/m = \frac{kq^2}{0.0025^2 \cdot m} = \frac{(9\cdot 10^9) (1.6 \cdot 10^{-19})^2}{(2.5 \cdot 10^{-3})^2 \cdot 1.67 \cdot 10^{-27}} = 2.2 \cdot 10^{4} m/s^2.
$$

(b): The force will decrease over time as the protons get further apart, so acceleration will decrease as well. But it will always be positive, so velocity will always increase, though at a decreasing rate. I hope this description is sufficient.\\

\textbf{Problem 16}: $q_3$ must be to the left of $q_1$ to make $q_1$ accelerate to the left, so we can let the position of $q_3$ be $-x$ where $x>0$. Then the total force on $q_1$ can be calculated by Coulomb's Law as
$$
F = \frac{kq_1q_2}{0.2^2} - \frac{kq_1q_3}{x^2}
$$
using the known values for $F,q_1,q_2,q_3$ gives
\begin{align*}
	-7 &= (9 \cdot 10^{-9})(3\cdot 10^{-6}) \bigg(\frac{(5\cdot 10^{-6})}{0.2^2} - \frac{(8 \cdot 10^{-6})}{x^2}\bigg)\\
	-7 &= 2.7 \cdot 10^{-14} \Big((1.25 \cdot 10^{-4}) - (8 \cdot 10^{-6})/x^2\Big)\\
	-2.6 \cdot 10^{14} &= (1.25 \cdot 10^{-4}) - (8 \cdot 10^{-6})/x^2\\
	-(8 \cdot 10^{-6})/x^2 &= 1.25 \cdot 10^{-4} - 2.6 \cdot 10^{14}\\
	x^2 &= -\frac{8 \cdot 10^{-6}}{1.25 \cdot 10^{-4} - 2.6 \cdot 10^{14}}\\
	x^2 &= 3.1 \cdot 10^{-20}\\
	x &= 1.8 \cdot 10^{-10}.\\
\end{align*}

\textbf{Problem 20}: \\

(a): Let $q,m$ be the charge and mass of a proton and $E$ the electric field. The force on the proton is $Eq$ and its acceleration is $Eq/m$. Since it begins at rest, its average speed over time $t$ is $t(Eq/m)/2$, and thus its displacement is $t^2(Eq/m)/2$, giving
$$
1.60 \cdot 10^{-2} = E \cdot (3.20 \cdot 10^{-6}) (q/2m) \implies E= 1.04 \cdot 10^{-4} N.
$$

(b): The final speed is twice the average speed, which is also $2d/t$, giving
 $$v = 2(1.60\cdot 10^{-2}) / (3.20\cdot 10^{-6}) = 10^4 m/s$$

\textbf{Problem 31}:\\

The force locally at any given point $(x,0)$ on the rod is given by Coulomb's Law as
$$
F = \frac{k(-2 \cdot 10^{-6})(4.8 \cdot 10^{-9})}{x^2+0.05^2}
$$
from which the $y$-component is 
$$
F_y = \frac{k(-2 \cdot 10^{-6})(4.8 \cdot 10^{-9})(0.05)}{(x^2+0.05^2)^{3/2}}
$$
By symmetry over the $y$ axis, the $x$ components of force cancel, so we only need to integrate $F_y$ over the rod, giving
$$
\int_{-0.1}^{0.1} \frac{k(-2 \cdot 10^{-6})(4.8 \cdot 10^{-9})(0.05)}{(x^2+0.05^2)^{3/2}} \; \d x = -3.1\cdot 10^{-3} N.
$$
This force is pulling the rod upward along the $+y$ axis.\\

\textbf{Problem 44}:\\

In the $y$ direction, the $q_1$ charge contributes no force and the $q_2$ charge contributes
$$
\frac{k(-1)(-4 \cdot 10^{-9})}{1^2} \cdot \frac{0.6}{1} = (9\cdot 10^9)(4\cdot 10^{-9})(0.6) = 21.6 N
$$
in the positive $y$ direction. In the $x$ direction, there are contributions from $q_1$ and $q_2$, giving
$$
\frac{k(-1)(-4 \cdot 10^{-9})}{1^2} \cdot \frac{0.8}{1} + \frac{k(-1)(6\cdot 10^{-9})}{0.6^2} = -121.2 N
$$
in the negative $x$ direction.\\

\textbf{Problem 54}: \\

Using the same method as example 21.14, we can approximate the electric field as $2pk/x^3$, and $F=Eq$, which gives
$$
F = \frac{2pkq}{x^3} = \frac{2\cdot (6.17\cdot 10^{-30}) \cdot (9\cdot 10^9)\cdot (-1.6\cdot 10^{-19})}{(3\cdot 10^{-9})^3} = -6.6 \cdot 10^{-13}
$$
and the charge will be pulled in the $-x$ direction.\\

\textbf{Problem 59}:\\

(a): In the $x$ component, the charge $q_1$ contributes
$$
F = \frac{kq_1q_2}{0.05^2}\cdot \frac{0.04}{0.05} = \frac{(9\cdot 10^9)(5\cdot 10^{-9})(6\cdot 10^{-9})}{0.05^2}\cdot \frac{0.04}{0.05} =  8.6 \cdot 10^{-5}
$$
in the $+x$ direction. In the $y$ component, both charges contribute, giving
$$
F = \frac{k(5\cdot 10^{-9})(6\cdot 10^{-9})}{0.05^2}\cdot \frac{0.03}{0.05} + \frac{k(-2\cdot 10^{-9})(6\cdot 10^{-9})}{0.03^2} = -5.5 \cdot 10^{-5}
$$
in the $-y$ direction.\\

(b): The total magnitude of this force is 
$$
\sqrt{( 8.6 \cdot 10^{-5})^2 + ( -5.5 \cdot 10^{-5})^2} = 1.0 \cdot 10^{-4}
$$
and its direction is $\arctan(-5.5/8.6) =-32.6$ degrees (with the positive $x$ axis being $0$ degrees). \\

\textbf{Problem 79}:\\

(a): The electric field can be computed by the integral
$$
\int_0^a \frac{k(Q/a)}{(x-t)^2} \; \d t = \frac{kQ}{a}\Big[-\frac{1}{3} (x-t)^3\Big]^a_0 = \frac{kQ}{3a}\Big(x^3 - (x-a)^3\Big).
$$

(b): The force on $q$ is 
$$F = Eq = \frac{kQq}{3a}\Big(x^3-r^3\Big).$$
and in the positive $x$ direction.

(c): When $r$ is large, $(r+a)^3-r^3 = 3r^2a + 3ra^2 + a^3$ is asymptotically close to $3r^2a$ (since $r\gg a$, making $3r^2a$ the dominant term), which gives
$$
F \approx \frac{kQq}{3a}\Big(3r^2 a\Big) = \frac{kQq}{r^2}.
$$
This makes sense because when we zoom out the rod looks like a point charge $Q$, giving back Coulomb's Law.


\end{document}