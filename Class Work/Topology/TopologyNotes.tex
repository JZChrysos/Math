\documentclass{amsart}
%\documentclass{amsart}
\usepackage[utf8]{inputenc}
\usepackage{amsfonts}
\usepackage{amsmath}
\usepackage{amssymb}
\usepackage{amsthm}
\usepackage{asymptote}
\usepackage{mathtools}
\usepackage{hhline}
\usepackage{graphicx,enumerate}
\usepackage{hyperref}
\usepackage[a4paper, margin=1.2in]{geometry}
%\usepackage{tcolorbox}
\usepackage{tikz-cd}
\usepackage{ytableau}
%\tcbuselibrary{skins,breakable,xparse}
\allowdisplaybreaks
\newcounter{count}
\hypersetup{
	colorlinks=true,
	linkcolor=teal,
	filecolor=magenta,      
	urlcolor=olive,
	citecolor=teal,
	pdfpagemode=FullScreen,
}

%\definecolor{defcolor}{HTML}{478EFF}
%\definecolor{thmcolor}{HTML}{CC0058}
%\definecolor{excolor}{HTML}{F5B400}
%\definecolor{probcolor}{HTML}{DD4803}
%\definecolor{lemcolor}{HTML}{741FEA}
%\definecolor{scarlet}{HTML}{A81111}
%
%\newtheoremstyle{definitionStyle}% Custom style for definitions
%{0.5em}% Space above
%{0.5em}% Space below
%{}% Body font
%{}% Indent amount
%{\bfseries\color{defcolor}}% Theorem head font: bold and red
%{.\\}% Punctuation after theorem head
%{0.5em}% Space after theorem head
%{\thmname{#1}\thmnumber{ #2 (#3)}}% Theorem head spec
%
%\theoremstyle{definitionStyle}
%\newtheorem{df}{Definition}[section]
%
%\newtheoremstyle{theoremStyle}% Custom style for definitions
%{0.5em}% Space above
%{0.5em}% Space below
%{}% Body font
%{}% Indent amount
%{\bfseries\color{thmcolor}}% Theorem head font: bold and red
%{.\\}% Punctuation after theorem head
%{0.5em}% Space after theorem head
%{\thmname{#1}\thmnumber{ #2 (#3)}}% Theorem head spec
%
%\theoremstyle{theoremStyle}
%\newtheorem{thm}{Theorem}[section]
%
%\newtheoremstyle{lemmaStyle}% Custom style for definitions
%{0.5em}% Space above
%{0.5em}% Space below
%{}% Body font
%{}% Indent amount
%{\bfseries\color{lemcolor}}% Theorem head font: bold and red
%{.\\}% Punctuation after theorem head
%{0.5em}% Space after theorem head
%{\thmname{#1}\thmnumber{ #2 (#3)}}% Theorem head spec
%
%\theoremstyle{lemmaStyle}
%\newtheorem{lem}{Lemma}[section]
%\newtheorem{cor}{Corollary}[section]
%
%\newtheoremstyle{exampleStyle}% Custom style for definitions
%{0.5em}% Space above
%{0.5em}% Space below
%{}% Body font
%{}% Indent amount
%{\bfseries\color{excolor}}% Theorem head font: bold and red
%{.\\}% Punctuation after theorem head
%{0.5em}% Space after theorem head
%{\thmname{#1}\thmnumber{ #2 (#3)}}% Theorem head spec
%
%\theoremstyle{exampleStyle}
%\newtheorem{ex}{Example}[section]
%
%\newtheoremstyle{problemStyle}% Custom style for definitions
%{0.5em}% Space above
%{0.5em}% Space below
%{}% Body font
%{}% Indent amount
%{\bfseries\color{probcolor}}% Theorem head font: bold and red
%{.\\}% Punctuation after theorem head
%{0.5em}% Space after theorem head
%{\thmname{#1}\thmnumber{ #2#3}}% Theorem head spec
%
%\theoremstyle{problemStyle}
%\newtheorem{prob}{Problem}[section]

% For Fun
\newcommand{\club}{\color{teal} \clubsuit}
\newcommand{\heart}{\color{red} \heartsuit}
\renewcommand{\star}{\color{scarlet} \bigstar}
\newcommand{\spade}{\color{violet} \spadesuit}

% Symbols
\newcommand{\A}{\mathcal{A}}
\newcommand{\B}{\mathcal{B}}
\newcommand{\C}{\mathbb{C}}
\newcommand{\D}{\mathcal{D}}
\newcommand{\E}{\mathbb{E}}
\newcommand{\F}{\mathbb{F}}
\newcommand{\G}{\mathcal{G}}
% \renewcommand{\H}{\mathcal{H}} Erdos o
\newcommand{\I}{\mathcal{I}}
\newcommand{\J}{\mathcal{J}}
\newcommand{\K}{\mathcal{K}}
% \renewcommand{\L}{\mathcal{L}}
\newcommand{\M}{\mathcal{M}}
\newcommand{\N}{\mathbb{N}}
\renewcommand{\O}{\mathcal{O}}
\renewcommand{\P}{\mathbb{P}}
\newcommand{\Q}{\mathbb{Q}}
\newcommand{\R}{\mathbb{R}}
\renewcommand{\S}{\mathbb{S}}
\newcommand{\T}{\mathbb{T}}
\newcommand{\U}{\mathcal{U}}
\newcommand{\V}{\mathcal{V}}
\newcommand{\W}{\mathcal{W}}
\newcommand{\X}{\mathcal{X}}
\newcommand{\Y}{\mathcal{Y}}
\newcommand{\Z}{\mathbb{Z}}

\renewcommand{\AA}{\mathcal{A}}
\newcommand{\BB}{\mathcal{B}}
\newcommand{\CC}{\mathcal{C}}
\newcommand{\DD}{\mathcal{D}}
\newcommand{\EE}{\mathcal{E}}
\newcommand{\FF}{\mathcal{F}}
\newcommand{\GG}{\mathbb{G}}
\newcommand{\HH}{\mathbb{H}}
\newcommand{\calH}{\mathcal{H}}
\newcommand{\II}{\mathcal{I}}
\newcommand{\JJ}{\mathcal{J}}
\newcommand{\KK}{\mathcal{K}}
\newcommand{\LL}{\mathcal{L}}
\newcommand{\MM}{\mathcal{M}}
\newcommand{\NN}{\mathcal{N}}
\newcommand{\OO}{\mathrm{O}}
\newcommand{\PP}{\mathcal{P}}
\newcommand{\QQ}{\mathcal{Q}}
\newcommand{\RR}{\mathcal{R}}
\renewcommand{\SS}{\mathcal{S}}
\newcommand{\TT}{\mathcal{T}}
\newcommand{\UU}{\mathcal{U}}
\newcommand{\VV}{\mathcal{V}}
\newcommand{\WW}{\mathcal{W}}
\newcommand{\XX}{\mathcal{X}}
\newcommand{\YY}{\mathcal{Y}}
\newcommand{\ZZ}{\mathcal{Z}}
\renewcommand{\d}{\textrm{d}}
% Greek letters
\newcommand{\ep}{\varepsilon}
\newcommand{\ph}{\varphi}
\newcommand{\de}{\delta}
\renewcommand{\a}{\alpha}
\renewcommand{\b}{\beta}
% Fraktur
\newcommand{\mm}{\mathfrak{m}}
\renewcommand{\aa}{\mathfrak{a}}
\newcommand{\bb}{\mathfrak{b}}
\newcommand{\pp}{\mathfrak{p}}
\newcommand{\qq}{\mathfrak{q}}
% Operators
\DeclareMathOperator{\Div}{div}
\DeclareMathOperator{\Gal}{Gal}
\DeclareMathOperator{\vol}{Vol}
\DeclareMathOperator{\Hom}{Hom}
\DeclareMathOperator{\End}{End}
\DeclareMathOperator{\Ext}{Ext}
\DeclareMathOperator{\Tor}{Tor}
\DeclareMathOperator{\tr}{tr}
\DeclareMathOperator{\rk}{rk}
\DeclareMathOperator{\curl}{curl}
\DeclareMathOperator{\mesh}{mesh}
\DeclareMathOperator{\im}{im}
\DeclareMathOperator{\coker}{coker}
\DeclareMathOperator{\width}{width}
\DeclareMathOperator{\diam}{diam}
\DeclareMathOperator{\maps}{Maps}
\DeclareMathOperator{\Frac}{Frac}
\DeclareMathOperator{\Sym}{Sym}
\DeclareMathOperator{\sgn}{sgn}
\DeclareMathOperator{\alt}{Alt}
\DeclareMathOperator{\supp}{supp}
\DeclareMathOperator{\Span}{span}
\DeclareMathOperator{\Var}{Var}
\DeclareMathOperator{\Spec}{Spec}

\newcommand{\nor}{\unlhd}
\DeclareMathOperator{\aut}{Aut}
\DeclareMathOperator{\orb}{Orb}
\DeclareMathOperator{\GL}{GL}
\DeclareMathOperator{\SL}{SL}
\DeclareMathOperator{\SO}{SO}
\DeclareMathOperator{\PGL}{PGL}
\DeclareMathOperator{\PSL}{PSL}
\DeclareMathOperator{\stab}{Stab}
\DeclareMathOperator{\fix}{Fix}
\DeclareMathOperator{\Th}{Th}
\DeclareMathOperator{\Ind}{Ind}
\DeclareMathOperator{\Res}{Res}
\DeclareMathOperator{\Ann}{Ann}
\DeclareMathOperator{\rad}{rad}
\DeclareMathOperator{\len}{len}
\DeclareMathOperator{\ord}{ord}

% \DeclareMathOperator{\arg}{arg}

%% misc
\newcommand{\<}{\langle}
\renewcommand{\>}{\rangle}
\renewcommand{\^}{\wedge}
\renewcommand{\v}{\vee}
\def\Xint#1{\mathchoice
	{\XXint\displaystyle\textstyle{#1}}%
	{\XXint\textstyle\scriptstyle{#1}}%
	{\XXint\scriptstyle\scriptscriptstyle{#1}}%
	{\XXint\scriptscriptstyle\scriptscriptstyle{#1}}%
	\!\int}
\def\XXint#1#2#3{{\setbox0=\hbox{$#1{#2#3}{\int}$ }
		\vcenter{\hbox{$#2#3$ }}\kern-.6\wd0}}
\def\ddashint{\Xint=}
\def\dashint{\Xint-}
%% arrows
\newcommand{\xhra}{\xhookrightarrow}
\newcommand{\xra}{\xrightarrow}
\newcommand{\ra}{\rightarrow}
\newcommand{\rra}{\rightrightarrows}
\newcommand{\lra}{\longrightarrow}
\newcommand{\Ra}{\Rightarrow}
\newcommand{\lRa}{\Longrightarrow}
\newcommand{\lrsa}{\leftrightsquiqarrow}
\newcommand{\ba}{\leftrightarrow}
%% lists
\newcommand{\be}{\begin{enumerate}[(i)]}
	\newcommand{\ee}{\end{enumerate}}
%% integration stuff
\newcommand{\calR}{\mathcal{R}}
\newcommand{\rint}{\calR\!\int}
\newcommand{\calL}{\mathcal{L}}
\newcommand{\lowerint}{\mbox{\b{$\int$}}}
\newcommand{\upperint}{{\textstyle\bar{\int}}}
%% end of proof
\def\endproof{{\hfill $\Box$}}
%% matrix shorthand

\title{Topology Notes}
\author{Jalen Chrysos}

\begin{document}
	
	\maketitle
	\begin{abstract}
		These are my notes for Topology I-II-III (Math 317-319) at UChicago, 2025-2026.
	\end{abstract}
	
	\tableofcontents
	
	\newpage
	
	\section*{Topology I with Danny Calegari}
	
	This is an Algebraic Topology course.\\
	
	Housekeeping:
	\begin{itemize}
		\item HW due Thursday midnight.
		\item Take-home midterm and final will replace HW.
		\item Textbook: \href{https://pi.math.cornell.edu/~hatcher/AT/AT.pdf}{Hatcher}.
		\item Collaboration is encouraged on homework (but give credit where it is due).
		\item Grades will be roughly 50\% homework 50\% exams, with some generous weighting.
		\item Office Hours: Thursday 5-6 p.m. in Eckhart E7 (basement).
	\end{itemize}
	
	
	\subsection{Homotopy}
	
	Rather than equivalence by homeomorphism, which is ``too fine to be useful,'' we'll use the coarser equivalence of homotopy.
	
	We'll also be looking at a lot of computable information about topological spaces. \\
	
	Suppose $f_0,f_1:X\to Y$ are two (continuous) maps between topological spaces $X$ and $Y$. We say $f_0,f_1$ are \textit{homotopic} if one can be continuously turned into the other, i.e. if there is a continuous map
	$$
	F:[0,1]\to \Hom(X,Y)
	$$
	for which $F(0)=f_0,F(1)=f_1$. Such an $F$ is a homotopy. We write $f_0\simeq f_1$.
	
	Two spaces $X$ and $Y$ are \textit{homotopy-equivalent} if there is a map $f:X\to Y$ that is an isomorphism ``up to homotopy,'' i.e. there is a map $g:Y\to X$ for which $f\circ g \simeq 1_Y$ and $g\circ f \simeq 1_X$. 
	
	Homotopy equivalence is indeed an equivalence relation (not too hard to show). Equivalence of maps is also stable under composition, which makes homotopy classes of spaces and maps a category.
	
	If $f_0,f_1:X\to Y$ and $A\subseteq X$ is a subset on which $f_0$ and $f_1$ agree, and additionally there is a homotopy $F$ which transforms $f_0$ into $f_1$ while remaining constant on $A$, then we say $f_0\simeq f_1$ relative to $A$. \\
	
	We say that a space $X$ is \textit{contractible} if it is homotopy-equivalent to a single point. For example, $\R^n$ is contractible, as constant maps on $\R^n$ are homotopic with the identity map by straight-line contraction. 
	
	Another example: given $f:X\to Y$, there is a \textit{mapping cylinder} $M_f$ which is $X\times [0,1] \coprod Y$ under the gluing equivalence $(x,1)\sim f(x)$. Then $M_f\simeq Y$ via the maps
	$$
	h_0: (x,t) \mapsto f(x), \;\;\; h_1 : y \mapsto y
	$$
	The thing that must be checked is that $h_1\circ h_0:M_f\to M_f$ is homotopy-equivalent to the identity on $M_f$. This is an example of \textit{deformation retraction}, which means that it is a homotopy relative to $Y$.\\
	
	\subsection{CW Complexes}
	
	General topology is difficult to say much about because of all the pathological cases. So we'll focus mainly on \textit{nice} topological spaces, and in particular \textit{CW-complexes}.\\
	
	A \textit{CW-complex} is built from cells of different dimensions and attaching maps. Each cell is a pair $(D^n,S^{n-1})$ consisting of a ball and its surface. We build up the complex by a ``skeleton'' $X_0\subseteq X_1\subseteq \dots$ where $X_n$ consists of all the cells of dimension at most $n$ and their gluing instructions. The attaching map $\ph$ for a cell maps its boundary $S^{n-1}$ into $X^{n-1}$. 
	
	The topology on a CW-complex is the \textit{weak topology} (no relation to functional analysis) which says that $A$ is open iff $A\cap X^n$ is open for all $n$.
	
	Examples:
	\begin{itemize}
		\item A 0-dimensional CW-complex is just a collection of discrete points.
		\item A 1-dimensional CW-complex is essentially a graph (with possibly loops and multiple edges).
		\item Klein bottle, torus, two-holed torus etc. all have presentations as 2-dim CW complexes.
		\item One can write $\C\P^n$ as the union of a 0-cell, a 2-cell, a 4-cell, $\dots$, and a $2n$-cell, where gluing takes the boundary of each to the infinite line of the previous.
	\end{itemize}
	
	Some operations on CW complexes:
	\begin{itemize}
		\item \textit{Product}: $X\times Y$ is given by the union of all products of a cell in $X$ and a cell in $Y$. Its topology as a CW-complex (i.e. the weak topology) is the same as the product topology in cases where there are only a \textit{countable} number of cells in each or if one is locally compact, but in general the topology is actually finer.
		\item \textit{Quotient}: $X/A$, where $A$ is a \textit{subcomplex} of $X$ (i.e. a closed union of cells in $X$) that is also \textit{contractible}, is given by the union of cells in $X-A$ plus an additional 0-cell representing the image of all cells in $A$. Such a pair $(X,A)$ is called a CW pair.
		\item \textit{Suspension}: $SX$ is $X\times [0,1]$ where $(X,0)$ is identified and $(X,1)$ is identified.
		\item \textit{Cone}: $CX$ is $X\times [0,1]$ where $(X,1)$ is identified.
		\item \textit{Join}: $X\ast Y$ is the space $X\times I \times Y$ quotiented such that all $(x,0,Y)$ are identified and all $(X,1,y)$ are identified. In the case $X=Y=[0,1]$, the resulting $X\ast Y$ looks like a tetrahedron. One can think of the points of $X\ast Y$ as pairs $(x,y)\in X\times Y$ along with a weight $t\in [0,1]$, such that $(x,y,0)=x$ and $(x,y,1)=y$.
		\item \textit{Wedge}: $X\vee Y$ is $X\coprod Y$ with two specific points $x$ and $y$ identified.
		\item \textit{Smash}: $X\wedge Y$ is $X\times Y$ with $X\vee Y$ all identified.
	\end{itemize}
	
	An important example of a CW complex obtained this way is the $n$-simplex, which is the join of $n$ discrete points. \\
	
	One thing to note about the quotient is that $X/A \simeq X$.
	
	A CW-complex $X$ is connected (and path-connected) iff $X^1$ is a connected graph. Thus, if $X$ is connected then we can give a spanning tree $T$ of its 1-skeleton $X^1$. Every tree is contractible, thus one can take the quotient $X/T \simeq X$.
	
	 Moreover, the quotient has a very simple structure in its low-dimension cells: $Y:= X/T$ has $Y^0$ a single point and $Y^1$ a wedge of some circles. So we've shown that one can always put a connected CW-complex into this nice form while preserving its homotopy class.\\
	 
	 If $(X,A)$ is a CW pair and $f:A\to Y$ is some map into another CW complex (or any topological space), then one can form the space
	 $$
	 X\cup_f Y := X\times Y / (a \sim f(a)).
	 $$
	 And if $f,g:A\to Y$ are two homotopy-equivalent maps, then $X\cup_fY \simeq X\cup_g Y$. This shows in particular that in the construction of CW complexes, the homotopy-type of the complex only depends on the homotopy-classes of the attaching maps.\\
	 
	 Both of these facts can be deduced from the \textit{Homotopy Extension Property} for CW-pairs (try this!). $(X,A)$ has the HEP if for all spaces $Y$, every map $f:X\times 0 \cup A\times I\to Y$ factors through the inclusion into $X\times I$:
	 $$
	 \begin{tikzcd}
	 	{X\times I} \\
	 	\\
	 	{X\times 0 \cup A\times I} && Y
	 	\arrow["{\exists g}", dashed, from=1-1, to=3-3]
	 	\arrow[hook', from=3-1, to=1-1]
	 	\arrow["f"', from=3-1, to=3-3]
	 \end{tikzcd}
	 $$
	 That is, a partial homotopy $f:A\to Y$ can always be extended to a homotopy $g:X\to Y$, hence the name. The HEP is equivalent to the specific case for $f$ the identity map on $X\times 0 \cup A\times I$. Thus, to prove the HEP for CW-pairs, it suffices to show the following: \\
	 
	 \textbf{Proposition}: If $(X,A)$ is a CW pair then there is a retraction from $X\times I$ to $X\times 0 \cup A\times I$.
	 \begin{proof}
	 	If $X$ has dimension $n$, then $X=X^n$. We will produce by a series of retractions:
	 	$$
	 	X\times I  = X^n\times I \cup A \times I \to (X\cup 0)\times (X^{n-1}\times I \cup A \times I) \to (X\cup 0)\times (X^{n-2}\times I \cup A \times I) \to \dots
	 	$$
	 	In each step we only need to retract every $j$-cell onto its boundary. We can do this because it has an \textit{open side}. (check Hatcher to get the details straight later).
	 \end{proof}\\
	 
	 \subsection{The Fundamental Group} Let $X$ be a space. A path $f$ in $X$ is a map $I\to X$. A homotopy between paths $f,g$ is a homotopy (in the sense defined before) which fixes the endpoints of the paths (so it must be that $f(0)=g(0)$ and $f(1)=g(1)$ for this to be possible). We say that $f,g$ are homotopy-equivalent if one exists.
	 
	 Two paths can be composed (concatenated) if the end point of one is the start point of the other. This is denoted $f\ast g$, and corresponds to a path which does $f$ from $[0,\tfrac12]$ and then does $g$ from $[\tfrac12,1]$. If $f$ and $g$ are both \textit{loops} with $f\simeq f'$ and $g\simeq g'$, then 
	 $$
	 f\ast g \simeq f'\ast g'.
	 $$
	 This can be proven by drawing a picture. Basically the homotopies $f\to f'$ and $g\to g'$ can be concatenated.\\
	 
	 The \textit{fundamental group} of $X$, denoted $\pi(X,x)$, is made up of homotopy-classes of loops beginning and ending at $x\in X$. The operation is concatenation. The identity is given by the constant map and the inverse is given by $f^{-1}(t) := f(1-t)$. We can check that this is a genuine inverse by drawing a picture.
	 
	 We also have to check that $\ast$ is associative, i.e. $f\ast (g\ast h) \simeq (f\ast g)\ast h$. This can also be shown by a simple picture (we're essentially just changing the rate of movement along the image of the path in different segments).
	 
	 If $\pi(X,x)$ is trivial, we say $X$ is \textit{simply connected} (note that this does not depend on $x$). In general, the fundamental group only depends on the path-connected component of $X$ in which $x$ lies. If there is a path $\b:x\to y$ in $X$ then $\pi(X,x)$ is just $\b^{-1} \pi(X,y) \b$. This gives a group isomorphism between $\pi(X,x)$ and $\pi(X,y)$.\\
	 
	 Any map $f:X\to Y$ induces a group homomorphism between the fundamental groups:
	 $$
	 f_{*}:\pi(X,x)\mapsto \pi(Y,f(x))
	 $$
	 given by $f_{*}:\a \mapsto f\circ \a$. If $f\simeq g:X\to Y$, then $f_*$ and $g_*$ differ by an inner automorphism. Suppose $f,g$ are homotopic via $F:X\times I \to Y$, and let $\b(t) = F(x,t)$. Then $f_* = \b^{-1}g_*\b$. And in particular, if $f(x)=g(x)$, $\b$ is the constant path at $x$, so $f_* = g_*$. 
	 
	 If $X,Y$ are homotopy-equivalent and path-connected, then their fundamental groups are isomorphic: the composition
	 $$
	 (X,x)\xrightarrow{f} (Y,f(x))\xrightarrow{g} (X,g\circ f(x))
	 $$
	 is an isomorphism up to homotopy equivalence, therefore $(X,x)$ and $(Y,f(x))$ have isomorphic fundamental group.\\
	 
	 An example: let $T^n := (S^1)^n$. Or equivalently, $T^n = \R^n/\Z^n$. $\R^n$ is a covering space of $T^n$. The fundamental group of $T^n$ is $\Z^n$. $\GL_n(\Z)$ acts on $T^n$ in a natural way (these are outer automorphisms).\\
	 
	 \subsection{Covering Spaces} $\hat{X}$ is a \textit{covering space} of $X$ when there is a map $p: \hat{X}\to X$ such that for every $x\in X$ there is a neighborhood $U\subset X$ such that $p^{-1}(U)$ is homeomorphic to a union of disjoint copies of $U$ in $\hat{X}$. We say that such a $U$ is \textit{evenly covered} by $\hat{X}$.\\
	 
	 Examples:
	 \begin{itemize}
	 \item The classic example is that the infinite helix covers $S^1$ by projection. 
	 \item One could also cover $S^1$ by $S^1$ with a map $z\mapsto z^n$. The order of the covering is $n$.
	 \item $\R^n/\Z^n$ (the $n$-torus) is covered by $\R^n$ in the natural way.
	 \item Graphs can be covered by other graphs. The only thing that must be obeyed by the covering space is the local behavior near vertices.
	 \end{itemize}
	 
	 \noindent \textbf{Homotopy Lifting Property}: Let $p:\hat{X}\to X$ be a covering of $X$. If $f_t$ is a homotopy from $Y$ to $X$, then there is a lifting of $f_t$ to a homotopy $\hat{f_t}$ from $Y$ to $\hat{X}$, and $\hat{f_t}$ is \textit{uniquely determined} by $\hat{f_0}$.
	 
	 \begin{proof}
	 	Let $y\in Y$. $f_t(y)$ for $t\in [0,1]$ is a path in $X$. This path is covered by finitely many neighborhoods which are evenly covered by $\hat{X}$. Given $\hat{f_0}$, we have a single preimage set that we must choose for $\hat{f_0}(y)$. Now, to choose $\hat{f_t}(y)$ for the next $t\in [0,1]$, we choose the preimage set which intersects with the previous, etc. The fact that $p:\hat{X}\to X$ is continuous guarantees that we can do this and result in a continuous path in $\hat{X}$.
	 \end{proof}\\
	 
	 In the case that $Y$ is a single point, this shows that individual paths can always be lifted through coverings. It is important to note that loops may not remain loops when they are lifted, since the first and last points may both map to $x$ while not being the same.
	 
	 In the case $Y=[0,1]$, this is saying that entire homotopies of paths can be lifted. In contrast to the case of paths in general, two paths that are \textit{homotopic} will maintain their homotopy (and as a result their endpoints will stay together) when lifted. This implies that loops that are the identity element in $\pi(X,x)$ stay as the identity element in $\pi(\hat{X},p^{-1}(x))$.\\
	 
	 As the previous example implies, we have a correspondence between the fundamental groups
	 $$
	 p_*: \pi_1(\hat{X},\hat{x}) \to \pi_1(X,x).
	 $$	 
	 The important property to know here is that $p_*$ is \textit{injective}. That is, we can think of $\pi_1(\hat{X},\hat{x})$ as a \textit{subgroup} of $\pi_1(X,x)$.
	 
	 To show this, we'll first show that if $p_*(\a)=1$ then $\a=1$. This follows from the preceding discussion about homotopy lifting: the homotopy $p(\a)\simeq 1_X$ lifts to a homotopy between $\a$ and $1_{\hat{X}}$.\\
	 
	 \textbf{Lifting Criterion}: Let $Y$ be path connected and locally-path connected (CW complex suffices). Given $f:(Y,y)\to (X,x)$ with a covering $p:(\hat{X},\hat{x})\to (X,x)$, when is there a lift $\hat{f}:(Y,y)\to (\hat{X},\hat{x})$? The answer is that it exists iff the image of 
	 $$f_*:\pi_1(Y,y)\to \pi_1(X,x)$$
	 lands inside the subgroup $\pi_1(\hat{X},\hat{x})$. It is clearly necessary, but the interesting thing is that it's sufficient.
	 
	 \begin{proof}
	 	We construct $\hat{f}(z)$ by lifting $f(z)$ through $p$. The thing that needs to be checked is that it preserves homotopy classes of paths. \textbf{fill in later} 
	 \end{proof}\\
	 
	 \textbf{Classification of Covering Spaces:} If $X$ satisfies some basic properties (which connected CW complexes satisfy) then covering spaces $\hat{X}$ correspond exactly with subgroups of $\pi_1(X,x)$.\\
	 
	 To show this we will construct a \textit{Universal Cover} of $X$, which we denote $\tilde{X}$, whose fundamental group is trivial. All other coverings will appear as quotients of this universal cover.
	 
	 \begin{proof}
	 	\textbf{Step 1}: We define $\tilde{X}$ as the set of homotopy classes of paths starting at $x$. The covering map comes from
	 	$$
	 	p: \gamma \mapsto \gamma(1).
	 	$$
	 	The topology of $\tilde{X}$ will have open sets 
	 	$$
	 	U_{[\gamma]} := \{[\gamma \cdot \eta] : \eta \text{ is a path in $U$ extending $\gamma$}\}.
	 	$$
	 	for all \textit{simply-connected} open sets $U\subset X$.
	 	
	 	$\tilde{X}$ is path-connected because every $\gamma$ (i.e. every point in $\tilde{X}$) can be contracted to the identity.
	 	
	 	Now to show $\tilde{X}$ is simply-connected. (\textit{omitted}).\\
	 	
	 	\textbf{Step 2}: A \textit{Deck transformation} is a specific case of lifting in which the space $Y$ is \textit{also} $\hat{X}$, but with a different basepoint. The Deck transformations form a group under composition, called the \textit{Deck group} of $\hat{X}$, which we denote $G(\hat{X})$.
	 	
	 	We can show that that 
	 	$$G(\hat{X}) = N(\pi_1(\hat{X},\hat{x}))/\pi_1(\hat{X},\hat{x}).$$ This implies, in the case of the universal cover we just constructed, that the $G(\tilde{X}) = \pi_1(X,x)$. Now for any subgroup $H$ of $\pi_1(X,x)$, we can look at $H$ as a subgroup of $G(\tilde{X})$, and thus take the quotient 
	 	$$
	 	X_H := \tilde{X}/H
	 	$$ 
	 	for which $\pi_1(X_H,x)=\pi_1(X,x)/H$. It remains to show that this $X_H$ is actually a covering space of $X$. (\textit{omitted}).
	 \end{proof}\\
	 
	 Examples:
	 \begin{itemize}
	 	\item In the covering of $S^1$ by $\R/\Z$, the deck group is $\Z$ and thus $\pi_1(S^1)=\Z$.
	 	\item For spaces $X,Y$ with universal covers $\tilde{X}$ and $\tilde{Y}$, $\tilde{X}\times \tilde{Y}$ is also the universal cover for $X\times Y$. Thus 
	 	$$
	 	\pi_1(X\times Y)= \pi_1(X)\oplus \pi_1(Y).
	 	$$
	 	This implies that $\pi_a(T^n)=\Z^n$.
	 	\item Let $\Gamma$ be a 1-dimensional connected CW complex (a connected graph). Up to homotopy we can assume that $\Gamma$ is a wedge of $n$ circles for some $n$. To calculate the fundamental group of $\Gamma$ then, we get the free group on $n$ generators. The universal covering is an infinite fractal tree. Ping Pong Lemma.
	 \end{itemize}
	 
	 Incidentally, our discussion so far proves that any subgroup of a free group is free, a highly nontrivial fact in group theory. Every subgroup of a free group $\<a,b\>$ has some generators in terms of $a,b$, and one can make a graph with these generators as loops. The resulting CW complex is homotopy-equivalent to a wedge of circles, and thus a free group.\footnote{Hatcher p. 58 has an excellent table of diagrams like this.}\\
	 
	 \textbf{Van Kampen Theorem}: Let $X=A\cup B$ where $A,B$ are both open in $X$, with basepoint $x\in A\cap B$. Assume $A,B,A\cup B$ are all path-connected. Then $\pi_1(X,x)$ is freely generated by $\pi_1(A,x),\pi_1(B,x)$ under the identification of $\pi_1(A\cap B,x)$. That is, it is the group such that the following diagram commutes for all groups $G$ such that the outer square commutes.
	 $$
	 \begin{tikzcd}
	 	& {\pi_1(A,x)} \\
	 	\\
	 	{\pi_1(A\cap B,x)} && {\pi_1(X,x)} && G \\
	 	\\
	 	& {\pi_1(B,x)}
	 	\arrow[from=1-2, to=3-3]
	 	\arrow[from=1-2, to=3-5]
	 	\arrow[from=3-1, to=1-2]
	 	\arrow[from=3-1, to=5-2]
	 	\arrow["{\exists!}"', from=3-3, to=3-5]
	 	\arrow[from=5-2, to=3-3]
	 	\arrow[from=5-2, to=3-5]
	 \end{tikzcd}
	 $$
	 Another way to say this is that it's the simplest group that $\pi_1(A,x),\pi_1(B,x)$ can both map into while commuting with the inclusions from $A\cap B$. We can explicitly determine $\pi_1(X,x)$ as 
	 $$
	 \pi_1(X,x)\cong \pi_1(A,x)\ast \pi_1(B,x) / \<\< \iota_A(w) = \iota_B(w)\>\>
	 $$
	 where $\<\<\bullet \>\>$ denotes the normal subgroup generated by the enclosed relations.
	 
	 The same will hold for more than two (even an arbitrary infinite family) open spaces, as long that any \textit{three} of them have path-connected intersection.
	 \begin{proof}
	 	There is a map $\pi_1(A,x)\ast \pi_1(B,x)\to \pi_1(X,x)$ by inclusion. We will show that it is surjective and that its kernel is exactly the words $w$ in $\pi_1(A\cap B,x)$ for which $\iota_A(w)=\iota_B(w)$.\\
	 	
	 	Step 1: Let $\a:[0,1]\to X$ be a loop at $x$ in $X$. We want to show that $\a$ is homotopy equivalent (rel endpoints) to a path which is a concatenation of paths entirely in $A$ and entirely in $B$. 
	 	
	 	Let $I_A\subset [0,1]$ be $\a^{-1}(A)$ and $I_B=\a^{-1}(B)$. Since $A,B$ are open, these sets are open, and they cover $[0,1]$. Thus by compactness, they are each expressible as finite collections of open intervals. So we can split $\a$ into finitely-many parts, each of which is in $A$ or $B$, and whose endpoints are in $A\cap B$. Because $A\cap B$ is path-connected, each of these are homotopic to a path which begins at $x$.\\
	 	
	 	Step 2: Suppose $\a,\b\in \pi_1(A,x)\ast \pi_1(B,x)$ are identified by this map. There is a nice diagram that looks like a brick wall. Basically, if $\a,\b$ are identified, there is a homotopy between their images, and one can factor this homotopy into individual steps (``bricks'') which correspond with identifying a path in $A\cap B$ in $\pi(A,x)$ with its other inclusion in $\pi(B,x)$.  
	 \end{proof}\\
	 
	 Examples:
	 \begin{itemize}
	 	\item Let $X=S^n$ decomposed into $D^n_-$ and $D^n_+$, with intersection $S^{n-1}$. Then Van Kampen implies that $\pi(S^n)$ is trivial.
	 	\item In any connected CW complex with $X^1=\vee_{\a} S^1_{\a}$, we have $\pi_1(X^2)$ is the free group on generators $\a$ quotiented by the attaching maps.
	 	\item Similarly, $\pi(X)=\pi(X^n)=\pi(X^2)$ for all connected CW complexes $X$. This is because attaching 3-cells yields 2-cell intersections, which have trivial fundamental group.
	 	\item Let $\Sigma_g$ be a $g$-holed torus in $\R^3$. This can be given a CW structure with $2g$ 1-cells and one 2-cell. The fundamental group $\pi_1(\Sigma_g)$ has a presentation 
	 	$$
	 	\pi_1(\Sigma_g) := \<\a_1,\b_1,\dots,\a_g,\b_g | \prod_{i=1}^g [\a_i,\b_i]\>.
	 	$$
	 \end{itemize}
	 \medskip
	 
	 Given a group $\pi$, A $K(\pi,1)$ is a path-connected space $X$ for which $\pi_1(X)\cong \pi$ and whose universal cover is contractible.
	 
	 Let $X$ be a connected CW complex and $Y$ a $K(\pi,1)$ where $\pi = \pi_1(Y,y)$. Then maps $X\to Y$ induce homomorphisms $\rho: \pi_1(X,x)\to \pi_1(Y,y)$. But moreover, \textit{every} such homomorphism is induced by a map $X\to Y$.
	 \begin{proof}
	 	Let $X$ have one $0$-cell $X^0=\{x\}$. For up to the 2-skeleton $X^2$, we can construct a map $X\to Y$ inducing $\rho$ without having any $K(\pi,1)$ property of $Y$. To get to $X^3$ and higher, we need this property. (details omitted)
	 \end{proof}\\
	 
	 For any group $G$ there is a CW complex that has $G$ as its fundamental group and has a contractible universal cover (i.e. a $K(G,1)$).
	 \begin{proof}
	 	Let $EG$ be the simplicial complex which has $n$-simplices corresponding to all $n$-tuples of elements of $G$, with triangles corresponding to the same three elements identified. $EG$ is deformation-retractable to $\id_G$.
	 	
	 	$G$ acts on $EG$ by acting on the vertices. Let $BG=EG/G$. This $BG$ is a $K(G,1)$.
	 \end{proof}\\
	 
	 \subsection{Homology} Homology groups are somewhat like fundamental groups (they are invariant properties of a topological space) but they are always Abelian. There are two equivalent definitions \textit{on CW complexes}: one will be easy to compute but non-trivially invariant (simplicial/cellular homology), and one will be clearly invariant but difficult to compute (singular homology).\\
	 
	 Let $\Delta^n$ be the canonical $n$-simplex, defined as the set of points in $\R^{n+1}$ whose coordinates are all non-negative and sum to $1$. We say that the $j$th face of $\Delta^n$, denoted $\Delta^n_j$, is the $(n-1)$-simplex obtained by taking all the vertices except for $j$. The orientation (i.e. the ordering on vertices) is preserved if $j$ is even and reversed if $j$ is odd. There may be other $n$-simplices in space which are images of $\Delta^n$ under some continuous map. A $\Delta$-complex is basically a CW complex but with simplices instead of balls and the attaching maps additionally have to obey some orientation stuff.\\
	 
	 If $X$ is a $\Delta$-complex, $\Delta_n(X)$ is the free abelian group generated by the $n$-simplices of $X$. We have a map $\partial_n:\Delta_n(X)\to \Delta_{n-1}(X)$ given by the oriented sum of all the faces:
	 $$
	 \partial_n : \sigma \mapsto \sum_{i=1}^n (-1)^i \sigma_i
	 $$
	 where $\sigma_i$ is the $i$th face of $\sigma$. This is the boundary operator. We can get the familiar property 
	 $$
	 \partial_{n-1} \circ \partial_n = 0.
	 $$
	 To check this,
	 $$
	 \partial_{n-1}\partial_n \sigma = \partial_{n-1}\sum_{i=1}^n (-1)^i \sigma_i = \sum_{i=1}^n\sum_j^{n-1} (-1)^{i+j} (\sigma_i)_j
	 $$
	 and note that $(\sigma_i)_j = - (\sigma_j)_i$ because $i$ and $j$ are swapped, which negates the sign of the permutation. The idea is essentially that $(n-2)$-edges all appear twice with opposite orientations (one can check this on the $3$-simplex).\\
	 
	 An element in $\ker(\partial_n)=:Z_n$ is an $n$-cycle, and element in $\im(\partial_{n+1})=:B_n$ is an $n$-boundary. Note that $B_n$ is a subgroup of $Z_n$, which is a subgroup of $\Delta_n$. The $n$th \textit{Homology Group} is 
	 $$H_n = Z_n/B_n.$$
	 Elements of $H_n$ are equivalence classes of cycles (which are equal mod boundaries). It's not clear a priori that this is well-defined (i.e. does not depend on the parametrization of $X$) but we will show this later. This is \textit{simplicial homology}.\\
	 
	 Alternatively, we could take a given $X$ which is not canonically given by a $\Delta$-complex, and form a simplicial complex $S(X)$ in such a way that it contains \textit{all} continuous maps from $\Delta_n$ into $X$, and define homology of $X$ to be $H_n(S(X))$. This is called the \textit{singular homology} of $X$.\\
	 
	 A sequence of Abelian groups $C_n$ with $\partial$ maps between them (these maps could actually be anything as  long as $\partial^2=0$, i.e. it is a long exact sequence) is called a \textit{chain complex}, and a \textit{chain map} is a sequence of maps $f_n:C_n\to C_n'$ from one chain complex to another such that the diagram commutes:
	 $$
	 \begin{tikzcd}
	 	\cdots && {C_{n+1}} && {C_n} && {C_{n-1}} && \cdots \\
	 	\\
	 	\cdots && {C_{n+1}'} && {C_n'} && {C_{n-1}'} && \cdots
	 	\arrow["\partial", from=1-1, to=1-3]
	 	\arrow["\partial", from=1-3, to=1-5]
	 	\arrow["{f_{n+1}}"{description}, from=1-3, to=3-3]
	 	\arrow["\partial", from=1-5, to=1-7]
	 	\arrow["{f_{n}}"{description}, from=1-5, to=3-5]
	 	\arrow["\partial", from=1-7, to=1-9]
	 	\arrow["{f_{n-1}}"{description}, from=1-7, to=3-7]
	 	\arrow["\partial", from=3-1, to=3-3]
	 	\arrow["\partial", from=3-3, to=3-5]
	 	\arrow["\partial", from=3-5, to=3-7]
	 	\arrow["\partial", from=3-7, to=3-9]
	 \end{tikzcd}
	 $$
	 This chain map induces a homomorphism between the homology groups $f_*:H_n(C_*)\to H_n(C_{*}')$ (this is diagram chasing).\\
	 
	 \textbf{Theorem}: If $f,g:X\to Y$ are homotopic maps of topological spaces, then the homomorphisms of homology groups induced by $f_*,g_*$ are equal.
	 \begin{proof}
	 	Let $f_{\sharp},g_{\sharp}:C_*(X)\to C_*(Y)$ be the induced chain maps. A \textit{dual homotopy} is a map $P:C_*(X)\to C_{*+1}(Y)$ with the property that
	 	$$
	 	\partial P + P\partial = g_{\sharp} - f_{\sharp}
	 	$$
	 	If such a $P$ exists, then $f_*=g_*$, since the boundary of any loop is $0$, so 
	 	$$
	 	(g_{\sharp} - f_{\sharp})(\a) = (\partial P + P\partial)(\a) = \partial P(\a) + 0 
	 	$$
	 	and this is a boundary, so in the homology group $f_{\sharp} = g_{\sharp}$.
	 	
	 	So it suffices to find such a $P$. To make this $P$, we begin with the homotopy between $f$ and $g$. \textit{add more details later}.
	 \end{proof}\\
	 
	 Examples:
	 \begin{itemize}
	 	\item If $X$ is a single point, then there is only a single map from the $n$-simplex to $X$, so each $C_n(X)$ is the Abelian group generated by a single generator $\sigma_n$, i.e. $C_n(X)=\Z$. For this $n$-simplex $\sigma_n$, we have 
	 	$$
	 	\partial_n (\sigma_n) = \sum_{i=0}^{n} (-1)^i \sigma_{n-1} = \begin{cases} 0 & n \text{ odd}\\
	 		\sigma_{n-1} & n \text{ even}
	 	\end{cases}
	 	$$
	 	so the maps in this complex are 
	 	$$
	 	\begin{tikzcd}
	 		\cdots && {\Z} && {\Z} && {\Z} && \cdots
	 		\arrow["\cong", from=1-1, to=1-3]
	 		\arrow["0", from=1-3, to=1-5]
	 		\arrow["\cong", from=1-5, to=1-7]
	 		\arrow["0", from=1-7, to=1-9]
	 	\end{tikzcd}
	 	$$
	 	thus giving $H_n(X) = 0$ for all $n$ (except $H_0(X)=\Z$).
	 \end{itemize}
	 
\end{document}