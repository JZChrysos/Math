\documentclass{amsart}
%\documentclass{amsart}
\usepackage[utf8]{inputenc}
\usepackage{amsfonts}
\usepackage{amsmath}
\usepackage{amssymb}
\usepackage{amsthm}
\usepackage{asymptote}
\usepackage{mathtools}
\usepackage{hhline}
\usepackage{graphicx,enumerate}
\usepackage{hyperref}
\usepackage[a4paper, margin=1.2in]{geometry}
%\usepackage{tcolorbox}
\usepackage{tikz-cd}
\usepackage{ytableau}
%\tcbuselibrary{skins,breakable,xparse}
\allowdisplaybreaks
\newcounter{count}
\hypersetup{
	colorlinks=true,
	linkcolor=teal,
	filecolor=magenta,      
	urlcolor=olive,
	citecolor=teal,
	pdfpagemode=FullScreen,
}

%\definecolor{defcolor}{HTML}{478EFF}
%\definecolor{thmcolor}{HTML}{CC0058}
%\definecolor{excolor}{HTML}{F5B400}
%\definecolor{probcolor}{HTML}{DD4803}
%\definecolor{lemcolor}{HTML}{741FEA}
%\definecolor{scarlet}{HTML}{A81111}
%
%\newtheoremstyle{definitionStyle}% Custom style for definitions
%{0.5em}% Space above
%{0.5em}% Space below
%{}% Body font
%{}% Indent amount
%{\bfseries\color{defcolor}}% Theorem head font: bold and red
%{.\\}% Punctuation after theorem head
%{0.5em}% Space after theorem head
%{\thmname{#1}\thmnumber{ #2 (#3)}}% Theorem head spec
%
%\theoremstyle{definitionStyle}
%\newtheorem{df}{Definition}[section]
%
%\newtheoremstyle{theoremStyle}% Custom style for definitions
%{0.5em}% Space above
%{0.5em}% Space below
%{}% Body font
%{}% Indent amount
%{\bfseries\color{thmcolor}}% Theorem head font: bold and red
%{.\\}% Punctuation after theorem head
%{0.5em}% Space after theorem head
%{\thmname{#1}\thmnumber{ #2 (#3)}}% Theorem head spec
%
%\theoremstyle{theoremStyle}
%\newtheorem{thm}{Theorem}[section]
%
%\newtheoremstyle{lemmaStyle}% Custom style for definitions
%{0.5em}% Space above
%{0.5em}% Space below
%{}% Body font
%{}% Indent amount
%{\bfseries\color{lemcolor}}% Theorem head font: bold and red
%{.\\}% Punctuation after theorem head
%{0.5em}% Space after theorem head
%{\thmname{#1}\thmnumber{ #2 (#3)}}% Theorem head spec
%
%\theoremstyle{lemmaStyle}
%\newtheorem{lem}{Lemma}[section]
%\newtheorem{cor}{Corollary}[section]
%
%\newtheoremstyle{exampleStyle}% Custom style for definitions
%{0.5em}% Space above
%{0.5em}% Space below
%{}% Body font
%{}% Indent amount
%{\bfseries\color{excolor}}% Theorem head font: bold and red
%{.\\}% Punctuation after theorem head
%{0.5em}% Space after theorem head
%{\thmname{#1}\thmnumber{ #2 (#3)}}% Theorem head spec
%
%\theoremstyle{exampleStyle}
%\newtheorem{ex}{Example}[section]
%
%\newtheoremstyle{problemStyle}% Custom style for definitions
%{0.5em}% Space above
%{0.5em}% Space below
%{}% Body font
%{}% Indent amount
%{\bfseries\color{probcolor}}% Theorem head font: bold and red
%{.\\}% Punctuation after theorem head
%{0.5em}% Space after theorem head
%{\thmname{#1}\thmnumber{ #2#3}}% Theorem head spec
%
%\theoremstyle{problemStyle}
%\newtheorem{prob}{Problem}[section]

% For Fun
\newcommand{\club}{\color{teal} \clubsuit}
\newcommand{\heart}{\color{red} \heartsuit}
\renewcommand{\star}{\color{scarlet} \bigstar}
\newcommand{\spade}{\color{violet} \spadesuit}

% Symbols
\newcommand{\A}{\mathcal{A}}
\newcommand{\B}{\mathcal{B}}
\newcommand{\C}{\mathbb{C}}
\newcommand{\D}{\mathcal{D}}
\newcommand{\E}{\mathbb{E}}
\newcommand{\F}{\mathbb{F}}
\newcommand{\G}{\mathcal{G}}
% \renewcommand{\H}{\mathcal{H}} Erdos o
\newcommand{\I}{\mathcal{I}}
\newcommand{\J}{\mathcal{J}}
\newcommand{\K}{\mathcal{K}}
% \renewcommand{\L}{\mathcal{L}}
\newcommand{\M}{\mathcal{M}}
\newcommand{\N}{\mathbb{N}}
\renewcommand{\O}{\mathcal{O}}
\renewcommand{\P}{\mathbb{P}}
\newcommand{\Q}{\mathbb{Q}}
\newcommand{\R}{\mathbb{R}}
\renewcommand{\S}{\mathbb{S}}
\newcommand{\T}{\mathbb{T}}
\newcommand{\U}{\mathcal{U}}
\newcommand{\V}{\mathcal{V}}
\newcommand{\W}{\mathcal{W}}
\newcommand{\X}{\mathcal{X}}
\newcommand{\Y}{\mathcal{Y}}
\newcommand{\Z}{\mathbb{Z}}

\renewcommand{\AA}{\mathcal{A}}
\newcommand{\BB}{\mathcal{B}}
\newcommand{\CC}{\mathcal{C}}
\newcommand{\DD}{\mathcal{D}}
\newcommand{\EE}{\mathcal{E}}
\newcommand{\FF}{\mathcal{F}}
\newcommand{\GG}{\mathbb{G}}
\newcommand{\HH}{\mathbb{H}}
\newcommand{\calH}{\mathcal{H}}
\newcommand{\II}{\mathcal{I}}
\newcommand{\JJ}{\mathcal{J}}
\newcommand{\KK}{\mathcal{K}}
\newcommand{\LL}{\mathcal{L}}
\newcommand{\MM}{\mathcal{M}}
\newcommand{\NN}{\mathcal{N}}
\newcommand{\OO}{\mathrm{O}}
\newcommand{\PP}{\mathcal{P}}
\newcommand{\QQ}{\mathcal{Q}}
\newcommand{\RR}{\mathcal{R}}
\renewcommand{\SS}{\mathcal{S}}
\newcommand{\TT}{\mathcal{T}}
\newcommand{\UU}{\mathcal{U}}
\newcommand{\VV}{\mathcal{V}}
\newcommand{\WW}{\mathcal{W}}
\newcommand{\XX}{\mathcal{X}}
\newcommand{\YY}{\mathcal{Y}}
\newcommand{\ZZ}{\mathcal{Z}}
\renewcommand{\d}{\textrm{d}}
% Greek letters
\newcommand{\ep}{\varepsilon}
\newcommand{\ph}{\varphi}
\newcommand{\de}{\delta}
\renewcommand{\a}{\alpha}
\renewcommand{\b}{\beta}
% Fraktur
\newcommand{\mm}{\mathfrak{m}}
\renewcommand{\aa}{\mathfrak{a}}
\newcommand{\bb}{\mathfrak{b}}
\newcommand{\pp}{\mathfrak{p}}
\newcommand{\qq}{\mathfrak{q}}
% Operators
\DeclareMathOperator{\Div}{div}
\DeclareMathOperator{\Gal}{Gal}
\DeclareMathOperator{\vol}{Vol}
\DeclareMathOperator{\Hom}{Hom}
\DeclareMathOperator{\End}{End}
\DeclareMathOperator{\Ext}{Ext}
\DeclareMathOperator{\Tor}{Tor}
\DeclareMathOperator{\tr}{tr}
\DeclareMathOperator{\rk}{rk}
\DeclareMathOperator{\curl}{curl}
\DeclareMathOperator{\mesh}{mesh}
\DeclareMathOperator{\im}{im}
\DeclareMathOperator{\coker}{coker}
\DeclareMathOperator{\width}{width}
\DeclareMathOperator{\diam}{diam}
\DeclareMathOperator{\maps}{Maps}
\DeclareMathOperator{\Frac}{Frac}
\DeclareMathOperator{\Sym}{Sym}
\DeclareMathOperator{\sgn}{sgn}
\DeclareMathOperator{\alt}{Alt}
\DeclareMathOperator{\supp}{supp}
\DeclareMathOperator{\Span}{span}
\DeclareMathOperator{\Var}{Var}
\DeclareMathOperator{\Spec}{Spec}

\newcommand{\nor}{\unlhd}
\DeclareMathOperator{\aut}{Aut}
\DeclareMathOperator{\orb}{Orb}
\DeclareMathOperator{\GL}{GL}
\DeclareMathOperator{\SL}{SL}
\DeclareMathOperator{\SO}{SO}
\DeclareMathOperator{\PGL}{PGL}
\DeclareMathOperator{\PSL}{PSL}
\DeclareMathOperator{\stab}{Stab}
\DeclareMathOperator{\fix}{Fix}
\DeclareMathOperator{\Th}{Th}
\DeclareMathOperator{\Ind}{Ind}
\DeclareMathOperator{\Res}{Res}
\DeclareMathOperator{\Ann}{Ann}
\DeclareMathOperator{\rad}{rad}
\DeclareMathOperator{\len}{len}
\DeclareMathOperator{\ord}{ord}

% \DeclareMathOperator{\arg}{arg}

%% misc
\newcommand{\<}{\langle}
\renewcommand{\>}{\rangle}
\renewcommand{\^}{\wedge}
\renewcommand{\v}{\vee}
\def\Xint#1{\mathchoice
	{\XXint\displaystyle\textstyle{#1}}%
	{\XXint\textstyle\scriptstyle{#1}}%
	{\XXint\scriptstyle\scriptscriptstyle{#1}}%
	{\XXint\scriptscriptstyle\scriptscriptstyle{#1}}%
	\!\int}
\def\XXint#1#2#3{{\setbox0=\hbox{$#1{#2#3}{\int}$ }
		\vcenter{\hbox{$#2#3$ }}\kern-.6\wd0}}
\def\ddashint{\Xint=}
\def\dashint{\Xint-}
%% arrows
\newcommand{\xhra}{\xhookrightarrow}
\newcommand{\xra}{\xrightarrow}
\newcommand{\ra}{\rightarrow}
\newcommand{\rra}{\rightrightarrows}
\newcommand{\lra}{\longrightarrow}
\newcommand{\Ra}{\Rightarrow}
\newcommand{\lRa}{\Longrightarrow}
\newcommand{\lrsa}{\leftrightsquiqarrow}
\newcommand{\ba}{\leftrightarrow}
%% lists
\newcommand{\be}{\begin{enumerate}[(i)]}
	\newcommand{\ee}{\end{enumerate}}
%% integration stuff
\newcommand{\calR}{\mathcal{R}}
\newcommand{\rint}{\calR\!\int}
\newcommand{\calL}{\mathcal{L}}
\newcommand{\lowerint}{\mbox{\b{$\int$}}}
\newcommand{\upperint}{{\textstyle\bar{\int}}}
%% end of proof
\def\endproof{{\hfill $\Box$}}
%% matrix shorthand

\title{Manifolds}
\author{Jalen Chrysos}

\begin{document}
	
\maketitle

These are my notes for Topology II taught by Eduard Looijenga at UChicago in Winter 2026.

\section{Basic Definitions}

A \textit{topological manifold} of dimension $m$ is a topological space $M$ that is Hausdorff and locally homeomorphic to $\R^m$. Such an $M$ has an open covering $\A = \{U_{\a}\}$ called an \textit{atlas} with associated homeomorphisms (\textit{charts}) $\kappa_{\a}:U_{\a}\to \R^m$ which are compatible, meaning that in each intersection $U_{\a}\cap U_{\b}$, we have a homeomorphic coordinate change map:
$$
\R^m \supset \kappa_{\a}(U_{\a}\cap U_{\b}) \xra{\kappa_{\b}\kappa_{\a}^{-1}} \kappa_{\b}(U_{\a}\cap U_{\b}) \subset \R^m
$$
The atlas is $C^k$ if all the coordinate change maps are $C^k$.\\

$\A$ must be $C^k$ in order to define the notion of a $C^k$ function $M\to \R$ (relative to $\A$); otherwise, we could have $f:M\to \R$ that is $C^k$ through one chart but not another. Naturally, which functions $M\to \R$ are $C^k$ depends on $\A$. And in fact, \textbf{atlases define the same notion of $C^k$ iff they are compatible}. That is, all possible notions of a $C^k$ function on $M$ correspond to maximal atlases, or ``$C^k$ \textit{structures}.''

The presence of a $C^k$ structure enriches $M$ and allows one to say more about it, so it is natural to ask whether a given $M$ has a $C^k$ structure. Whitney showed that all manifolds with a $C^k$ structure also have a $C^{\infty}$ structure that can be obtained by restricting the corresponding atlas (and hence a $C^j$ structure for $j>0$). So the $C^k$ structures come together. However, there are topological manifolds with no $C^1$ structure, and hence no $C^k$ structure for any $k>0$. Thus the only distinction is between smooth manifolds and non-differentiable manifolds. We will be concerned only with the former.\\

A bijection $f:M\to N$ between smooth manifolds is called a \textit{diffeomorphism} if it and its inverse are both $C^1$. This is more strict than a \textit{homeomorphism}, which is only required to be continuous in both directions.

While in Algebraic topology we are concerned with the homotopy types of spaces, which is a coarser characterization than homeomorphism type, when studying smooth manifolds we can also ask about the diffeomorphism type, which is finer\footnote{For example, the homeomorphism type of $S^7$ splits into 28 diffeomorphism types, as shown by Milnor.}.

\subsection{Submanifolds, Immersions, Submersions, and Embeddings}
All $m$-manifolds $M$, because they are locally homeomorphic to $\R^m$, have an associated \textit{tangent space} $T_pM$ at each point $p$. This is literally the set of tangent vectors to $M$ at $p$. It is in fact a vector space of dimension $m$. For smooth maps $f:M\to N$, we can think of the derivative $D_pf$ as literally a linear map between tangent spaces $D_pf :T_p M \to T_{f(p)}N$.\\

The quality of $D_pf$ locally tells us a lot about its overall properties.
\begin{itemize}
	\item $f$ is called an \textit{immersion} if $D_pf$ is injective at all points $p$.
	\item $f$ is called a \textit{submersion} if $D_pf$ is surjective (i.e. full rank) at all points $p$.
	\item $f$ is called a \textit{local diffeomorphism} if $D_pf$ is invertible at all points $p$.
\end{itemize}
Note that a local diffeomorphism need not be a diffeomorphism because though it is locally invertible it might not be globally (consider e.g. the map $\R\to S^1$ given by $x\mapsto e^{ix}$).

The image of a local diffeomorphism need not even be a manifold. Consider the map taking $S^1$ to a figure-eight. Locally, every section of the circle is sent to a segment of the figure-eight, yet near the crossing point there is no homeomorphism to $\R$, so the figure-eight isn't a manifold. Another way of looking at this is that the tangent space to the figure-eight at the crossing point is \textit{not} a vector-space. For a higher-dimensional example, take the Klein bottle. There is a local diffeomorphism to a ``fake Klein bottle'' in $\R^3$ which self-intersects in a circle.

If an immersion is injective (so its image has no self-intersections), it is called an \textit{embedding}. The image of an embedding $f:M\to N$ is always a manifold, and is called a \textit{submanifold} of $N$. Submanifolds can be equivalently characterized in another way: a manifold $A\subset N$ is a submanifold of $N$ if the charts $\kappa:N\to \R^n$ send $A$ to a linear subspace $\R^k\subset \R^n$.\\

Question: Can a given $N$ be realized as a submanifold of a given $M$ (especially when $M=\R^m$)? Or in another way, can we classify all submanifolds of a given $M$ up to diffeomorphism type?
\begin{itemize}
	\item Any smooth manifold (as long as its topology has a countable basis) can be embedded in $\R^m$ for sufficiently large $m$, though exactly what $m$ is the minimum is not always easy to determine. 
	\begin{itemize} 
		\item \textbf{Whitney's Embedding Theorem}: a smooth $m$-manifold can always be embedded in $\R^{2m}$ and immersed in $\R^{2m-1}$ (smaller dimensions may be possible). 
		\item Cohen proved more generally that a smooth $m$-manifold could be immersed in $\R^{2m-a(m)}$, where $a(m)$ is the number of 1's in the binary expansion of $m$.
		\item For example, the Klein bottle $K$ can be described without reference to an underlying space via gluing instructions, and the lowest-dimensional space it can be embedded in is $\R^4$. In $\R^3$ it can be \textit{immersed}, but not embedded.
	\end{itemize}
\item \textbf{Inverse/Implicit Function Theorem}:
\begin{itemize}
	\item If $f:M\to N$ is an immersion at $p$, $f(U)$ is a submanifold of $N$ for some open $U\ni p$.
	\item If $f:M\to N$ is a submersion at $p\mapsto q$, then the fiber $f^{-1}(q)$ is a submanifold of $M$.
\end{itemize}
\item \textbf{Transverse Intersections}:
\begin{itemize}
	\item We say $f:M\to N$ is \textit{transverse} to a submanifold $Q\subseteq N$ if for $q=f(p)\in Q$, $D_qf(T_p(M))$ and $T_q(Q)$ span $T_q(N)$. This is denoted $f\pitchfork Q$.
	\item In this case, $f^{-1}(Q)$ is a submanifold of $M$ with the same codimension as $Q$ in $N$.\\
\end{itemize}
\end{itemize}

\subsection{Tangent Bundles and Vector Fields} Every smooth manifold $M$ has an associated \textit{tangent bundle} $TM$ whose elements are pairs $(p,v)$ where $v\in T_pM$. It is a $2m$-dimensional manifold.\\

A \textit{vector field} over a manifold is a smooth map $V:M\to TM$ with $V(p)\in T_p(M)$. For some manifolds $M$, it is possible to give a basis for $TM$ by vector fields; that is, to give vector fields $V_1,V_2,\dots,V_m$ such that their values at $p\in M$ are always a basis of $T_pM$.\\

Question: For which $M$ is there a basis of vector spaces?
\begin{itemize}
	\item Examples:
	\begin{itemize}
		\item For $M=S^1$ there is a basis, given by a $90$-degree rotation at every $p$. Thinking of $S^1$ as $\C^{\times}$, this corresponds algebraically to a multiplication by $i$.
		\item For $M=S^2$, there is not. There isn't even a nonzero vector field $V:M\to TM$. Suppose there were such $V:M\to TM$. Such $V$ induces by projection a map $V':S^2\to S^2$ with $V'(p)\bot p$ for all $p\in S^2$. But we know from Algebraic Topology that every map $S^n\to S^n$ either maps at least a point to its antipode or has a fixed point (if $p,V'(p)$ are not antipodes then there is a unique shortest path between them, so we can homotope every $V'(p)$ continuously along this shortest path to $p$, giving a homotopy to the identity map, and hence there is a fixed point).
		\item For $M=S^3$, there is a basis. It is given by analogy to the Quaternions: thinking of $S^3$ as $\HH^{\times}$, then we have a basis via the three vector fields $p\mapsto ip,p\mapsto jp,p\mapsto kp$. 
	\end{itemize}
\end{itemize}


%\textbf{Lemma}: Given two $C^k$ atlases $\{U_{\a},\kappa_{\a}\}$ and $\{V_{\b},\lambda_{\b}\}$ of the same manifold $M$, the following are equivalent:
%\begin{enumerate}[(i)]
%	\item The two atlases define the same notion of a $C^k$ function.
%	\item $\{U_{\a}\} \cup \{V_{\a}\}$ is a $C^k$ atlas.
%\end{enumerate}
%\begin{proof}
%	$(i)\implies (ii)$: It suffices to show that the coordinate change maps $\kappa_{\a}\lambda_{\b}^{-1}$ are all $C^k$. do later
%\end{proof}\\
%
%This result implies that all atlases giving the same notion of a $C^k$ function can be combined into a maximal atlas. Such an atlas is called a ``$C^k$ structure'' on $M$.\\

\end{document}