\documentclass{amsart}
%\documentclass{amsart}
\usepackage[utf8]{inputenc}
\usepackage{amsfonts}
\usepackage{amsmath}
\usepackage{amssymb}
\usepackage{amsthm}
\usepackage{asymptote}
\usepackage{mathtools}
\usepackage{hhline}
\usepackage{graphicx,enumerate}
\usepackage{hyperref}
\usepackage[a4paper, margin=1.2in]{geometry}
%\usepackage{tcolorbox}
\usepackage{tikz-cd}
\usepackage{ytableau}
%\tcbuselibrary{skins,breakable,xparse}
\allowdisplaybreaks
\newcounter{count}
\hypersetup{
	colorlinks=true,
	linkcolor=teal,
	filecolor=magenta,      
	urlcolor=olive,
	citecolor=teal,
	pdfpagemode=FullScreen,
}

%\definecolor{defcolor}{HTML}{478EFF}
%\definecolor{thmcolor}{HTML}{CC0058}
%\definecolor{excolor}{HTML}{F5B400}
%\definecolor{probcolor}{HTML}{DD4803}
%\definecolor{lemcolor}{HTML}{741FEA}
%\definecolor{scarlet}{HTML}{A81111}
%
%\newtheoremstyle{definitionStyle}% Custom style for definitions
%{0.5em}% Space above
%{0.5em}% Space below
%{}% Body font
%{}% Indent amount
%{\bfseries\color{defcolor}}% Theorem head font: bold and red
%{.\\}% Punctuation after theorem head
%{0.5em}% Space after theorem head
%{\thmname{#1}\thmnumber{ #2 (#3)}}% Theorem head spec
%
%\theoremstyle{definitionStyle}
%\newtheorem{df}{Definition}[section]
%
%\newtheoremstyle{theoremStyle}% Custom style for definitions
%{0.5em}% Space above
%{0.5em}% Space below
%{}% Body font
%{}% Indent amount
%{\bfseries\color{thmcolor}}% Theorem head font: bold and red
%{.\\}% Punctuation after theorem head
%{0.5em}% Space after theorem head
%{\thmname{#1}\thmnumber{ #2 (#3)}}% Theorem head spec
%
%\theoremstyle{theoremStyle}
%\newtheorem{thm}{Theorem}[section]
%
%\newtheoremstyle{lemmaStyle}% Custom style for definitions
%{0.5em}% Space above
%{0.5em}% Space below
%{}% Body font
%{}% Indent amount
%{\bfseries\color{lemcolor}}% Theorem head font: bold and red
%{.\\}% Punctuation after theorem head
%{0.5em}% Space after theorem head
%{\thmname{#1}\thmnumber{ #2 (#3)}}% Theorem head spec
%
%\theoremstyle{lemmaStyle}
%\newtheorem{lem}{Lemma}[section]
%\newtheorem{cor}{Corollary}[section]
%
%\newtheoremstyle{exampleStyle}% Custom style for definitions
%{0.5em}% Space above
%{0.5em}% Space below
%{}% Body font
%{}% Indent amount
%{\bfseries\color{excolor}}% Theorem head font: bold and red
%{.\\}% Punctuation after theorem head
%{0.5em}% Space after theorem head
%{\thmname{#1}\thmnumber{ #2 (#3)}}% Theorem head spec
%
%\theoremstyle{exampleStyle}
%\newtheorem{ex}{Example}[section]
%
%\newtheoremstyle{problemStyle}% Custom style for definitions
%{0.5em}% Space above
%{0.5em}% Space below
%{}% Body font
%{}% Indent amount
%{\bfseries\color{probcolor}}% Theorem head font: bold and red
%{.\\}% Punctuation after theorem head
%{0.5em}% Space after theorem head
%{\thmname{#1}\thmnumber{ #2#3}}% Theorem head spec
%
%\theoremstyle{problemStyle}
%\newtheorem{prob}{Problem}[section]

% For Fun
\newcommand{\club}{\color{teal} \clubsuit}
\newcommand{\heart}{\color{red} \heartsuit}
\renewcommand{\star}{\color{scarlet} \bigstar}
\newcommand{\spade}{\color{violet} \spadesuit}

% Symbols
\newcommand{\A}{\mathcal{A}}
\newcommand{\B}{\mathcal{B}}
\newcommand{\C}{\mathbb{C}}
\newcommand{\D}{\mathcal{D}}
\newcommand{\E}{\mathbb{E}}
\newcommand{\F}{\mathbb{F}}
\newcommand{\G}{\mathcal{G}}
% \renewcommand{\H}{\mathcal{H}} Erdos o
\newcommand{\I}{\mathcal{I}}
\newcommand{\J}{\mathcal{J}}
\newcommand{\K}{\mathcal{K}}
% \renewcommand{\L}{\mathcal{L}}
\newcommand{\M}{\mathcal{M}}
\newcommand{\N}{\mathbb{N}}
\renewcommand{\O}{\mathcal{O}}
\renewcommand{\P}{\mathbb{P}}
\newcommand{\Q}{\mathbb{Q}}
\newcommand{\R}{\mathbb{R}}
\renewcommand{\S}{\mathbb{S}}
\newcommand{\T}{\mathbb{T}}
\newcommand{\U}{\mathcal{U}}
\newcommand{\V}{\mathcal{V}}
\newcommand{\W}{\mathcal{W}}
\newcommand{\X}{\mathcal{X}}
\newcommand{\Y}{\mathcal{Y}}
\newcommand{\Z}{\mathbb{Z}}

\renewcommand{\AA}{\mathcal{A}}
\newcommand{\BB}{\mathcal{B}}
\newcommand{\CC}{\mathcal{C}}
\newcommand{\DD}{\mathcal{D}}
\newcommand{\EE}{\mathcal{E}}
\newcommand{\FF}{\mathcal{F}}
\newcommand{\GG}{\mathbb{G}}
\newcommand{\HH}{\mathbb{H}}
\newcommand{\calH}{\mathcal{H}}
\newcommand{\II}{\mathcal{I}}
\newcommand{\JJ}{\mathcal{J}}
\newcommand{\KK}{\mathcal{K}}
\newcommand{\LL}{\mathcal{L}}
\newcommand{\MM}{\mathcal{M}}
\newcommand{\NN}{\mathcal{N}}
\newcommand{\OO}{\mathrm{O}}
\newcommand{\PP}{\mathcal{P}}
\newcommand{\QQ}{\mathcal{Q}}
\newcommand{\RR}{\mathcal{R}}
\renewcommand{\SS}{\mathcal{S}}
\newcommand{\TT}{\mathcal{T}}
\newcommand{\UU}{\mathcal{U}}
\newcommand{\VV}{\mathcal{V}}
\newcommand{\WW}{\mathcal{W}}
\newcommand{\XX}{\mathcal{X}}
\newcommand{\YY}{\mathcal{Y}}
\newcommand{\ZZ}{\mathcal{Z}}
\renewcommand{\d}{\textrm{d}}
% Greek letters
\newcommand{\ep}{\varepsilon}
\newcommand{\ph}{\varphi}
\newcommand{\de}{\delta}
\renewcommand{\a}{\alpha}
\renewcommand{\b}{\beta}
% Fraktur
\newcommand{\mm}{\mathfrak{m}}
\renewcommand{\aa}{\mathfrak{a}}
\newcommand{\bb}{\mathfrak{b}}
\newcommand{\pp}{\mathfrak{p}}
\newcommand{\qq}{\mathfrak{q}}
% Operators
\DeclareMathOperator{\Div}{div}
\DeclareMathOperator{\Gal}{Gal}
\DeclareMathOperator{\vol}{Vol}
\DeclareMathOperator{\Hom}{Hom}
\DeclareMathOperator{\End}{End}
\DeclareMathOperator{\Ext}{Ext}
\DeclareMathOperator{\Tor}{Tor}
\DeclareMathOperator{\tr}{tr}
\DeclareMathOperator{\rk}{rk}
\DeclareMathOperator{\curl}{curl}
\DeclareMathOperator{\mesh}{mesh}
\DeclareMathOperator{\im}{im}
\DeclareMathOperator{\coker}{coker}
\DeclareMathOperator{\width}{width}
\DeclareMathOperator{\diam}{diam}
\DeclareMathOperator{\maps}{Maps}
\DeclareMathOperator{\Frac}{Frac}
\DeclareMathOperator{\Sym}{Sym}
\DeclareMathOperator{\sgn}{sgn}
\DeclareMathOperator{\alt}{Alt}
\DeclareMathOperator{\supp}{supp}
\DeclareMathOperator{\Span}{span}
\DeclareMathOperator{\Var}{Var}
\DeclareMathOperator{\Spec}{Spec}

\newcommand{\nor}{\unlhd}
\DeclareMathOperator{\aut}{Aut}
\DeclareMathOperator{\orb}{Orb}
\DeclareMathOperator{\GL}{GL}
\DeclareMathOperator{\SL}{SL}
\DeclareMathOperator{\SO}{SO}
\DeclareMathOperator{\PGL}{PGL}
\DeclareMathOperator{\PSL}{PSL}
\DeclareMathOperator{\stab}{Stab}
\DeclareMathOperator{\fix}{Fix}
\DeclareMathOperator{\Th}{Th}
\DeclareMathOperator{\Ind}{Ind}
\DeclareMathOperator{\Res}{Res}
\DeclareMathOperator{\Ann}{Ann}
\DeclareMathOperator{\rad}{rad}
\DeclareMathOperator{\len}{len}
\DeclareMathOperator{\ord}{ord}

% \DeclareMathOperator{\arg}{arg}

%% misc
\newcommand{\<}{\langle}
\renewcommand{\>}{\rangle}
\renewcommand{\^}{\wedge}
\renewcommand{\v}{\vee}
\def\Xint#1{\mathchoice
	{\XXint\displaystyle\textstyle{#1}}%
	{\XXint\textstyle\scriptstyle{#1}}%
	{\XXint\scriptstyle\scriptscriptstyle{#1}}%
	{\XXint\scriptscriptstyle\scriptscriptstyle{#1}}%
	\!\int}
\def\XXint#1#2#3{{\setbox0=\hbox{$#1{#2#3}{\int}$ }
		\vcenter{\hbox{$#2#3$ }}\kern-.6\wd0}}
\def\ddashint{\Xint=}
\def\dashint{\Xint-}
%% arrows
\newcommand{\xhra}{\xhookrightarrow}
\newcommand{\xra}{\xrightarrow}
\newcommand{\ra}{\rightarrow}
\newcommand{\rra}{\rightrightarrows}
\newcommand{\lra}{\longrightarrow}
\newcommand{\Ra}{\Rightarrow}
\newcommand{\lRa}{\Longrightarrow}
\newcommand{\lrsa}{\leftrightsquiqarrow}
\newcommand{\ba}{\leftrightarrow}
%% lists
\newcommand{\be}{\begin{enumerate}[(i)]}
	\newcommand{\ee}{\end{enumerate}}
%% integration stuff
\newcommand{\calR}{\mathcal{R}}
\newcommand{\rint}{\calR\!\int}
\newcommand{\calL}{\mathcal{L}}
\newcommand{\lowerint}{\mbox{\b{$\int$}}}
\newcommand{\upperint}{{\textstyle\bar{\int}}}
%% end of proof
\def\endproof{{\hfill $\Box$}}
%% matrix shorthand

\title{Manifolds}
\author{Jalen Chrysos}

\begin{document}
	
\maketitle

These are my notes for Topology II taught by Eduard Looijenga at UChicago in Winter 2026.

\section{Basic Definitions}

A \textit{topological manifold} of dimension $m$ is a topological space $M$ that is Hausdorff and locally homeomorphic to $\R^m$. Such an $M$ has an open covering $\A = \{U_{\a}\}$ called an \textit{atlas} with associated homeomorphisms (\textit{charts}) $\kappa_{\a}:U_{\a}\to \R^m$ which are compatible, meaning that in each intersection $U_{\a}\cap U_{\b}$, we have a homeomorphic coordinate change map:
$$
\R^m \supset \kappa_{\a}(U_{\a}\cap U_{\b}) \xra{\kappa_{\b}\kappa_{\a}^{-1}} \kappa_{\b}(U_{\a}\cap U_{\b}) \subset \R^m
$$
The atlas is $C^k$ if all the coordinate change maps are $C^k$.\\

$\A$ must be $C^k$ in order to define the notion of a $C^k$ function $M\to \R$ (relative to $\A$); otherwise, we could have $f:M\to \R$ that is $C^k$ through one chart but not another. Naturally, which functions $M\to \R$ are $C^k$ depends on $\A$. And in fact, \textbf{atlases define the same notion of $C^k$ iff they are compatible}. That is, all possible notions of a $C^k$ function on $M$ correspond to maximal atlases, or ``$C^k$ \textit{structures}.''\\

The presence of a $C^k$ structure enriches $M$ and allows one to say more about it, so the natural question is which $M$ have such structures. Whitney showed that all manifolds with a $C^k$ structure also have a $C^{\infty}$ structure that can be obtained by restricting the corresponding atlas (and hence a $C^j$ structure for $j>0$). So the $C^k$ structures come together. However, there are topological manifolds with no $C^1$ structure, and hence no $C^k$ structure for any $k>0$. Thus the only distinction is between smooth manifolds and non-differentiable manifolds. We will be concerned only with the former.\\

Smooth manifolds are always (given some cardinality restrictions) diffeomorphic to smooth submanifolds of $\R^n$ for some $n$.\\

\subsection{Submanifolds}
A \textit{submanifold} $N\subset M$ is a subset of $M$ for which the charts $\kappa:M\to \R^m$ send $N$ to a linear subspace $\R^k\subset \R^m$. These charts naturally give $N$ the structure of a $k$-manifold.\\  

We are interested in questions of the form: can a given $N$ be realized as a submanifold of a given $M$? Especially when $M=\R^m$. Another way to say this is: can $N$ be \textit{embedded} in $M$?\\

\subsection{Tangent Space} If manifold $M$ is smooth, it has the additional structure of \textit{tangent spaces} $T_pM$ at each point $p\in M$. These are vector spaces of the same dimension as $M$. Specifically, $T_pM$ is the set of vectors that are $\gamma'(0)$ for curves $\gamma:I\to M$ with $\gamma(0)=p$.\\

For each map $f:M\to N$ where $M,N$ are smooth manifolds, we can see the derivative of $f$ at $p\in M$ as a linear map between the tangent spaces $D_pf :T_pM\to T_{f(p)}N$.\\

\subsection{Basic Results} 

%\textbf{Lemma}: Given two $C^k$ atlases $\{U_{\a},\kappa_{\a}\}$ and $\{V_{\b},\lambda_{\b}\}$ of the same manifold $M$, the following are equivalent:
%\begin{enumerate}[(i)]
%	\item The two atlases define the same notion of a $C^k$ function.
%	\item $\{U_{\a}\} \cup \{V_{\a}\}$ is a $C^k$ atlas.
%\end{enumerate}
%\begin{proof}
%	$(i)\implies (ii)$: It suffices to show that the coordinate change maps $\kappa_{\a}\lambda_{\b}^{-1}$ are all $C^k$. do later
%\end{proof}\\
%
%This result implies that all atlases giving the same notion of a $C^k$ function can be combined into a maximal atlas. Such an atlas is called a ``$C^k$ structure'' on $M$.\\

\end{document}