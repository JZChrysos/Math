\documentclass{amsart}

%\documentclass{amsart}
\usepackage[utf8]{inputenc}
\usepackage{amsfonts}
\usepackage{amsmath}
\usepackage{amssymb}
\usepackage{amsthm}
\usepackage{asymptote}
\usepackage{mathtools}
\usepackage{hhline}
\usepackage{graphicx,enumerate}
\usepackage{hyperref}
\usepackage[a4paper, margin=1.2in]{geometry}
%\usepackage{tcolorbox}
\usepackage{tikz-cd}
\usepackage{ytableau}
%\tcbuselibrary{skins,breakable,xparse}
\allowdisplaybreaks
\newcounter{count}
\hypersetup{
	colorlinks=true,
	linkcolor=teal,
	filecolor=magenta,      
	urlcolor=olive,
	citecolor=teal,
	pdfpagemode=FullScreen,
}

%\definecolor{defcolor}{HTML}{478EFF}
%\definecolor{thmcolor}{HTML}{CC0058}
%\definecolor{excolor}{HTML}{F5B400}
%\definecolor{probcolor}{HTML}{DD4803}
%\definecolor{lemcolor}{HTML}{741FEA}
%\definecolor{scarlet}{HTML}{A81111}
%
%\newtheoremstyle{definitionStyle}% Custom style for definitions
%{0.5em}% Space above
%{0.5em}% Space below
%{}% Body font
%{}% Indent amount
%{\bfseries\color{defcolor}}% Theorem head font: bold and red
%{.\\}% Punctuation after theorem head
%{0.5em}% Space after theorem head
%{\thmname{#1}\thmnumber{ #2 (#3)}}% Theorem head spec
%
%\theoremstyle{definitionStyle}
%\newtheorem{df}{Definition}[section]
%
%\newtheoremstyle{theoremStyle}% Custom style for definitions
%{0.5em}% Space above
%{0.5em}% Space below
%{}% Body font
%{}% Indent amount
%{\bfseries\color{thmcolor}}% Theorem head font: bold and red
%{.\\}% Punctuation after theorem head
%{0.5em}% Space after theorem head
%{\thmname{#1}\thmnumber{ #2 (#3)}}% Theorem head spec
%
%\theoremstyle{theoremStyle}
%\newtheorem{thm}{Theorem}[section]
%
%\newtheoremstyle{lemmaStyle}% Custom style for definitions
%{0.5em}% Space above
%{0.5em}% Space below
%{}% Body font
%{}% Indent amount
%{\bfseries\color{lemcolor}}% Theorem head font: bold and red
%{.\\}% Punctuation after theorem head
%{0.5em}% Space after theorem head
%{\thmname{#1}\thmnumber{ #2 (#3)}}% Theorem head spec
%
%\theoremstyle{lemmaStyle}
%\newtheorem{lem}{Lemma}[section]
%\newtheorem{cor}{Corollary}[section]
%
%\newtheoremstyle{exampleStyle}% Custom style for definitions
%{0.5em}% Space above
%{0.5em}% Space below
%{}% Body font
%{}% Indent amount
%{\bfseries\color{excolor}}% Theorem head font: bold and red
%{.\\}% Punctuation after theorem head
%{0.5em}% Space after theorem head
%{\thmname{#1}\thmnumber{ #2 (#3)}}% Theorem head spec
%
%\theoremstyle{exampleStyle}
%\newtheorem{ex}{Example}[section]
%
%\newtheoremstyle{problemStyle}% Custom style for definitions
%{0.5em}% Space above
%{0.5em}% Space below
%{}% Body font
%{}% Indent amount
%{\bfseries\color{probcolor}}% Theorem head font: bold and red
%{.\\}% Punctuation after theorem head
%{0.5em}% Space after theorem head
%{\thmname{#1}\thmnumber{ #2#3}}% Theorem head spec
%
%\theoremstyle{problemStyle}
%\newtheorem{prob}{Problem}[section]

% For Fun
\newcommand{\club}{\color{teal} \clubsuit}
\newcommand{\heart}{\color{red} \heartsuit}
\renewcommand{\star}{\color{scarlet} \bigstar}
\newcommand{\spade}{\color{violet} \spadesuit}

% Symbols
\newcommand{\A}{\mathcal{A}}
\newcommand{\B}{\mathcal{B}}
\newcommand{\C}{\mathbb{C}}
\newcommand{\D}{\mathcal{D}}
\newcommand{\E}{\mathbb{E}}
\newcommand{\F}{\mathbb{F}}
\newcommand{\G}{\mathcal{G}}
% \renewcommand{\H}{\mathcal{H}} Erdos o
\newcommand{\I}{\mathcal{I}}
\newcommand{\J}{\mathcal{J}}
\newcommand{\K}{\mathcal{K}}
% \renewcommand{\L}{\mathcal{L}}
\newcommand{\M}{\mathcal{M}}
\newcommand{\N}{\mathbb{N}}
\renewcommand{\O}{\mathcal{O}}
\renewcommand{\P}{\mathbb{P}}
\newcommand{\Q}{\mathbb{Q}}
\newcommand{\R}{\mathbb{R}}
\renewcommand{\S}{\mathbb{S}}
\newcommand{\T}{\mathbb{T}}
\newcommand{\U}{\mathcal{U}}
\newcommand{\V}{\mathcal{V}}
\newcommand{\W}{\mathcal{W}}
\newcommand{\X}{\mathcal{X}}
\newcommand{\Y}{\mathcal{Y}}
\newcommand{\Z}{\mathbb{Z}}

\renewcommand{\AA}{\mathcal{A}}
\newcommand{\BB}{\mathcal{B}}
\newcommand{\CC}{\mathcal{C}}
\newcommand{\DD}{\mathcal{D}}
\newcommand{\EE}{\mathcal{E}}
\newcommand{\FF}{\mathcal{F}}
\newcommand{\GG}{\mathbb{G}}
\newcommand{\HH}{\mathbb{H}}
\newcommand{\calH}{\mathcal{H}}
\newcommand{\II}{\mathcal{I}}
\newcommand{\JJ}{\mathcal{J}}
\newcommand{\KK}{\mathcal{K}}
\newcommand{\LL}{\mathcal{L}}
\newcommand{\MM}{\mathcal{M}}
\newcommand{\NN}{\mathcal{N}}
\newcommand{\OO}{\mathrm{O}}
\newcommand{\PP}{\mathcal{P}}
\newcommand{\QQ}{\mathcal{Q}}
\newcommand{\RR}{\mathcal{R}}
\renewcommand{\SS}{\mathcal{S}}
\newcommand{\TT}{\mathcal{T}}
\newcommand{\UU}{\mathcal{U}}
\newcommand{\VV}{\mathcal{V}}
\newcommand{\WW}{\mathcal{W}}
\newcommand{\XX}{\mathcal{X}}
\newcommand{\YY}{\mathcal{Y}}
\newcommand{\ZZ}{\mathcal{Z}}
\renewcommand{\d}{\textrm{d}}
% Greek letters
\newcommand{\ep}{\varepsilon}
\newcommand{\ph}{\varphi}
\newcommand{\de}{\delta}
\renewcommand{\a}{\alpha}
\renewcommand{\b}{\beta}
% Fraktur
\newcommand{\mm}{\mathfrak{m}}
\renewcommand{\aa}{\mathfrak{a}}
\newcommand{\bb}{\mathfrak{b}}
\newcommand{\pp}{\mathfrak{p}}
\newcommand{\qq}{\mathfrak{q}}
% Operators
\DeclareMathOperator{\Div}{div}
\DeclareMathOperator{\Gal}{Gal}
\DeclareMathOperator{\vol}{Vol}
\DeclareMathOperator{\Hom}{Hom}
\DeclareMathOperator{\End}{End}
\DeclareMathOperator{\Ext}{Ext}
\DeclareMathOperator{\Tor}{Tor}
\DeclareMathOperator{\tr}{tr}
\DeclareMathOperator{\rk}{rk}
\DeclareMathOperator{\curl}{curl}
\DeclareMathOperator{\mesh}{mesh}
\DeclareMathOperator{\im}{im}
\DeclareMathOperator{\coker}{coker}
\DeclareMathOperator{\width}{width}
\DeclareMathOperator{\diam}{diam}
\DeclareMathOperator{\maps}{Maps}
\DeclareMathOperator{\Frac}{Frac}
\DeclareMathOperator{\Sym}{Sym}
\DeclareMathOperator{\sgn}{sgn}
\DeclareMathOperator{\alt}{Alt}
\DeclareMathOperator{\supp}{supp}
\DeclareMathOperator{\Span}{span}
\DeclareMathOperator{\Var}{Var}
\DeclareMathOperator{\Spec}{Spec}

\newcommand{\nor}{\unlhd}
\DeclareMathOperator{\aut}{Aut}
\DeclareMathOperator{\orb}{Orb}
\DeclareMathOperator{\GL}{GL}
\DeclareMathOperator{\SL}{SL}
\DeclareMathOperator{\SO}{SO}
\DeclareMathOperator{\PGL}{PGL}
\DeclareMathOperator{\PSL}{PSL}
\DeclareMathOperator{\stab}{Stab}
\DeclareMathOperator{\fix}{Fix}
\DeclareMathOperator{\Th}{Th}
\DeclareMathOperator{\Ind}{Ind}
\DeclareMathOperator{\Res}{Res}
\DeclareMathOperator{\Ann}{Ann}
\DeclareMathOperator{\rad}{rad}
\DeclareMathOperator{\len}{len}
\DeclareMathOperator{\ord}{ord}

% \DeclareMathOperator{\arg}{arg}

%% misc
\newcommand{\<}{\langle}
\renewcommand{\>}{\rangle}
\renewcommand{\^}{\wedge}
\renewcommand{\v}{\vee}
\def\Xint#1{\mathchoice
	{\XXint\displaystyle\textstyle{#1}}%
	{\XXint\textstyle\scriptstyle{#1}}%
	{\XXint\scriptstyle\scriptscriptstyle{#1}}%
	{\XXint\scriptscriptstyle\scriptscriptstyle{#1}}%
	\!\int}
\def\XXint#1#2#3{{\setbox0=\hbox{$#1{#2#3}{\int}$ }
		\vcenter{\hbox{$#2#3$ }}\kern-.6\wd0}}
\def\ddashint{\Xint=}
\def\dashint{\Xint-}
%% arrows
\newcommand{\xhra}{\xhookrightarrow}
\newcommand{\xra}{\xrightarrow}
\newcommand{\ra}{\rightarrow}
\newcommand{\rra}{\rightrightarrows}
\newcommand{\lra}{\longrightarrow}
\newcommand{\Ra}{\Rightarrow}
\newcommand{\lRa}{\Longrightarrow}
\newcommand{\lrsa}{\leftrightsquiqarrow}
\newcommand{\ba}{\leftrightarrow}
%% lists
\newcommand{\be}{\begin{enumerate}[(i)]}
	\newcommand{\ee}{\end{enumerate}}
%% integration stuff
\newcommand{\calR}{\mathcal{R}}
\newcommand{\rint}{\calR\!\int}
\newcommand{\calL}{\mathcal{L}}
\newcommand{\lowerint}{\mbox{\b{$\int$}}}
\newcommand{\upperint}{{\textstyle\bar{\int}}}
%% end of proof
\def\endproof{{\hfill $\Box$}}
%% matrix shorthand

\title{Simple Guide to Solving Problems in Algebraic Topology}
\author{Jalen Chrysos}
\begin{document}
	\begin{abstract}This is a consolidation of my algebraic topology notes. I don't intend to replicate any proofs of important statements unless they're immediate. I'm just aiming to make a dummyproof flowchart-style guide to solving Hatcher questions for my own use.
	\end{abstract}
	\maketitle
	
	\section{Basic Terminology} 
	The most basic questions one asks in topology are 
	\begin{center}
		``Does there exist a map $X\to Y$ with property $P$?"
	\end{center} 
	To be able to answer these questions positively, we need methods of constructing maps, but this is usually the simple part. To answer them negatively, we need \textit{algebraic invariants}, properties of a space which are preserved by sufficiently nice maps.\\
	
	\textbf{Types of maps} (in order of increasing strictness):
	\begin{itemize}
		\item Homotopy equivalence: continuous maps with continuous ``inverses'' \textit{up to homotopy equivalence}.
		\item Homeomorphism: continuous maps with continuous inverses.
	\end{itemize}
	
	It's a lot easier to work with homotopy equivalence in practice because there's more you can do (contract simply-connected subspaces for example). The invariants of interest are all preserved by homotopy equivalence:\\
	
	\textbf{Algebraic Invariants}:
	\begin{itemize}
		\item Fundamental group: $\pi_1(X)$
		\item Homology groups: $H_0(X),H_1(X),\dots$
		\item Cohomology groups: $H^0(X;G),H^1(X;G),\dots$
		\item Cohomology ring: $H^*(X;G)$ as a graded ring with $\smile$.
		\item Higher homotopy groups: $\pi_2(X),\pi_3(X),\dots$
	\end{itemize}
	
	First, we'll discuss how to compute all of these algebraic invariants, or at least glean information about them. We'll assume $X$ is a connected CW complex.\\
	
	\section{Fundamental Group}
	
	The fundamental group of $X$ is a group $\pi_1(X)$ whose elements are homotopy equivalence classes of loops in $X$, and whose multiplication is the concatenation of loops (at some base point, but if $X$ is connected then the basepoint doesn't matter).\\
	
	\underline{Quick Facts}:
	\begin{itemize}
		\item If $\pi_1(X)=0$ then we say $X$ is ``simply connected,'' i.e. all loops can be contracted.
		\item $\pi_1(X)$ is in general non-Abelian.
	\end{itemize}
	
	\underline{How to Calculate}: Use \textbf{Van Kampen's Theorem}:
	\begin{itemize}
		\item Write $X$ as a CW complex.
		\item If $X$ has more than one 0-cell, quotient by a maximal spanning tree in $X^1$ so that $X$ has a single 0-cell (this is a homotopy equivalence so it preserves $\pi_1(X)$).
		\item Let the 1-cells (loops) at this 0-cell be $\a_1,\a_2,\dots,\a_k$. The boundaries of the 2-cells are loops in $X^1$, so they can each be written as products of the $\a_j$. Let these boundaries be $\b_1,\b_2,\dots,\b_{\ell} \in \<\a_1,\dots,\a_k\>$. 
		\item Now $\pi_1(X)$ is the free group on $\a_1,\dots,\a_k$ mod the boundaries of the 2-cells:
		$$
		\pi_1(X) := \<\a_1,\dots,\a_k \; | \; \b_1=\b_2=\cdots=\b_{\ell}=1\>.
		$$
	\end{itemize}
	
	\section{Homology}
	
	The Homology Groups of $X$ are a sequence $H_0(X),H_1(X),\dots$ of \textit{Abelian groups}, so they are of the form $\Z^a\oplus \Z_2^{b_2}\oplus \Z_3^{b_3}\oplus \Z_5^{b_5}\oplus \cdots$. \\
	
	\underline{Quick Facts}:
	\begin{itemize}
	\item $H_0(X)=\Z^a$ where $a$ is the number of connected components of $X$. 
	\item $H_1(X)$ is the Abelianization of $\pi_1(X)$.
	\item $H_n(X)=0$ for $n>\dim(X)$. 
	\item If $X$ is a manifold of dimension $n$:
		\begin{itemize}
			\item $H_n(X)=\Z$ if $X$ is oriented and $H_n(X)=0$ otherwise.
			\item $H_{n-1}(X)$ has torsion subgroup $0$ if $X$ is oriented and $\Z_2$ otherwise.
		\end{itemize}
	\end{itemize}
	
	\underline{How to Calculate}: There are a couple of useful methods.
	\begin{enumerate}[(I)]
		\item \textbf{Quotients}: 
		\begin{itemize}
			\item Express $X$ as a quotient $Y/A$ where $Y$ is a CW complex and $A$ is a subcomplex. Try to choose $Y$ and $A$ such that $H_*(Y)$ and $H_*(A)$ are both simpler than $H_*(X)$.
			\item There is a long exact sequence relating $\tilde{H}_*(A),\tilde{H}_*(Y),$ and $\tilde{H}_*(X)$:
			$$
			\cdots \to \tilde{H}_j(A) \to \tilde{H}_j(Y) \to \tilde{H}_j(Y/A) \to \tilde{H}_{j-1}(A)\to \tilde{H}_{j-1}(Y)\to \tilde{H}_{j-1}(Y/A)\to \cdots 
			$$
			\item Now use $\tilde{H}_*(Y)$ and $\tilde{H}_*(A)$, along with exactness of the sequence, to deduce $\tilde{H}_*(X)$.
			\item \textit{Special Cases / Notable Examples}: 
			\begin{enumerate}[(i)]
				\item If $A$ is contractible then $H_*(A)=0$, so $H_*(X)=H_*(Y)$.
				\item When $X=S^n$, we can express it as $X=D^n/S^{n-1}$. Because $H_*(D^n)=0$, the exact sequence yields $\tilde{H}_j(S^n)\cong \tilde{H}_{j-1}(S^{n-1})$, so by induction we can see that $\tilde{H}_j(S^n)=\Z$ iff $j=n$, otherwise it is 0.
			\end{enumerate}
		\end{itemize} 
		\item \textbf{Mayer-Vietoris}:
		\begin{itemize}
			\item Express $X$ as the union of two \textit{open} subcomplexes $A\cup B$. Try to choose $A$ and $B$ so that $A$,$B$, and $A\cap B$ are all simpler than $X$.
			\item There is a long exact sequence relating $H_*(A),H_*(B),H_*(A\cap B),$ and $H_*(X)$:
			$$
			\cdots \to H_j(A\cap B) \to H_j(A)\oplus H_j(B) \to H_j(X) \to H_{j-1}(A\cap B)\to \cdots
			$$ 
			\item Now use $H_*(A),H_*(B),H_*(A\cap B)$, and the exactness of the sequence to deduce $H_*(X)$.
			\item \textit{Special Cases / Notable Examples}:
			\begin{enumerate}[(i)]
				\item If $A\cap B$ is contractible, then $H_j(X)\cong H_j(A)\oplus H_j(B)$. In particular, we see that $H_j(A\vee B) = H_j(A)\oplus H_j(B)$.
				\item 
			\end{enumerate}
		\end{itemize}
		\item \textbf{Cellular Homology}: {\color{red} Not done}.
		\begin{itemize}
			\item Write $X$ as a CW complex.
			\item For each $j$,
		\end{itemize}
		\item \textbf{K\"unneth Formula}:
		\begin{itemize}
			\item Use this if $X$ happens to be a product $A\times B$ and $H_*(B)$ is free in every dimension.
			\item There is an isomorphism 
			$$
			H^*(A\times B;\Z) \cong H^*(A;\Z) \otimes H^*(B;\Z) 
			$$
			of the cohomology rings, and in particular 
			$$
			H^n(A\times B;\Z) \cong \bigoplus_{j=0}^n H^j(A;\Z) \oplus H^{n-j}(B;\Z)
			$$
			\item Putting this through the universal coefficient theorem,
			$$
			\Hom(H_n(A\times B;\Z);\Z) \cong \bigoplus_{j=0}^n \Hom(H_j(A),\Z) \oplus \Ext(H_{j-1}(A),\Z) \oplus \Hom(H_{n-j}(B),\Z)
			$$
		\end{itemize}
	\end{enumerate}
	
	\section{Cohomology}
	
	The Cohomology of $X$ (wrt $G$) is a sequence of Abelian groups $H^0(X;G),H^1(X;G),\dots$.\\
	
	\underline{How to Calculate}: If you know $H_*(X)$, then use \textbf{Universal Coefficient Theorem}:
	$$
	H^j(X;G) = \Hom(H_j(X),G)\oplus \Ext(H_{j-1}(X),G).
	$$
	How to calculate Ext? Each $H_k(X)$ is an Abelian group, so it has some free parts and some torsion parts. If $H_k(X)=\Z^a \oplus (\Z_2)^{b_2}\oplus (\Z_3)
	^{b_3} \oplus (\Z_5)^{b_5}\oplus \cdots $ then 
	$$
	\Ext(H_k(X);G) = (G/(2G))^{b_2} \oplus (G/(3G))^{b_3}\oplus (G/(5G))^{b_5}\cdots
	$$
	(note that the free part $\Z^a$ is dropped entirely).\\
	
	If it's not easy to calculate $H_*(X)$ then there's basically only one other thing you can do. If you're in the specific case that $G=R$ a ring, and $X=A\times B$, and $H^*(A;R),H^*(B;R)$ are both known, and finally $H^*(B;R)$ is free in every dimension, then you can apply the \textbf{K\"unneth formula}:
	$$
	H^*(X;R) = H^*(A\times B;R) = H^*(A;R)\otimes_R H^*(B;R).
	$$
	
	The cohomology ring is a strictly finer invariant. Two spaces might have the same cohomology groups in every dimension but different ring structures
	
	\section{Homotopy Groups}
	
	\section{Invariants of Common Spaces}
\end{document}