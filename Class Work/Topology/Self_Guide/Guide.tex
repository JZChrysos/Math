\documentclass{amsart}

%\documentclass{amsart}
\usepackage[utf8]{inputenc}
\usepackage{amsfonts}
\usepackage{amsmath}
\usepackage{amssymb}
\usepackage{amsthm}
\usepackage{asymptote}
\usepackage{mathtools}
\usepackage{hhline}
\usepackage{graphicx,enumerate}
\usepackage{hyperref}
\usepackage[a4paper, margin=1.2in]{geometry}
%\usepackage{tcolorbox}
\usepackage{tikz-cd}
\usepackage{ytableau}
%\tcbuselibrary{skins,breakable,xparse}
\allowdisplaybreaks
\newcounter{count}
\hypersetup{
	colorlinks=true,
	linkcolor=teal,
	filecolor=magenta,      
	urlcolor=olive,
	citecolor=teal,
	pdfpagemode=FullScreen,
}

%\definecolor{defcolor}{HTML}{478EFF}
%\definecolor{thmcolor}{HTML}{CC0058}
%\definecolor{excolor}{HTML}{F5B400}
%\definecolor{probcolor}{HTML}{DD4803}
%\definecolor{lemcolor}{HTML}{741FEA}
%\definecolor{scarlet}{HTML}{A81111}
%
%\newtheoremstyle{definitionStyle}% Custom style for definitions
%{0.5em}% Space above
%{0.5em}% Space below
%{}% Body font
%{}% Indent amount
%{\bfseries\color{defcolor}}% Theorem head font: bold and red
%{.\\}% Punctuation after theorem head
%{0.5em}% Space after theorem head
%{\thmname{#1}\thmnumber{ #2 (#3)}}% Theorem head spec
%
%\theoremstyle{definitionStyle}
%\newtheorem{df}{Definition}[section]
%
%\newtheoremstyle{theoremStyle}% Custom style for definitions
%{0.5em}% Space above
%{0.5em}% Space below
%{}% Body font
%{}% Indent amount
%{\bfseries\color{thmcolor}}% Theorem head font: bold and red
%{.\\}% Punctuation after theorem head
%{0.5em}% Space after theorem head
%{\thmname{#1}\thmnumber{ #2 (#3)}}% Theorem head spec
%
%\theoremstyle{theoremStyle}
%\newtheorem{thm}{Theorem}[section]
%
%\newtheoremstyle{lemmaStyle}% Custom style for definitions
%{0.5em}% Space above
%{0.5em}% Space below
%{}% Body font
%{}% Indent amount
%{\bfseries\color{lemcolor}}% Theorem head font: bold and red
%{.\\}% Punctuation after theorem head
%{0.5em}% Space after theorem head
%{\thmname{#1}\thmnumber{ #2 (#3)}}% Theorem head spec
%
%\theoremstyle{lemmaStyle}
%\newtheorem{lem}{Lemma}[section]
%\newtheorem{cor}{Corollary}[section]
%
%\newtheoremstyle{exampleStyle}% Custom style for definitions
%{0.5em}% Space above
%{0.5em}% Space below
%{}% Body font
%{}% Indent amount
%{\bfseries\color{excolor}}% Theorem head font: bold and red
%{.\\}% Punctuation after theorem head
%{0.5em}% Space after theorem head
%{\thmname{#1}\thmnumber{ #2 (#3)}}% Theorem head spec
%
%\theoremstyle{exampleStyle}
%\newtheorem{ex}{Example}[section]
%
%\newtheoremstyle{problemStyle}% Custom style for definitions
%{0.5em}% Space above
%{0.5em}% Space below
%{}% Body font
%{}% Indent amount
%{\bfseries\color{probcolor}}% Theorem head font: bold and red
%{.\\}% Punctuation after theorem head
%{0.5em}% Space after theorem head
%{\thmname{#1}\thmnumber{ #2#3}}% Theorem head spec
%
%\theoremstyle{problemStyle}
%\newtheorem{prob}{Problem}[section]

% For Fun
\newcommand{\club}{\color{teal} \clubsuit}
\newcommand{\heart}{\color{red} \heartsuit}
\renewcommand{\star}{\color{scarlet} \bigstar}
\newcommand{\spade}{\color{violet} \spadesuit}

% Symbols
\newcommand{\A}{\mathcal{A}}
\newcommand{\B}{\mathcal{B}}
\newcommand{\C}{\mathbb{C}}
\newcommand{\D}{\mathcal{D}}
\newcommand{\E}{\mathbb{E}}
\newcommand{\F}{\mathbb{F}}
\newcommand{\G}{\mathcal{G}}
% \renewcommand{\H}{\mathcal{H}} Erdos o
\newcommand{\I}{\mathcal{I}}
\newcommand{\J}{\mathcal{J}}
\newcommand{\K}{\mathcal{K}}
% \renewcommand{\L}{\mathcal{L}}
\newcommand{\M}{\mathcal{M}}
\newcommand{\N}{\mathbb{N}}
\renewcommand{\O}{\mathcal{O}}
\renewcommand{\P}{\mathbb{P}}
\newcommand{\Q}{\mathbb{Q}}
\newcommand{\R}{\mathbb{R}}
\renewcommand{\S}{\mathbb{S}}
\newcommand{\T}{\mathbb{T}}
\newcommand{\U}{\mathcal{U}}
\newcommand{\V}{\mathcal{V}}
\newcommand{\W}{\mathcal{W}}
\newcommand{\X}{\mathcal{X}}
\newcommand{\Y}{\mathcal{Y}}
\newcommand{\Z}{\mathbb{Z}}

\renewcommand{\AA}{\mathcal{A}}
\newcommand{\BB}{\mathcal{B}}
\newcommand{\CC}{\mathcal{C}}
\newcommand{\DD}{\mathcal{D}}
\newcommand{\EE}{\mathcal{E}}
\newcommand{\FF}{\mathcal{F}}
\newcommand{\GG}{\mathbb{G}}
\newcommand{\HH}{\mathbb{H}}
\newcommand{\calH}{\mathcal{H}}
\newcommand{\II}{\mathcal{I}}
\newcommand{\JJ}{\mathcal{J}}
\newcommand{\KK}{\mathcal{K}}
\newcommand{\LL}{\mathcal{L}}
\newcommand{\MM}{\mathcal{M}}
\newcommand{\NN}{\mathcal{N}}
\newcommand{\OO}{\mathrm{O}}
\newcommand{\PP}{\mathcal{P}}
\newcommand{\QQ}{\mathcal{Q}}
\newcommand{\RR}{\mathcal{R}}
\renewcommand{\SS}{\mathcal{S}}
\newcommand{\TT}{\mathcal{T}}
\newcommand{\UU}{\mathcal{U}}
\newcommand{\VV}{\mathcal{V}}
\newcommand{\WW}{\mathcal{W}}
\newcommand{\XX}{\mathcal{X}}
\newcommand{\YY}{\mathcal{Y}}
\newcommand{\ZZ}{\mathcal{Z}}
\renewcommand{\d}{\textrm{d}}
% Greek letters
\newcommand{\ep}{\varepsilon}
\newcommand{\ph}{\varphi}
\newcommand{\de}{\delta}
\renewcommand{\a}{\alpha}
\renewcommand{\b}{\beta}
% Fraktur
\newcommand{\mm}{\mathfrak{m}}
\renewcommand{\aa}{\mathfrak{a}}
\newcommand{\bb}{\mathfrak{b}}
\newcommand{\pp}{\mathfrak{p}}
\newcommand{\qq}{\mathfrak{q}}
% Operators
\DeclareMathOperator{\Div}{div}
\DeclareMathOperator{\Gal}{Gal}
\DeclareMathOperator{\vol}{Vol}
\DeclareMathOperator{\Hom}{Hom}
\DeclareMathOperator{\End}{End}
\DeclareMathOperator{\Ext}{Ext}
\DeclareMathOperator{\Tor}{Tor}
\DeclareMathOperator{\tr}{tr}
\DeclareMathOperator{\rk}{rk}
\DeclareMathOperator{\curl}{curl}
\DeclareMathOperator{\mesh}{mesh}
\DeclareMathOperator{\im}{im}
\DeclareMathOperator{\coker}{coker}
\DeclareMathOperator{\width}{width}
\DeclareMathOperator{\diam}{diam}
\DeclareMathOperator{\maps}{Maps}
\DeclareMathOperator{\Frac}{Frac}
\DeclareMathOperator{\Sym}{Sym}
\DeclareMathOperator{\sgn}{sgn}
\DeclareMathOperator{\alt}{Alt}
\DeclareMathOperator{\supp}{supp}
\DeclareMathOperator{\Span}{span}
\DeclareMathOperator{\Var}{Var}
\DeclareMathOperator{\Spec}{Spec}

\newcommand{\nor}{\unlhd}
\DeclareMathOperator{\aut}{Aut}
\DeclareMathOperator{\orb}{Orb}
\DeclareMathOperator{\GL}{GL}
\DeclareMathOperator{\SL}{SL}
\DeclareMathOperator{\SO}{SO}
\DeclareMathOperator{\PGL}{PGL}
\DeclareMathOperator{\PSL}{PSL}
\DeclareMathOperator{\stab}{Stab}
\DeclareMathOperator{\fix}{Fix}
\DeclareMathOperator{\Th}{Th}
\DeclareMathOperator{\Ind}{Ind}
\DeclareMathOperator{\Res}{Res}
\DeclareMathOperator{\Ann}{Ann}
\DeclareMathOperator{\rad}{rad}
\DeclareMathOperator{\len}{len}
\DeclareMathOperator{\ord}{ord}

% \DeclareMathOperator{\arg}{arg}

%% misc
\newcommand{\<}{\langle}
\renewcommand{\>}{\rangle}
\renewcommand{\^}{\wedge}
\renewcommand{\v}{\vee}
\def\Xint#1{\mathchoice
	{\XXint\displaystyle\textstyle{#1}}%
	{\XXint\textstyle\scriptstyle{#1}}%
	{\XXint\scriptstyle\scriptscriptstyle{#1}}%
	{\XXint\scriptscriptstyle\scriptscriptstyle{#1}}%
	\!\int}
\def\XXint#1#2#3{{\setbox0=\hbox{$#1{#2#3}{\int}$ }
		\vcenter{\hbox{$#2#3$ }}\kern-.6\wd0}}
\def\ddashint{\Xint=}
\def\dashint{\Xint-}
%% arrows
\newcommand{\xhra}{\xhookrightarrow}
\newcommand{\xra}{\xrightarrow}
\newcommand{\ra}{\rightarrow}
\newcommand{\rra}{\rightrightarrows}
\newcommand{\lra}{\longrightarrow}
\newcommand{\Ra}{\Rightarrow}
\newcommand{\lRa}{\Longrightarrow}
\newcommand{\lrsa}{\leftrightsquiqarrow}
\newcommand{\ba}{\leftrightarrow}
%% lists
\newcommand{\be}{\begin{enumerate}[(i)]}
	\newcommand{\ee}{\end{enumerate}}
%% integration stuff
\newcommand{\calR}{\mathcal{R}}
\newcommand{\rint}{\calR\!\int}
\newcommand{\calL}{\mathcal{L}}
\newcommand{\lowerint}{\mbox{\b{$\int$}}}
\newcommand{\upperint}{{\textstyle\bar{\int}}}
%% end of proof
\def\endproof{{\hfill $\Box$}}
%% matrix shorthand

\title{Simple Guide to Solving Problems in Algebraic Topology}
\author{Jalen Chrysos}
\begin{document}
	\begin{abstract}This is a consolidation of my algebraic topology notes. I don't intend to replicate any proofs of important statements unless they're immediate. I'm just aiming to make a dummyproof flowchart-style guide to solving Hatcher questions for my own use.
	\end{abstract}
	\maketitle
	
	\section{Basic Terminology} 
	The most basic questions one asks in topology are 
	\begin{center}
		``Does there exist a map $X\to Y$ with property $P$?"
	\end{center} 
	To be able to answer these questions positively, we need methods of constructing maps, but this is usually the simple part. To answer them negatively, we need \textit{algebraic invariants}, properties of a space which are preserved by sufficiently nice maps.\\
	
	\textbf{Types of maps} (in order of increasing strictness):
	\begin{itemize}
		\item Homotopy equivalence: continuous maps with continuous ``inverses'' \textit{up to homotopy equivalence}.
		\item Homeomorphism: continuous maps with continuous inverses.
	\end{itemize}
	
	It's a lot easier to work with homotopy equivalence in practice because there's more you can do (contract simply-connected subspaces for example). The invariants of interest are all preserved by homotopy equivalence:\\
	
	\textbf{Algebraic Invariants}:
	\begin{itemize}
		\item Fundamental group: $\pi_1(X)$
		\item Homology groups: $H_0(X),H_1(X),\dots$
		\item Cohomology groups: $H^0(X;G),H^1(X;G),\dots$
		\item Cohomology ring: $H^*(X;G)$ as a graded ring with $\smile$.
		\item Higher homotopy groups: $\pi_2(X),\pi_3(X),\dots$
	\end{itemize}
	
	First, we'll discuss how to compute all of these algebraic invariants, or at least glean information about them. We'll assume $X$ is a connected CW complex.\\
	
	\section{Fundamental Group}
	
	The fundamental group of $X$ is a group $\pi_1(X)$ whose elements are homotopy equivalence classes of loops in $X$, and whose multiplication is the concatenation of loops (at some base point, but if $X$ is connected then the basepoint doesn't matter).\\
	
	\underline{Quick Facts}:
	\begin{itemize}
		\item If $\pi_1(X)=0$ then we say $X$ is ``simply connected,'' i.e. all loops can be contracted.
		\item $\pi_1(X)$ is in general non-Abelian.
	\end{itemize}
	
	\underline{How to Calculate}: Use \textbf{Van Kampen's Theorem}:
	\begin{itemize}
		\item Write $X$ as a CW complex.
		\item If $X$ has more than one 0-cell, quotient by a maximal spanning tree in $X^1$ so that $X$ has a single 0-cell (this is a homotopy equivalence so it preserves $\pi_1(X)$).
		\item Let the 1-cells (loops) at this 0-cell be $\a_1,\a_2,\dots,\a_k$. The boundaries of the 2-cells are loops in $X^1$, so they can each be written as products of the $\a_j$. Let these boundaries be $\b_1,\b_2,\dots,\b_{\ell} \in \<\a_1,\dots,\a_k\>$. 
		\item Now $\pi_1(X)$ is the free group on $\a_1,\dots,\a_k$ mod the boundaries of the 2-cells:
		$$
		\pi_1(X) := \<\a_1,\dots,\a_k \; | \; \b_1=\b_2=\cdots=\b_{\ell}=1\>.
		$$
	\end{itemize}
	
	\section{Homology}
	
	The Homology Groups of $X$ are a sequence $H_0(X),H_1(X),\dots$ of \textit{Abelian groups}, so they are of the form $\Z^a\oplus \Z_2^{b_2}\oplus \Z_3^{b_3}\oplus \Z_5^{b_5}\oplus \cdots$. \\
	
	\underline{Quick Facts}:
	\begin{itemize}
	\item $H_0(X)=\Z^a$ where $a$ is the number of connected components of $X$. 
	\item $H_1(X)$ is the Abelianization of $\pi_1(X)$.
	\item $H_n(X)=0$ for $n>\dim(X)$. 
	\item If $X$ is a manifold of dimension $n$:
		\begin{itemize}
			\item $H_n(X)=\Z$ if $X$ is oriented and $H_n(X)=0$ otherwise.
			\item $H_{n-1}(X)$ has torsion subgroup $0$ if $X$ is oriented and $\Z_2$ otherwise.
		\end{itemize}
	\end{itemize}
	
	\underline{How to Calculate}: There are a couple of useful methods.
	\begin{enumerate}[(I)]
		\item \textbf{Quotients}: 
		\begin{itemize}
			\item Express $X$ as a quotient $Y/A$ where $Y$ is a CW complex and $A$ is a subcomplex. Try to choose $Y$ and $A$ such that $H_*(Y)$ and $H_*(A)$ are both simpler than $H_*(X)$.
			\item There is a long exact sequence relating $\tilde{H}_*(A),\tilde{H}_*(Y),$ and $\tilde{H}_*(X)$:
			$$
			\cdots \to \tilde{H}_j(A) \to \tilde{H}_j(Y) \to \tilde{H}_j(Y/A) \to \tilde{H}_{j-1}(A)\to \tilde{H}_{j-1}(Y)\to \tilde{H}_{j-1}(Y/A)\to \cdots 
			$$
			\item Now use $\tilde{H}_*(Y)$ and $\tilde{H}_*(A)$, along with exactness of the sequence, to deduce $\tilde{H}_*(X)$.
			\item \textit{Special Cases / Notable Examples}: 
			\begin{enumerate}[(i)]
				\item If $A$ is contractible then $H_*(A)=0$, so $H_*(X)=H_*(Y)$.
				\item When $X=S^n$, we can express it as $X=D^n/S^{n-1}$. Because $H_*(D^n)=0$, the exact sequence yields $\tilde{H}_j(S^n)\cong \tilde{H}_{j-1}(S^{n-1})$, so by induction we can see that $\tilde{H}_j(S^n)=\Z$ iff $j=n$, otherwise it is 0.
			\end{enumerate}
		\end{itemize} 
		\item \textbf{Mayer-Vietoris}:
		\begin{itemize}
			\item Express $X$ as the union of two \textit{open} subcomplexes $A\cup B$. Try to choose $A$ and $B$ so that $A$,$B$, and $A\cap B$ are all simpler than $X$.
			\item There is a long exact sequence relating $H_*(A),H_*(B),H_*(A\cap B),$ and $H_*(X)$:
			$$
			\cdots \to H_j(A\cap B) \to H_j(A)\oplus H_j(B) \to H_j(X) \to H_{j-1}(A\cap B)\to \cdots
			$$ 
			\item Now use $H_*(A),H_*(B),H_*(A\cap B)$, and the exactness of the sequence to deduce $H_*(X)$.
			\item \textit{Special Cases / Notable Examples}:
			\begin{enumerate}[(i)]
				\item If $A\cap B$ is contractible, then $H_j(X)\cong H_j(A)\oplus H_j(B)$. In particular, we see that $H_j(A\vee B) = H_j(A)\oplus H_j(B)$.
				\item 
			\end{enumerate}
		\end{itemize}
		\item \textbf{Cellular Homology}: 
		\begin{itemize}
			\item Use this if $X$ is a CW complex with relatively few cells.
			\item For each $j$, let $c_j$ be the number of $j$-cells in $X$, and let $C_j(X)\cong \Z^{c_j}$ be the Abelian group with generators corresponding to the $j$-cells of $X$.
			\item We have the long exact sequence
			$$
			C_n(X) \xrightarrow{\d_n} C_{n-1}(X) \xrightarrow{\d_{n-1}}\cdots \xrightarrow{\d_1} C_0(X) \to 0
			$$
			in which $\d_n$ acts by taking cells to their boundaries. $\d_n$ is a linear map $\Z^{c_n}\to \Z^{c_{n-1}}$, so we can see $\d_n$ as a $c_n\times c_{n-1}$ matrix.
			\item $H_j(X)$ is the quotient
			$$
			H_j(X) \cong \frac{\ker(\d_j)}{\im(\d_{j+1})}
			$$
		\end{itemize}
	\end{enumerate}
	
	\section{Cohomology}
	
	The Cohomology of $X$ (wrt $G$) is a sequence of Abelian groups $H^0(X;G),H^1(X;G),\dots$.\\
	
	\underline{How to Calculate $H^*(X;G)$}: Use \textbf{Universal Coefficient Theorem}:
		\begin{itemize}
			\item Use this if you know $H_*(X)$.
			\item The Cohomology can be determined from the Homology via the isomorphism
			$$
			H^j(X;G) = \Hom(H_j(X),G)\oplus \Ext(H_{j-1}(X),G).
			$$
			\item To calculate Ext: each $H_k(X)$ is an Abelian group, so it has some free parts and some torsion parts. If $H_k(X)=\Z^a \oplus (\Z_2)^{b_2}\oplus (\Z_3)
			^{b_3} \oplus (\Z_5)^{b_5}\oplus \cdots $ then 
			$$
			\Ext(H_k(X);G) = (G/(2G))^{b_2} \oplus (G/(3G))^{b_3}\oplus (G/(5G))^{b_5}\cdots
			$$
			(note that the free part $\Z^a$ is dropped entirely).\\
		\end{itemize}
		
		If $G=R$, a ring, then the cohomology groups $H^*(X;R)$ together form a graded ring under the cup product operation. With the additional structure of the cup product, the cohomology ring makes a strictly finer invariant than the groups alone. Two spaces might have the same cohomology groups in every dimension but different ring structures, e.g. with $S^2\vee S^4$ and $\CP^2$.\\
	
	\underline{How to Calculate $H^*(X;R)$ (as a ring)}:
	\begin{enumerate}[(I)]
		\item \textbf{K\"unneth Formula}:
	\begin{itemize}
	\item Use this if $X=A\times B$, and $H^*(A;R),H^*(B;R)$ are both known, and finally $H^*(B;R)$ is free in every dimension.
	\item There is a ring isomorphism
	$$
	H^*(A\times B;R) \cong H^*(A;R)\otimes_R H^*(B;R).
	$$
	\item In particular this also implies that 
	$$
	H^n(A\times B;R) \cong \bigoplus_{j=0}^n H^j(A;R)\otimes_R H^{n-j}(B;R) 
	$$
	as groups.
	\end{itemize}
	\item \textbf{Poincar\'e Duality}:
	\begin{itemize}
		\item Use this if $X$ is a closed, connected $R$-oriented $n$-manifold. 
		\item In this case $H^j(X;R)=H^{n-j}(X;R)$ for all $j$. We can sometimes guarantee that the generators $\pm 1 \in H^n(X;R)=\Z$ factor as cup products of elements in these two groups.
		\item Suppose $R=\Z$. If $H^j(X;\Z)=H^{n-j}(X;\Z)$ has positive rank, then for $\a$ a $\Z$-generator in $H^j(X;\Z)$ and $\b$ a $\Z$-generator in $H^{n-j}(X;\Z)$, 
		$$
		\a \smile \b = \pm 1 \in H^n(X;\Z).
		$$
		\item Suppose $R$ is a field. If $H^j(X;R)=H^{n-j}(X;R)$ are nontrivial, then for any nonzero $\a\in H^j(X;R)$ and $\b\in H^j(X;R)$,
		$$
		\a \smile \b = \pm 1 \in H^n(X;R).
		$$
	\end{itemize}
	\end{enumerate}
	
	\section{Higher Homotopy Groups}
	
	The $n$th homotopy group of $X$ is denoted $\pi_n(X)$. It is like a higher-dimensional analogue of $\pi_1(X)$, where the elements are equivalence classes of maps $S^n\to X$.\\
	
	\underline{Quick Facts}:
	\begin{itemize}
		\item $\pi_n(X)$ is Abelian for $n \geq 2$.
		\item $\pi_j(S^n)=0$ for $j<n$, $\pi_n(S^n)=\Z$.
		\item \textit{Hurewicz}: If $\pi_1(X)=0$, the first nonzero $\pi_i(X)$ and $H_j(X)$ occur at the same dimension $i=j$, and $\pi_j(X)=H_j(X)$.
		\item For any collection of path-connected spaces $\{X_{\a}\}$, $\pi_n\big(\prod_{\a} X_{\a}\big) \cong \prod_{\a}\pi_n(X_{\a})$.
	\end{itemize}
	
	\underline{How To Calculate $\pi_n(X)$}: 
	\begin{enumerate}[(I)]
		\item \textbf{Covering Space}:
		\begin{itemize}
			\item If $p:\tilde{X}\to X$ is a covering space of $X$, then 
			$$p_*:\pi_n(\tilde{X})\to \pi_n(X)$$
			is an isomorphism for all $n\geq 2$.
			\item \textit{Special Cases}:
			\begin{itemize}
				\item If $X$ has a contractible universal cover then $\pi_n(X)=0$ for $n\geq 2$.
			\end{itemize}
		\end{itemize}
		\item \textbf{Excision}:
		\begin{itemize}
			\item Express $X$ as a CW complex that is the union of two subcomplexes $A$ and $B$ whose intersection is subcomplex $C$.
			\item If $(A,C)$ is $m$-connected and $(B,C)$ is $n$-connected, then the inclusion $(A,C)\xhra{} (X,B)$ induces the map
			$$
			\pi_j(A,C)\to \pi_j(X,B) 
			$$ 
			and this map is an isomorphism for $j<m+n$, and a surjection for $j=m+n$.
			\item \textit{Special Cases}:
			\begin{itemize}
				\item If $C$ is a single point, i.e. $X=A\vee B$, then the map is $\pi_j(A)\to \pi_j(X,B)$.
				\item One can represent $X$ as $(X/A)\vee A$. Applying the theorem in this case gives
				$$
				\pi_j(X/A)\to \pi_j(X,A)
				$$
				so we can identify quotients with relative homotopy groups in sufficiently good conditions.
			\end{itemize}
		\end{itemize}
	\end{enumerate}
	
	\section{Existence Results}
	
	\underline{How to Build a Space with Given Algebraic Properties}: 
	\begin{itemize}
%		\item For any sequence of groups $G_1,G_2,\dots$ where $G_n$ is Abelian for $n\geq 2$, there is a  
		\item \textit{Moore Spaces}: For any Abelian group $G$, there is a Moore space $M(G,n)$ for which
		$$
		\tilde{H}_j(X) = \begin{cases}
			G & j=n\\
			0 & j\neq n
		\end{cases}
		$$
		\item By taking wedge products of Moore spaces $M(G_1,1),M(G_2,2),\dots$ one can construct a space with any arbitrary homology sequence.
		\item \textit{Eilenberg-MacLane}: For any group $G$, there is a space $X$ of type $K(G,1)$, i.e. for which 
		$$
		\pi_j(X) = \begin{cases}
			G & j=1\\
			0 & j>1
		\end{cases}
		$$
		Moreover, if $G$ is Abelian, then for any $n$ there is a space $X$ of type $K(G,n)$, i.e. for which
		$$
		\pi_j(X) = \begin{cases}
			G & j=n\\
			0 & j\neq n
		\end{cases}
		$$
		All $K(G,n)$ spaces are weak-homotopy equivalent, and all CW complexes of type $K(G,n)$ are fully homotopy equivalent by Whitehead's Theorem.
		\item \textit{Postnikov Tower}: For any connected CW complex $X$, there is decreasing sequence of CW complexes $X_1,X_2,\dots$ such that 
		\begin{itemize}
			\item $X$ is a subcomplex of each $X_j$,
			\item The inclusion $X\xhra{} X_j$ induces an isomorphism on $\pi_i$ for $i\leq j$,
			\item $\pi_i(X_j)=0$ for $i>j$.
		\end{itemize} 
	\end{itemize}
	
	\section{Strategies for Broad Question-Types} 
	
	\textit{``Are $X$ and $Y$ homotopy equivalent?"}
	\begin{itemize}
		\item If the answer is \textit{no}:
		\begin{itemize}
			\item You can probably determine this from their algebraic invariants. 
			\item If the fundamental group, homology, cohomology ring, or any higher homotopy groups are different, then $X$ and $Y$ are not homotopy equivalent.
			\end{itemize}
		\item If the answer is \textit{yes}:
		\begin{itemize} 
			\item Then you should construct the map. 
			\item If $X\subset Y$, then you may be able to find a deformation retraction.
			\item If you can show that $f:X\to Y$ induces an isomorphism $f_*: H_*(X)\to H_*(Y)$, then this automatically implies that $f$ is a homotopy equivalence by Whitehead's + Hurewicz's Theorems.\\
		\end{itemize}
	\end{itemize}
	
	\textit{``Is there a space $X$ with algebraic property $P$?"}
	\begin{itemize}
		\item If the answer is \textit{no}:
		\item If the answer is \textit{yes}:
	\end{itemize}
	
	\section{Worked Examples} 
	
	\textbf{Example 1}: The Klein Bottle.\\
	
	\textit{Fundamental Group}: The Klein bottle can be represented as a CW complex with one 0-cell, two 1-cells, and one 2-cell as depicted below:
	
	$$\textbf{DIAGRAM}$$
	
	Letting the 1-cells be $\a,\b$, the attaching map of the 2-cell has boundary $\a\b\a^{-1}\b$, so by Van Kampen's Theorem we have
	$$
	\pi_1(X) = \<\a,\b \; | \; \a\b\a^{-1}\b = 1\>
	$$
	In other words $\a\b = \b^{-1}\a$. Every string of $\a$ and $\b$ can be rearranged using this rule into a canonical representative $\a^n\b^m$ with $n,m\in \Z$. These multiply by
	$$
	\a^{n_1}\b^{m_1} \cdot \a^{n_2}\b^{m_2} = \a^{n_1+n_2}\b^{(-1)^{n_2}m_1 + m_2}
	$$
	(each $\b$ in $\b^{m_1}$ has its sign swapped $n_2$ times when swapping with the $\a^{n_2}$ block). Thus we can consider $\pi_1(X)$ as $\Z \ltimes \Z$ with the action $n:m\mapsto (-1)^nm$. (revisit)\\
	
	\textit{Homology}: A natural way to calculate the homology of $K$ is via cellular homology. We have the exact sequence
	$$
	0 \to \Z \xrightarrow{\d_2} \Z^2 \xrightarrow{\d_1} \Z\xrightarrow{\d_0} 0
	$$
	where the boundary maps look like 
	$$
	\d_0 : \begin{pmatrix} 0 \end{pmatrix}, \;\;\; \d_1 = \begin{pmatrix}0 & 0 \end{pmatrix},\;\;\; \d_2 = \begin{pmatrix} 0 \\ 2 \end{pmatrix} 
	$$
	noting in $\d_2$ that the boundary of the 2-cell is $\a + \b - \a + \b = 0\a + 2\b$. So we have 
	\begin{align*}
	H_0(K) &= \frac{\ker(\d_0)}{\im(\d_1)} = \frac{\Z}{0} = \Z,\\
	H_1(K) &= \frac{\ker(\d_1)}{\im(\d_2)} = \frac{\Z^2}{0\times 2\Z} = \Z\oplus \Z_2,\\
	H_2(K) &= \frac{\ker(\d_2)}{\im(\d_3)} = \frac{0}{0} = 0.
	\end{align*}
	These also could have been calculated faster by using other methods:
	\begin{itemize}
		\item $H_0(K)=\Z$ because $K$ is connected.
		\item $H_1(K)=\Z\oplus \Z_2$ because this is the abelianization of $\pi_1(K)$:
		$$
		\Ab\big(\<\a,\b\; | \; \a\b\a^{-1}\b = 1\>\big) = \Z\<\a,\b\>/(2\b) = \Z \oplus \Z_2.   
		$$ 
	\end{itemize}
	Also note that $H_2(K)=0$ implies, by Poincar\'e Duality, that $K$ is non-orientable.\\
	
	\textit{Cohomology}: Let the coefficient ring be $G$. Having solved $H_*(K)$ already, we can use Universal Coefficient Theorem:
	$$
	H^j(K;G) = \Hom(H_j(K),G) \oplus \Ext(H_{j-1}(K),G).
	$$
	In the case $G=\Z$, we have 
	\begin{align*}	
	H^0(K;\Z) &= \Hom(\Z,\Z)\oplus \Ext(0,\Z) =(\Z)\oplus (0)= \Z\\
	H^1(K;\Z) &= \Hom(\Z\oplus \Z_2,\Z)\oplus \Ext(\Z,\Z) = (\Z)\oplus(0)= \Z,\\ 
	H^2(K;\Z) &= \Hom(0,\Z) \oplus \Ext(\Z\oplus \Z_2,\Z) = (0)\oplus(\Z_2) = \Z_2.
	\end{align*}
	For the ring structure, beginning with $\Z[x,y]$ where $x$ generates $H^1(K;\Z)=\Z$ and $y$ generates $H^2(K;\Z)=\Z_2$. We clearly have the relation $2y=0$ because $H^2(K;\Z)=\Z_2$, and $xy=y^2=0$ because $H^3=H^4=0$. The only product that could be nontrivial is $x^2$.\\
	
	When $G=\Z_2$,
	\begin{align*}	
	H^0(K;\Z_2) &= \Hom(\Z,\Z_2)\oplus \Ext(0,\Z_2) =(\Z_2)\oplus (0)= \Z_2\\
	H^1(K;\Z_2) &= \Hom(\Z\oplus \Z_2,\Z_2)\oplus \Ext(\Z,\Z_2) = (\Z_2^2)\oplus(0)= \Z_2^2,\\ 
	H^2(K;\Z_2) &= \Hom(0,\Z_2) \oplus \Ext(\Z\oplus \Z_2,\Z_2) = (0)\oplus(\Z_2) = \Z_2.
	\end{align*}
	
	Now for the ring structure. \\
	
	\textit{Homotopy Groups}: There is a covering $\R^2\to K$ as depicted below:
	$$
	\textbf{Diagram}
	$$
	Xovering induces isomorphism on homotopy groups above dimension $1$, so $\pi_n(K)=0$ for $n\geq 2$.
	
	\section{Table of Computation Results}
\end{document}