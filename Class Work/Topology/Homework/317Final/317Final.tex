\documentclass{amsart}

%\documentclass{amsart}
\usepackage[utf8]{inputenc}
\usepackage{amsfonts}
\usepackage{amsmath}
\usepackage{amssymb}
\usepackage{amsthm}
\usepackage{asymptote}
\usepackage{mathtools}
\usepackage{hhline}
\usepackage{graphicx,enumerate}
\usepackage{hyperref}
\usepackage[a4paper, margin=1.2in]{geometry}
%\usepackage{tcolorbox}
\usepackage{tikz-cd}
\usepackage{ytableau}
%\tcbuselibrary{skins,breakable,xparse}
\allowdisplaybreaks
\newcounter{count}
\hypersetup{
	colorlinks=true,
	linkcolor=teal,
	filecolor=magenta,      
	urlcolor=olive,
	citecolor=teal,
	pdfpagemode=FullScreen,
}

%\definecolor{defcolor}{HTML}{478EFF}
%\definecolor{thmcolor}{HTML}{CC0058}
%\definecolor{excolor}{HTML}{F5B400}
%\definecolor{probcolor}{HTML}{DD4803}
%\definecolor{lemcolor}{HTML}{741FEA}
%\definecolor{scarlet}{HTML}{A81111}
%
%\newtheoremstyle{definitionStyle}% Custom style for definitions
%{0.5em}% Space above
%{0.5em}% Space below
%{}% Body font
%{}% Indent amount
%{\bfseries\color{defcolor}}% Theorem head font: bold and red
%{.\\}% Punctuation after theorem head
%{0.5em}% Space after theorem head
%{\thmname{#1}\thmnumber{ #2 (#3)}}% Theorem head spec
%
%\theoremstyle{definitionStyle}
%\newtheorem{df}{Definition}[section]
%
%\newtheoremstyle{theoremStyle}% Custom style for definitions
%{0.5em}% Space above
%{0.5em}% Space below
%{}% Body font
%{}% Indent amount
%{\bfseries\color{thmcolor}}% Theorem head font: bold and red
%{.\\}% Punctuation after theorem head
%{0.5em}% Space after theorem head
%{\thmname{#1}\thmnumber{ #2 (#3)}}% Theorem head spec
%
%\theoremstyle{theoremStyle}
%\newtheorem{thm}{Theorem}[section]
%
%\newtheoremstyle{lemmaStyle}% Custom style for definitions
%{0.5em}% Space above
%{0.5em}% Space below
%{}% Body font
%{}% Indent amount
%{\bfseries\color{lemcolor}}% Theorem head font: bold and red
%{.\\}% Punctuation after theorem head
%{0.5em}% Space after theorem head
%{\thmname{#1}\thmnumber{ #2 (#3)}}% Theorem head spec
%
%\theoremstyle{lemmaStyle}
%\newtheorem{lem}{Lemma}[section]
%\newtheorem{cor}{Corollary}[section]
%
%\newtheoremstyle{exampleStyle}% Custom style for definitions
%{0.5em}% Space above
%{0.5em}% Space below
%{}% Body font
%{}% Indent amount
%{\bfseries\color{excolor}}% Theorem head font: bold and red
%{.\\}% Punctuation after theorem head
%{0.5em}% Space after theorem head
%{\thmname{#1}\thmnumber{ #2 (#3)}}% Theorem head spec
%
%\theoremstyle{exampleStyle}
%\newtheorem{ex}{Example}[section]
%
%\newtheoremstyle{problemStyle}% Custom style for definitions
%{0.5em}% Space above
%{0.5em}% Space below
%{}% Body font
%{}% Indent amount
%{\bfseries\color{probcolor}}% Theorem head font: bold and red
%{.\\}% Punctuation after theorem head
%{0.5em}% Space after theorem head
%{\thmname{#1}\thmnumber{ #2#3}}% Theorem head spec
%
%\theoremstyle{problemStyle}
%\newtheorem{prob}{Problem}[section]

% For Fun
\newcommand{\club}{\color{teal} \clubsuit}
\newcommand{\heart}{\color{red} \heartsuit}
\renewcommand{\star}{\color{scarlet} \bigstar}
\newcommand{\spade}{\color{violet} \spadesuit}

% Symbols
\newcommand{\A}{\mathcal{A}}
\newcommand{\B}{\mathcal{B}}
\newcommand{\C}{\mathbb{C}}
\newcommand{\D}{\mathcal{D}}
\newcommand{\E}{\mathbb{E}}
\newcommand{\F}{\mathbb{F}}
\newcommand{\G}{\mathcal{G}}
% \renewcommand{\H}{\mathcal{H}} Erdos o
\newcommand{\I}{\mathcal{I}}
\newcommand{\J}{\mathcal{J}}
\newcommand{\K}{\mathcal{K}}
% \renewcommand{\L}{\mathcal{L}}
\newcommand{\M}{\mathcal{M}}
\newcommand{\N}{\mathbb{N}}
\renewcommand{\O}{\mathcal{O}}
\renewcommand{\P}{\mathbb{P}}
\newcommand{\Q}{\mathbb{Q}}
\newcommand{\R}{\mathbb{R}}
\renewcommand{\S}{\mathbb{S}}
\newcommand{\T}{\mathbb{T}}
\newcommand{\U}{\mathcal{U}}
\newcommand{\V}{\mathcal{V}}
\newcommand{\W}{\mathcal{W}}
\newcommand{\X}{\mathcal{X}}
\newcommand{\Y}{\mathcal{Y}}
\newcommand{\Z}{\mathbb{Z}}

\renewcommand{\AA}{\mathcal{A}}
\newcommand{\BB}{\mathcal{B}}
\newcommand{\CC}{\mathcal{C}}
\newcommand{\DD}{\mathcal{D}}
\newcommand{\EE}{\mathcal{E}}
\newcommand{\FF}{\mathcal{F}}
\newcommand{\GG}{\mathbb{G}}
\newcommand{\HH}{\mathbb{H}}
\newcommand{\calH}{\mathcal{H}}
\newcommand{\II}{\mathcal{I}}
\newcommand{\JJ}{\mathcal{J}}
\newcommand{\KK}{\mathcal{K}}
\newcommand{\LL}{\mathcal{L}}
\newcommand{\MM}{\mathcal{M}}
\newcommand{\NN}{\mathcal{N}}
\newcommand{\OO}{\mathrm{O}}
\newcommand{\PP}{\mathcal{P}}
\newcommand{\QQ}{\mathcal{Q}}
\newcommand{\RR}{\mathcal{R}}
\renewcommand{\SS}{\mathcal{S}}
\newcommand{\TT}{\mathcal{T}}
\newcommand{\UU}{\mathcal{U}}
\newcommand{\VV}{\mathcal{V}}
\newcommand{\WW}{\mathcal{W}}
\newcommand{\XX}{\mathcal{X}}
\newcommand{\YY}{\mathcal{Y}}
\newcommand{\ZZ}{\mathcal{Z}}
\renewcommand{\d}{\textrm{d}}
% Greek letters
\newcommand{\ep}{\varepsilon}
\newcommand{\ph}{\varphi}
\newcommand{\de}{\delta}
\renewcommand{\a}{\alpha}
\renewcommand{\b}{\beta}
% Fraktur
\newcommand{\mm}{\mathfrak{m}}
\renewcommand{\aa}{\mathfrak{a}}
\newcommand{\bb}{\mathfrak{b}}
\newcommand{\pp}{\mathfrak{p}}
\newcommand{\qq}{\mathfrak{q}}
% Operators
\DeclareMathOperator{\Div}{div}
\DeclareMathOperator{\Gal}{Gal}
\DeclareMathOperator{\vol}{Vol}
\DeclareMathOperator{\Hom}{Hom}
\DeclareMathOperator{\End}{End}
\DeclareMathOperator{\Ext}{Ext}
\DeclareMathOperator{\Tor}{Tor}
\DeclareMathOperator{\tr}{tr}
\DeclareMathOperator{\rk}{rk}
\DeclareMathOperator{\curl}{curl}
\DeclareMathOperator{\mesh}{mesh}
\DeclareMathOperator{\im}{im}
\DeclareMathOperator{\coker}{coker}
\DeclareMathOperator{\width}{width}
\DeclareMathOperator{\diam}{diam}
\DeclareMathOperator{\maps}{Maps}
\DeclareMathOperator{\Frac}{Frac}
\DeclareMathOperator{\Sym}{Sym}
\DeclareMathOperator{\sgn}{sgn}
\DeclareMathOperator{\alt}{Alt}
\DeclareMathOperator{\supp}{supp}
\DeclareMathOperator{\Span}{span}
\DeclareMathOperator{\Var}{Var}
\DeclareMathOperator{\Spec}{Spec}

\newcommand{\nor}{\unlhd}
\DeclareMathOperator{\aut}{Aut}
\DeclareMathOperator{\orb}{Orb}
\DeclareMathOperator{\GL}{GL}
\DeclareMathOperator{\SL}{SL}
\DeclareMathOperator{\SO}{SO}
\DeclareMathOperator{\PGL}{PGL}
\DeclareMathOperator{\PSL}{PSL}
\DeclareMathOperator{\stab}{Stab}
\DeclareMathOperator{\fix}{Fix}
\DeclareMathOperator{\Th}{Th}
\DeclareMathOperator{\Ind}{Ind}
\DeclareMathOperator{\Res}{Res}
\DeclareMathOperator{\Ann}{Ann}
\DeclareMathOperator{\rad}{rad}
\DeclareMathOperator{\len}{len}
\DeclareMathOperator{\ord}{ord}

% \DeclareMathOperator{\arg}{arg}

%% misc
\newcommand{\<}{\langle}
\renewcommand{\>}{\rangle}
\renewcommand{\^}{\wedge}
\renewcommand{\v}{\vee}
\def\Xint#1{\mathchoice
	{\XXint\displaystyle\textstyle{#1}}%
	{\XXint\textstyle\scriptstyle{#1}}%
	{\XXint\scriptstyle\scriptscriptstyle{#1}}%
	{\XXint\scriptscriptstyle\scriptscriptstyle{#1}}%
	\!\int}
\def\XXint#1#2#3{{\setbox0=\hbox{$#1{#2#3}{\int}$ }
		\vcenter{\hbox{$#2#3$ }}\kern-.6\wd0}}
\def\ddashint{\Xint=}
\def\dashint{\Xint-}
%% arrows
\newcommand{\xhra}{\xhookrightarrow}
\newcommand{\xra}{\xrightarrow}
\newcommand{\ra}{\rightarrow}
\newcommand{\rra}{\rightrightarrows}
\newcommand{\lra}{\longrightarrow}
\newcommand{\Ra}{\Rightarrow}
\newcommand{\lRa}{\Longrightarrow}
\newcommand{\lrsa}{\leftrightsquiqarrow}
\newcommand{\ba}{\leftrightarrow}
%% lists
\newcommand{\be}{\begin{enumerate}[(i)]}
	\newcommand{\ee}{\end{enumerate}}
%% integration stuff
\newcommand{\calR}{\mathcal{R}}
\newcommand{\rint}{\calR\!\int}
\newcommand{\calL}{\mathcal{L}}
\newcommand{\lowerint}{\mbox{\b{$\int$}}}
\newcommand{\upperint}{{\textstyle\bar{\int}}}
%% end of proof
\def\endproof{{\hfill $\Box$}}
%% matrix shorthand

\DeclareMathOperator{\ab}{ab}

\title{Math 317 Final}
\author{Jalen Chrysos}

\begin{document}
	\maketitle
	
	\textbf{Problem 1}: Let $F$ be the closed oriented surface of genus 2.
	\begin{enumerate}[(a)]
		\item Give a CW complex structure for $F$ with one 0-cell, four 1-cells, and one 2-cell.
		\item From this CW complex structure, write down a presentation for the fundamental group and compute the homology and cohomology in every dimension.
		\item Compute the ring structure on cohomology with coefficients in $\Z$.
	\end{enumerate}
	\begin{proof}
	(a):
	$$
	\begin{tikzcd}[sep=small]
		& \bullet & \bullet \\
		\bullet &&& \bullet \\
		\bullet &&& \bullet \\
		& \bullet & \bullet
		\arrow["a", color={rgb,255:red,214;green,92;blue,92}, from=1-2, to=1-3]
		\arrow["{d}"', color={rgb,255:red,214;green,92;blue,214}, from=1-2, to=2-1]
		\arrow["b", color={rgb,255:red,92;green,214;blue,92}, from=1-3, to=2-4]
		\arrow["{c}"', color={rgb,255:red,92;green,92;blue,214}, from=2-1, to=3-1]
		\arrow["a", color={rgb,255:red,214;green,92;blue,92}, tail reversed, no head, from=2-4, to=3-4]
		\arrow["b", color={rgb,255:red,92;green,214;blue,92}, tail reversed, no head, from=3-4, to=4-3]
		\arrow["{d}", color={rgb,255:red,214;green,92;blue,214}, from=4-2, to=3-1]
		\arrow["{c}", color={rgb,255:red,92;green,92;blue,214}, from=4-3, to=4-2]
	\end{tikzcd}
	$$
	Consider the CW structure depicted above. $a,b,c,d$ are the 1-cells, the black dots are all identified and that's the 0-cell, and the octagonal region is the 2-cell, attached via $aba^{-1}b^{-1}cdc^{-1}d^{-1}$.\\
	
	(b): By Van Kampen's Theorem, the fundamental group is generated by the 1-cells with relations given by the 2-cell's attaching map, giving
	$$
	\pi_1(\Sigma_2) = \<a,b,c,d \; | \; aba^{-1}b^{-1}cdc^{-1}d^{-1} = 1\>.
	$$
	To calculate the homology groups, $H_0(\Sigma_2)=\Z$ because $\Sigma_2$ is connected, $H_1(\Sigma_2)$ is the abelianization of $\pi_1(\Sigma_2)$, which is $\Z^4$, and $H_2(\Sigma_2)=\Z$ because $\Sigma_2$ is an oriented 2-manifold (by Poincar\'e duality). In higher dimensions the homology is 0. In summary,
	$$
	H_*(\Sigma_2) = (\Z,\Z^4,\Z,0,0,\dots).
	$$
	The cohomology can now be computed by universal coefficient theorem:
	$$
	H^j(\Sigma_2;G) = \Hom(H_{j}(\Sigma_2),G)\oplus \Ext(H_{j-1}(\Sigma_2),G) = \Hom(H_j(\Sigma_2),G)
	$$
	noting that $H_*$ is free in all dimensions so the ext vanishes. $\Hom(\Z,G)=G$ for all abelian groups $G$, as a unique homomorphism exists bringing $1$ to a given group element $g$. Likewise $\Hom(\Z^4,G)=G^4$. So we have the cohomology
	$$
	H^*(\Sigma_2;G) = (G,G^4,G,0,0,\dots).
	$$
	
	(c): Let $\a_1,\a_2,\b_1,\b_2$ be generators of $H^1(\Sigma_2;\Z)$ and let $\gamma$ be a generator of $H^2(\Sigma_2;\Z)$. We can associate to each of these cocycles a loop on $\Sigma_2$. I really don't have time to make a diagram for this but say $\a_1$ goes ``around'' hole 1, $\b_1$ goes ``through'' hole 1, and similarly for $\a_2,\b_2$. Cup product is dual to intersection, so the ones which intersect transversely will have nontrivial cup product and the others will have trivial cup product. This is of course invariant under homotopies of these loops, as we showed in class. This gives the following cup product structure (noting that orientation is reversed when swapping the order):
	\begin{center}
	\begin{tabular}{|c|c|c|c|c|}
		\hline
		$\smile$ & $\a_1$ & $\a_2$ & $\b_1$ & $\b_2$\\
		\hline
		$\a_1$ & 0 & 0 & $\gamma$ & 0 \\
		\hline
		$\a_2$ & 0 & 0 & 0 & $\gamma$ \\
		\hline 
		$\b_1$ & $-\gamma$ & 0 & 0 & 0 \\
		\hline 
		$\b_2$ & 0 & $-\gamma$ & 0 & 0\\
		\hline 
	\end{tabular}
	\end{center}
	\end{proof}

	\newpage
	\textbf{Problem 2}: Let $W$ be the space obtained from $\C^2$ by removing three complex lines: the line $x=0$, the line $y=0$, and the line $x+y=1$. 
	\begin{enumerate}[(a)]
		\item Compute the fundamental group of $W$.
		\item Show that $W$ is not a $K(\pi,1)$.
	\end{enumerate}
	\begin{proof}
%		(a): 
%		We'll calculate the fundamental group locally near each of the points $(0,0)$, $(0,1)$, and $(1,0)$ where two of the lines intersect. Near $(0,0)$, $W$ is $(\C-\{0\})^2$, so its fundamental group is $\pi_1((S^1)^2) = \Z^2$. Similarly, near $(0,1)$ we can deformation retract the space to the torus by mapping a point $(x,y)$ as follows: assume $|x|,|y|<\ep$. First project $y$ away from $1-x$ and onto a circle of radius $\ep$ around $1\in \C$. Then project $x$ onto the circle of radius $\ep/2$ around $0\in \C$ (note that the second step avoids $x+y=1$ because $|x|<|1-y|=\ep$ during the retraction). This ends on the product of two circles, another torus with fundamental group $\Z^2$. By symmetry, the same is true near $(1,0)$. So the fundamental group is locally $\Z^2$ near each of these three points.\\
		(a): We use Van Kampen's theorem to deduce $\pi_1(W)$ by breaking it into three pieces. Consider the following three open subsets $P,Q,R\subset W$:
		\begin{align*}
			P &:= \{(x,y)\in \C^2 \; | \; x\neq 0, y \neq 0, \Re(x+y)<1\}\\
			Q &:= \{(x,y)\in \C^2 \; | \; x\neq 0, x + y \neq 1, \Re(y) > 0\}\\
			R &:= \{(x,y)\in \C^2 \; | \; y\neq 0, x + y \neq 1, \Re(x)> 0\}.
		\end{align*}
		Note that $P\cup Q\cup R=W$; clearly all three are subsets of $W$, and if $(x,y)\in W$ but not in $Q,R$ then $\Re(x),\Re(y)<0$ so $\Re(x+y)<0< 1$, thus $(x,y)\in P$. 
		
		First, we show that $\pi_1(P)=\pi_1(Q)=\pi_1(R)=\Z^2$ by deformation-retracting each of $P,Q,R$ to a torus. But note that $P,Q,R$ are all symmetric; taking a linear change of variables $z=1-x-y$, we have
		$$
		Q = \{(x,z) \in \C^2 \; |\; x\neq 0, z\neq 0, \Re(x+z)<1\} = P
		$$
		so in fact it suffices to just show this for $P$.
		\begin{itemize}
			\item $P$: Project $(x,y)$ through $(0,0)$ to $(\ep x/|x|, \ep y/|y|) \in S^1\times S^1$. Note that as $x+y$ begins with real part $<1$, and its real part linearly shifts to something at most $2\ep$, it cannot go above 1 during this transformation. Also $x,y$ never become 0. So we stay in $P$.
		\end{itemize}
		
		Now we can show that each of the pairwise intersections has fundamental group $\Z$ by showing that each can be deformation retracted to a circle. Again, they are symmetric so I didn't really need to do all three, but I'll leave them in anyways:
		\begin{itemize}
			\item $P\cap Q$: Send $(x,y)$ linearly to $(\ep x/|x|,\ep)\in S^1\times \{\ep\}$ where $\ep\in \R^+$ (this relies on the fact that $x\neq 0$ initially). Note that $\Re(x+y)$ is never 1 during this, nor is $\Re(y)$ ever 0.
			\item $P\cap R$: Send $(x,y)$ linearly to $(\ep,\ep y/|y|)\in \{\ep\}\times S^1$.
			\item $Q\cap R$: Let $S_{1/2}(\ep)$ be the circle of radius $\ep$ centered at $\tfrac12\in \R$. We are given that $x,y$ both have positive real part and that $m:=\tfrac12(x+y)$ is not $\tfrac12$. Thus project $m$ from $\tfrac12$ onto $S_{1/2}(\ep)$ to get $m^*$ and send $(x,y)$ linearly to $(m^*,m^*)$. One can easily see that $m$ is going away from $\tfrac12$ during this retraction and thus cannot become $\tfrac12$, and similarly $x,y$ retain positive real part.
		\end{itemize}
		
		We can also easily check that $P\cap Q \cap R$ is path-connected by linearly mapping $(x,y)$ to $(\tfrac14,\tfrac14)$.
		
		Let $a,b,c$ be generators of $\pi_1(Q\cap R),\pi_1(P\cap R),\pi_1(P\cap Q)$. $\pi_1(P)=\Z^2$ is generated by the inclusions of $b$ and $c$, which we can call $b_P,c_P$, and likewise for $Q$ and $R$. By Van Kampen's theorem, we calculate $\pi_1(W)$ as 
		$$
		\pi_1(W) = \frac{\pi_1(P)\ast \pi_1(Q)\ast \pi_1(R)}{\<a_Q=a_R,b_P=b_R,c_P=c_Q\>} = \<a\>\oplus \<b\> \oplus \<c\> = \Z^3. 
		$$
		I think it makes sense to think of this visually as: if you have a composition of loops $ab$ going around the lines $x+y=1$ and $y=0$, you can homotope this loop and bring it near the point $(1,0)$ (in $R$) where locally the fundamental group is like a torus, so the loops $a$ and $b$ commute, and you can swap their order to get $ba$.\\
		
		(b): The homotopy type of a $K(G,1)$ is determined by $G$, and we already know that the space $(S^1)^3$ is a $K(\Z^3,1)$, so it suffices to show that $W$ differs from $(S^1)^3$ in any homotopy-invariant. We can calculate $H_*(W)$ using Alexander duality: considering $W$ as a subspace of $S^4$, a one-point compactification of $\C^2$ (which has the same homology by excision), let $K=S^4\setminus W$. $K$ is the union of three complex lines, which are 2-cells all attached at the infinite point, giving $K$ a CW structure with three 2-cells and one 0-cell. Thus $K$ has homology
		$$
		H_*(K) = (\Z,0,\Z^3,0,\dots)
		$$
		and the exact same integral cohomology because all of the homology groups are free abelian. By Alexander duality,
		$$
		\tilde{H}_i(W) \cong \tilde{H}^{3-i}(K) \cong \tilde{H}_{3-i}(K) 
		$$
		which gives
		$$
		H_*(W) = (\Z,\Z^3,0,0,\dots)
		$$
		so in particular $\chi(W)=-2$, but this differs from $\chi((S^1)^3)=\chi(S^1)^3=0^3=0$, so the spaces must be different, and $W$ is therefore not a $K(\Z^3,1)$.
	\end{proof}
	
	\newpage
	\textbf{Problem 3}: Give an example of a connected CW complex $X$ with fundamental group $\Z$ and vanishing homology in dimensions above 1 but for which $X$ is not homotopy-equivalent to $S^1$.
	\begin{proof}
		Consider the following 3-dimensional CW complex $X$: $X$ has one 0-cell, one 1-cell attached trivially, one 2-cell attached trivially, and one 3-cell attached via a map $\ph: S^2\to S^1\vee S^2$ corresponding to the polynomial $2-t\in \Z[t,t^{-1}]\cong \pi_2(S^1\vee S^2)$.
		
		To be more concrete, this attaching map (viewed as a loop of loops $S^1\times I \to S^1\vee S^2$) goes over the $S^2$ twice, then cycles around the $S^1$ once, then goes back over the $S^2$ in the opposite direction once, and then cycles back around the $S^1$. Note that the degree of $\ph$ restricted to $S^2$ is $(2)+(-1)=1$. 
		
		We can verify the homology groups of $X$ using cellular homology: the cellular chain of $X$ is
		$$
		\underbrace{H_3(X^3,X^2)}_{\Z}\xra{\d_3} \underbrace{H_2(X^2,X^1)}_{\Z} \xrightarrow{0} \underbrace{H_1(X^1,X^0)}_{\Z} \xrightarrow{0} \underbrace{H_0(X^0)}_{\Z}\to 0.
		$$
		Because $\ph$ has degree 1 restricted to the 2-cell, the first map is surjective, so $H_3(X)=H_2(X)=0$. And we also see that $H_1(X)=H_0(X)=\Z$.\\
		
		But $\pi_2(X)=\Z[t,t^{-1}]/(2-t) = \Z[\tfrac12]$. To see this, consider the universal cover $\overline{X}$, which is a line with 2-cells $e^2_{\a}$ attached at integer points and 3-cells $e^3_{\b}$ attached to each neighboring pair of 2-cells. $\overline{X}$ is simply connected, as is its 2-skeleton, so by limited-range excision we can say
		$$
		\pi_j(\overline{X},\overline{X}^2) \cong \pi_j(\overline{X} / \overline{X}^2)
		$$
		for $j\leq 3$. Applied to the exact sequence of the pair $(\overline{X},\overline{X}^2)$, this yields
		$$
		\pi_3(\overline{X}/\overline{X}^2) \xra{\partial} \pi_2(\overline{X}^2) \to \pi_2(\overline{X})\to \pi_2(\overline{X}/\overline{X}^2) \cong 0
		$$
		noting that $\overline{X}/\overline{X}^2$ has no 2-cells and thus by cellular approximation it has no $\pi_2$. This gives 
		$$
		\pi_2(\overline{X}) = \pi_2(\overline{X}^2) / \im(\partial) = \Z[t,t^{-1}] / (2-t) = \Z[\tfrac12].
		$$
		And $\pi_n$ for $n\geq 2$ is shared by spaces and their covering spaces, so in particular $\pi_2(X)=\pi_2(\overline{X})= \Z[\tfrac12]$. Thus, $\pi_2(X)\neq \pi_2(S^1)=0$, so $X$ is not homotopy-equivalent to $S^1$.
		
	\end{proof}
	
	\newpage
	\textbf{Problem 4}: Give an example of two connected CW complexes $X$ and $Y$ with isomorphic homotopy groups in every dimension but with different homology groups in some dimensions.
	\begin{proof}
		Similarly to the previous problem, let $X,Y$ be CW complexes defined in the following way: for $X$, start with $S^1\vee S^2$ and attach a 3-cell via an attaching map $S^2\to S^1\vee S^2$ corresponding to the polynomial $2 -t \in \Z[t,t^{-1}] \cong \pi_2(S^1\vee S^2)$, just as in problem 3. For $Y$, do the same but with the polynomial $4-t$. As established in problem 3,
		$$
		\pi_*(X) = (\Z,\Z[\tfrac12],0,\dots), \;\; H_*(X) = (\Z,0,0,\dots).
		$$
		For $Y$, $\pi_2(Y)=\Z[t,t^{-1}]/(4-t) = \Z[\tfrac14] = \Z[\tfrac12]$, so all homotopy groups are the same. However, the degree of the attaching map of the 3-cell in $Y$, restricted to $S^2$, is $(4) + (-1) = 3$, so the map
		$$
		\d_3: H_3(Y^3,Y^2)\to H_2(Y^2,Y^1)
		$$
		is multiplication by 3. Thus, $H_2(Y) = \ker(\d_2)/\im(\d_3) = \Z/3\Z$. So although $X,Y$ have the same homotopy groups, they differ in $H_2$!
	\end{proof}
	
	\newpage
	\textbf{Problem 5}: Show that the real projective plane $\RP^2$ is not the boundary of any compact 3-manifold $M$.
	\begin{proof}
%		Let $N$ be the closed 3-manifold $M/\partial M$.
%		Recall that every manifold $N$ has a 2-sheeted cover $p:\tilde{N}\to N$ where $\tilde{N}$ is orientable. This implies, in particular, that $2\chi(N)=\chi(\tilde{N})$. But any oriented 3-manifold $\tilde{N}$ has Euler characteristic 0; Poincar\'e duality and universal coefficient theorem give
%		$$
%		H_j(\tilde{N}) \cong H^{n-j}(\tilde{N};\Z) \cong \Hom(H_{n-j}(\tilde{N}),\Z) \oplus \Ext(H_{n-j-1}(\tilde{N}),\Z)
%		$$
%		which in case $j=3,2$ yield
%		\begin{align*}
%		H_3(\tilde{N}) &\cong \Hom(H_0(\tilde{N}),\Z), \\ H_2(\tilde{N}) &\cong \Hom(H_1(\tilde{N}),\Z) \oplus \Ext(H_0(\tilde{N})=\Z,\Z) = \Hom(H_1(\tilde{N}),\Z).
%		\end{align*}
%		In either case, $\rk(H_j(\tilde{N}))=\rk(H_{n-j}(\tilde{N}))$, as $\Hom(A,\Z)$ is just the free part of $A$. So in the alternating sum, opposite terms cancel and $\chi(\tilde{N})=0$. Thus, any 3-manifold $N$ has Euler characteristic $\chi(N)=\tfrac12 \chi(\tilde{N})=0$.\\
%		
%		Now we can relate $\chi(N)$ to $\chi(M)$ and $\chi(\partial M)$. 
%		First, any $n$-dimensional manifold has the long exact sequence
%		$$
%		0\to H_n(\partial M) \to H_n(M) \to H_{n-1}(M,\partial M) \to H_{n-1}(\partial M)\to \cdots \to 0
%		$$
%		which implies, because the alternating sum of ranks is 0, that 
%		$$
%		\chi(\partial M) - \chi(M) + \chi(M,\partial M) = 0.
%		$$
%		Using excision, $\tilde{H}_j(M / \partial M) = H_j(M,\partial M)$, which gives
%		$$
%		\chi(M/\partial M) = \chi(M,\partial M) + 1.
%		$$
%		This is easier to work with because $M/\partial M$ is a manifold without boundary. 
		
		In the case where $n$ is odd, any $n$-dimensional manifold $N$ without boundary (oriented or not) has Euler characteristic 0. In the oriented case this follows from Poincar\'e duality and the universal coefficient theorem:
		\begin{align*}
			\chi(N) &= \sum_{j=0}^n (-1)^j \rk H_j(N)\\
			&= \sum_{j=0}^n (-1)^j \rk H^{n-j}(N;\Z)\\
			&= \sum_{j=0}^n (-1)^j \Big(\rk \Hom(H_{n-j}(N),\Z) + \rk \Ext(H_{n-j-1}(N),\Z)\Big)\\
			&= \sum_{j=0}^n (-1)^j \rk H_{n-j}(N)\\
			&= (-1)^n\sum_{j=0}^n (-1)^{n-j}\rk H_{n-j}(N)\\
			&= (-1)^n \chi(N).
		\end{align*}
		So $\chi(N)=-\chi(N)$, implying $\chi(N)=0$. And in the general case we can use the fact that $N$ has a 2-sheeted orientable cover $\tilde{N}$, so $\chi(N)=\tfrac12 \chi(\tilde{N}) = 0$.
%		
%		Applying this fact to $M/\partial M$, a 3-dimensional manifold without boundary, yields
%		$$
%		\chi(M,\partial M) = -1 \implies \chi(\partial M) = \chi(M) + 1.
%		$$ 
%		So if $\partial M=\RP^2$ then $\chi(M)=0$, . \\
		
		Now let $D$ be the 3-manifold obtained by gluing together two copies of $M$ along their shared boundary. $D$ has two cells for every cell of $M$, except those in the boundary which are only included in $D$ once, giving
		$$
		\chi(D) = 2\chi(M) - \chi(\partial M).
		$$
		Since $D$ is a 3-manifold without boundary, $\chi(D)=0$, which yields
		$$
		\chi(\partial M) = 2\chi(M).
		$$
		Yet $\chi(\RP^2)=1$, as $\chi(\RP^2)=\tfrac12 \chi(S^2)=1$ due to the 2-sheeted covering $S^2\to \RP^2$, so it cannot be that $\partial M=\RP^2$.
		
%		But we will show that if $M$ is a 3-manifold \textit{with boundary}, then $\chi(M)=1$:\\
%		
%		First, some notation: for any space or CW pair $A$, define $\chi_2,\overline{\chi_2}$ as
%		\begin{align*}
%			\chi_2(A) &:= \sum_{j=0}^n (-1)^j\rk_{\Z_2} H_j(A;\Z_2)\\
%			\overline{\chi_2}(A) &:= \sum_{j=0}^n (-1)^j\rk_{\Z_2} H^j(A;\Z_2).
%		\end{align*}
%		When $A$ is a manifold of dimension $n$ (or pair of manifolds, following by excision) these are both equal to $\chi$: first, using universal coefficient theorem yields
%		\begin{align*}
%		H^j(A;\Z_2) &\cong \Hom(H_j(A),\Z_2)\oplus \Ext(H_{j-1}(A),\Z_2)\\
%		\rk_{\Z_2} H^j(A;\Z_2)&= \big(\rk_{\Z} H_j(A) + \rk_{\Z_2} H_j(A)\big) + \big(\rk_{\Z_2} H_{j-1}(A)\big)\\
%		\overline{\chi_2}(A)&= \chi(A) + \rk_{\Z_2}H_n(A)=\chi(A)
%		\end{align*}
%		using the fact that, by Poincar\'e duality, $H_n(A)\in \{\Z,0\}$ and in either case $\rk_{\Z_2}H_n(A)=0$. And similarly for $\chi_2$, we can use Poincar\'e duality (noting that $A$ is $\Z_2$-orientable) to get
%		\begin{align*}
%			H_j(A;\Z_2) &\cong H^{n-j}(A,\Z_2)\\
%			\chi_2(A) &= \overline{\chi_2}(A) = \chi(A)
%		\end{align*}
%		\\
		
%		By Theorem 3.43 in Hatcher (in the case $A=\partial M$, $B=\emptyset$, and $R=\Z_2$), between a $\Z_2$-oriented $n$-manifold $M$ (which is true of any manifold) and its boundary $\partial M$, we have 
%		$$
%		H^{j}(M,\partial M;\Z_2) \cong H_{n-j}(M;\Z_2) \cong H^{n-j}(M;\Z_2)
%		$$
%		for all $j$. Taking an alternating sum of ranks on both sides yields 
%		$$
%		\overline{\chi_2}(M,\partial M) = (-1)^n\chi_2(M).
%		$$
%		On the left, note that $\overline{\chi_2}(M,\partial M) = \overline{\chi_2}(M/\partial M) - 1$ by excision and UCT (they're the same except when $j=0$, for which $\Hom(H_0(M/\partial M),\Z_2)=\Z_2$). And because $A:=M/\partial M$ is a manifold with boundary, we can use Poincar\'e duality to see
%		\begin{align*}
%			H^j(A;\Z_2) &\cong \Hom(H_j(A),\Z_2)\oplus \Ext(H_{j-1}(A),\Z_2)\\
%			\rk_{\Z_2} H^j(A;\Z_2)&= \big(\rk_{\Z} H_j(A) + \rk_{\Z_2} H_j(A)\big) + \big(\rk_{\Z_2} H_{j-1}(A)\big)\\
%			\overline{\chi_2}(A)&= \chi(A) + \rk_{\Z_2}H_n(A)=\chi(A).
%		\end{align*}
%		This gives the left side the value $\overline{\chi_2}(M,\partial M) = \overline{\chi_2}(M/\partial M) - 1 = \chi(M/\partial M) - 1 = 0 - 1 = -1$, again using the fact that 3-manifolds without boundary have Euler characteristic 0.
%		
%		One the right side, we can show $\chi_2(M)=\chi(M)$, again following from the universal coefficient theorem. So $-1=(-1)^n\chi(M)$, which in the case $n=3$ gives $\chi(M)=1$.
%		With the identities shown above, this reduces to
%		$$
%		\chi(M,\partial M) = (-1)^n\chi(M).
%		$$
%		And as we showed earlier, all 3-manifolds $M$ have $\chi(M,\partial M)=-1$, so $\chi(M)=1$. This contradicts the assumption that $\chi(\partial M)=1$.
		
		
%		By universal coefficient theorem, 
%		$$
%		H_{n-j}(M;\Z_2) \cong H^j(M,\partial M;\Z_2) \cong \Hom(H_j(M,\partial M);\Z_2) \oplus \Ext(H_j(M,\partial M);\Z_2).
%		$$
%		Taking the rank of both sides yields
%		$$
%		\rk(H_{n-j}(M)) = \rk(\Hom(H_j(M,\partial M),\Z)) + \rk(\Ext(H_j(M,\partial M),\Z)) = \rk(H_j(M,\partial M)).
%		$$
%		So we can compute $\chi(M,\partial M)$ in terms of $\chi(M)$:
%		\begin{align*}
%			\chi(M,\partial M) &= \sum_{j=0}^n (-1)^j \rk(H_j(M,\partial M))\\
%			&=  \sum_{j=0}^n (-1)^j \rk(H_{n-j}(M))\\
%			&= (-1)^n \sum_{j=0}^n (-1)^{n-j}\rk(H_{n-j}(M))\\
%			&= (-1)^n\chi(M).
%		\end{align*}
%		So in this case with $n=3$, $\chi(M,\partial M)=-\chi(M)$. Now we can derive $\chi(\partial M)$ from $\chi(M)$ and $\chi(M,\partial M)$ via the long exact sequence
%		$$
%		0\to H_3(\partial M) \to H_3(M) \to H_3(M,\partial M) \to H_2(\partial M)\to \cdots \to 0.
%		$$
%		Using the fact that the alternating sum of ranks is 0, we have
%		$$
%		\chi(\partial M) - \chi(M) + \chi(M,\partial M) = 0 \; \implies \; \chi(\partial M) = 2\chi(M).
%		$$
%		Now by excision, $H_j(M/\partial M) = H_j(M,\partial M)$ except in the case $j=0$, where $H_0(M/\partial M) = \Z$ and $H_0(M,\partial M)=0$. This gives the Euler characteristic relation
%		$$
%		\chi(M,\partial M) = \chi(M/\partial M) - 1 = -1.
%		$$
%		So we have $\chi(M)=1$ and hence
%		$$
%		\chi(\partial M) = 2\chi(M) = 2.
%		$$
%		
%		But $\RP^2$ has a 2-sheeted cover by $S^2$ given by identifying antipodes, so $\chi(\RP^2)=\tfrac12 \chi(S^2) = 1$. Thus, $\partial M$ cannot be $\RP^2$.\\
%		
%		In fact, through similar steps we can also show that any odd-dimensional manifold with boundary $M$ must have $\chi(M/\partial M)=0$, $\chi(M,\partial M)=-1$, $\chi(M)=1$, and $\chi(\partial M) = 2$.
	\end{proof}
	\newpage
	\textbf{Problem 6}: Let $W$ be a closed oriented simply-connected 4-manifold and let $f:W\to W$ be a self-map which is homotopic to the identity. Show that $f$ has a fixed point.
	\begin{proof}
		We will show that the Lefschetz trace $\tau(f)$ is nonzero, from which it follows that $f$ has a fixed point by the Lefschetz fixed point theorem.\\
		
		$W$ is simply-connected, which implies that $H_1(W)=0$ (since $H_1$ is the abelianization of $\pi_1$). Because $W$ is an oriented manifold, Poincar\'e duality also gives some information about the homology groups: $H_4(W)=\Z$, 
		$$
		H_3(W) = H^1(W,\Z) = \Hom(0,\Z)\oplus \Ext(\Z,\Z) = 0
		$$
		and 
		$$
		H_2(W) = H^2(W,\Z) = \Hom(H_2(W),\Z)\oplus \Ext(0,\Z) = \Hom(H_2(W),\Z)
		$$
		which shows that $H_2(W)$ is free. Thus $W$ has homology only in even dimensions, and in particular its Euler characteristic is at least 2. And since $f$ is homotopic to the identity, $\tau(f)=\chi(W)$, as the trace of the identity map is the rank of the matrix. Thus $\tau(f)>0$, so $f$ has a fixed point.
	\end{proof}
	
	\newpage
	\textbf{Problem 7}: Exhibit a nontrivial element of $\pi_5(S^3\vee S^3)$.
	\begin{proof}
%		$S^3\times S^3$ has a CW complex structure given by one 0-cell, two 3-cells attached trivially, and one 6-cell. This 6-cell has some attaching map $\ph:S^5\to S^3\vee S^3$.
%		
%		I claim that $\ph$ is nontrivial. If not, then we can homotopically map $S^3\times S^3$ to $S^3\vee S^3\vee S^6$ by collapsing the image of the attaching map inside $S^3\vee S^3$. But these have different homology groups; $\pi_5(S^3\times S^3)=\pi_5(S^3)^2$, but $\pi_5(S^3\vee S^3\vee S^6) = \pi_5(S^3\vee S^3)$.\\
		
		In the long exact sequence of the homotopy groups of the pair $(S^3\times S^3,S^3\vee S^3)$, we can see that the generators of $\pi_n(S^3\times S^3)=\pi_n(S^3)\oplus \pi_n(S^3)$, restricted to one or the other $S^3$, are also contained in $\pi_n(S^3\vee S^3)$, so the map between them is surjective, which implies that the map $\pi_n(S^3\times S^3)\to \pi_{n-1}(S^3\times S^3,S^3\vee S^3)$ is trivial. Moreover, there is section taking these generators back. So in particular at $n=5$, the sequence
		$$
		\cdots \xra{0} \pi_{6}(S^3\times S^3,S^3\vee S^3) \xhra{\partial} \pi_5(S^3\vee S^3) \twoheadrightarrow \pi_5(S^3\times S^3)\xra{0}\cdots 
		$$
		is short exact and \textit{splits} (because of the section, by the splitting lemma) to give
		$$
		\pi_5(S^3\vee S^3) \cong \pi_6(S^3\times S^3,S^3\vee S^3) \oplus \pi_5(S^3) \oplus \pi_5(S^3).
		$$
		All three of these component parts are nontrivial. $\pi_5(S^3)=\Z_2$ via the table in Hatcher, but I don't really know what this element is. Instead, let's consider $\pi_6(S^3\times S^3,S^3\vee S^3)$. $(S^3\times S^3,S^3\vee S^3)$ is 5-connected and $S^3\vee S^3$ is 2-connected, so we can apply limited-range excision to get
		$$
		\pi_6(S^3\times S^3,S^3\vee S^3) \cong \pi_6(S^3\times S^3 / (S^3\vee S^3)) = \pi_6(S^6) = \Z.
		$$
		A generator of this group is a map including the 6-cell into $S^3\times S^3$, and its image in $\pi_5(S^3\vee S^3)$ is the 5-cell boundary of the 6-cell, i.e. the attaching map where that 5-cell is glued onto $S^3\vee S^3$.\\
		
		What \textit{is} this map $S^5\to S^3\vee S^3$? If we view $S^3\times S^3$ as a quotient of the 6-cube $I^6$,
		$$
		S^3\times S^3 = \frac{(a,b,c,d,e,f) \in I^6}{(a,b,c,d,e,f)=(a',b',c',d,e,f) \text{ for } (a,b,c),(a',b',c')\in \partial I^3, \text{ etc.}}
		$$
		then the boundary of the 6-cell is the quotient of $\partial I^6$, which is the union of 12 5-cell faces. Each of these cells corresponds to a locus where one of the six indices is fixed at 0 or 1. Each of the twelve cells maps entirely inside either $S^3_{\a}$ or $S^3_{\b}$. A cell mapping to $S^3_{\a}$ has ten neighbors, six of which map to $S^3_{\b}$, and the other four to $S^3_{\a}$. This is a pretty difficult-to-describe map.
	\end{proof}
	
	\newpage
	\textbf{Problem 8 (Hatcher 4.2.31)}: For a fiber bundle $F\to E \to B$ such that the inclusion $F\xhra{} E$ is homotopic to a contant map, show that the long exact sequence of homotopy groups breaks up into split short exact sequences giving isomorphisms $\pi_n(B)\cong \pi_n(E)\oplus \pi_{n-1}(F)$. In particular, for the Hopf bundles $S^3\to S^7\to S^4$ and $S^7\to S^{15}\to S^8$ this yields isomorphisms 
	$$
	\pi_n(S^4)\cong \pi_n(S^7)\oplus \pi_{n-1}(S^3), \;\; \pi_n(S^8)\cong \pi_n(S^{15})\oplus \pi_{n-1}(S^7)
	$$
	Thus $\pi_7(S^4)$ and $\pi_{15}(S^8)$ contain $\Z$ summands.
	\begin{proof}
		Normally, the long exact sequence is
		$$
		\pi_n(F)\to \pi_n(E)\to\pi_n(B)\to\pi_{n-1}(F)\to\pi_{n-1}(E)\to \cdots 
		$$
		If the inclusion $F\xhra{} E$ is homotopic to a constant map, then it induces the trivial map on $\pi_i(F)\to \pi_i(E)$ for all $i$. So the sequence becomes
		$$
		\pi_n(F)\xra{0} \pi_n(E)\to\pi_n(B)\to\pi_{n-1}(F)\xra{0}\pi_{n-1}(E)\to \cdots 
		$$
		making the middle part exact. 
		
		To show that it's split, by the splitting lemma it suffices to produce a group homomorphism $r:\pi_{n-1}(F)\to \pi_{n}(B)$ such that the composition 
		$$
		\pi_{n-1}(F)\xra{r} \pi_n(B) \to \pi_{n-1}(F)
		$$
		is the identity on $\pi_{n-1}(F)$. Let $r$ be the map which takes $\ph:S^{n-1}\to F$ to $\psi$ which attaches an $n$-cell to $E$ via $\ph$ on the fiber $F_b$ (where $b$ is the basepoint of $B$), and then project this down via the fiber bundle to get an $n$-cell in $B$ attached at $b$, i.e. an element of $\pi_n(B)$. It is clear to see that this is a retract of the boundary map.\\
		
		The map $S^3\to S^7$ in the Hopf fibration is homotopic to a constant map because $\pi_3(S^7)=0$ via cellular approximation, and likewise for $S^7\to S^{15}$. Setting $n$ to be 7 and 15 respectively gives 
		$$
		\pi_7(S^4) \cong \Z \oplus \pi_6(S^3), \;\; \pi_{15}(S^8) \cong \Z\oplus \pi_{14}(S^7).
		$$
		
	\end{proof}
	
	\newpage
	\textbf{Problem 9 (Hatcher 4.3.3)}: Suppose that a CW complex $X$ contains a subcomplex $S^1$ such that the inclusion $S^1\xhra{} X$ induces an injection $H_1(S^1)\to H_1(X)$ with image a direct summand of $H_1(X)$. Show that $S^1$ is a retract of $X$.
	\begin{proof}
		It suffices to prove this for connected CW complexes $X$, as every connected component not containing the $S^1$ can be crushed to any point of $S^1$ without affecting the continuity of the map from other components.\\
		
		Let $s:S^1\xhra{} X$ be the inclusion of the $S^1$ into $X$ and let $A=\im(s)$. We are given that $s$ induces an inclusion of homology groups
		$$
		H_1(S^1) = \Z \to \Z \oplus G = H_1(X)
		$$
		where $G$ is some Abelian group. Let $p:\Z\oplus G\to \Z$ be the projection in the first coordinate, forming the sequence
		$$
		H_1(S^1) \xra{s*} H_1(X) \xra{p} H_1(S^1).
		$$
		where the composition $p\circ s*$ is the identity on $\Z$.\\
		
		Because $S^1$ is a $K(\Z,1)$, by proposition 1B.9 in Hatcher (which requires that $X$ is connected), every map $\pi_1(X)\to \pi_1(S^1)=\Z$ is induced by a map $f:X\to S^1$. Using this fact, we can show that there is a map $f$ which induces $p$ on $H_1(X)$.
		
		Let $\ab:\pi_1(X)\to H_1(X)$ be the surjective group homomorphism given by abelianization. Then there is a group homomorphism $p\circ \ab:\pi_1(X)\to H_1(S^1)=\pi_1(S^1)$, and by 1B.9 it must be induced by some $f:X\to S^1$. This $f$ induces the map $p$ on $H_1(X)\to H_1(S^1)$.\\
		
		It follows that the composition $s\circ f|_A: A\to A$ induces the identity map $p\circ s^*$ on $H_1(A)=\Z$ and thus also on $\pi_1(A)=\Z$, so it is homotopic to the identity on $A$ via some homotopy $h_t:A\to A$. By the homotopy extension property for CW pairs (using the fact that $A$ is assumed to be a subcomplex of $X$), this homotopy can be extended to a homotopy $\tilde{h}_t:X\to A$. Then $\tilde{h}_1:X\to A$ is a continuous map which restricts to the identity on $A$, i.e. a retract of $X$ onto $A$. 
	\end{proof}
\end{document}