\documentclass{amsart}

%\documentclass{amsart}
\usepackage[utf8]{inputenc}
\usepackage{amsfonts}
\usepackage{amsmath}
\usepackage{amssymb}
\usepackage{amsthm}
\usepackage{asymptote}
\usepackage{mathtools}
\usepackage{hhline}
\usepackage{graphicx,enumerate}
\usepackage{hyperref}
\usepackage[a4paper, margin=1.2in]{geometry}
%\usepackage{tcolorbox}
\usepackage{tikz-cd}
\usepackage{ytableau}
%\tcbuselibrary{skins,breakable,xparse}
\allowdisplaybreaks
\newcounter{count}
\hypersetup{
	colorlinks=true,
	linkcolor=teal,
	filecolor=magenta,      
	urlcolor=olive,
	citecolor=teal,
	pdfpagemode=FullScreen,
}

%\definecolor{defcolor}{HTML}{478EFF}
%\definecolor{thmcolor}{HTML}{CC0058}
%\definecolor{excolor}{HTML}{F5B400}
%\definecolor{probcolor}{HTML}{DD4803}
%\definecolor{lemcolor}{HTML}{741FEA}
%\definecolor{scarlet}{HTML}{A81111}
%
%\newtheoremstyle{definitionStyle}% Custom style for definitions
%{0.5em}% Space above
%{0.5em}% Space below
%{}% Body font
%{}% Indent amount
%{\bfseries\color{defcolor}}% Theorem head font: bold and red
%{.\\}% Punctuation after theorem head
%{0.5em}% Space after theorem head
%{\thmname{#1}\thmnumber{ #2 (#3)}}% Theorem head spec
%
%\theoremstyle{definitionStyle}
%\newtheorem{df}{Definition}[section]
%
%\newtheoremstyle{theoremStyle}% Custom style for definitions
%{0.5em}% Space above
%{0.5em}% Space below
%{}% Body font
%{}% Indent amount
%{\bfseries\color{thmcolor}}% Theorem head font: bold and red
%{.\\}% Punctuation after theorem head
%{0.5em}% Space after theorem head
%{\thmname{#1}\thmnumber{ #2 (#3)}}% Theorem head spec
%
%\theoremstyle{theoremStyle}
%\newtheorem{thm}{Theorem}[section]
%
%\newtheoremstyle{lemmaStyle}% Custom style for definitions
%{0.5em}% Space above
%{0.5em}% Space below
%{}% Body font
%{}% Indent amount
%{\bfseries\color{lemcolor}}% Theorem head font: bold and red
%{.\\}% Punctuation after theorem head
%{0.5em}% Space after theorem head
%{\thmname{#1}\thmnumber{ #2 (#3)}}% Theorem head spec
%
%\theoremstyle{lemmaStyle}
%\newtheorem{lem}{Lemma}[section]
%\newtheorem{cor}{Corollary}[section]
%
%\newtheoremstyle{exampleStyle}% Custom style for definitions
%{0.5em}% Space above
%{0.5em}% Space below
%{}% Body font
%{}% Indent amount
%{\bfseries\color{excolor}}% Theorem head font: bold and red
%{.\\}% Punctuation after theorem head
%{0.5em}% Space after theorem head
%{\thmname{#1}\thmnumber{ #2 (#3)}}% Theorem head spec
%
%\theoremstyle{exampleStyle}
%\newtheorem{ex}{Example}[section]
%
%\newtheoremstyle{problemStyle}% Custom style for definitions
%{0.5em}% Space above
%{0.5em}% Space below
%{}% Body font
%{}% Indent amount
%{\bfseries\color{probcolor}}% Theorem head font: bold and red
%{.\\}% Punctuation after theorem head
%{0.5em}% Space after theorem head
%{\thmname{#1}\thmnumber{ #2#3}}% Theorem head spec
%
%\theoremstyle{problemStyle}
%\newtheorem{prob}{Problem}[section]

% For Fun
\newcommand{\club}{\color{teal} \clubsuit}
\newcommand{\heart}{\color{red} \heartsuit}
\renewcommand{\star}{\color{scarlet} \bigstar}
\newcommand{\spade}{\color{violet} \spadesuit}

% Symbols
\newcommand{\A}{\mathcal{A}}
\newcommand{\B}{\mathcal{B}}
\newcommand{\C}{\mathbb{C}}
\newcommand{\D}{\mathcal{D}}
\newcommand{\E}{\mathbb{E}}
\newcommand{\F}{\mathbb{F}}
\newcommand{\G}{\mathcal{G}}
% \renewcommand{\H}{\mathcal{H}} Erdos o
\newcommand{\I}{\mathcal{I}}
\newcommand{\J}{\mathcal{J}}
\newcommand{\K}{\mathcal{K}}
% \renewcommand{\L}{\mathcal{L}}
\newcommand{\M}{\mathcal{M}}
\newcommand{\N}{\mathbb{N}}
\renewcommand{\O}{\mathcal{O}}
\renewcommand{\P}{\mathbb{P}}
\newcommand{\Q}{\mathbb{Q}}
\newcommand{\R}{\mathbb{R}}
\renewcommand{\S}{\mathbb{S}}
\newcommand{\T}{\mathbb{T}}
\newcommand{\U}{\mathcal{U}}
\newcommand{\V}{\mathcal{V}}
\newcommand{\W}{\mathcal{W}}
\newcommand{\X}{\mathcal{X}}
\newcommand{\Y}{\mathcal{Y}}
\newcommand{\Z}{\mathbb{Z}}

\renewcommand{\AA}{\mathcal{A}}
\newcommand{\BB}{\mathcal{B}}
\newcommand{\CC}{\mathcal{C}}
\newcommand{\DD}{\mathcal{D}}
\newcommand{\EE}{\mathcal{E}}
\newcommand{\FF}{\mathcal{F}}
\newcommand{\GG}{\mathbb{G}}
\newcommand{\HH}{\mathbb{H}}
\newcommand{\calH}{\mathcal{H}}
\newcommand{\II}{\mathcal{I}}
\newcommand{\JJ}{\mathcal{J}}
\newcommand{\KK}{\mathcal{K}}
\newcommand{\LL}{\mathcal{L}}
\newcommand{\MM}{\mathcal{M}}
\newcommand{\NN}{\mathcal{N}}
\newcommand{\OO}{\mathrm{O}}
\newcommand{\PP}{\mathcal{P}}
\newcommand{\QQ}{\mathcal{Q}}
\newcommand{\RR}{\mathcal{R}}
\renewcommand{\SS}{\mathcal{S}}
\newcommand{\TT}{\mathcal{T}}
\newcommand{\UU}{\mathcal{U}}
\newcommand{\VV}{\mathcal{V}}
\newcommand{\WW}{\mathcal{W}}
\newcommand{\XX}{\mathcal{X}}
\newcommand{\YY}{\mathcal{Y}}
\newcommand{\ZZ}{\mathcal{Z}}
\renewcommand{\d}{\textrm{d}}
% Greek letters
\newcommand{\ep}{\varepsilon}
\newcommand{\ph}{\varphi}
\newcommand{\de}{\delta}
\renewcommand{\a}{\alpha}
\renewcommand{\b}{\beta}
% Fraktur
\newcommand{\mm}{\mathfrak{m}}
\renewcommand{\aa}{\mathfrak{a}}
\newcommand{\bb}{\mathfrak{b}}
\newcommand{\pp}{\mathfrak{p}}
\newcommand{\qq}{\mathfrak{q}}
% Operators
\DeclareMathOperator{\Div}{div}
\DeclareMathOperator{\Gal}{Gal}
\DeclareMathOperator{\vol}{Vol}
\DeclareMathOperator{\Hom}{Hom}
\DeclareMathOperator{\End}{End}
\DeclareMathOperator{\Ext}{Ext}
\DeclareMathOperator{\Tor}{Tor}
\DeclareMathOperator{\tr}{tr}
\DeclareMathOperator{\rk}{rk}
\DeclareMathOperator{\curl}{curl}
\DeclareMathOperator{\mesh}{mesh}
\DeclareMathOperator{\im}{im}
\DeclareMathOperator{\coker}{coker}
\DeclareMathOperator{\width}{width}
\DeclareMathOperator{\diam}{diam}
\DeclareMathOperator{\maps}{Maps}
\DeclareMathOperator{\Frac}{Frac}
\DeclareMathOperator{\Sym}{Sym}
\DeclareMathOperator{\sgn}{sgn}
\DeclareMathOperator{\alt}{Alt}
\DeclareMathOperator{\supp}{supp}
\DeclareMathOperator{\Span}{span}
\DeclareMathOperator{\Var}{Var}
\DeclareMathOperator{\Spec}{Spec}

\newcommand{\nor}{\unlhd}
\DeclareMathOperator{\aut}{Aut}
\DeclareMathOperator{\orb}{Orb}
\DeclareMathOperator{\GL}{GL}
\DeclareMathOperator{\SL}{SL}
\DeclareMathOperator{\SO}{SO}
\DeclareMathOperator{\PGL}{PGL}
\DeclareMathOperator{\PSL}{PSL}
\DeclareMathOperator{\stab}{Stab}
\DeclareMathOperator{\fix}{Fix}
\DeclareMathOperator{\Th}{Th}
\DeclareMathOperator{\Ind}{Ind}
\DeclareMathOperator{\Res}{Res}
\DeclareMathOperator{\Ann}{Ann}
\DeclareMathOperator{\rad}{rad}
\DeclareMathOperator{\len}{len}
\DeclareMathOperator{\ord}{ord}

% \DeclareMathOperator{\arg}{arg}

%% misc
\newcommand{\<}{\langle}
\renewcommand{\>}{\rangle}
\renewcommand{\^}{\wedge}
\renewcommand{\v}{\vee}
\def\Xint#1{\mathchoice
	{\XXint\displaystyle\textstyle{#1}}%
	{\XXint\textstyle\scriptstyle{#1}}%
	{\XXint\scriptstyle\scriptscriptstyle{#1}}%
	{\XXint\scriptscriptstyle\scriptscriptstyle{#1}}%
	\!\int}
\def\XXint#1#2#3{{\setbox0=\hbox{$#1{#2#3}{\int}$ }
		\vcenter{\hbox{$#2#3$ }}\kern-.6\wd0}}
\def\ddashint{\Xint=}
\def\dashint{\Xint-}
%% arrows
\newcommand{\xhra}{\xhookrightarrow}
\newcommand{\xra}{\xrightarrow}
\newcommand{\ra}{\rightarrow}
\newcommand{\rra}{\rightrightarrows}
\newcommand{\lra}{\longrightarrow}
\newcommand{\Ra}{\Rightarrow}
\newcommand{\lRa}{\Longrightarrow}
\newcommand{\lrsa}{\leftrightsquiqarrow}
\newcommand{\ba}{\leftrightarrow}
%% lists
\newcommand{\be}{\begin{enumerate}[(i)]}
	\newcommand{\ee}{\end{enumerate}}
%% integration stuff
\newcommand{\calR}{\mathcal{R}}
\newcommand{\rint}{\calR\!\int}
\newcommand{\calL}{\mathcal{L}}
\newcommand{\lowerint}{\mbox{\b{$\int$}}}
\newcommand{\upperint}{{\textstyle\bar{\int}}}
%% end of proof
\def\endproof{{\hfill $\Box$}}
%% matrix shorthand

\title{Math 317 HW 5}
\author{Jalen Chrysos}

\begin{document}
 \textbf{Problem 1 (Hatcher 3.2:1)}: Assuming the cup product structure on the torus $S^1\times S^1$, compute the cup product structure in $H^*(M_g)$ for $M_g$ the closed orientable surface of genus $g$, by using the quotient map from $M_g$ to the wedge of $g$ tori.
 \begin{proof}
 	Let $T=S^1\times S^1$. By the quotient map, we have
 	$$
 	H^*(M_g) = H^*(\vee_g T) = \prod_g H^*(T).
 	$$ 
 	Thus, using the fact that $H^*(T;R)$ is the exterior algebra over $R$ generated by two elements,
 	$$
 	H^*(M_g;R) = \prod_{i=1}^g \Lambda_R[\a_i,\b_i].
 	$$
 \end{proof}
 \newpage 
 \textbf{Problem 2 (Hatcher 3.2:2)}: Using the cup product $$H^k(X,A;R)\times H^{\ell}(X,B;R) \to H^{k+\ell}(X,A\cup B;R),$$ show that if $X$ is the union of contractible open subsets $A$ and $B$, then all cup products of positive-dimensional classes in $H^*(X;R)$ are 0. This applies in particular if $X$ is a suspension. Generalize to the situation that $X$ is the union of $n$ contractible open subsets, to show that all $n$-fold cup products of positive-dimensional classes are 0.
 \begin{proof}
 	If $A,B$ are contractible then $H^k(A;R)=H^{\ell}(B;R)=0$, so in the long exact sequence defining $H^k(X,A;R)$ we have
 	$$
 	0 \cong H^{k+1}(A;R) \to H^k(X,A;R) \to H^k(X;R) \to H^k(A;R) \cong 0
 	$$
 	so $H^k(X,A;R)\cong H^k(X;R)$ and similarly $H^{\ell}(X,B;R)=H^{\ell}(X;R)$. If $X=A\cup B$ then $$H^{k+\ell}(X,A\cup B;R)=H^{k+\ell}(X,X;R)=0.$$
 	So the cup product yields a map
 	$$
 	H^k(X;R)\times H^{\ell}(X;R)\to 0
 	$$
 	i.e. it is always 0.\\
 	
 	Similarly, if $X=A_1\cup A_2 \cup \cdots \cup A_n$, a finite union of contractible open sets, then $$H^{k_1+\cdots+k_n}(X,A_1\cup \cdots \cup A_n;R) = 0$$
 	and $H^{k_j}(X,A_j;R)=H^{k_j}(X;R)$, so the $n$-fold cup product is
 	$$
 	H^{k_1}(X;R)\times \cdots \times H^{k_n}(X;R)\to 0
 	$$
 	and thus it is always 0.
 \end{proof}
 \newpage 
 \textbf{Problem 3 (Hatcher 3.2:3)}: 
 \begin{enumerate}[(a)]
 	\item Using the cup product structure, show that there is no map $\RP^n\to \RP^m$ which induces a nontrivial map $H^1(\RP^m;\Z_2)\to H^1(\RP^n;\Z_2)$ if $n>m$. What is the corresponding result for maps $\CP^n\to\CP^m$?
 	\item Prove the Borsuk-Ulam theorem by the following argument: suppose on the contrary that $f:S^n\to \R^n$ satisfies $f(x)\neq f(-x)$ for all $x$. Then define $g:S^n\to S^{n-1}$ by 
 	$$
 	g(x) := \frac{f(x)-f(-x)}{|f(x)-f(-x)|}
 	$$
 	so that $g(-x)=-g(x)$ and $g$ induces a map $\RP^n\to \RP^{n-1}$. Show that part (a) applies to this map.
 \end{enumerate}
 \begin{proof}
 	(a): We know that $H^*(\RP^n;\Z_2)=\Z_2[\a]/(\a^{n+1})$. A map from $\RP^n\to \RP^m$ induces a map on the cohomology, and thus a ring homomorphism
 	$$
 	\Z_2[\b]/(\b^{m+1}) \to \Z_2[\a]/(\a^{n+1}).
 	$$
 	This map is determined by where it sends $\b$. To induce a nontrivial map on $H^1$, it would have to send $\b$ to $\a$ (the only degree-1 term in $\Z_2[\a]$), but then the map would be an isomorphism which is impossible since $n\neq m$.
 	
 	For $\CP$, we can take coefficients in $\Z$ to get the cohomology ring $H^*(\CP^n;\Z)=\Z[\a]/(\a^{n+1})$. In this case a map $\CP^n\to \CP^m$ would induce a ring map
 	$$
 	\Z[\b]/(\b^{m+1})\to \Z[\a]/(\a^{n+1}).
 	$$
 	If this is nontrivial on $H^1$, then $\b$ must be mapped to $t\a$ for some $t\in \Z$. But $\b^{m+1}$ must be mapped to 0, so $(t\a)^{m+1}=t^{m+1}\a^{m+1}=0$, which implies $\a^{m+1}=0$, but $m<n$ so this is false in $\Z[\a]/(\a^{n+1})$.\\
 	
 	(b): We can show that this map $g$ (if it exists) induces a nontrivial map on the first cohomology groups $H^1(\RP^{n-1};\Z_2)\to H^1(\RP^n;\Z_2)$. That is, $g$ sends a nontrivial cocycle (which corresponds dually to a nontrivial loop in $\RP^n$, i.e. a path in $S^n$ whose endpoints are antipodes) to a nontrivial cocycle. And this is indeed true, as $g(x)=-g(-x)$, so $g$ preserves antipodes.
 \end{proof}
 \newpage 
 \textbf{Problem 4 (Hatcher 3.2:4)}: Apply the Lefschetz fixed point theorem to show that every map $f:\CP^n\to\CP^n$ has a fixed point if $n$ is even, using the fact that $f^*:H^*(\CP^n;\Z)\to H^*(\CP^n;\Z)$ is a ring homomorphism. When $n$ is odd, show that there is a fixed point unless $f^*(\a)=-\a$ for some $\a$ that is a generator of $H^2(\CP^n;\Z)$.
 \begin{proof}
 	Recall that $H^*(\CP^n;\Z)=\Z[\a]/(\a^{n+1})$ as a ring with the cup product, where $\a$ is a generator of $H^2$. Since $f$ induces a ring map on $\Z[\a]/(\a^{n+1})$, it must send $\a$ to some $t\a$ with nonzero $t\in \Z$. Then $f^*(\a^k)=t^k\a^k$ etc. So the Lefschetz trace of $f$ is 
 	$$
 	\tau(f)= \sum_{k=0}^n t^k = \frac{1-t^{n+1}}{1-t}.
 	$$ 
 	If there are no fixed-points then $\tau(f)=0$ by Lefschetz fixed point theorem. This can only be 0 if $t\neq 1$ and $t^{n+1}=1$, which for $t\in \Z$ is only possible if $t=-1$ and $n$ is odd. In the case that $n$ is odd and $t=-1$, it follows that $f^*(\a)=-\a$.
 \end{proof}
 \newpage 
 \textbf{Problem 5 (Hatcher 3.2:8)}: Let $X$ be $\CP^2$ with a cell $e^3$ attached by a map $S^2\to \CP^1\subset \CP^2$ of degree $p$, and let $Y=M(\Z_p,2)\vee S^4$ (where $M(\Z_p,2)$ is the \textit{Moore Space} given by $S^2$ with an $S^3$ attached via a degree-$p$ attaching map). Thus $X$ and $Y$ have the same 3-skeleton but differ in the way their $4$-cells are attached. Show that $X$ and $Y$ have isomorphic cohomology rings with $\Z$ coefficients but not with $\Z_p$ coefficients.
 \begin{proof}
 	Recall that the homology of a Moore space $M(G,n)$ is 
 	$$
 	H_j(M(G,n)) = (\Z,0,0,\dots,0,n)
 	$$
 	so in particular
 	$$
 	H_j(M(\Z_p,2)) = (\Z,0,\Z_p).
 	$$
 	Using the universal coefficient theorem we can calculate the cohomology of $M(\Z_p,2)$ over $\Z$ as 
 	$$
 	H^j(M(Z_p,2);\Z) = \Hom(H_j(M(\Z_p,2)),\Z) \oplus \Ext(H_{j-1}(M(\Z_p,2)),\Z).
 	$$
 	This $\Ext$ is 0 when $H_{j-1}$ is free, i.e. for $j=1,2$. For $j=3$, we have $\Ext(H_{2},\Z)=\Ext(\Z_p,\Z)=\Z_p$ and $\Hom(\Z_p,\Z)=0$, resulting in $H^3=\Z_p$. Thus,
 	$$
 	H^j(M(Z_p,2);\Z) = (\Z,0,0,\Z_p).
 	$$
 	Now we can calculate the cohomology of $Y=M(\Z_p,2)\vee S^4$:
 	$$
 	H^*(Y;\Z) = H^*(M(\Z_p,2);\Z) \oplus H^*(S^4;\Z) = (\Z, 0 , 0 , \Z_p , \Z).
 	$$
 	It is easy to check via cellular homology that $H^*(X;\Z)=(\Z,0,0,\Z_p,\Z)$ as well. We immediately see that the square (cup product with itself) of every ring-element in $H^*(X)$ or $H^*(Y)$ is 0, since there is no nonempty pair $H^n,H^{2n}$. Thus the rings $H^*(X;\Z)$ and $H^*(Y;\Z)$ are automatically isomorphic; one can just send generators in one to same-grade generators in the other.\\
 	
 	Over $\Z_p$, things are different because $\Hom(\Z_p,\Z_p)=\Z_p$ whereas before $\Hom(\Z_p,\Z)=0$. This makes $H^2(X;\Z_p)=\Z_p$ rather than 0, and similarly for $Y$. So now there is (potentially) a non-trivial cup product $H^2\times H^2\to H^4$. In $Y$, the cup product of any two 2-cocycles is 0, as it is 0 in $M(\Z_p,2)$ (which has no $H^4$) and the ring splits:
 	$$
 	H^*(M(\Z_p,2)\vee S^4;\Z_p) = H^*(M(\Z_p,2);\Z_p) \oplus H^*(S^4;\Z_p).
 	$$
 	However, in $\CP^2$, the ring structure is $\Z_p[\a]/(\a^3)$, so there is a 2-cocycle class $\a$ with a nontrivial square, and this remains true in $X$. Thus, $H^*(X;\Z_p)$ and $H^*(Y;\Z_p)$ are non-isomorphic rings.
 \end{proof}
 \newpage 
 \textbf{Problem 6 (Hatcher 3.2:11)}: Using cup products, show that every map $S^{k+\ell}\to S^k\times S^{\ell}$ induces the trivial homomorphism $H_{k+\ell}(S^{k+\ell})\to H_{k+\ell}(S^k\times S^{\ell})$ if $k,\ell>0$.
 \begin{proof}
 	First we'll show the corresponding result on cohomology:
% 	 by the K\"unneth formula,
% 	$$
% 	H^{k+\ell}(S^k\times S^{\ell}) = \bigoplus_j H^j(S^k) \times H^{k+\ell - j}(S^{\ell}) \cong H^k(S^{k}) \times H^{\ell}(S^{\ell}) = \Z^2
% 	$$
% 	noting that $H^j(S^k)$ is trivial unless $j\in \{0,k\}$, in which case it is $\Z$.
	Let $f:S^{k+\ell}\to S^k\times S^{\ell}$. Then $f$ induces maps on cohomology:
	$$
	H^{j}(S^{k}\times S^{\ell})\to H^{j}(S^{k+\ell}).
	$$
	All of these maps are necessarily trivial for $j\not\in \{0,k+\ell\}$. In particular we can break it into $j=k,\ell$ via the cup product. $f^*$ maps 
	$$
	H^k(S^k\times S^{\ell})\times H^{\ell}(S^k\times S^{\ell})\to H^{k+\ell}(S^{k}\times S^{\ell})
	$$
	to
	$$
	H^k(S^{k+\ell})\times H^{\ell}(S^{k+\ell}) \to H^{k+\ell}(S^{k+\ell})
	$$
	and since the maps on $H^k,H^{\ell}$ are trivial, the map on $H^{k+\ell}$ must also be trivial.
	\\
	
 	Now we will transfer this result to homology. By the Universal Coefficient theorem, 
 	$$
 	H^{j}(S^{k}\times S^{\ell};\Z) = \Hom(H_j(S^k\times S^{\ell}),\Z) \oplus \Ext(H_{j}(S^k\times S^{\ell}),\Z) = \Hom(H_j(S^k\times S^{\ell}),\Z) 
 	$$
 	noting that $H_j(S^k\times S^{\ell})$ is free, whatever $j$ is, so the $\Ext$ has to be 0. So the Hom functor takes $f^*$ to $f_*$, showing that the latter is also trivial on $H_{k+\ell}$.
 \end{proof}
 \newpage 
 \textbf{Problem 7 (Hatcher 3.2:14)}: Let $q:\RP^{\infty}\to \CP^{\infty}$ be the natural quotient map obtained by regarding both spaces as quotients of $S^{\infty}$. Show that the induced map
 $$
 q^*:H^*(\CP^{\infty};\Z) \to H^*(\RP^{\infty};\Z)
 $$
 is surjective in even dimensions by showing first by a geometric argument that the restriction $q:\RP^2\to \CP^1$ induces a surjection on $H^2$, and then appealing to the cup product structures. Next, form a quotient space $X$ of $\RP^{\infty}\coprod \CP^n$ by identifying each point $x\in \RP^{2n}$ with $q(x)\in \CP^n$. Show that there are ring isomorphisms $$H^*(X;\Z) \cong \Z[\a]/(2\a^{n+1}) \; \;\; \text{and} \;\;\; H^*(X;\Z_2)\cong \Z_2[\a,\b]/(\b^2-\a^{2n+1})$$ 
 where $|\a|=2$ and $|\b|=2n+1$. Make a similar construction and analysis for the quotient map $q:\CP^{\infty}\to \HP^{\infty}$.
 \begin{proof}
 	The map $q$ specifically acts on projective points in $\RP^{\infty}$ via
 	$$
 	q: [a_1:b_1:a_2:b_2:\cdots] \mapsto [a_1 + ib_1 : a_2 + i b_2 : \cdots].
 	$$
 	The restriction to $\RP^2\to \CP^1$ looks like
 	$$
 	[a_1:b_1] \mapsto [a_1+ib_1].
 	$$
 	Looking at the induced map, we see that $q^*:H^2(\CP^1;\Z)\to H^2(\RP^2;\Z)$ is a map $q^*:\Z\to \Z_2$, thus is surjective simply because it is nontrivial. Likewise, $H^{2j}(\RP^{\infty};\Z)=\Z_2$ for all $j$, so by a similar argument we can say that $q^*$ is surjective everywhere.\\
 	
 	$X$ is the pushout of $q:\RP^{2n}\to \CP^n$ and $\RP^{2n}\xhra{} \RP^{\infty}$:
 	$$
 	\begin{tikzcd}
 		{\RP^{2n}} && {\RP^{\infty}} && {\Z[\a]/(2\a,\a^{n+1})} && {\Z[\a]/(2\a)} \\
 		\\
 		{\CP^n} && X && {\Z[\a]/(\a^{n+1})} && {H^*(X;\Z)}
 		\arrow[hook, from=1-1, to=1-3]
 		\arrow["q"', from=1-1, to=3-1]
 		\arrow[""{name=0, anchor=center, inner sep=0}, from=1-3, to=3-3]
 		\arrow[from=1-7, to=1-5]
 		\arrow[from=3-1, to=3-3]
 		\arrow[""{name=1, anchor=center, inner sep=0}, "{q^*}"', from=3-5, to=1-5]
 		\arrow[from=3-7, to=1-7]
 		\arrow[from=3-7, to=3-5]
 		\arrow["{H^*(\bullet,\Z)}", between={0.2}{0.8}, Rightarrow, from=0, to=1]
 	\end{tikzcd}
 	$$
 	The cohomologies are given by Hatcher (p. 222). Note that because $q^*$ is surjective, the generator of $H^2(\CP^n;\Z)$ does map to a generator of $H^2(\RP^{2n};\Z)$. This yields $H^*(X;\Z)=\Z[\a]/(2\a^{n+1})$.\\
 	
 	For $R=\Z_2$, $q^*$ is no longer necessarily surjective, so $H^*(X;\Z_2)$ has two generators now, $\a$ and $\b$. Calculating the cohomology groups $H^*(X;\Z_2)$, we get $\Z_2$ in even dimensions below $2n$ and in all dimensions above $2n$. Thus we can let $\a$ generate the even dimensions and $\b$ generate the odd dimensions above $2n$, so $\a\in H^2(X;\Z_2)$ and $\b\in H^{2n+1}(X;\Z_2)$. But $\a^{2n+1},\b^2\in H^{4n+2}(X;\Z_2)\cong \Z_2$, and as neither is 0 they must be equal. Thus we have $H^*(X;\Z_2) = \Z_2[\a,\b]/(\a^{2n+1}-\b^2)$.\\
 	
 	Similarly if $X'$ is constructed via the quotient map to the quaternions $\HP^{\infty}$, then the cohomology is $\Z$ in dimensions $4j$ for $j\leq n$ and in even dimensions above $4n$. One can show that $H^*(X';\Z) = \Z[\a,\b]/(\a^{2n+1}-\b^2)$ in the same way, except with $|\a|=4$ and $|\b|=4n+2$.
 \end{proof}
 \newpage 
 \textbf{Problem 8 (Hatcher 3.2:15)}: For a fixed coefficient field $F$, define the \textit{Poincar\'e series} of a space $X$ to be the formal power series $$p(t)= \sum_k a_kt^k$$
 where $a_k$ is the dimension of $H^k(X;F)$ as a vector space over $F$, assuming this dimension is finite for all $k$. Show that $p(X\times Y)=p(X)p(Y)$. Compute the Poincar\'e series for $S^n$, $\RP^n$, $\RP^{\infty}$, $\CP^n$, $\CP^{\infty}$, and the space in the preceding exercise.
 \begin{proof}
 	First, to show that $p(X\times Y)=p(X)p(Y)$ will mean that 
 	$$
 	\dim(H^k(X\times Y;F)) = \sum_{j=0}^k \dim(H^j(X;F))\cdot \dim(H^{k-j}(Y;F))
 	$$
 	This follows from the K\"unneth formula (which we can always apply for $F$ a field, since $F$-modules are vector spaces and thus free), as 
 	$$
 	H^k(X\times Y;F) \cong \bigoplus_j H^j(X;F) \otimes_F H^{k-j}(Y;F) \implies 
 	$$
 	$$
 	\dim(H^k(X\times Y;F))= \sum_j \dim(H^j(X;F) \otimes_F H^{k-j}(Y;F))= \sum_j \dim(H^j(X;F))\cdot\dim(H^{k-j}(Y;F)).
 	$$
 	
 	The Poincar\'e series are as follows:\\
 	
 	For $S^n$, $H^k(S^n;F)$ is $F$ for $k=0,n$ and $0$ otherwise, giving $p(t)=1+t^n$.\\
 	
 	For $\RP^n$, recall that the homology is
 	$$
 	H_*(\RP^n) = (\Z,\Z_2,0,\Z_2,0,\dots)
 	$$
 	ending in $\Z$ if $n$ is odd and 0 otherwise. Now, extending this to cohomology over $F$, by universal coefficient theorem we get
 	$$
 	H^j(\RP^n) = \Hom(H_j(\RP^n),F) \oplus \Ext(H_{j-1}(\RP^n),F).
 	$$
 	In the case that $F=\Z_2$, then $\Ext(\Z_2,\Z_2)=\Z_2$, but otherwise $\Ext(\Z_2,F)=F/2F=0$ and $\Hom(\Z_2,F)=0$. So in the $\Z_2$ case we get the cohomology 
 	$$
 	p(t) = 1 + t + t^2 + \cdots + t^n = \frac{1-t^{n+1}}{1-t}
 	$$
 	and in the general case $F\neq \Z_2$ we get
 	$$
 	p(t) =\begin{cases} 1 + t^n & n \text{ odd}\\
 		1 & n \text{ even}.
 		\end{cases}
 	$$
 	These become $1/(1-t)$ for $H^*(\RP^{\infty};\Z_2)$ and $1$ for $H^*(\RP^{\infty},F)$ with $F\neq \Z_2$.\\
 	
 	For $\CP^n$, the homology is 
 	$$
 	H_*(\CP^n) = (\Z,0,\Z,0,\dots,\Z).
 	$$
 	Thus by UCT, noting that $\Ext(\Z,F)=0$ and $\Hom(\Z,F)=\Z$, the cohomology is the same as homology for any field $F$. This gives the Poincar\'e series
 	$$
 	p(t) = 1 + t^2 + t^4 + \dots + t^{2n} = \frac{1-t^{2n+2}}{1-t^2}.
 	$$
 	Finally, let $X$ be the quotient space constructed in the previous problem. In that problem we saw that $H^*(X;\Z)\cong \Z[\a]/(2\a^{n+1})$ and $H^*(X;\Z_2)\cong \Z_2[\a,\b]/(\b^2-\a^{2n+1})$. In the first case, the Poincar\'e series is 
 	$$
 	1 + t^2 + t^4 + \cdots = \frac{1}{1-t^2}
 	$$
 	and in the second case it is
 	$$
 	1 + t^2 + t^4 + \cdots + t^{2n} + t^{2n+1} + t^{2n+2} + \cdots = \frac{1}{1-t} - t\Big(\frac{1-t^{2n}}{1-t^2}\Big).
 	$$
 	For the $\HP^{\infty}$ version, we get the series
 	$$
 	p(t) = 1 + t^4 + t^8 + \cdots + t^{4n} + t^{4n+2} + \cdots = \frac{1}{1-t^2}-t^2\Big(\frac{1-t^{4n}}{1-t^4}\Big).
 	$$
 \end{proof}
 \newpage 
\end{document}