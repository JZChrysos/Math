\documentclass{amsart}

%\documentclass{amsart}
\usepackage[utf8]{inputenc}
\usepackage{amsfonts}
\usepackage{amsmath}
\usepackage{amssymb}
\usepackage{amsthm}
\usepackage{asymptote}
\usepackage{mathtools}
\usepackage{hhline}
\usepackage{graphicx,enumerate}
\usepackage{hyperref}
\usepackage[a4paper, margin=1.2in]{geometry}
%\usepackage{tcolorbox}
\usepackage{tikz-cd}
\usepackage{ytableau}
%\tcbuselibrary{skins,breakable,xparse}
\allowdisplaybreaks
\newcounter{count}
\hypersetup{
	colorlinks=true,
	linkcolor=teal,
	filecolor=magenta,      
	urlcolor=olive,
	citecolor=teal,
	pdfpagemode=FullScreen,
}

%\definecolor{defcolor}{HTML}{478EFF}
%\definecolor{thmcolor}{HTML}{CC0058}
%\definecolor{excolor}{HTML}{F5B400}
%\definecolor{probcolor}{HTML}{DD4803}
%\definecolor{lemcolor}{HTML}{741FEA}
%\definecolor{scarlet}{HTML}{A81111}
%
%\newtheoremstyle{definitionStyle}% Custom style for definitions
%{0.5em}% Space above
%{0.5em}% Space below
%{}% Body font
%{}% Indent amount
%{\bfseries\color{defcolor}}% Theorem head font: bold and red
%{.\\}% Punctuation after theorem head
%{0.5em}% Space after theorem head
%{\thmname{#1}\thmnumber{ #2 (#3)}}% Theorem head spec
%
%\theoremstyle{definitionStyle}
%\newtheorem{df}{Definition}[section]
%
%\newtheoremstyle{theoremStyle}% Custom style for definitions
%{0.5em}% Space above
%{0.5em}% Space below
%{}% Body font
%{}% Indent amount
%{\bfseries\color{thmcolor}}% Theorem head font: bold and red
%{.\\}% Punctuation after theorem head
%{0.5em}% Space after theorem head
%{\thmname{#1}\thmnumber{ #2 (#3)}}% Theorem head spec
%
%\theoremstyle{theoremStyle}
%\newtheorem{thm}{Theorem}[section]
%
%\newtheoremstyle{lemmaStyle}% Custom style for definitions
%{0.5em}% Space above
%{0.5em}% Space below
%{}% Body font
%{}% Indent amount
%{\bfseries\color{lemcolor}}% Theorem head font: bold and red
%{.\\}% Punctuation after theorem head
%{0.5em}% Space after theorem head
%{\thmname{#1}\thmnumber{ #2 (#3)}}% Theorem head spec
%
%\theoremstyle{lemmaStyle}
%\newtheorem{lem}{Lemma}[section]
%\newtheorem{cor}{Corollary}[section]
%
%\newtheoremstyle{exampleStyle}% Custom style for definitions
%{0.5em}% Space above
%{0.5em}% Space below
%{}% Body font
%{}% Indent amount
%{\bfseries\color{excolor}}% Theorem head font: bold and red
%{.\\}% Punctuation after theorem head
%{0.5em}% Space after theorem head
%{\thmname{#1}\thmnumber{ #2 (#3)}}% Theorem head spec
%
%\theoremstyle{exampleStyle}
%\newtheorem{ex}{Example}[section]
%
%\newtheoremstyle{problemStyle}% Custom style for definitions
%{0.5em}% Space above
%{0.5em}% Space below
%{}% Body font
%{}% Indent amount
%{\bfseries\color{probcolor}}% Theorem head font: bold and red
%{.\\}% Punctuation after theorem head
%{0.5em}% Space after theorem head
%{\thmname{#1}\thmnumber{ #2#3}}% Theorem head spec
%
%\theoremstyle{problemStyle}
%\newtheorem{prob}{Problem}[section]

% For Fun
\newcommand{\club}{\color{teal} \clubsuit}
\newcommand{\heart}{\color{red} \heartsuit}
\renewcommand{\star}{\color{scarlet} \bigstar}
\newcommand{\spade}{\color{violet} \spadesuit}

% Symbols
\newcommand{\A}{\mathcal{A}}
\newcommand{\B}{\mathcal{B}}
\newcommand{\C}{\mathbb{C}}
\newcommand{\D}{\mathcal{D}}
\newcommand{\E}{\mathbb{E}}
\newcommand{\F}{\mathbb{F}}
\newcommand{\G}{\mathcal{G}}
% \renewcommand{\H}{\mathcal{H}} Erdos o
\newcommand{\I}{\mathcal{I}}
\newcommand{\J}{\mathcal{J}}
\newcommand{\K}{\mathcal{K}}
% \renewcommand{\L}{\mathcal{L}}
\newcommand{\M}{\mathcal{M}}
\newcommand{\N}{\mathbb{N}}
\renewcommand{\O}{\mathcal{O}}
\renewcommand{\P}{\mathbb{P}}
\newcommand{\Q}{\mathbb{Q}}
\newcommand{\R}{\mathbb{R}}
\renewcommand{\S}{\mathbb{S}}
\newcommand{\T}{\mathbb{T}}
\newcommand{\U}{\mathcal{U}}
\newcommand{\V}{\mathcal{V}}
\newcommand{\W}{\mathcal{W}}
\newcommand{\X}{\mathcal{X}}
\newcommand{\Y}{\mathcal{Y}}
\newcommand{\Z}{\mathbb{Z}}

\renewcommand{\AA}{\mathcal{A}}
\newcommand{\BB}{\mathcal{B}}
\newcommand{\CC}{\mathcal{C}}
\newcommand{\DD}{\mathcal{D}}
\newcommand{\EE}{\mathcal{E}}
\newcommand{\FF}{\mathcal{F}}
\newcommand{\GG}{\mathbb{G}}
\newcommand{\HH}{\mathbb{H}}
\newcommand{\calH}{\mathcal{H}}
\newcommand{\II}{\mathcal{I}}
\newcommand{\JJ}{\mathcal{J}}
\newcommand{\KK}{\mathcal{K}}
\newcommand{\LL}{\mathcal{L}}
\newcommand{\MM}{\mathcal{M}}
\newcommand{\NN}{\mathcal{N}}
\newcommand{\OO}{\mathrm{O}}
\newcommand{\PP}{\mathcal{P}}
\newcommand{\QQ}{\mathcal{Q}}
\newcommand{\RR}{\mathcal{R}}
\renewcommand{\SS}{\mathcal{S}}
\newcommand{\TT}{\mathcal{T}}
\newcommand{\UU}{\mathcal{U}}
\newcommand{\VV}{\mathcal{V}}
\newcommand{\WW}{\mathcal{W}}
\newcommand{\XX}{\mathcal{X}}
\newcommand{\YY}{\mathcal{Y}}
\newcommand{\ZZ}{\mathcal{Z}}
\renewcommand{\d}{\textrm{d}}
% Greek letters
\newcommand{\ep}{\varepsilon}
\newcommand{\ph}{\varphi}
\newcommand{\de}{\delta}
\renewcommand{\a}{\alpha}
\renewcommand{\b}{\beta}
% Fraktur
\newcommand{\mm}{\mathfrak{m}}
\renewcommand{\aa}{\mathfrak{a}}
\newcommand{\bb}{\mathfrak{b}}
\newcommand{\pp}{\mathfrak{p}}
\newcommand{\qq}{\mathfrak{q}}
% Operators
\DeclareMathOperator{\Div}{div}
\DeclareMathOperator{\Gal}{Gal}
\DeclareMathOperator{\vol}{Vol}
\DeclareMathOperator{\Hom}{Hom}
\DeclareMathOperator{\End}{End}
\DeclareMathOperator{\Ext}{Ext}
\DeclareMathOperator{\Tor}{Tor}
\DeclareMathOperator{\tr}{tr}
\DeclareMathOperator{\rk}{rk}
\DeclareMathOperator{\curl}{curl}
\DeclareMathOperator{\mesh}{mesh}
\DeclareMathOperator{\im}{im}
\DeclareMathOperator{\coker}{coker}
\DeclareMathOperator{\width}{width}
\DeclareMathOperator{\diam}{diam}
\DeclareMathOperator{\maps}{Maps}
\DeclareMathOperator{\Frac}{Frac}
\DeclareMathOperator{\Sym}{Sym}
\DeclareMathOperator{\sgn}{sgn}
\DeclareMathOperator{\alt}{Alt}
\DeclareMathOperator{\supp}{supp}
\DeclareMathOperator{\Span}{span}
\DeclareMathOperator{\Var}{Var}
\DeclareMathOperator{\Spec}{Spec}

\newcommand{\nor}{\unlhd}
\DeclareMathOperator{\aut}{Aut}
\DeclareMathOperator{\orb}{Orb}
\DeclareMathOperator{\GL}{GL}
\DeclareMathOperator{\SL}{SL}
\DeclareMathOperator{\SO}{SO}
\DeclareMathOperator{\PGL}{PGL}
\DeclareMathOperator{\PSL}{PSL}
\DeclareMathOperator{\stab}{Stab}
\DeclareMathOperator{\fix}{Fix}
\DeclareMathOperator{\Th}{Th}
\DeclareMathOperator{\Ind}{Ind}
\DeclareMathOperator{\Res}{Res}
\DeclareMathOperator{\Ann}{Ann}
\DeclareMathOperator{\rad}{rad}
\DeclareMathOperator{\len}{len}
\DeclareMathOperator{\ord}{ord}

% \DeclareMathOperator{\arg}{arg}

%% misc
\newcommand{\<}{\langle}
\renewcommand{\>}{\rangle}
\renewcommand{\^}{\wedge}
\renewcommand{\v}{\vee}
\def\Xint#1{\mathchoice
	{\XXint\displaystyle\textstyle{#1}}%
	{\XXint\textstyle\scriptstyle{#1}}%
	{\XXint\scriptstyle\scriptscriptstyle{#1}}%
	{\XXint\scriptscriptstyle\scriptscriptstyle{#1}}%
	\!\int}
\def\XXint#1#2#3{{\setbox0=\hbox{$#1{#2#3}{\int}$ }
		\vcenter{\hbox{$#2#3$ }}\kern-.6\wd0}}
\def\ddashint{\Xint=}
\def\dashint{\Xint-}
%% arrows
\newcommand{\xhra}{\xhookrightarrow}
\newcommand{\xra}{\xrightarrow}
\newcommand{\ra}{\rightarrow}
\newcommand{\rra}{\rightrightarrows}
\newcommand{\lra}{\longrightarrow}
\newcommand{\Ra}{\Rightarrow}
\newcommand{\lRa}{\Longrightarrow}
\newcommand{\lrsa}{\leftrightsquiqarrow}
\newcommand{\ba}{\leftrightarrow}
%% lists
\newcommand{\be}{\begin{enumerate}[(i)]}
	\newcommand{\ee}{\end{enumerate}}
%% integration stuff
\newcommand{\calR}{\mathcal{R}}
\newcommand{\rint}{\calR\!\int}
\newcommand{\calL}{\mathcal{L}}
\newcommand{\lowerint}{\mbox{\b{$\int$}}}
\newcommand{\upperint}{{\textstyle\bar{\int}}}
%% end of proof
\def\endproof{{\hfill $\Box$}}
%% matrix shorthand

\title{Math 317 HW 1}
\author{Jalen Chrysos}

\begin{document}
	
	\maketitle

\noindent \textbf{Problem 1 (Hatcher 0.4)}: A deformation retraction ``in the weak sense'' of $X\to A$ is a homotopy $f_t:X\to X$ such that $f_0$ is the identity map on $X$, $f_1$ is a map $X\to A$, and $f_t(A)\subseteq A$ for all $t\in [0,1]$ (this is weaker because it does not require $f_t$ be \textit{constant} on $A$). Show that if such a map exists, then the inclusion $\iota:A\to X$ is a homotopy equivalence.

\begin{proof}
	To show that $\iota$ is a homotopy equivalence it is sufficient and necessary to produce an inverse up to homotopy. This inverse will be $f_1$. The composition $\iota \circ f_1:X\to X$ is homotopic to $\id_X$ via $f_t$. $f_1\circ \iota:A\to A$ is also not necessarily $\id_A$, but is homotopic to $\id_A$ via $f_t$ (here we need the fact that $f_t$ is always a map $A\to A$).
\end{proof}

\newpage 


\noindent \textbf{Problem 2 (Hatcher 0.6)}: 
\begin{enumerate}[(a)]
	\item Let $X$ be the subspace of $\R^2$ consisting defined by
	$$
	X := \Big([0,1]\times \{0\} \Big) \cup \bigcup_{q\in \Q[0,1]} \{q\}\times [0,1-q].
	$$
	Show that $X$ deformation retracts to any point in $[0,1]\times \{0\}$, but not to other points in $X$.
	\item Let $Y$ be the union of an infinite number of copies of $X$ arranged as in the figure below. Show that $Y$ is contractible but does not deformation retract onto any point.
	\item Let $Z$ be the zigzag subspace of $Y$ homeomorphic to $\R$ indicated by the heavier line. Show that there is a deformation retraction $Y\to Z$ in the weak sense, but not in the regular sense.
\end{enumerate}

\includegraphics[width=\textwidth]{C:/Users/jzchr/MathDocuments/Class Work/Topology/Homework/Fig_0.6(b)_Hatcher.png}

\begin{proof}
	(a): For any point $r\in [0,1]$, we can construct a deformation retraction from $X$ to $\{r\}\times \{0\}$ as follows: let $d_t:X\to X$ be given by
	$$
	d_t(q,h) = \begin{cases}
		(q,h(1 - 2t)) & t\in [0,\tfrac12]\\
		((2t-1)r+(2-2t)q,0) & t\in [\tfrac12,1]
	\end{cases}
	$$
	It is easy to check that $d_t$ is continuous and that $d_t((r,0)) = (r,0)$ for all $t\in[0,1]$.
	
	On the other hand, every point $x:=(q,h)$ with $h\neq 0$ has the property that there is an $\ep$ (namely $\ep=h/2$) such that within every neighborhood of $x$ there are points $y$ such that there are no paths between $y$ and $x$ within $B_{x}(\ep)$. Any point with this property cannot be the target of a deformation retraction:
	
	If there was such a deformation retraction, then for every point $y$ there is a closed time interval $T_y$ for which $d_t(y)\not\in B_{x}(\ep)$ for $t\in T_y$. Let $y_1,y_2,\dots$ be a sequence in $X$ approaching $x$. By compactness of $[0,1]$, there is a time $t$ such that $t\in T_{y_j}$ for infinitely many $y_j$. This shows $d_t$ is not continuous in $t$ at $x$, since $|d_t(x)-d_t(y)|>\ep$ for arbitrarily small $|x-y|$. Thus such a deformation retraction cannot exist.\\
	
	(b): The difference between being contractible to a point $x$ and being \textit{deformation retractable} to $x$ is that in the former case, the homotopy need not keep $x$ fixed. $Y$ is contractible because it is a 1-dimensional CW complex with no cycles (i.e. a tree). Yet it is not deformation retractable to any point because every point of $Y$ has the property described in the second part of (a).\\
	
	(c): Here is a somewhat vague description of a weak deformation retraction to $Z$: every point $y\in Y$ has a single always-rightward path which extends to infinity through $Y$, so we let every point simultaneously walk along this path at the same constant rate. As a result, the ``bristles'' move together with the parallel section of $Z$, making the map continuous. At time $t=1$, the bristles will have all rejoined $Z$, making this a weak deformation retraction. But all points in $Z$ also moved, so it is not a true deformation retraction.
\end{proof}

\newpage 


\noindent \textbf{Problem 3 (Hatcher 0.16)}: Show that $S^{\infty}$ is contractible.
\begin{proof}
	Let $x$ be the single point in the $0$-skeleton of $S^{\infty}$. Let $A_k^+,A_k^-$ be the upper and lower open half-$k$-sphere, so that $A_k^+\cap A_k^-=S^{k-1}$ and all $A_k^{\pm}$ contain $x$. Every loop at $x$ in $S^{\infty}$ is entirely contained in $A_k^+$ or $A_k^-$ for some finite $k$, and thus contractible to $x$. Thus $S^{\infty}$ is also contractible.
\end{proof}
\newpage 


\noindent \textbf{Problem 4 (Hatcher 0.20)}: Show that the space $X\subset \R^3$ formed by a Klein bottle intersecting itself in a circle is homotopy equivalent to $S^1\vee S^1 \vee S^2$.
\begin{proof}
	We can think of this Klein bottle embedded in $\R^3$ as a square with the usual Klein bottle edge-gluing and an additional identification of two disjoint circles.
\end{proof}
\newpage 


\noindent \textbf{Problem 5 (Hatcher 0.23)}: Show that a CW complex is contractible if it is the union of two contractible subcomplexes whose intersection is also contractible.

\begin{proof}
	Let $X,Y$ be two contractible CW-complexes with intersection $Z$, and let $x\in Z$. If $\ell$ is a loop at $x$ in $X\cup Y$, then 
\end{proof}

\newpage 


\noindent \textbf{Problem 6 (Hatcher 1.1.6)}: 

\newpage 


\noindent \textbf{Problem 7 (Hatcher 1.1.13)}:

\newpage 


\noindent \textbf{Problem 8 (Hatcher 1.1.20)}:


\end{document}