\documentclass{amsart}

%\documentclass{amsart}
\usepackage[utf8]{inputenc}
\usepackage{amsfonts}
\usepackage{amsmath}
\usepackage{amssymb}
\usepackage{amsthm}
\usepackage{asymptote}
\usepackage{mathtools}
\usepackage{hhline}
\usepackage{graphicx,enumerate}
\usepackage{hyperref}
\usepackage[a4paper, margin=1.2in]{geometry}
%\usepackage{tcolorbox}
\usepackage{tikz-cd}
\usepackage{ytableau}
%\tcbuselibrary{skins,breakable,xparse}
\allowdisplaybreaks
\newcounter{count}
\hypersetup{
	colorlinks=true,
	linkcolor=teal,
	filecolor=magenta,      
	urlcolor=olive,
	citecolor=teal,
	pdfpagemode=FullScreen,
}

%\definecolor{defcolor}{HTML}{478EFF}
%\definecolor{thmcolor}{HTML}{CC0058}
%\definecolor{excolor}{HTML}{F5B400}
%\definecolor{probcolor}{HTML}{DD4803}
%\definecolor{lemcolor}{HTML}{741FEA}
%\definecolor{scarlet}{HTML}{A81111}
%
%\newtheoremstyle{definitionStyle}% Custom style for definitions
%{0.5em}% Space above
%{0.5em}% Space below
%{}% Body font
%{}% Indent amount
%{\bfseries\color{defcolor}}% Theorem head font: bold and red
%{.\\}% Punctuation after theorem head
%{0.5em}% Space after theorem head
%{\thmname{#1}\thmnumber{ #2 (#3)}}% Theorem head spec
%
%\theoremstyle{definitionStyle}
%\newtheorem{df}{Definition}[section]
%
%\newtheoremstyle{theoremStyle}% Custom style for definitions
%{0.5em}% Space above
%{0.5em}% Space below
%{}% Body font
%{}% Indent amount
%{\bfseries\color{thmcolor}}% Theorem head font: bold and red
%{.\\}% Punctuation after theorem head
%{0.5em}% Space after theorem head
%{\thmname{#1}\thmnumber{ #2 (#3)}}% Theorem head spec
%
%\theoremstyle{theoremStyle}
%\newtheorem{thm}{Theorem}[section]
%
%\newtheoremstyle{lemmaStyle}% Custom style for definitions
%{0.5em}% Space above
%{0.5em}% Space below
%{}% Body font
%{}% Indent amount
%{\bfseries\color{lemcolor}}% Theorem head font: bold and red
%{.\\}% Punctuation after theorem head
%{0.5em}% Space after theorem head
%{\thmname{#1}\thmnumber{ #2 (#3)}}% Theorem head spec
%
%\theoremstyle{lemmaStyle}
%\newtheorem{lem}{Lemma}[section]
%\newtheorem{cor}{Corollary}[section]
%
%\newtheoremstyle{exampleStyle}% Custom style for definitions
%{0.5em}% Space above
%{0.5em}% Space below
%{}% Body font
%{}% Indent amount
%{\bfseries\color{excolor}}% Theorem head font: bold and red
%{.\\}% Punctuation after theorem head
%{0.5em}% Space after theorem head
%{\thmname{#1}\thmnumber{ #2 (#3)}}% Theorem head spec
%
%\theoremstyle{exampleStyle}
%\newtheorem{ex}{Example}[section]
%
%\newtheoremstyle{problemStyle}% Custom style for definitions
%{0.5em}% Space above
%{0.5em}% Space below
%{}% Body font
%{}% Indent amount
%{\bfseries\color{probcolor}}% Theorem head font: bold and red
%{.\\}% Punctuation after theorem head
%{0.5em}% Space after theorem head
%{\thmname{#1}\thmnumber{ #2#3}}% Theorem head spec
%
%\theoremstyle{problemStyle}
%\newtheorem{prob}{Problem}[section]

% For Fun
\newcommand{\club}{\color{teal} \clubsuit}
\newcommand{\heart}{\color{red} \heartsuit}
\renewcommand{\star}{\color{scarlet} \bigstar}
\newcommand{\spade}{\color{violet} \spadesuit}

% Symbols
\newcommand{\A}{\mathcal{A}}
\newcommand{\B}{\mathcal{B}}
\newcommand{\C}{\mathbb{C}}
\newcommand{\D}{\mathcal{D}}
\newcommand{\E}{\mathbb{E}}
\newcommand{\F}{\mathbb{F}}
\newcommand{\G}{\mathcal{G}}
% \renewcommand{\H}{\mathcal{H}} Erdos o
\newcommand{\I}{\mathcal{I}}
\newcommand{\J}{\mathcal{J}}
\newcommand{\K}{\mathcal{K}}
% \renewcommand{\L}{\mathcal{L}}
\newcommand{\M}{\mathcal{M}}
\newcommand{\N}{\mathbb{N}}
\renewcommand{\O}{\mathcal{O}}
\renewcommand{\P}{\mathbb{P}}
\newcommand{\Q}{\mathbb{Q}}
\newcommand{\R}{\mathbb{R}}
\renewcommand{\S}{\mathbb{S}}
\newcommand{\T}{\mathbb{T}}
\newcommand{\U}{\mathcal{U}}
\newcommand{\V}{\mathcal{V}}
\newcommand{\W}{\mathcal{W}}
\newcommand{\X}{\mathcal{X}}
\newcommand{\Y}{\mathcal{Y}}
\newcommand{\Z}{\mathbb{Z}}

\renewcommand{\AA}{\mathcal{A}}
\newcommand{\BB}{\mathcal{B}}
\newcommand{\CC}{\mathcal{C}}
\newcommand{\DD}{\mathcal{D}}
\newcommand{\EE}{\mathcal{E}}
\newcommand{\FF}{\mathcal{F}}
\newcommand{\GG}{\mathbb{G}}
\newcommand{\HH}{\mathbb{H}}
\newcommand{\calH}{\mathcal{H}}
\newcommand{\II}{\mathcal{I}}
\newcommand{\JJ}{\mathcal{J}}
\newcommand{\KK}{\mathcal{K}}
\newcommand{\LL}{\mathcal{L}}
\newcommand{\MM}{\mathcal{M}}
\newcommand{\NN}{\mathcal{N}}
\newcommand{\OO}{\mathrm{O}}
\newcommand{\PP}{\mathcal{P}}
\newcommand{\QQ}{\mathcal{Q}}
\newcommand{\RR}{\mathcal{R}}
\renewcommand{\SS}{\mathcal{S}}
\newcommand{\TT}{\mathcal{T}}
\newcommand{\UU}{\mathcal{U}}
\newcommand{\VV}{\mathcal{V}}
\newcommand{\WW}{\mathcal{W}}
\newcommand{\XX}{\mathcal{X}}
\newcommand{\YY}{\mathcal{Y}}
\newcommand{\ZZ}{\mathcal{Z}}
\renewcommand{\d}{\textrm{d}}
% Greek letters
\newcommand{\ep}{\varepsilon}
\newcommand{\ph}{\varphi}
\newcommand{\de}{\delta}
\renewcommand{\a}{\alpha}
\renewcommand{\b}{\beta}
% Fraktur
\newcommand{\mm}{\mathfrak{m}}
\renewcommand{\aa}{\mathfrak{a}}
\newcommand{\bb}{\mathfrak{b}}
\newcommand{\pp}{\mathfrak{p}}
\newcommand{\qq}{\mathfrak{q}}
% Operators
\DeclareMathOperator{\Div}{div}
\DeclareMathOperator{\Gal}{Gal}
\DeclareMathOperator{\vol}{Vol}
\DeclareMathOperator{\Hom}{Hom}
\DeclareMathOperator{\End}{End}
\DeclareMathOperator{\Ext}{Ext}
\DeclareMathOperator{\Tor}{Tor}
\DeclareMathOperator{\tr}{tr}
\DeclareMathOperator{\rk}{rk}
\DeclareMathOperator{\curl}{curl}
\DeclareMathOperator{\mesh}{mesh}
\DeclareMathOperator{\im}{im}
\DeclareMathOperator{\coker}{coker}
\DeclareMathOperator{\width}{width}
\DeclareMathOperator{\diam}{diam}
\DeclareMathOperator{\maps}{Maps}
\DeclareMathOperator{\Frac}{Frac}
\DeclareMathOperator{\Sym}{Sym}
\DeclareMathOperator{\sgn}{sgn}
\DeclareMathOperator{\alt}{Alt}
\DeclareMathOperator{\supp}{supp}
\DeclareMathOperator{\Span}{span}
\DeclareMathOperator{\Var}{Var}
\DeclareMathOperator{\Spec}{Spec}

\newcommand{\nor}{\unlhd}
\DeclareMathOperator{\aut}{Aut}
\DeclareMathOperator{\orb}{Orb}
\DeclareMathOperator{\GL}{GL}
\DeclareMathOperator{\SL}{SL}
\DeclareMathOperator{\SO}{SO}
\DeclareMathOperator{\PGL}{PGL}
\DeclareMathOperator{\PSL}{PSL}
\DeclareMathOperator{\stab}{Stab}
\DeclareMathOperator{\fix}{Fix}
\DeclareMathOperator{\Th}{Th}
\DeclareMathOperator{\Ind}{Ind}
\DeclareMathOperator{\Res}{Res}
\DeclareMathOperator{\Ann}{Ann}
\DeclareMathOperator{\rad}{rad}
\DeclareMathOperator{\len}{len}
\DeclareMathOperator{\ord}{ord}

% \DeclareMathOperator{\arg}{arg}

%% misc
\newcommand{\<}{\langle}
\renewcommand{\>}{\rangle}
\renewcommand{\^}{\wedge}
\renewcommand{\v}{\vee}
\def\Xint#1{\mathchoice
	{\XXint\displaystyle\textstyle{#1}}%
	{\XXint\textstyle\scriptstyle{#1}}%
	{\XXint\scriptstyle\scriptscriptstyle{#1}}%
	{\XXint\scriptscriptstyle\scriptscriptstyle{#1}}%
	\!\int}
\def\XXint#1#2#3{{\setbox0=\hbox{$#1{#2#3}{\int}$ }
		\vcenter{\hbox{$#2#3$ }}\kern-.6\wd0}}
\def\ddashint{\Xint=}
\def\dashint{\Xint-}
%% arrows
\newcommand{\xhra}{\xhookrightarrow}
\newcommand{\xra}{\xrightarrow}
\newcommand{\ra}{\rightarrow}
\newcommand{\rra}{\rightrightarrows}
\newcommand{\lra}{\longrightarrow}
\newcommand{\Ra}{\Rightarrow}
\newcommand{\lRa}{\Longrightarrow}
\newcommand{\lrsa}{\leftrightsquiqarrow}
\newcommand{\ba}{\leftrightarrow}
%% lists
\newcommand{\be}{\begin{enumerate}[(i)]}
	\newcommand{\ee}{\end{enumerate}}
%% integration stuff
\newcommand{\calR}{\mathcal{R}}
\newcommand{\rint}{\calR\!\int}
\newcommand{\calL}{\mathcal{L}}
\newcommand{\lowerint}{\mbox{\b{$\int$}}}
\newcommand{\upperint}{{\textstyle\bar{\int}}}
%% end of proof
\def\endproof{{\hfill $\Box$}}
%% matrix shorthand

\title{Math 317 HW 7}
\author{Jalen Chrysos}

\begin{document}
	\maketitle
	\textbf{Problem 1 (Hatcher 4.1.5)}: For a pair $(X,A)$ of path-connected spaces, show that $\pi_1(X,A,x_0)$ can be identified in a natural way with the set of cosets $\a H$ of the subgroup $H\subset \pi_1(X,x_0)$ represented by loops in $A$ at $x_0$.
	\begin{proof}
		By definition, $\pi_1(X,A,x_0)$ is the group of (homotopy classes of) maps $I\to X$ beginning at $x_0$ and ending in $A$. In each of those classes there is a loop at $x_0$, since $A$ is path connected and paths can be extended by homotopy.
		
		If $\a$ is a loop at $x_0$ in $X$, then everything in $\a H$ is homotopic to $\a$ (do later)
	\end{proof}
	
	\newpage
	
	\textbf{Problem 2 (Hatcher 4.1.8)}: Show that the sequence 
	$$
	\pi_1(X,x_0) \xrightarrow{f} \pi_1(X,A,x_0) \xrightarrow{\partial} \pi_0(A,x_0) \xrightarrow{g} \pi_0(X,x_0)
	$$
	is exact.
	\begin{proof}
		$\ker(\partial)$ is exactly the set of paths in $\pi_1(X,A,x_0)$ which are loops, equivalently the paths whose endpoint is $x_0$, equivalently the image of $\pi_1(X,x_0)$.\\
		
		Next, $\im(\partial)$ is the points in $A$ that are the other endpoint of a path in $X$ from $x_0$, i.e. the points in the same connected component of $X$ as $x_0$. And this is also the kernel of $g$, so indeed the sequence is exact here too.
	\end{proof}
	
	\newpage 
	\textbf{Problem 3 (Hatcher 4.1.15)}: Show that every map $f:S^n\to S^n$ is homotopic to a multiple of the identity map by the following steps:
	\begin{enumerate}[(a)]
		\item Use Lemma 4.10 to reduce to the case that there is a point $q\in S^n$ with $f^{-1}(q)=\{p_1,\dots,p_k\}$ and $f$ is an invertible linear map near each $p_i$.
		\item For $f$ as in (a), consider the composition $gf$ where $g:S^n\to S^n$ collapses the complement of a small ball about $q$ to the basepoint. Use this to reduce (a) further to the case $k=1$.
		\item Finish the argument by showing that an invertible $n\times n$ matrix can be joined by a path of such matrices to either the identity matrix or the matrix of a reflection. 
	\end{enumerate}
	\begin{proof}
		(a): Lemma 4.10 says that $f$ is homotopic to a map $f_1:S^n\to S^n$ which is piecewise linear on some polyhedron $K\subset S^n$.\\
		
		(b): Supposing the result is true for $k=1$,\\
		
		(c): The space of $n\times n$ matrices with positive determinant is path-connected, as a subspace of $\R^{n^2}$. One way to see this is that every matrix $a\in M_n(\R)$ has a well defined logarithm
		$$
		\log(1 + a) = a + a^2/2 + a^3/3 + a^4/4 + \cdots
		$$
		and exponential 
		$$
		e^{a} = I + a + a^2/2 + a^3/6 + a^4/24 + \cdots
		$$
		so one can let $x=\log(a)$ and take the path $\gamma(t) = e^{tx}$. Then $\gamma(0)=I$, $\gamma(1)=e^x=a$, and $e^{tx}\in \GL_n(\R)$ because it has inverse $e^{-tx}$.  
	\end{proof}
	
	\newpage
	\textbf{Problem 4 (Hatcher 4.1.17)}: Show that if $X$ and $Y$ are CW complexes with $X$ $m$-connected and $Y$ $n$-connected, then $(X\times Y, X\vee Y)$ is $(m+n+1)$-connected, as is the smash product $X\wedge Y$.
	\begin{proof}
		We have the long exact sequence
		$$
		\cdots \to \pi_j(X\vee Y) \to \pi_j(X\times Y)\to \pi_j(X\times Y,X\vee Y) \to \pi_{j-1}(X\vee Y)\to \cdots 
		$$
		
		$X$ is homotopy equivalent to a CW complex $X'$ with one 0-cell and no other cells of dimension $\leq m$, and likewise $Y$ has a corresponding $Y'$. $(X\times Y,X\vee Y)\simeq (X'\times Y',X'\vee Y')$ so it suffices to show this for the latter. 
		
		The cells of $X'\times Y'$ are the same as $X'\vee Y'$ in small dimensions. The smallest-dimension cells in $X'\times Y'$ not in $X'\vee Y'$ are of dimension $m+n$, so $\pi_j(X'\vee Y')\to \pi_j(X'\times Y')$ is automatically an isomorphism for $j\leq m+n$, and thus in the long exact sequence
		$$
		\cdots \to \underbrace{\pi_{m+n+1}(X' \times Y',X'\vee Y')}_{0} \to \pi_{m+n}(X'\vee Y') \cong \pi_{m+n}(X'\times Y') \to \underbrace{\pi_{m+n}(X' \times Y',X'\vee Y')}_{0}\to \cdots
		$$
		implies $\pi_{m+n+1}(X' \times Y',X'\vee Y')=0$ and likewise for all smaller dimensions.\\
		
		The smash product $X\wedge Y$ is the quotient $X\times Y/X\vee Y$. Excision for homotopy groups implies that 
		$$
		\pi_j(X\times Y / X\vee Y) \cong \pi_j(X\times Y, X\vee Y)
		$$
		for $j\leq m+n$. For $j=m+n+1$, the inclusion induces a surjection, and the only thing 0 can surject onto is 0, so $\pi_{m+n+1}(X\wedge Y)=0$ as well.
	\end{proof}
	
	\newpage
	\textbf{Problem 5 (Hatcher 4.2.1)}: Use homotopy groups to show that there is no retraction $\RP^n\to \RP^k$ for $n>k>0$.
	\begin{proof}
		$\RP^n$ has a 2-sheeted cover from $S^n$, as we've seen before. So $\pi_k(\RP^n)=\pi_k(S^n)=0$, but $\pi_k(\RP^k)=\pi_k(S^k)=\Z$ (assuming $k\geq 2$). And in the case $k=1$, $\RP^1=S^1$, so it is still true that $\pi_1(\RP^1)=\Z$. A retraction would induce an isomorphism between $\pi_k(S^n)$ and $\pi_k(S^k)$, but they are not isomorphic, so there is no retraction.
	\end{proof}
	
	\newpage
	\textbf{Problem 6 (Hatcher 4.2.5)}: Let $f:S^2_{\a}\vee S^2_{\b} \to S^2_{\a}\vee S^2_{\b}$ be the map which is the identity on $S^2_{\a}$, and the sum of the identity and a map $S^2_{\b}\to S^2_{\a}$ on $S^2_{\b}$. Let $X$ be the mapping torus of $f$, i.e.
	$$
	X := \frac{(S^2_{\a} \vee S^2_{\b} )\times I}{(x,0)\sim (f(x),1)}
	$$	
	The mapping torus of the restriction of $f$ to $S^2_{\a}$ forms a subspace $A=S^1\times S^2_{\a}\subset X$. Show that the maps $\pi_2(A)\to \pi_2(X)\to \pi_2(X,A)$ form a short exact sequence 
	$$
	0 \to \underbrace{\pi_2(A)}_{\Z}\to \underbrace{\pi_2(X)}_{\Z^2}\to \underbrace{\pi_2(X,A)}_{\Z}\to 0
	$$
	and compute the action of $\pi_1(A)$ on these three groups. In particular show that the action is trivial on $\pi_2(A)$ and $\pi_2(X,A)$ but nontrivial on $\pi_2(X)$.
	\begin{proof}
		$\pi_3(X,A)=0$ by cellular approximation, as $X$ has no cells of dimension 3 so one could deform any map $S^3\to X$ to a point. And we know that 
		$$
		\pi_3(X,A)\to \pi_2(A)\to \pi_2(X)
		$$
		is exact in general. This shows that $\pi_2(A)\to \pi_2(X)$ is injective.
		
		Now to show that $\pi_2(X)\to \pi_2(X,A)$ is surjective, it suffices to show that $\pi_1(A)\to\pi_1(X)$ is injective, i.e. that if a loop in $A$ is nullhomotopic in $X$ then it is nullhomotopic in $A$ too. $\pi_1(A)=\pi_1(X)=\Z$, and the inclusion of the generator of $\pi_1(A)$ (Say more precisely)\\
		
		$\pi_2(A) = \pi_2(S^1\times S^2) = 0\times \Z= \Z$.\\
		
		$\pi_1(A)$ acts on $\pi_2(A)$ by
	\end{proof}
	
	\newpage
	\textbf{Problem 7 (Hatcher 4.2.10)}: Let $X$ be the CW complex obtained from $S^2\vee S^n$ (where $n\geq 2$) by attaching $e^{n+1}$ by a map representing the polynomial $p(t)\in \Z[t,t^{-1}]\cong \pi_n(S^1\vee S^n)$, so that $\pi_n(X)\cong \Z[t,t^{-1}]/(p(t))$. Show that $\pi_n'(X)$ is cyclic and compute its order in terms of $p(t)$. Give examples showing that the group $\pi_n(X)$ can be finitely-generated or not, independently of whether $\pi_n'(X)$ is finite or infinite.
	\begin{proof}
		By the relative Hurewicz theorem, 
	\end{proof}
	
	\newpage
	\textbf{Problem 8 (Hatcher 4.2.13)}: Show that a map between connected $n$-dimensional CW complexes is a homotopy equivalence if it induces an isomorphism on $\pi_i$ for $i\leq n$.
	\begin{proof}
		Whitehead's theorem implies that if $f:X\to Y$ induces an isomorphism on $\pi_i$ for all $i$ then it is a homotopy equivalence. If $X,Y$ are $n$-dimensional, then $\pi_i(X)=\pi_i(Y)=0$ for $i>n$, so any such $f$ automatically induces an isomorphism on these groups. Thus it suffices to know that $f_*:\pi_i(X)\to \pi_i(Y)$ is an isomorphism for $i\leq n$, as desired.
	\end{proof}
	
	
\end{document}