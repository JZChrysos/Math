\documentclass{amsart}
%\documentclass{amsart}
\usepackage[utf8]{inputenc}
\usepackage{amsfonts}
\usepackage{amsmath}
\usepackage{amssymb}
\usepackage{amsthm}
\usepackage{asymptote}
\usepackage{mathtools}
\usepackage{hhline}
\usepackage{graphicx,enumerate}
\usepackage{hyperref}
\usepackage[a4paper, margin=1.2in]{geometry}
%\usepackage{tcolorbox}
\usepackage{tikz-cd}
\usepackage{ytableau}
%\tcbuselibrary{skins,breakable,xparse}
\allowdisplaybreaks
\newcounter{count}
\hypersetup{
	colorlinks=true,
	linkcolor=teal,
	filecolor=magenta,      
	urlcolor=olive,
	citecolor=teal,
	pdfpagemode=FullScreen,
}

%\definecolor{defcolor}{HTML}{478EFF}
%\definecolor{thmcolor}{HTML}{CC0058}
%\definecolor{excolor}{HTML}{F5B400}
%\definecolor{probcolor}{HTML}{DD4803}
%\definecolor{lemcolor}{HTML}{741FEA}
%\definecolor{scarlet}{HTML}{A81111}
%
%\newtheoremstyle{definitionStyle}% Custom style for definitions
%{0.5em}% Space above
%{0.5em}% Space below
%{}% Body font
%{}% Indent amount
%{\bfseries\color{defcolor}}% Theorem head font: bold and red
%{.\\}% Punctuation after theorem head
%{0.5em}% Space after theorem head
%{\thmname{#1}\thmnumber{ #2 (#3)}}% Theorem head spec
%
%\theoremstyle{definitionStyle}
%\newtheorem{df}{Definition}[section]
%
%\newtheoremstyle{theoremStyle}% Custom style for definitions
%{0.5em}% Space above
%{0.5em}% Space below
%{}% Body font
%{}% Indent amount
%{\bfseries\color{thmcolor}}% Theorem head font: bold and red
%{.\\}% Punctuation after theorem head
%{0.5em}% Space after theorem head
%{\thmname{#1}\thmnumber{ #2 (#3)}}% Theorem head spec
%
%\theoremstyle{theoremStyle}
%\newtheorem{thm}{Theorem}[section]
%
%\newtheoremstyle{lemmaStyle}% Custom style for definitions
%{0.5em}% Space above
%{0.5em}% Space below
%{}% Body font
%{}% Indent amount
%{\bfseries\color{lemcolor}}% Theorem head font: bold and red
%{.\\}% Punctuation after theorem head
%{0.5em}% Space after theorem head
%{\thmname{#1}\thmnumber{ #2 (#3)}}% Theorem head spec
%
%\theoremstyle{lemmaStyle}
%\newtheorem{lem}{Lemma}[section]
%\newtheorem{cor}{Corollary}[section]
%
%\newtheoremstyle{exampleStyle}% Custom style for definitions
%{0.5em}% Space above
%{0.5em}% Space below
%{}% Body font
%{}% Indent amount
%{\bfseries\color{excolor}}% Theorem head font: bold and red
%{.\\}% Punctuation after theorem head
%{0.5em}% Space after theorem head
%{\thmname{#1}\thmnumber{ #2 (#3)}}% Theorem head spec
%
%\theoremstyle{exampleStyle}
%\newtheorem{ex}{Example}[section]
%
%\newtheoremstyle{problemStyle}% Custom style for definitions
%{0.5em}% Space above
%{0.5em}% Space below
%{}% Body font
%{}% Indent amount
%{\bfseries\color{probcolor}}% Theorem head font: bold and red
%{.\\}% Punctuation after theorem head
%{0.5em}% Space after theorem head
%{\thmname{#1}\thmnumber{ #2#3}}% Theorem head spec
%
%\theoremstyle{problemStyle}
%\newtheorem{prob}{Problem}[section]

% For Fun
\newcommand{\club}{\color{teal} \clubsuit}
\newcommand{\heart}{\color{red} \heartsuit}
\renewcommand{\star}{\color{scarlet} \bigstar}
\newcommand{\spade}{\color{violet} \spadesuit}

% Symbols
\newcommand{\A}{\mathcal{A}}
\newcommand{\B}{\mathcal{B}}
\newcommand{\C}{\mathbb{C}}
\newcommand{\D}{\mathcal{D}}
\newcommand{\E}{\mathbb{E}}
\newcommand{\F}{\mathbb{F}}
\newcommand{\G}{\mathcal{G}}
% \renewcommand{\H}{\mathcal{H}} Erdos o
\newcommand{\I}{\mathcal{I}}
\newcommand{\J}{\mathcal{J}}
\newcommand{\K}{\mathcal{K}}
% \renewcommand{\L}{\mathcal{L}}
\newcommand{\M}{\mathcal{M}}
\newcommand{\N}{\mathbb{N}}
\renewcommand{\O}{\mathcal{O}}
\renewcommand{\P}{\mathbb{P}}
\newcommand{\Q}{\mathbb{Q}}
\newcommand{\R}{\mathbb{R}}
\renewcommand{\S}{\mathbb{S}}
\newcommand{\T}{\mathbb{T}}
\newcommand{\U}{\mathcal{U}}
\newcommand{\V}{\mathcal{V}}
\newcommand{\W}{\mathcal{W}}
\newcommand{\X}{\mathcal{X}}
\newcommand{\Y}{\mathcal{Y}}
\newcommand{\Z}{\mathbb{Z}}

\renewcommand{\AA}{\mathcal{A}}
\newcommand{\BB}{\mathcal{B}}
\newcommand{\CC}{\mathcal{C}}
\newcommand{\DD}{\mathcal{D}}
\newcommand{\EE}{\mathcal{E}}
\newcommand{\FF}{\mathcal{F}}
\newcommand{\GG}{\mathbb{G}}
\newcommand{\HH}{\mathbb{H}}
\newcommand{\calH}{\mathcal{H}}
\newcommand{\II}{\mathcal{I}}
\newcommand{\JJ}{\mathcal{J}}
\newcommand{\KK}{\mathcal{K}}
\newcommand{\LL}{\mathcal{L}}
\newcommand{\MM}{\mathcal{M}}
\newcommand{\NN}{\mathcal{N}}
\newcommand{\OO}{\mathrm{O}}
\newcommand{\PP}{\mathcal{P}}
\newcommand{\QQ}{\mathcal{Q}}
\newcommand{\RR}{\mathcal{R}}
\renewcommand{\SS}{\mathcal{S}}
\newcommand{\TT}{\mathcal{T}}
\newcommand{\UU}{\mathcal{U}}
\newcommand{\VV}{\mathcal{V}}
\newcommand{\WW}{\mathcal{W}}
\newcommand{\XX}{\mathcal{X}}
\newcommand{\YY}{\mathcal{Y}}
\newcommand{\ZZ}{\mathcal{Z}}
\renewcommand{\d}{\textrm{d}}
% Greek letters
\newcommand{\ep}{\varepsilon}
\newcommand{\ph}{\varphi}
\newcommand{\de}{\delta}
\renewcommand{\a}{\alpha}
\renewcommand{\b}{\beta}
% Fraktur
\newcommand{\mm}{\mathfrak{m}}
\renewcommand{\aa}{\mathfrak{a}}
\newcommand{\bb}{\mathfrak{b}}
\newcommand{\pp}{\mathfrak{p}}
\newcommand{\qq}{\mathfrak{q}}
% Operators
\DeclareMathOperator{\Div}{div}
\DeclareMathOperator{\Gal}{Gal}
\DeclareMathOperator{\vol}{Vol}
\DeclareMathOperator{\Hom}{Hom}
\DeclareMathOperator{\End}{End}
\DeclareMathOperator{\Ext}{Ext}
\DeclareMathOperator{\Tor}{Tor}
\DeclareMathOperator{\tr}{tr}
\DeclareMathOperator{\rk}{rk}
\DeclareMathOperator{\curl}{curl}
\DeclareMathOperator{\mesh}{mesh}
\DeclareMathOperator{\im}{im}
\DeclareMathOperator{\coker}{coker}
\DeclareMathOperator{\width}{width}
\DeclareMathOperator{\diam}{diam}
\DeclareMathOperator{\maps}{Maps}
\DeclareMathOperator{\Frac}{Frac}
\DeclareMathOperator{\Sym}{Sym}
\DeclareMathOperator{\sgn}{sgn}
\DeclareMathOperator{\alt}{Alt}
\DeclareMathOperator{\supp}{supp}
\DeclareMathOperator{\Span}{span}
\DeclareMathOperator{\Var}{Var}
\DeclareMathOperator{\Spec}{Spec}

\newcommand{\nor}{\unlhd}
\DeclareMathOperator{\aut}{Aut}
\DeclareMathOperator{\orb}{Orb}
\DeclareMathOperator{\GL}{GL}
\DeclareMathOperator{\SL}{SL}
\DeclareMathOperator{\SO}{SO}
\DeclareMathOperator{\PGL}{PGL}
\DeclareMathOperator{\PSL}{PSL}
\DeclareMathOperator{\stab}{Stab}
\DeclareMathOperator{\fix}{Fix}
\DeclareMathOperator{\Th}{Th}
\DeclareMathOperator{\Ind}{Ind}
\DeclareMathOperator{\Res}{Res}
\DeclareMathOperator{\Ann}{Ann}
\DeclareMathOperator{\rad}{rad}
\DeclareMathOperator{\len}{len}
\DeclareMathOperator{\ord}{ord}

% \DeclareMathOperator{\arg}{arg}

%% misc
\newcommand{\<}{\langle}
\renewcommand{\>}{\rangle}
\renewcommand{\^}{\wedge}
\renewcommand{\v}{\vee}
\def\Xint#1{\mathchoice
	{\XXint\displaystyle\textstyle{#1}}%
	{\XXint\textstyle\scriptstyle{#1}}%
	{\XXint\scriptstyle\scriptscriptstyle{#1}}%
	{\XXint\scriptscriptstyle\scriptscriptstyle{#1}}%
	\!\int}
\def\XXint#1#2#3{{\setbox0=\hbox{$#1{#2#3}{\int}$ }
		\vcenter{\hbox{$#2#3$ }}\kern-.6\wd0}}
\def\ddashint{\Xint=}
\def\dashint{\Xint-}
%% arrows
\newcommand{\xhra}{\xhookrightarrow}
\newcommand{\xra}{\xrightarrow}
\newcommand{\ra}{\rightarrow}
\newcommand{\rra}{\rightrightarrows}
\newcommand{\lra}{\longrightarrow}
\newcommand{\Ra}{\Rightarrow}
\newcommand{\lRa}{\Longrightarrow}
\newcommand{\lrsa}{\leftrightsquiqarrow}
\newcommand{\ba}{\leftrightarrow}
%% lists
\newcommand{\be}{\begin{enumerate}[(i)]}
	\newcommand{\ee}{\end{enumerate}}
%% integration stuff
\newcommand{\calR}{\mathcal{R}}
\newcommand{\rint}{\calR\!\int}
\newcommand{\calL}{\mathcal{L}}
\newcommand{\lowerint}{\mbox{\b{$\int$}}}
\newcommand{\upperint}{{\textstyle\bar{\int}}}
%% end of proof
\def\endproof{{\hfill $\Box$}}
%% matrix shorthand

\title{MATH 318 HW 1}
\author{Jalen Chrysos}

\begin{document}
	
	\maketitle
% OPTIONAL QUESTIONS
%\textbf{Problem 1 (2.3)}: Prove that two $C^k$-manifolds (with $k\geq 1$) that are $C^k$-diffeomorphic have the same dimension.
%
%\newpage 
%
%\textbf{Problem 2 (2.5)}: Prove that for $m\leq n$, $S^m$ is a smooth submanifold of $S^n$ if we identify $S^m$ with $S^n\cap (\R^{m+1}\times 0)$. Conclude the same for $P^m$ and $P^n$.
%
%\newpage 

\textbf{Problem 1 (2.6)}: Let $f:\R^n\to \R$ be a homogeneous function of degree $d$. Prove that $f^{-1}(1)$ is a (possibly empty) submanifold of dimension $n-1$.
\begin{proof}
	By Example 2.6, it suffices to show that the derivative of a homogeneous function $f:\R^n\to \R$ is surjective on points within $f^{-1}(1)$. In this case because the dimension of the output space is 1, it is equivalent to show that the derivative is nonzero. 
	
	For $f$ to be homogeneous of degree $d$ means that $f(\lambda v)=\lambda^df(v)$ for all $v\in \R^n$ and $\lambda \in \R$. In particular, where $f(v)=1$, $f(\lambda v) = \lambda^df(v) = \lambda^d$, so the derivative in direction $v$ (which is defined because $f(0)=0$ so $0\not\in f^{-1}(1)$) at $v$ is nonzero. In particular 
	$$\partial_{t} f(v+t v) = \partial_{t} f((1+t)v) = \partial_{t} (1+t)^d = d(1+t)^{d-1} = d \; \text{(at $t=0$)}$$ 
	so $f$ has nonzero derivative at all points in $f^{-1}(1)$, thus making $f^{-1}(1)$ a submanifold of dimension $n-1$ as desired.
\end{proof}

\newpage

\textbf{Problem 2 (2.7)}: Show that $\SL_n(\R)$ is a smooth submanifold of $\R^{n^2}$ and determine its dimension. Prove also that the map $\SL_n(\R)\times \SL_n(\R)\to \SL_n(\R)$ via $(\sigma,\tau)\mapsto \sigma\tau^{-1}$ is smooth. Do the same for $\SO_n(\R)$.
\begin{proof}
	$\SL_n(\R)$ is the preimage $\det^{-1}(1)$ of $\det:\R^{n^2}\to \R$. The determinant is homogeneous of degree $n$. Thus, by the previous problem, $\SL_n(\R)$ is a submanifold of dimension $n^2-1$.
	
	$\tau\mapsto \tau^{-1}$ is smooth on $\SL_n(\R)$ as it is given by the adjugate matrix, so each coordinate is just the determinant of one of the minors, a polynomial in the matrix entries in $\tau$ and thus smooth. Similarly, each entry of $\sigma\tau^{-1}$ is a polynomial in the entries of $\sigma$ and $\tau$ and thus smooth.\\
	
	$\SO_n(\R)$ can be seen as the fiber $f^{-1}(I)$ where $f:\GL_n(\R)\to \Sym_n(\R)$ is defined $f:a \mapsto a a^{\top}$, with the codomain being the symmetric matrices. So by Corollary 4.3, it suffices to show that $D_pf$ is surjective at every special orthogonal matrix $p$.
	
	To do this, wlog suppose we want to show that the symmetric matrix with 1 in the $(i,j)$ and $(j,i)$ places and 0 elsewhere is in the image of $D_pf$ (the diagonal case is easy). Let $c_k$ be the $k$th column of $p$. Assume wlog that $c_j$ is not in the direction $(1,1,\dots,1)$; otherwise swap $i,j$ and the following construction will work (note that they cannot both be in the same direction because $p$ has determinant 1). Consider $D_p(f)$ of the vector $v$ that perturbs the matrix by 
	$$p_{i1} \mapsto p_{i1} + v_1, \;\; p_{i2}\mapsto p_{i2} + v_2, \;\; \dots \;\; p_{in}\mapsto p_{in} + v_n.$$
	 We can choose $v$ to be orthogonal to $c_k$ for $k\neq i,j$ and also orthogonal to $(1,1,\dots,1)$. Due to this choice, the perturbation by $v$ doesn't affect $c_i\cdot c_k$ for any $k\neq i,j$, nor any dot product that doesn't involve $c_i$. $c_i\cdot c_i$ changes by $2(v_1+v_2+\cdots+v_n)$, which is 0 because $v$ to be orthogonal to $(1,1,\dots,1)$. Now, because we assumed that $c_j$ is not in the direction $(1,1,\dots,1)$, and also $p$ is orthogonal so it is not in the direction of any other $c_k$, $c_i\cdot c_j$ is changed by $v\cdot c_j$, which is nonzero. Thus, this (and its symmetric opposite) is the only entry that has nonzero determinant. So $D_pf$ is surjective, as desired.
	
	Proving that the product and inverse are smooth is the same as for $\SL_n(\R)$.
	
	As for the dimension, 4.3 gives that the codimension should be $\dim(\Sym_n(\R))$, which is $n + (n-1) + \cdots + 1 = \tfrac12(n)(n+1)$, so $\dim(\SO_n(\R)) = n^2 - \tfrac12(n)(n+1) = \tfrac12(n)(n-1)$. 
	
%	$\SO_n(\R)$ is homeomorphic to
%	$$S^{n-1}\times S^{n-2}\times \cdots \times S^1$$
%	as each $(n-1)$-tuple $(a_1,a_2,\dots,a_{n-1})$ in this product can be uniquely and continuously mapped to a $\SO_n(\R)$ matrix as follows: take the first column to be $a_1$, let the $(n-1)$-plane orthogonal to $a_1$ be $V_1$, take the second column to be $a_2\in V_1$, let the $(n-2)$-plane orthogonal to both $a_1$ and $a_2$ be $V_2$, $\dots$, and finally when $V_{n-1}$ is a 1-dimensional space, take whichever of the two unit vectors will make the resulting matrix det 1 (the other will give det -1). 
%	
%	This is continuous and bijective. This parametrization gives the dimension of $\SO_n(\R)$ as
%	$$
%	\dim(\SO_n(\R)) = (n-1) + (n-2) + \cdots + 1 = \tfrac12 n(n-1).
%	$$
\end{proof}

\newpage 

\textbf{Problem 3 (4.2)}: Find an embedding of $S^n\times S^m$ in $\R^{n+m+1}$.
\begin{proof}
	Since $S^n$ and $S^m$ are compact, it suffices to produce an injective immersion. Take $S^n$ and $S^m$ to be the submanifolds of $\R^{n+1},\R^{m+1}$ given by
	$$
	S^n = \{x_1,\dots,x_{n+1} \; : \; x_1^2 + \cdots + x_{n+1}^2 = 1\}, \;\;\; 	S^m = \{z_1,\dots,z_{m+1} \; : \; z_1^2 + \cdots + z_{n+1}^2 = 1\}. 
	$$
	Fixing some $R>1$, define the map $f:\R^{n+1}\times \R^{m+1} \to \R^{n+m+1}$ by
	$$
	f:(x_1,\dots,x_{n+1},z_1,\dots,z_{m+1}) \mapsto \Big(x_1(R+z_1),\dots,x_{n+1}(R+z_1),z_2,z_3,\dots,z_{m+1}\Big).
	$$
	$f$ is polynomial in every coordinate and thus smooth. I claim that $f$ is injective as a map restricted to $S^n\times S^m$.
	
	Suppose $(x_1',\dots,x_{n+1}',z_1',\dots,z_{m+1}')$ is another point in $S^n\times S^m$ mapped to the same output by $f$. The first $n+1$ coordinates give
	$$
	x_j(R+z_1) = x_{j}'(R+z_1') \implies \frac{x_j}{x_j'} = \frac{R+z_1'}{R+z_1} =: \lambda
	$$
	for all $1\leq j\leq n+1$. Because $R>1$ and $|z_1|,|z_1'|\leq 1$, we have $\lambda > 0$. But then
	$$
	1= x_1^2 + \cdots + x_{n+1}^2 = \lambda^2(x_1'^2 + \cdots + x_{n+1}'^2) = \lambda^2 \implies \lambda = 1. 
	$$
	This gives $z_1=z_1'$ and hence $x_j=x_j'$ for all $j$. From the remaining $m$ coordinates, it immediately follows that $z_j=z_j'$ for $j\geq 2$, so the two points are indeed equal. That is, $f$ is injective on $S^n\times S^m$, and thus it is an embedding of $S^n\times S^m$ into $\R^{n+m+1}$.
	
	%Then immediately $z_j'=z_j$ for $j\geq 2$, and hence because $z_1^2 = 1 - z_2^2-\cdots-z_{m+1}^2$ we have $z_1^2=z_1'^2$, giving $z_1=\pm z_1'$.
\end{proof}
	
\end{document}