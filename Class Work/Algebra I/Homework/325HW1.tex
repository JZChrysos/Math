\documentclass{amsart}

%\documentclass{amsart}
\usepackage[utf8]{inputenc}
\usepackage{amsfonts}
\usepackage{amsmath}
\usepackage{amssymb}
\usepackage{amsthm}
\usepackage{asymptote}
\usepackage{mathtools}
\usepackage{hhline}
\usepackage{graphicx,enumerate}
\usepackage{hyperref}
\usepackage[a4paper, margin=1.2in]{geometry}
%\usepackage{tcolorbox}
\usepackage{tikz-cd}
\usepackage{ytableau}
%\tcbuselibrary{skins,breakable,xparse}
\allowdisplaybreaks
\newcounter{count}
\hypersetup{
	colorlinks=true,
	linkcolor=teal,
	filecolor=magenta,      
	urlcolor=olive,
	citecolor=teal,
	pdfpagemode=FullScreen,
}

%\definecolor{defcolor}{HTML}{478EFF}
%\definecolor{thmcolor}{HTML}{CC0058}
%\definecolor{excolor}{HTML}{F5B400}
%\definecolor{probcolor}{HTML}{DD4803}
%\definecolor{lemcolor}{HTML}{741FEA}
%\definecolor{scarlet}{HTML}{A81111}
%
%\newtheoremstyle{definitionStyle}% Custom style for definitions
%{0.5em}% Space above
%{0.5em}% Space below
%{}% Body font
%{}% Indent amount
%{\bfseries\color{defcolor}}% Theorem head font: bold and red
%{.\\}% Punctuation after theorem head
%{0.5em}% Space after theorem head
%{\thmname{#1}\thmnumber{ #2 (#3)}}% Theorem head spec
%
%\theoremstyle{definitionStyle}
%\newtheorem{df}{Definition}[section]
%
%\newtheoremstyle{theoremStyle}% Custom style for definitions
%{0.5em}% Space above
%{0.5em}% Space below
%{}% Body font
%{}% Indent amount
%{\bfseries\color{thmcolor}}% Theorem head font: bold and red
%{.\\}% Punctuation after theorem head
%{0.5em}% Space after theorem head
%{\thmname{#1}\thmnumber{ #2 (#3)}}% Theorem head spec
%
%\theoremstyle{theoremStyle}
%\newtheorem{thm}{Theorem}[section]
%
%\newtheoremstyle{lemmaStyle}% Custom style for definitions
%{0.5em}% Space above
%{0.5em}% Space below
%{}% Body font
%{}% Indent amount
%{\bfseries\color{lemcolor}}% Theorem head font: bold and red
%{.\\}% Punctuation after theorem head
%{0.5em}% Space after theorem head
%{\thmname{#1}\thmnumber{ #2 (#3)}}% Theorem head spec
%
%\theoremstyle{lemmaStyle}
%\newtheorem{lem}{Lemma}[section]
%\newtheorem{cor}{Corollary}[section]
%
%\newtheoremstyle{exampleStyle}% Custom style for definitions
%{0.5em}% Space above
%{0.5em}% Space below
%{}% Body font
%{}% Indent amount
%{\bfseries\color{excolor}}% Theorem head font: bold and red
%{.\\}% Punctuation after theorem head
%{0.5em}% Space after theorem head
%{\thmname{#1}\thmnumber{ #2 (#3)}}% Theorem head spec
%
%\theoremstyle{exampleStyle}
%\newtheorem{ex}{Example}[section]
%
%\newtheoremstyle{problemStyle}% Custom style for definitions
%{0.5em}% Space above
%{0.5em}% Space below
%{}% Body font
%{}% Indent amount
%{\bfseries\color{probcolor}}% Theorem head font: bold and red
%{.\\}% Punctuation after theorem head
%{0.5em}% Space after theorem head
%{\thmname{#1}\thmnumber{ #2#3}}% Theorem head spec
%
%\theoremstyle{problemStyle}
%\newtheorem{prob}{Problem}[section]

% For Fun
\newcommand{\club}{\color{teal} \clubsuit}
\newcommand{\heart}{\color{red} \heartsuit}
\renewcommand{\star}{\color{scarlet} \bigstar}
\newcommand{\spade}{\color{violet} \spadesuit}

% Symbols
\newcommand{\A}{\mathcal{A}}
\newcommand{\B}{\mathcal{B}}
\newcommand{\C}{\mathbb{C}}
\newcommand{\D}{\mathcal{D}}
\newcommand{\E}{\mathbb{E}}
\newcommand{\F}{\mathbb{F}}
\newcommand{\G}{\mathcal{G}}
% \renewcommand{\H}{\mathcal{H}} Erdos o
\newcommand{\I}{\mathcal{I}}
\newcommand{\J}{\mathcal{J}}
\newcommand{\K}{\mathcal{K}}
% \renewcommand{\L}{\mathcal{L}}
\newcommand{\M}{\mathcal{M}}
\newcommand{\N}{\mathbb{N}}
\renewcommand{\O}{\mathcal{O}}
\renewcommand{\P}{\mathbb{P}}
\newcommand{\Q}{\mathbb{Q}}
\newcommand{\R}{\mathbb{R}}
\renewcommand{\S}{\mathbb{S}}
\newcommand{\T}{\mathbb{T}}
\newcommand{\U}{\mathcal{U}}
\newcommand{\V}{\mathcal{V}}
\newcommand{\W}{\mathcal{W}}
\newcommand{\X}{\mathcal{X}}
\newcommand{\Y}{\mathcal{Y}}
\newcommand{\Z}{\mathbb{Z}}

\renewcommand{\AA}{\mathcal{A}}
\newcommand{\BB}{\mathcal{B}}
\newcommand{\CC}{\mathcal{C}}
\newcommand{\DD}{\mathcal{D}}
\newcommand{\EE}{\mathcal{E}}
\newcommand{\FF}{\mathcal{F}}
\newcommand{\GG}{\mathbb{G}}
\newcommand{\HH}{\mathbb{H}}
\newcommand{\calH}{\mathcal{H}}
\newcommand{\II}{\mathcal{I}}
\newcommand{\JJ}{\mathcal{J}}
\newcommand{\KK}{\mathcal{K}}
\newcommand{\LL}{\mathcal{L}}
\newcommand{\MM}{\mathcal{M}}
\newcommand{\NN}{\mathcal{N}}
\newcommand{\OO}{\mathrm{O}}
\newcommand{\PP}{\mathcal{P}}
\newcommand{\QQ}{\mathcal{Q}}
\newcommand{\RR}{\mathcal{R}}
\renewcommand{\SS}{\mathcal{S}}
\newcommand{\TT}{\mathcal{T}}
\newcommand{\UU}{\mathcal{U}}
\newcommand{\VV}{\mathcal{V}}
\newcommand{\WW}{\mathcal{W}}
\newcommand{\XX}{\mathcal{X}}
\newcommand{\YY}{\mathcal{Y}}
\newcommand{\ZZ}{\mathcal{Z}}
\renewcommand{\d}{\textrm{d}}
% Greek letters
\newcommand{\ep}{\varepsilon}
\newcommand{\ph}{\varphi}
\newcommand{\de}{\delta}
\renewcommand{\a}{\alpha}
\renewcommand{\b}{\beta}
% Fraktur
\newcommand{\mm}{\mathfrak{m}}
\renewcommand{\aa}{\mathfrak{a}}
\newcommand{\bb}{\mathfrak{b}}
\newcommand{\pp}{\mathfrak{p}}
\newcommand{\qq}{\mathfrak{q}}
% Operators
\DeclareMathOperator{\Div}{div}
\DeclareMathOperator{\Gal}{Gal}
\DeclareMathOperator{\vol}{Vol}
\DeclareMathOperator{\Hom}{Hom}
\DeclareMathOperator{\End}{End}
\DeclareMathOperator{\Ext}{Ext}
\DeclareMathOperator{\Tor}{Tor}
\DeclareMathOperator{\tr}{tr}
\DeclareMathOperator{\rk}{rk}
\DeclareMathOperator{\curl}{curl}
\DeclareMathOperator{\mesh}{mesh}
\DeclareMathOperator{\im}{im}
\DeclareMathOperator{\coker}{coker}
\DeclareMathOperator{\width}{width}
\DeclareMathOperator{\diam}{diam}
\DeclareMathOperator{\maps}{Maps}
\DeclareMathOperator{\Frac}{Frac}
\DeclareMathOperator{\Sym}{Sym}
\DeclareMathOperator{\sgn}{sgn}
\DeclareMathOperator{\alt}{Alt}
\DeclareMathOperator{\supp}{supp}
\DeclareMathOperator{\Span}{span}
\DeclareMathOperator{\Var}{Var}
\DeclareMathOperator{\Spec}{Spec}

\newcommand{\nor}{\unlhd}
\DeclareMathOperator{\aut}{Aut}
\DeclareMathOperator{\orb}{Orb}
\DeclareMathOperator{\GL}{GL}
\DeclareMathOperator{\SL}{SL}
\DeclareMathOperator{\SO}{SO}
\DeclareMathOperator{\PGL}{PGL}
\DeclareMathOperator{\PSL}{PSL}
\DeclareMathOperator{\stab}{Stab}
\DeclareMathOperator{\fix}{Fix}
\DeclareMathOperator{\Th}{Th}
\DeclareMathOperator{\Ind}{Ind}
\DeclareMathOperator{\Res}{Res}
\DeclareMathOperator{\Ann}{Ann}
\DeclareMathOperator{\rad}{rad}
\DeclareMathOperator{\len}{len}
\DeclareMathOperator{\ord}{ord}

% \DeclareMathOperator{\arg}{arg}

%% misc
\newcommand{\<}{\langle}
\renewcommand{\>}{\rangle}
\renewcommand{\^}{\wedge}
\renewcommand{\v}{\vee}
\def\Xint#1{\mathchoice
	{\XXint\displaystyle\textstyle{#1}}%
	{\XXint\textstyle\scriptstyle{#1}}%
	{\XXint\scriptstyle\scriptscriptstyle{#1}}%
	{\XXint\scriptscriptstyle\scriptscriptstyle{#1}}%
	\!\int}
\def\XXint#1#2#3{{\setbox0=\hbox{$#1{#2#3}{\int}$ }
		\vcenter{\hbox{$#2#3$ }}\kern-.6\wd0}}
\def\ddashint{\Xint=}
\def\dashint{\Xint-}
%% arrows
\newcommand{\xhra}{\xhookrightarrow}
\newcommand{\xra}{\xrightarrow}
\newcommand{\ra}{\rightarrow}
\newcommand{\rra}{\rightrightarrows}
\newcommand{\lra}{\longrightarrow}
\newcommand{\Ra}{\Rightarrow}
\newcommand{\lRa}{\Longrightarrow}
\newcommand{\lrsa}{\leftrightsquiqarrow}
\newcommand{\ba}{\leftrightarrow}
%% lists
\newcommand{\be}{\begin{enumerate}[(i)]}
	\newcommand{\ee}{\end{enumerate}}
%% integration stuff
\newcommand{\calR}{\mathcal{R}}
\newcommand{\rint}{\calR\!\int}
\newcommand{\calL}{\mathcal{L}}
\newcommand{\lowerint}{\mbox{\b{$\int$}}}
\newcommand{\upperint}{{\textstyle\bar{\int}}}
%% end of proof
\def\endproof{{\hfill $\Box$}}
%% matrix shorthand
\DeclareMathOperator{\res}{res}

\title{Math 325 HW 1}
\author{Jalen Chrysos}

\begin{document}
	
	\maketitle
	
\noindent \textbf{Problem 1}:  Let $V=k^n$ and $P_d\subseteq k[x_1,\dots,x_n]$ be the space of degree-$d$ polynomials in $n$ variables over $k$. Let $a:V\to V$ be some linear map with eigenvalues $\lambda_1,\dots,\lambda_n$. Show the following generating-function identities:
\begin{enumerate}[(a)]
	\item $$\sum_{d\geq 0} \tr_{V^{\otimes d}}(a^{\otimes d})\cdot t^d = \frac{1}{1-\tr_V(a)\cdot t}.$$
	\item $$\sum_{d\geq 0} \tr_{\Sym^dV}(\Sym^d(a)) \cdot t^d = \prod_{i=1}^n \frac{1}{1-\lambda_it}.$$
	\item $$\sum_{d\geq 0} \tr_{\wedge^dV}(\wedge^d a)\cdot t^d = \prod_{i=1}^n (1+\lambda_i t).$$
\end{enumerate}
	\begin{proof}
		(a): The trace of $a^{\otimes d}$ is $\tr(a)^d$. It suffices to take some basis where $a$ is in Jordan normal form, as trace is invariant under conjugation. In this case, $a^{\otimes d}$ acts by
		$$a^{\otimes d}(e_{k_1}\otimes \dots \otimes e_{k_d}) = \Big(\prod_j \lambda_{k_j}\Big) (e_{k_1}\otimes \dots \otimes e_{k_d}) + \text{off-diagonal terms}$$
		and the trace is thus
		$$
		\sum_{(k_j)\in [n]^d} \prod_j \lambda_{k_j} = \Big(\sum_k \lambda_k\Big)^d = \tr(a)^d
		$$
		as desired. The generating function immediately follows from the geometric sum formula.\\
		
		(b): The trace of $\Sym^d(a)$ is the sum of all degree-$d$ monomials in $\lambda_1,\dots,\lambda_n$. Again we can assume that $a$ is in Jordan normal form wrt some basis. Then $\Sym^d(a)$ acts by
		$$
		e_1^{d_1}e_2^{d_2}\cdots e_n^{d_n} \mapsto \lambda_1^{d_1}\lambda_2^{d_2}\cdots \lambda_n^{d_n} (e_1^{d_1}e_2^{d_2}\cdots e_n^{d_n}) + \text{off-diagonal terms}
		$$
		giving the trace as desired. From there, the generating function is
		$$
		\sum_{d\geq 0} \tr(\Sym^d(a))\cdot t^d = \prod_i \sum_{d_i \geq 0} \lambda_i^{d_i} t^{d_i} = \prod_i \frac{1}{1-\lambda_i t}.
		$$
		
		(c): The trace of $\wedge^d a$ is the sum of all degree $d$ monomials in $\lambda_1,\dots,\lambda_n$ without repeating terms. Again we assume that $a$ is in Jordan normal form wrt some basis. $\wedge^d a$ acts by
		$$
		e_{k_1}\wedge \dots \wedge e_{k_d} \mapsto \lambda_{k_1}\cdots\lambda_{k_d} (e_1\wedge \dots \wedge e_d)
		$$
		giving the desired trace. The generating function follows. It is like in (b) except that terms cannot repeat, thus we have $1+\lambda_i t$ instead of the full series $1+\lambda_it+\lambda_i^2t^2+\dots$
	\end{proof}
	
	\newpage
	
	\noindent \textbf{Problem 2}: Prove the following claims made in class:
	\begin{enumerate}[(a)]
	\item The algebra $\C[x_1,\dots,x_n]^{\SO_n}$ is the free algebra generated by $R:= x_1^2+\dots+x_n^2$.
	\item The algebra $\C[x_1,\dots,x_n]^{\SO_{n-1}}$ is the free algebra generated by $R$ and $x_n$.
	\end{enumerate}
	
	\begin{proof}
		(a): Let $p$ be a polynomial fixed by $\SO_n$. Let $q(t)$ be the polynomial
		$$
		q(t) := p(t,0,0,\dots,0).
		$$
		Because $p$ is fixed by $\SO_n$, we have $p(x_1,\dots,x_n) = q(\sqrt{R})$ for all $x_1,\dots,x_n$. Moreover, $q$ has no odd-degree terms, since $q(t)$ is invariant under switching the sign of $t$ (as this corresponds to a $\pi$-rotation about the $x_2$ axis, which is in $\SO_n$). Thus, $q(\sqrt{R})$ is a polynomial in $R$, and hence $p$ is as well.\\
		
		(b) If $p$ is fixed by $\SO_{n-1}$, then we can think of $p$ as a polynomial in $\C[x_n][x_1,\dots,x_{n-1}]$, from which the same method from part (a) shows that $p$ is generated by $x_1^2+\dots+x_{n-1}^2$ over $\C[x_n]$, and thus that $p$ is generated by $x_1^2+\dots+x_{n-1}^2$ and $x_n$ over $\C$, or equivalently $R$ and $x_n$. 
	\end{proof}
	
	\newpage
	
	\noindent \textbf{Problem 3}: The case $n=2$: Let $H_d\subseteq P_d$ be the space of degree-$d$ harmonic homogeneous polynomials in $x,y$ over $\C$.
	\begin{enumerate}[(a)]
	\item Find an explicit $\C$-basis for $H_d$.
	\item Show that the representation of $\SO_2$ in $H_d$ is not irreducible.
	\end{enumerate}
	
	\begin{proof}
		(a): For all $d$, $\dim(H_d)=2$. Let $A$ be the homogeneous polynomial
		$$
		A(x,y) := \sum_{k=0}^d a_k x^ky^{d-k}.
		$$
		If $A\in H_d$, then 
		\begin{align*}
			0 &= \Delta(A)\\
		 &= \sum_{k=0}^d a_k \cdot (k)(k-1) x^{k-2}y^{d-k} + a_k\cdot (d-k)(d-k-1) x^ky^{d-k-2}\\
		&= \sum_{k=0}^{d-2} \Big(a_{k}(d-k)(d-k-1) + a_{k+2}(k+2)(k+1)\Big) x^kx^{d-k-2}
		\end{align*}
		and thus 
		$$
		a_{k}(d-k)(d-k-1) + a_{k+2}(k+2)(k+1) = 0
		$$
		for all $0\leq k \leq d-2$. Thus, given that $A$ is harmonic, the ratios between all even coefficients are fixed, and the ratios between all odd coefficients are fixed, i.e. $A$ is a linear combination of the two harmonic polynomials
		$$
		y^d - \Big(\frac{d(d-1)}{2}\Big)x^2y^{d-2} + \Big(\frac{d(d-1)(d-2)(d-3)}{(4)(3)(2)}\Big)x^4y^{d-4} - \cdots
		$$
		and 
		$$
		xy^{d-1} - \Big(\frac{(d-1)(d-2)}{(3)(2)}\Big)x^3y^{d-3} + \Big(\frac{(d-1)(d-2)(d-3)(d-4)}{(5)(4)(3)(2)}\Big)x^5y^{d-5} - \cdots 
		$$
		which are also proportional to the real and imaginary parts of $(ix+y)^d$.\\
		
		(b): The action of $\SO_2$ on $H_d$ preserves the 1-dimensional subspace $\C\<(ix+y)^d\>\subset H_d$. All elements of $\SO_2$ are rotations by some $\theta$, corresponding to matrices
		$$
		\begin{bmatrix}
			\cos \theta & \sin \theta \\
			-\sin \theta & \cos\theta
		\end{bmatrix}
		$$
		which maps $(ix+y)^d$ to 
		$$
		(i(\cos\theta x + \sin\theta y) + \cos\theta y - \sin\theta x)^d = (e^{i\theta}(ix + y))^d = e^{i\theta d} (ix+y)^d
		$$
		and thus preserves the subspace. Thus the action of $\SO_2$ is reducible.
		
	\end{proof}
		
	\newpage
	\noindent \textbf{Problem 4}: In the case $n=3$: 
	\begin{enumerate}[(a)]
		\item Check that $\dim (H_d)=2d+1$ and find a nonzero element of $H_2$ which is fixed by $\SO_2$.
		\item For each $d\geq 0$, find the trace of the operator $g_{\theta}$, which is the rotation by $\theta$ about the $z$ axis. 
	\end{enumerate}
	\begin{proof}
		(a): To show that $\dim(H_d)=2d+1$, we use the decomposition of $P_d$ as
		$$
		P_d = H_d\oplus R\cdot P_{d-2}
		$$
		which gives
		\begin{align*}
		\dim(P_d) &= \dim(H_d) + \dim(P_{d-2})\\
		{d+2 \choose 2} &= \dim(H_2) + {d \choose 2}\\
		\dim(H_2) &= \frac{(d+2)(d+1)-d(d-1)}{2} \\
		&= \frac{4d+2}{2}=2d+1
		\end{align*}
		as expected.
		
		
%		note that among the ${d+2\choose 2}$ monomial coefficients of a polynomial in $P_d$, harmonicity adds one linear condition on these coefficients for every monomial that can appear in the Laplacian, of which there are ${d\choose 2}$ (because the degree is reduced by 2). This yields a dimension of at least
%		$$
%		{d+2\choose 2} - {d \choose 2} = \frac{(d+2)(d+1)-d(d-1)}{2} = \frac{4d+2}{2}=2d+1
%		$$
%		giving $\dim(H_d)\geq 2d+1$. In the other direction, we can use the decomposition of $P_d$ to get
%		$$
%		{d+2 \choose 2} = \dim(P_d) = \sum_{2j\leq d} \dim(H_{d-2j}) \geq \sum_{2j\leq d} 2(d-2j) + 1 = {d+2 \choose 2}
%		$$ 
%		so $\dim(H_{d-2j})$ can be no larger than $2(d-2j) + 1$ for each $j$. Thus $\dim(H_{d})=2d+1$ for all $d$.
		
		To give a nonzero element of $H_2$ fixed by $\SO_2$, take $x^2+y^2-2z^2$. Clearly $x^2+y^2$ and $z^2$ are each fixed by $\SO_2$, so $x^2+y^2-2z^2$ is also fixed, and it is harmonic.\\
		
		(b): $g_{\theta}$ can be represented by the matrix
		$$
		g_{\theta} = \begin{pmatrix}
			e^{i\theta} & 0 & 0 \\
			0 & e^{-i\theta} & 0\\
			0 & 0 & 1
		\end{pmatrix}
		$$
		and its action on the basis elements of $P_d$ is
		$$
		g_{\theta}: x^ay^bz^c \mapsto (e^{i\theta}x)^a(e^{-i\theta}y)^bz^c = e^{(a-b)i\theta} \cdot x^ay^bz^c.
		$$
		This gives the trace
		$$
		\tr_{P_d}(g_{\theta}) = \sum_{a+b\leq d} e^{(a-b)i\theta}.
		$$
		Now to determine $\tr_{H_d}(g_{\theta})$ from this. We can decompose $P_d$ into subspaces $H_d\oplus R\cdot P_{d-2}$. Thus, the trace of $g_{\theta}$ is the sum of its trace on the subspaces $H_d$ and $R\cdot P_{d-2}$. In the latter subspace, the trace is not affected by $R$, since $g_{\theta}$ preserves $R$. Thus,
		\begin{align*}
		\tr_{H_d}(g_{\theta}) &= \tr_{P_d}(g_{\theta}) - \tr_{P_{d-2}}(g_{\theta}) \\
		&= \sum_{a+b \in \{d-1,d\}} e^{i\theta(a-b)}\\
		&= \sum_{k=-d}^{d} e^{ki\theta}\\
		&= 1 + 2\Big(\sum_{k=1}^{d} \cos(k\theta)\Big).\\
		\end{align*}
		In the case $\theta = 0$, this yields $\tr_{H_d}(\id)=\dim(H_d)=2d+1$ as expected.
		
	\end{proof}
	
	\newpage 
	\noindent \textbf{Problem 5}: For $n\geq 3$, prove the following generating function identity:
	$$
	\sum_{d\geq 0} \dim (H_d) \cdot t^d = \frac{1+t}{(1-t)^{n-1}}.
	$$
	\begin{proof}
%		First we'll show the case $n=3$. By the previous problem, $\dim(H_d)=2d+1$, so we have
%		\begin{align*}
%		\sum_{d\geq 0} \dim(H_d)\cdot t^d &= \sum_{d\geq 0} (2d+1)t^d\\
%		&= \bigg(\sum_{d\geq 0} t^d\bigg) + 2\bigg(\sum_{d\geq 0} dt^d\bigg)\\
%		&= \bigg(\frac{1}{1-t}\bigg) + 2\bigg(\frac{t}{(1-t)^2}\bigg)\\
%		&= \frac{1+t}{(1-t)^2}
%		\end{align*}
%		as expected. 
		For general $n$, $\dim(H_d)$ is 
		$$
		\dim(H_d) = \dim(P_d) - \dim(P_{d-2}) = {d+n-1 \choose n-1} - {d + n - 3 \choose n - 1}
		$$
		which yields the generating function
		\begin{align*}
			\sum_{d\geq 0} \dim(H_d) t^d &= \sum_{d\geq 0} \bigg({d+n-1 \choose n-1} - {d + n - 3 \choose n - 1}\bigg) t^d\\
			&= \frac{1}{(1-t)^{n}} - \frac{t^2}{(1-t)^{n}} \\
			&= \frac{1+t}{(1-t)^{n-1}}
		\end{align*}
		as desired. Here we used the generating function identity
		$$
		\frac{1}{(1-t)^m} = \sum_{d\geq 0} {d + m \choose m} t^d
		$$
		which can be proven by the generalized binomial theorem:
		$$
		(1-t)^{-m} = \sum_{d\geq 0}{-m\choose d} (-t)^d = \sum_{d\geq 0} {m+d \choose d} t^d = \sum_{d\geq 0} {m+d \choose m} t^d.
		$$
	\end{proof}
	
	\newpage
	\noindent \textbf{Problem 6}: Let $\res:\C[x_1,\dots,x_n]\to C(S^{n-1})$ be the restriction map to $S^{n-1}$. If $\overline{P_d}:=\res(P_d)$, let $M_d$ be the orthogonal complement of $\overline{P_{d-2}}$ in $\overline{P_{d}}$. Show that $\res$ induces an isomorphism $H_d\to M_d$ of $\SO_n$-representations.
	
	\begin{proof}
		First, we'll show that $\res(H_d)\subseteq M_d$ by showing that for $h\in H_d,p\in P_{d-2}$, 
		$$
		\int_{S^{n-1}} h\overline{p} = 0.
		$$
		Because of the decomposition of $P_{d-2}$, and since $R=1$ on $S^{n-1}$, it suffices to show this for $p\in H_{d-2j}$ for all $j\geq 1$. We will prove this be induction on $d$. For $d=1,2$ the result is clear. For the inductive step, assume that this holds for $d-1$, and we'll show that it does for $d$ as well.
		
		To do this we will use the divergence theorem
		$$
		\int_{S^{n-1}} \nabla f \cdot \vec{x} = \int_{B^{n}} \Delta f,
		$$
		Euler's identity for homogeneous functions,
		$$
		\nabla f \cdot \vec{x} = \deg(f) \cdot f,
		$$
		and the Laplacian identity
		$$
		\Delta(ab) = a\Delta(b) + b\Delta(a) + 2\nabla a \cdot \nabla b.
		$$
		Using these, we get
		\begin{align*}
		\int_{S^{n-1}} h\overline{p} &= \frac{1}{2d-2} \int_{S^{n-1}} \nabla (h\overline{p}) \cdot \vec{x}\\
		&= \frac{1}{2d-2} \int_{B^{n}} \Delta (h\overline{p})\\
		&= \frac{1}{2d-2} \int_{B^{n}} \overline{p}\Delta (h) + h \Delta(\overline{p}) + 2\nabla(h)\cdot \nabla (\overline{p})\\
		&= \frac{1}{d-1}\int_{B^{n}} \nabla(h)\cdot \nabla (\overline{p}).
		\end{align*}
		Now, since $h,\overline{p}$ are harmonic, $\partial_j h$ and $\partial_j \overline{p}$ are also harmonic for each $j\in [n]$, thus $\nabla(h)\cdot \nabla(\overline{p})$ is the sum of products in $H_{d-1}\cdot H_{d-3}$. By the inductive hypothesis, we know these integrate to $0$ over $S^{n-1}$ and thus over $B^n$ as well. So $\<h,\overline{p}\>=0$, as desired.\\
		
		Conversely, if $q\in P_d$, and $\<q,p\>=0$ for all $p\in P_{d-2}$, then $q$ must be harmonic: otherwise, $q=Rq_{d-2}$ where $q_{d-2}\in P_{d-2}$, so setting $p=q_{d-2}$ yields
		$$
		\<q,p\> = \<Rq_{d-2},q_{d-2}\> = \int_{S^{n-1}} |q_{d-2}|^2 > 0.
		$$
		So we see that $\res$ induces an isomorphism on the vector spaces $H_d$ and $M_d$. And noting that $\res$ clearly commutes with the action of $\SO_n$ on $H_d$ and $M_d$, this map is also an isomorphism of representations.
	\end{proof}
	
	\newpage 
	\noindent \textbf{Problem 7}: Prove that for $d\geq 0$, there is a constant $c_d>0$ such that for all $p,q\in H_d$, 
	$$p(\partial)(\overline{q})(0) = c_d \cdot \<\res(p),\res(q)\>_{L^2}$$
	where the inner product $\<\bullet,\bullet\>_{L^2}$ is given by 
	$$
	\<f,g\> := \int_{S^{n-1}} f(x)\overline{g}(x)
 \; \d x.$$
 
 \begin{proof}
 	We will show this by induction on $d$. In the case $d=1$, $p$ and $q$ are each linear combinations of $x_1,\dots,x_n$, so 
 	$$
 	p = p_1x_1 + p_2x_2+\dots +p_n x_n, \;\;\; 	q = q_1x_1 + q_2x_2+\dots +q_n x_n.
 	$$
 	The LHS is equal to $p\cdot q:= \sum_j p_j\overline{q_j}$, naturally. On the RHS, we integrate a quadratic over $S^{n-1}$, and the cross-terms cancel because they are odd, yielding
 	\begin{align*}
 		\int_{S^{n-1}} p\overline{q} &= \sum_{i,j=1}^n \int_{S^{n-1}} p_i\overline{q_j} x_ix_j\\
 		&= \sum_j \int_{S^{n-1}} p_j\overline{q_j} x_j^2\\
 		&= \bigg(\int_{S^{n-1}} x_1^2 \bigg) \cdot (p \cdot q).
 	\end{align*}
 	Indeed, the RHS and LHS differ by a constant factor, as desired. Now, for the inductive step, assume that there is such a $c_{d-1}$, and we will show that there is a $c_d$.\\
 	
 	As explained in the solution to Problem 6, we can express $\<\res(p),\res(q)\>$ as 
 	\begin{align*}
 		\<\res(p),\res(q)\> &= \int_{S^{n-1}} p\overline{q} \\
 		&= \frac{1}{d}\int_{B^n} \nabla(p) \cdot \nabla(\overline{q}).
 	\end{align*}
 	Now, $\nabla(p)$ and $\nabla(\overline{q})$ are both sums of polynomials in $H_{d-1}$, so by the inductive hypothesis,
 	\begin{align*}
 		\<\res(p),\res(q)\> &= \frac{1}{d}\int_{B^n} \nabla(p) \cdot \nabla(\overline{q})\\
 		&= \frac{1}{d} \int_{0}^1 \int_{S^{n-1}} \nabla(p(r\vec{x})) \cdot \nabla(\overline{q}(r\vec{x})) \;\d \vec{x} \d r\\
 		&= \frac{1}{d} \int_{0}^1 r^{2d-2} \int_{S^{n-1}} \nabla(p(\vec{x})) \cdot \nabla(\overline{q}(\vec{x})) \;\d \vec{x} \d r\\
 		&= \frac{1}{d} \int_{0}^1 r^{2d-2} c_{d-1} (\nabla p,\nabla\overline{q})\;\d r\\
 		&= \frac{c_{d-1}}{d(2d-1)} (\nabla p, \nabla \overline{q}).
 	\end{align*}
 	Finally, we have
 	$$
 	(\nabla p,\nabla \overline{q}) = (\nabla p \cdot \vec{x},\overline{q}) = (dp,\overline{q}) = d(p,\overline{q})
 	$$
 	giving 
 	$$
 	\<\res(p),\res(q)\> = \frac{c_{d-1}}{(2d-1)} \cdot (p,\overline{q})
 	$$
 	Thus, $c_d$ exists and is equal to $c_{d-1}/(2d-1)$.
 \end{proof}
 
 \newpage
 
\noindent \textbf{Problem 8 (Optional)}: Show that the kernel of $\res$ is exactly the principal ideal generated by $R-1$.

\begin{proof}
	By Hilbert's Nullstellensatz, the ideal of polynomials vanishing on $S^{n-1}$ is exactly $\sqrt{R-1}$. But $R-1$ is irreducible so $\sqrt{R-1}=(R-1)$. One can see this by using Eisenstein's criterion:\\
	
	We'll show this by induction on $n$. For the base case $n=2$, $x^2+y^2 - 1$ is irreducible as a polynomial in $\C[x][y]$, as $(x-1)$ is prime and divides (only once) the constant term $(x^2-1)$ and the linear term $0$, but not the leading term $y^2$ (Eisenstein's Criterion). Thus it is irreducible as a polynomial in $\C[x,y]$ as well.
	
	For the inductive step, the $x_n^0$ term $x_1^2+\dots+x_{n-1}^2-1$ is prime by the inductive hypothesis, and it also divides the $x_n^1$ term (which is 0), but not the $x_n^2$ term, so $x_1^2+\dots+x_n^2-1$ is also irreducible.
\end{proof} 
		 
\end{document}