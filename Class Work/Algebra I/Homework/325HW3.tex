\documentclass{amsart}

%\documentclass{amsart}
\usepackage[utf8]{inputenc}
\usepackage{amsfonts}
\usepackage{amsmath}
\usepackage{amssymb}
\usepackage{amsthm}
\usepackage{asymptote}
\usepackage{mathtools}
\usepackage{hhline}
\usepackage{graphicx,enumerate}
\usepackage{hyperref}
\usepackage[a4paper, margin=1.2in]{geometry}
%\usepackage{tcolorbox}
\usepackage{tikz-cd}
\usepackage{ytableau}
%\tcbuselibrary{skins,breakable,xparse}
\allowdisplaybreaks
\newcounter{count}
\hypersetup{
	colorlinks=true,
	linkcolor=teal,
	filecolor=magenta,      
	urlcolor=olive,
	citecolor=teal,
	pdfpagemode=FullScreen,
}

%\definecolor{defcolor}{HTML}{478EFF}
%\definecolor{thmcolor}{HTML}{CC0058}
%\definecolor{excolor}{HTML}{F5B400}
%\definecolor{probcolor}{HTML}{DD4803}
%\definecolor{lemcolor}{HTML}{741FEA}
%\definecolor{scarlet}{HTML}{A81111}
%
%\newtheoremstyle{definitionStyle}% Custom style for definitions
%{0.5em}% Space above
%{0.5em}% Space below
%{}% Body font
%{}% Indent amount
%{\bfseries\color{defcolor}}% Theorem head font: bold and red
%{.\\}% Punctuation after theorem head
%{0.5em}% Space after theorem head
%{\thmname{#1}\thmnumber{ #2 (#3)}}% Theorem head spec
%
%\theoremstyle{definitionStyle}
%\newtheorem{df}{Definition}[section]
%
%\newtheoremstyle{theoremStyle}% Custom style for definitions
%{0.5em}% Space above
%{0.5em}% Space below
%{}% Body font
%{}% Indent amount
%{\bfseries\color{thmcolor}}% Theorem head font: bold and red
%{.\\}% Punctuation after theorem head
%{0.5em}% Space after theorem head
%{\thmname{#1}\thmnumber{ #2 (#3)}}% Theorem head spec
%
%\theoremstyle{theoremStyle}
%\newtheorem{thm}{Theorem}[section]
%
%\newtheoremstyle{lemmaStyle}% Custom style for definitions
%{0.5em}% Space above
%{0.5em}% Space below
%{}% Body font
%{}% Indent amount
%{\bfseries\color{lemcolor}}% Theorem head font: bold and red
%{.\\}% Punctuation after theorem head
%{0.5em}% Space after theorem head
%{\thmname{#1}\thmnumber{ #2 (#3)}}% Theorem head spec
%
%\theoremstyle{lemmaStyle}
%\newtheorem{lem}{Lemma}[section]
%\newtheorem{cor}{Corollary}[section]
%
%\newtheoremstyle{exampleStyle}% Custom style for definitions
%{0.5em}% Space above
%{0.5em}% Space below
%{}% Body font
%{}% Indent amount
%{\bfseries\color{excolor}}% Theorem head font: bold and red
%{.\\}% Punctuation after theorem head
%{0.5em}% Space after theorem head
%{\thmname{#1}\thmnumber{ #2 (#3)}}% Theorem head spec
%
%\theoremstyle{exampleStyle}
%\newtheorem{ex}{Example}[section]
%
%\newtheoremstyle{problemStyle}% Custom style for definitions
%{0.5em}% Space above
%{0.5em}% Space below
%{}% Body font
%{}% Indent amount
%{\bfseries\color{probcolor}}% Theorem head font: bold and red
%{.\\}% Punctuation after theorem head
%{0.5em}% Space after theorem head
%{\thmname{#1}\thmnumber{ #2#3}}% Theorem head spec
%
%\theoremstyle{problemStyle}
%\newtheorem{prob}{Problem}[section]

% For Fun
\newcommand{\club}{\color{teal} \clubsuit}
\newcommand{\heart}{\color{red} \heartsuit}
\renewcommand{\star}{\color{scarlet} \bigstar}
\newcommand{\spade}{\color{violet} \spadesuit}

% Symbols
\newcommand{\A}{\mathcal{A}}
\newcommand{\B}{\mathcal{B}}
\newcommand{\C}{\mathbb{C}}
\newcommand{\D}{\mathcal{D}}
\newcommand{\E}{\mathbb{E}}
\newcommand{\F}{\mathbb{F}}
\newcommand{\G}{\mathcal{G}}
% \renewcommand{\H}{\mathcal{H}} Erdos o
\newcommand{\I}{\mathcal{I}}
\newcommand{\J}{\mathcal{J}}
\newcommand{\K}{\mathcal{K}}
% \renewcommand{\L}{\mathcal{L}}
\newcommand{\M}{\mathcal{M}}
\newcommand{\N}{\mathbb{N}}
\renewcommand{\O}{\mathcal{O}}
\renewcommand{\P}{\mathbb{P}}
\newcommand{\Q}{\mathbb{Q}}
\newcommand{\R}{\mathbb{R}}
\renewcommand{\S}{\mathbb{S}}
\newcommand{\T}{\mathbb{T}}
\newcommand{\U}{\mathcal{U}}
\newcommand{\V}{\mathcal{V}}
\newcommand{\W}{\mathcal{W}}
\newcommand{\X}{\mathcal{X}}
\newcommand{\Y}{\mathcal{Y}}
\newcommand{\Z}{\mathbb{Z}}

\renewcommand{\AA}{\mathcal{A}}
\newcommand{\BB}{\mathcal{B}}
\newcommand{\CC}{\mathcal{C}}
\newcommand{\DD}{\mathcal{D}}
\newcommand{\EE}{\mathcal{E}}
\newcommand{\FF}{\mathcal{F}}
\newcommand{\GG}{\mathbb{G}}
\newcommand{\HH}{\mathbb{H}}
\newcommand{\calH}{\mathcal{H}}
\newcommand{\II}{\mathcal{I}}
\newcommand{\JJ}{\mathcal{J}}
\newcommand{\KK}{\mathcal{K}}
\newcommand{\LL}{\mathcal{L}}
\newcommand{\MM}{\mathcal{M}}
\newcommand{\NN}{\mathcal{N}}
\newcommand{\OO}{\mathrm{O}}
\newcommand{\PP}{\mathcal{P}}
\newcommand{\QQ}{\mathcal{Q}}
\newcommand{\RR}{\mathcal{R}}
\renewcommand{\SS}{\mathcal{S}}
\newcommand{\TT}{\mathcal{T}}
\newcommand{\UU}{\mathcal{U}}
\newcommand{\VV}{\mathcal{V}}
\newcommand{\WW}{\mathcal{W}}
\newcommand{\XX}{\mathcal{X}}
\newcommand{\YY}{\mathcal{Y}}
\newcommand{\ZZ}{\mathcal{Z}}
\renewcommand{\d}{\textrm{d}}
% Greek letters
\newcommand{\ep}{\varepsilon}
\newcommand{\ph}{\varphi}
\newcommand{\de}{\delta}
\renewcommand{\a}{\alpha}
\renewcommand{\b}{\beta}
% Fraktur
\newcommand{\mm}{\mathfrak{m}}
\renewcommand{\aa}{\mathfrak{a}}
\newcommand{\bb}{\mathfrak{b}}
\newcommand{\pp}{\mathfrak{p}}
\newcommand{\qq}{\mathfrak{q}}
% Operators
\DeclareMathOperator{\Div}{div}
\DeclareMathOperator{\Gal}{Gal}
\DeclareMathOperator{\vol}{Vol}
\DeclareMathOperator{\Hom}{Hom}
\DeclareMathOperator{\End}{End}
\DeclareMathOperator{\Ext}{Ext}
\DeclareMathOperator{\Tor}{Tor}
\DeclareMathOperator{\tr}{tr}
\DeclareMathOperator{\rk}{rk}
\DeclareMathOperator{\curl}{curl}
\DeclareMathOperator{\mesh}{mesh}
\DeclareMathOperator{\im}{im}
\DeclareMathOperator{\coker}{coker}
\DeclareMathOperator{\width}{width}
\DeclareMathOperator{\diam}{diam}
\DeclareMathOperator{\maps}{Maps}
\DeclareMathOperator{\Frac}{Frac}
\DeclareMathOperator{\Sym}{Sym}
\DeclareMathOperator{\sgn}{sgn}
\DeclareMathOperator{\alt}{Alt}
\DeclareMathOperator{\supp}{supp}
\DeclareMathOperator{\Span}{span}
\DeclareMathOperator{\Var}{Var}
\DeclareMathOperator{\Spec}{Spec}

\newcommand{\nor}{\unlhd}
\DeclareMathOperator{\aut}{Aut}
\DeclareMathOperator{\orb}{Orb}
\DeclareMathOperator{\GL}{GL}
\DeclareMathOperator{\SL}{SL}
\DeclareMathOperator{\SO}{SO}
\DeclareMathOperator{\PGL}{PGL}
\DeclareMathOperator{\PSL}{PSL}
\DeclareMathOperator{\stab}{Stab}
\DeclareMathOperator{\fix}{Fix}
\DeclareMathOperator{\Th}{Th}
\DeclareMathOperator{\Ind}{Ind}
\DeclareMathOperator{\Res}{Res}
\DeclareMathOperator{\Ann}{Ann}
\DeclareMathOperator{\rad}{rad}
\DeclareMathOperator{\len}{len}
\DeclareMathOperator{\ord}{ord}

% \DeclareMathOperator{\arg}{arg}

%% misc
\newcommand{\<}{\langle}
\renewcommand{\>}{\rangle}
\renewcommand{\^}{\wedge}
\renewcommand{\v}{\vee}
\def\Xint#1{\mathchoice
	{\XXint\displaystyle\textstyle{#1}}%
	{\XXint\textstyle\scriptstyle{#1}}%
	{\XXint\scriptstyle\scriptscriptstyle{#1}}%
	{\XXint\scriptscriptstyle\scriptscriptstyle{#1}}%
	\!\int}
\def\XXint#1#2#3{{\setbox0=\hbox{$#1{#2#3}{\int}$ }
		\vcenter{\hbox{$#2#3$ }}\kern-.6\wd0}}
\def\ddashint{\Xint=}
\def\dashint{\Xint-}
%% arrows
\newcommand{\xhra}{\xhookrightarrow}
\newcommand{\xra}{\xrightarrow}
\newcommand{\ra}{\rightarrow}
\newcommand{\rra}{\rightrightarrows}
\newcommand{\lra}{\longrightarrow}
\newcommand{\Ra}{\Rightarrow}
\newcommand{\lRa}{\Longrightarrow}
\newcommand{\lrsa}{\leftrightsquiqarrow}
\newcommand{\ba}{\leftrightarrow}
%% lists
\newcommand{\be}{\begin{enumerate}[(i)]}
	\newcommand{\ee}{\end{enumerate}}
%% integration stuff
\newcommand{\calR}{\mathcal{R}}
\newcommand{\rint}{\calR\!\int}
\newcommand{\calL}{\mathcal{L}}
\newcommand{\lowerint}{\mbox{\b{$\int$}}}
\newcommand{\upperint}{{\textstyle\bar{\int}}}
%% end of proof
\def\endproof{{\hfill $\Box$}}
%% matrix shorthand

\title{Math 325 HW 3}
\author{Jalen Chrysos}

\begin{document}
	\maketitle	
	
	
	\textbf{Problem 1}: Prove that $P^{\sign_{\lambda}}$ is a rank 1 free $P^{S_{\lambda}}$-module with generator $\Delta_{\lambda}$.
	\begin{proof}
		$\Delta_{\lambda}\in P^{\sign_{\lambda}}$ because each $\sigma \in S_{\lambda}$ is the product of permutations $\sigma_m$ on each index set $I_m$, each of which acts on $\Delta(I_m)$ as multiplication by $\sign(\sigma_m)$ (because the determinant is alternating), so their product also acts on $\Delta_{\lambda}$ as multiplication by $\sign(\sigma_1)\cdot \sign(\sigma_2)\cdots \sign(\sigma_k) = \sign(\sigma)$.\\
		
		Moreover, I claim that \textit{every} polynomial in $P^{\sign_{\lambda}}$ is a multiple of $\Delta_{\lambda}$. For any $p\in P^{\sign_{\lambda}}$, $(x_j-x_i)|p$ for all pairs $i<j \in I_m$, as swapping $x_i,x_j$ inverts $p$. To show this, express $p$ as a polynomial in $x_i,x_j$ with coefficients in $k[x_1,\dots,\hat{x_{i}},\dots, \hat{x_j},\dots,x_n]$, and note that the coefficients of terms $x_i^{e_1}x_j^{e_2}$ and $x_i^{e_2}x_j^{e_1}$ must add to 0 (since swapping $x_i,x_j$ inverts $p$). Thus, $p$ is a linear combination of terms $(x_i^{e_1}x_j^{e_2}-x_i^{e_2}x_j^{e_1})$, each of which is a multiple of $(x_i-x_j)$. Now, since $p$ must be a multiple of $(x_i-x_j)$ for all such $i,j$, and these polynomials are all irreducible, their product $\Delta_{\lambda}$ must also divide $p$, which was the desired result.
	\end{proof}
	
	\newpage 
	
	\textbf{Problem 2}: 
	\begin{enumerate}[(a)]
		\item Prove that all polynomials $f\in V(\lambda)$ are $S_n$-harmonic.
		\item (Optional) Show that there is no nonzero homogeneous $S_n$-harmonic polynomial $f$ of degree greater than $\deg(\Delta_n)$ (which is ${n \choose 2}$).
	\end{enumerate}
	\begin{proof}
		(a): To be $S_n$-harmonic means that for any \textit{symmetric} polynomial $p$, $\<p(\partial),f\>=0$.
		
%		To show this for $f\in V(\lambda)$, it suffices to check only $\Delta_{\lambda}$, since every basis element of $V(\lambda)$ is $\sigma(\Delta_{\lambda})$ for some $\sigma\in S_n$ and 
%		$$
%		\<p(\partial),f\> = \<\sigma(p)(\partial),\sigma(f)\> = \<p(\partial),\sigma(f)\> 
%		$$
%		for symmetric $p$.\\
		
		For $p\in P_d^{S_n}$, $p$ induces a linear map $p(\partial): V(\lambda)\to P_{d_{\lambda}-d}$ which is an $S_n$-intertwining map because $p$ is symmetric. Thus, by Lemma 4.2.7, this map must be the constant zero map.
	\end{proof}
	
	\newpage
	\textbf{Problem 3}: Using Lemmas 4.2.7 and 4.2.9, deduce Corollary 4.2.11:
	\begin{enumerate}[(a)]
		\item The representation $V(\lambda)$ is irreducible.
		\item If $d_{\lambda} = d_{\mu}$ and $V(\lambda) \cong V(\mu)$ then $V(\lambda) = V(\mu)$ as subspaces of $P_{d_{\lambda}}=P_{d{\mu}}$.
		\item If $d_{\lambda} \neq d_{\mu}$ then $V(\lambda)\not\cong V(\mu)$.
	\end{enumerate}
	\begin{proof}
		(a): Every permutation is a unitary operator, so every $S_n$-representation is unitary and hence completely reducible. Thus $V(\lambda)$ is completely reducible. So we can use Lemma 4.2.9 to say that $V(\lambda)$ is irreducible iff $\dim_k(\End_{S_n}V(\lambda)) = 1$. And Lemma 4.2.7 showed that the only $S_n$-intertwining maps $V(\lambda)\to P_{d_{\lambda}}$ are scaling by some constant in $\C$, which implies in particular that maps in $\End_{S_n}V(\lambda)$ are also scaling by constants, and hence it is dimension 1. Thus, $V(\lambda)$ is irreducible. \\
		
		(b): By Lemma 4.2.7, if $d_{\mu}=d_{\lambda}$, any $S_n$-intertwiner $V(\lambda)\to V(\mu)$ must be scaling by a constant. Thus, if $V(\mu)\cong V(\lambda)$ then the intertwiner between them is a constant, so the subspaces $V(\mu)$ and $V(\lambda)$ are actually the same.\\
		
		(c): Likewise, if $d_{\mu}\neq d_{\lambda}$ then any $S_n$-intertwiner $V(\lambda)\to P_{d_{\mu}}$ must be trivial, so in particular there can be nonzero map $V(\lambda)\to V(\mu)$.
	\end{proof}
	
	\newpage
	\textbf{Problem 4}: Prove the generating function identity
	$$
	\sum_{d\geq 0} \dim(P_d^{S_n}) t^d = \frac{1}{(1-t)(1-t^2)\cdots(1-t^n)}.
	$$
	\begin{proof}
		$P_d^{S_n}$ has a basis consisting of the $S_n$-orbits of monomials in $P_d$, and these correspond to partitions of $d$ with at most $n$ parts (for the degrees of the $n$ variables $x_1,\dots,x_n$). To express the number of such partitions as a generating function, it is easier to count the transposed partitions, i.e. those which have parts of size no larger than $n$ (the count will be the same, naturally). To do this, take the infinite power series  
		$$
		Q(t) := (1+t+t^2+\cdots)(1+t^2+t^4+\cdots)\cdots (1+t^n+t^{2n}+\cdots).
		$$
		One can see by expanding the products that the $t^d$ term of $Q(t)$ is equal to the number of ways to partition $d$ into parts of size no larger than $n$. And $Q$ can be expressed using the geometric series identity as
		$$
		Q(t) = \bigg(\frac{1}{1-t}\bigg)\bigg(\frac{1}{1-t^2}\bigg)\cdots \bigg(\frac{1}{1-t^n}\bigg)
		$$
		as desired.
	\end{proof}
	
	\newpage 
	\textbf{Problem 5}: (Optional) Show that $P$ is a free $P^{S_n}$ module with basis 
	$$
	\{x_2^{m_2}x_3^{m_3}\cdots x_n^{m_n} | m_j \in [0,j-1] \forall j\}.
	$$
	
	\newpage 
	\textbf{Problem 6}: Let $g\in \GL_n(\C)$ be a diagonal matrix with diagonal entries $z_1,\dots,z_n$, and $\sigma\in S_d$ a permutation. Consider the linear operator $(\C^n)^{\otimes d}\to(\C^n)^{\otimes d}$ given by composing the action of $\sigma$ with that of $g$, i.e.
	$$
	v_1\otimes \cdots \otimes v_d \mapsto g(v_{\sigma^{-1}(1)})\otimes \cdots \otimes g(v_{\sigma^{-1}(d)}).
	$$
	Show that its trace is
	$$
	\prod_{j\geq 1} ((z_1)^j + \cdots + (z_n)^j)^{m_j}
	$$
	where $m_j$ is the number of cycles of length $j$ in the cycle type of $\sigma$.
	\begin{proof}
		For any basis element $e_{i_1}\otimes e_{i_2}\otimes \cdots \otimes e_{i_d}$, this operator $\Phi$ acts as
		$$
		e_{i_1}\otimes \cdots \otimes e_{i_d} \mapsto g(e_{i_{\sigma^{-1}(1)}}) \otimes \cdots \otimes g(e_{i_{\sigma^{-1}(d)}}) = (z_{i_{\sigma^{-1}(1)}}\cdots z_{i_{\sigma^{-1}(d)}}) (e_{i_{\sigma^{-1}(1)}}\otimes \cdots \otimes e_{i_{\sigma^{-1}(d)}}).
		$$
		To get the trace, we only need to consider these products for basis elements which are scaled by $\Phi$, i.e. those for which $i_t = i_{\sigma^{-1}(t)}$ for all $1\leq t \leq d$. That is, for each cycle $(t_1 \; t_2 \; t_3 \; \dots \; t_j)$ in $\sigma$, $$i_{t_1}=i_{t_2}=\dots=i_{t_j} \in \{1,\dots,n\}.$$
		In choosing such a basis element, then, there is one choice to be made for each cycle in $\sigma$. If $i$ is chosen as the index of a cycle, that contributes a multiplication by $z_i^j$ to the corresponding diagonal element (where $j$ is the length of the cycle). Let $\rm{cyc}(\sigma)$ denote the set of disjoint cycles whose product is $\sigma$. The sum of all possible diagonal elements, i.e. the trace, is the sum over all choice functions $f:\rm{cyc}(\sigma)\to \{1,\dots,n\}$ of
		$$
		\prod_{c\in \rm{cyc}(\sigma)} z_{f(c)}^{|c|}
		$$
		which is the expanded form of the product
		$$\prod_{c\in \rm{cyc}(\sigma)} (z_1^{|c|} + z_2^{|c|} + \dots + z_n^{|c|}) =
		\prod_{j\geq 1} ((z_1)^j + \cdots + (z_n)^j)^{m_j}
		$$
		as desired.
	\end{proof}
	
	\newpage 
	\textbf{Problem 7}: (Optional)
	
	\newpage 
	\textbf{Problem 8}: Let $A$ be an algebra over an ACF $k$. Let $V_1,\dots,V_r$ be pairwise non-isomorphic simple finite-dimensional $A$-modules, and 
	$$
	N := (V_1)^{\ell_1}\oplus \cdots \oplus (V_r)^{\ell_r}
	$$ 
	for some positive integers $\ell_j$. Use Schur's Lemma to prove:
	\begin{enumerate}[(a)]
		\item Any simple $A$-submodule $V\subseteq N$ is isomorphic to $V_i$ for some $i$, and in this case $V$ is contained in $(V_i)^{\ell_i}$.
		\item The algebra $\End_A(N)$ is isomorphic to $M_{\ell_1}(k)\oplus \cdots \oplus M_{\ell_r}$ (where $M_n$ is the matrix algebra).
	\end{enumerate}
	
	\begin{proof}
		(a): Let $V\subseteq N$ be a simple $A$-module. Then $V$ has a projection map onto each copy of $V_j$ for $1\leq j \leq r$, and by Schur's Lemma each of these maps must be either trivial or an isomorphism. They cannot all be trivial because they span the entire space of $N$, and each $V_j$ is pairwise non-isomorphic to the others, so $V$ is isomorphic with exactly one of them.\\
		
		(b): Every $A$-algebra homomorphism $(V_j)^{\ell_j}\to (V_i)^{\ell_i}$ where $i\neq j$ is trivial by Schur's Lemma, so any such endomorphism on $N$ preserves $(V_j)^{\ell_j}$, i.e. is a direct sum of matrices in $M_{\ell_j}(k)$ for each $j$. 
		
		And conversely, any linear map $V_j^{\ell_j}\to V_j^{\ell_j}$ is an $A$-algebra endomorphism, since the action of $A$ is scaling in $V_j$ (again by Schur's Lemma). Thus, these direct sums of matrices are \textit{exactly} the endomorphisms on $N$.
	\end{proof}
\end{document}
