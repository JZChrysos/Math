\documentclass{amsart}

%\documentclass{amsart}
\usepackage[utf8]{inputenc}
\usepackage{amsfonts}
\usepackage{amsmath}
\usepackage{amssymb}
\usepackage{amsthm}
\usepackage{asymptote}
\usepackage{mathtools}
\usepackage{hhline}
\usepackage{graphicx,enumerate}
\usepackage{hyperref}
\usepackage[a4paper, margin=1.2in]{geometry}
%\usepackage{tcolorbox}
\usepackage{tikz-cd}
\usepackage{ytableau}
%\tcbuselibrary{skins,breakable,xparse}
\allowdisplaybreaks
\newcounter{count}
\hypersetup{
	colorlinks=true,
	linkcolor=teal,
	filecolor=magenta,      
	urlcolor=olive,
	citecolor=teal,
	pdfpagemode=FullScreen,
}

%\definecolor{defcolor}{HTML}{478EFF}
%\definecolor{thmcolor}{HTML}{CC0058}
%\definecolor{excolor}{HTML}{F5B400}
%\definecolor{probcolor}{HTML}{DD4803}
%\definecolor{lemcolor}{HTML}{741FEA}
%\definecolor{scarlet}{HTML}{A81111}
%
%\newtheoremstyle{definitionStyle}% Custom style for definitions
%{0.5em}% Space above
%{0.5em}% Space below
%{}% Body font
%{}% Indent amount
%{\bfseries\color{defcolor}}% Theorem head font: bold and red
%{.\\}% Punctuation after theorem head
%{0.5em}% Space after theorem head
%{\thmname{#1}\thmnumber{ #2 (#3)}}% Theorem head spec
%
%\theoremstyle{definitionStyle}
%\newtheorem{df}{Definition}[section]
%
%\newtheoremstyle{theoremStyle}% Custom style for definitions
%{0.5em}% Space above
%{0.5em}% Space below
%{}% Body font
%{}% Indent amount
%{\bfseries\color{thmcolor}}% Theorem head font: bold and red
%{.\\}% Punctuation after theorem head
%{0.5em}% Space after theorem head
%{\thmname{#1}\thmnumber{ #2 (#3)}}% Theorem head spec
%
%\theoremstyle{theoremStyle}
%\newtheorem{thm}{Theorem}[section]
%
%\newtheoremstyle{lemmaStyle}% Custom style for definitions
%{0.5em}% Space above
%{0.5em}% Space below
%{}% Body font
%{}% Indent amount
%{\bfseries\color{lemcolor}}% Theorem head font: bold and red
%{.\\}% Punctuation after theorem head
%{0.5em}% Space after theorem head
%{\thmname{#1}\thmnumber{ #2 (#3)}}% Theorem head spec
%
%\theoremstyle{lemmaStyle}
%\newtheorem{lem}{Lemma}[section]
%\newtheorem{cor}{Corollary}[section]
%
%\newtheoremstyle{exampleStyle}% Custom style for definitions
%{0.5em}% Space above
%{0.5em}% Space below
%{}% Body font
%{}% Indent amount
%{\bfseries\color{excolor}}% Theorem head font: bold and red
%{.\\}% Punctuation after theorem head
%{0.5em}% Space after theorem head
%{\thmname{#1}\thmnumber{ #2 (#3)}}% Theorem head spec
%
%\theoremstyle{exampleStyle}
%\newtheorem{ex}{Example}[section]
%
%\newtheoremstyle{problemStyle}% Custom style for definitions
%{0.5em}% Space above
%{0.5em}% Space below
%{}% Body font
%{}% Indent amount
%{\bfseries\color{probcolor}}% Theorem head font: bold and red
%{.\\}% Punctuation after theorem head
%{0.5em}% Space after theorem head
%{\thmname{#1}\thmnumber{ #2#3}}% Theorem head spec
%
%\theoremstyle{problemStyle}
%\newtheorem{prob}{Problem}[section]

% For Fun
\newcommand{\club}{\color{teal} \clubsuit}
\newcommand{\heart}{\color{red} \heartsuit}
\renewcommand{\star}{\color{scarlet} \bigstar}
\newcommand{\spade}{\color{violet} \spadesuit}

% Symbols
\newcommand{\A}{\mathcal{A}}
\newcommand{\B}{\mathcal{B}}
\newcommand{\C}{\mathbb{C}}
\newcommand{\D}{\mathcal{D}}
\newcommand{\E}{\mathbb{E}}
\newcommand{\F}{\mathbb{F}}
\newcommand{\G}{\mathcal{G}}
% \renewcommand{\H}{\mathcal{H}} Erdos o
\newcommand{\I}{\mathcal{I}}
\newcommand{\J}{\mathcal{J}}
\newcommand{\K}{\mathcal{K}}
% \renewcommand{\L}{\mathcal{L}}
\newcommand{\M}{\mathcal{M}}
\newcommand{\N}{\mathbb{N}}
\renewcommand{\O}{\mathcal{O}}
\renewcommand{\P}{\mathbb{P}}
\newcommand{\Q}{\mathbb{Q}}
\newcommand{\R}{\mathbb{R}}
\renewcommand{\S}{\mathbb{S}}
\newcommand{\T}{\mathbb{T}}
\newcommand{\U}{\mathcal{U}}
\newcommand{\V}{\mathcal{V}}
\newcommand{\W}{\mathcal{W}}
\newcommand{\X}{\mathcal{X}}
\newcommand{\Y}{\mathcal{Y}}
\newcommand{\Z}{\mathbb{Z}}

\renewcommand{\AA}{\mathcal{A}}
\newcommand{\BB}{\mathcal{B}}
\newcommand{\CC}{\mathcal{C}}
\newcommand{\DD}{\mathcal{D}}
\newcommand{\EE}{\mathcal{E}}
\newcommand{\FF}{\mathcal{F}}
\newcommand{\GG}{\mathbb{G}}
\newcommand{\HH}{\mathbb{H}}
\newcommand{\calH}{\mathcal{H}}
\newcommand{\II}{\mathcal{I}}
\newcommand{\JJ}{\mathcal{J}}
\newcommand{\KK}{\mathcal{K}}
\newcommand{\LL}{\mathcal{L}}
\newcommand{\MM}{\mathcal{M}}
\newcommand{\NN}{\mathcal{N}}
\newcommand{\OO}{\mathrm{O}}
\newcommand{\PP}{\mathcal{P}}
\newcommand{\QQ}{\mathcal{Q}}
\newcommand{\RR}{\mathcal{R}}
\renewcommand{\SS}{\mathcal{S}}
\newcommand{\TT}{\mathcal{T}}
\newcommand{\UU}{\mathcal{U}}
\newcommand{\VV}{\mathcal{V}}
\newcommand{\WW}{\mathcal{W}}
\newcommand{\XX}{\mathcal{X}}
\newcommand{\YY}{\mathcal{Y}}
\newcommand{\ZZ}{\mathcal{Z}}
\renewcommand{\d}{\textrm{d}}
% Greek letters
\newcommand{\ep}{\varepsilon}
\newcommand{\ph}{\varphi}
\newcommand{\de}{\delta}
\renewcommand{\a}{\alpha}
\renewcommand{\b}{\beta}
% Fraktur
\newcommand{\mm}{\mathfrak{m}}
\renewcommand{\aa}{\mathfrak{a}}
\newcommand{\bb}{\mathfrak{b}}
\newcommand{\pp}{\mathfrak{p}}
\newcommand{\qq}{\mathfrak{q}}
% Operators
\DeclareMathOperator{\Div}{div}
\DeclareMathOperator{\Gal}{Gal}
\DeclareMathOperator{\vol}{Vol}
\DeclareMathOperator{\Hom}{Hom}
\DeclareMathOperator{\End}{End}
\DeclareMathOperator{\Ext}{Ext}
\DeclareMathOperator{\Tor}{Tor}
\DeclareMathOperator{\tr}{tr}
\DeclareMathOperator{\rk}{rk}
\DeclareMathOperator{\curl}{curl}
\DeclareMathOperator{\mesh}{mesh}
\DeclareMathOperator{\im}{im}
\DeclareMathOperator{\coker}{coker}
\DeclareMathOperator{\width}{width}
\DeclareMathOperator{\diam}{diam}
\DeclareMathOperator{\maps}{Maps}
\DeclareMathOperator{\Frac}{Frac}
\DeclareMathOperator{\Sym}{Sym}
\DeclareMathOperator{\sgn}{sgn}
\DeclareMathOperator{\alt}{Alt}
\DeclareMathOperator{\supp}{supp}
\DeclareMathOperator{\Span}{span}
\DeclareMathOperator{\Var}{Var}
\DeclareMathOperator{\Spec}{Spec}

\newcommand{\nor}{\unlhd}
\DeclareMathOperator{\aut}{Aut}
\DeclareMathOperator{\orb}{Orb}
\DeclareMathOperator{\GL}{GL}
\DeclareMathOperator{\SL}{SL}
\DeclareMathOperator{\SO}{SO}
\DeclareMathOperator{\PGL}{PGL}
\DeclareMathOperator{\PSL}{PSL}
\DeclareMathOperator{\stab}{Stab}
\DeclareMathOperator{\fix}{Fix}
\DeclareMathOperator{\Th}{Th}
\DeclareMathOperator{\Ind}{Ind}
\DeclareMathOperator{\Res}{Res}
\DeclareMathOperator{\Ann}{Ann}
\DeclareMathOperator{\rad}{rad}
\DeclareMathOperator{\len}{len}
\DeclareMathOperator{\ord}{ord}

% \DeclareMathOperator{\arg}{arg}

%% misc
\newcommand{\<}{\langle}
\renewcommand{\>}{\rangle}
\renewcommand{\^}{\wedge}
\renewcommand{\v}{\vee}
\def\Xint#1{\mathchoice
	{\XXint\displaystyle\textstyle{#1}}%
	{\XXint\textstyle\scriptstyle{#1}}%
	{\XXint\scriptstyle\scriptscriptstyle{#1}}%
	{\XXint\scriptscriptstyle\scriptscriptstyle{#1}}%
	\!\int}
\def\XXint#1#2#3{{\setbox0=\hbox{$#1{#2#3}{\int}$ }
		\vcenter{\hbox{$#2#3$ }}\kern-.6\wd0}}
\def\ddashint{\Xint=}
\def\dashint{\Xint-}
%% arrows
\newcommand{\xhra}{\xhookrightarrow}
\newcommand{\xra}{\xrightarrow}
\newcommand{\ra}{\rightarrow}
\newcommand{\rra}{\rightrightarrows}
\newcommand{\lra}{\longrightarrow}
\newcommand{\Ra}{\Rightarrow}
\newcommand{\lRa}{\Longrightarrow}
\newcommand{\lrsa}{\leftrightsquiqarrow}
\newcommand{\ba}{\leftrightarrow}
%% lists
\newcommand{\be}{\begin{enumerate}[(i)]}
	\newcommand{\ee}{\end{enumerate}}
%% integration stuff
\newcommand{\calR}{\mathcal{R}}
\newcommand{\rint}{\calR\!\int}
\newcommand{\calL}{\mathcal{L}}
\newcommand{\lowerint}{\mbox{\b{$\int$}}}
\newcommand{\upperint}{{\textstyle\bar{\int}}}
%% end of proof
\def\endproof{{\hfill $\Box$}}
%% matrix shorthand

\title{Math 325 Midterm}
\author{Jalen Chrysos}

\begin{document}
	\maketitle
	
	\textbf{Problem 1}:
	
	\begin{proof}
		(i): Suppose for the sake of contradiction that there is some $W\subsetneq V$ stable under $\SL_6(\C)$. Then I claim that it is also stable under the action of $\GL_6(\C)$, contradicting the irreducibility of $V$. Indeed, every $g\in \GL_6(\C)$ is of the form $a g'$ where $g'\in \SL_6(\C)$ and $a=\det(g)^{1/6}\cdot \id \in \GL_6(\C)$. So 
		$$
		g(W) = ag'(W) = aW = W.
		$$
	\end{proof}
	
	\newpage 
	
	\textbf{Problem 2}:
	
	\begin{proof}
		A conjugacy class of $\GL_n(\C)$ is determined by its eigenspaces and eigenvalues. In this set $x^2=I$ so all eigenvalues must be $\pm 1$. The eigenspaces themselves are indistinguishable, as they can be permuted by change of basis. Thus, the conjugacy class is determined uniquely by the number $N(d,\lambda)$ of eigenspaces of dimension $d$ and eigenvalue $\lambda$ for each $d\in \{1,2,\dots,n\}$ and $\lambda \in \{-1,1\}$. The dimensions must all add to $n$.\\
		
		To calculate this, we can go by partitions of $n$. For each partition of $n$, let $m_j$ be the number of pieces of size $j$. Then choosing a conjugacy class of $\GL_n(\C)$ is equivalent to choosing, for each $j$, the number of the $m_j$ that have eigenvalue 1 and the number that have $-1$ (of which there are $m_j+1$ ways). Thus, we get
		$$
		\sum_{\lambda \in P_n} \prod_{j=1}^n m_j+1.
		$$
		The first couple of values are $2,5,10,20,36$ for $n=1,2,3,4,5$. I'm sure there's a generating function for this as well.
	\end{proof}
	\newpage 
	
	\textbf{Problem 3}:
	\begin{proof}
		If $(x,y)\in (\C^2)^G$ then
		$$
		(x,y) = (x+ay,by)
		$$
		for all $a,b$ with $b\neq 0$, which in the case $a\neq 0$ implies $y=0$. Thus, this subspace is 1-dimensional and spanned by $(1,0)$.\\
		
		To show that there is no $G$-stable complement, note that the $G$-orbit of $(0,1)$ is $(a,b)$ for all $b\neq 0$, i.e. the entire space except for the $G$-fixed subspace. Thus, the smallest $G$-stable subspace outside of $(\C^2)^G$ must include all of $\C^2$.\\
		
		If $\ph:\C^2\to \C^2$ is a $G$-intertwiner, then we can use the fact that it commutes with the action of $g$ in particular on the input $(0,1)$ to get
		$$
		\ph_1(a,b) = \ph_1(0,1) + a\ph_2(0,1), \;\;\; \ph_2(a,b) = b\ph_2(0,1)
		$$ 
		for all $a,b\in \C$ with $b\neq 0$. The second of these equations shows that $\ph_2(a,b)$ depends only on $b$ and scales it by $c:=\ph_2(0,1)$. Then the first equation shows that $\ph_1(a,b)$ depends only on $a$. In the case $(a,b)=(0,1)$, it gives
		$$
		\ph_1(0,1) = \ph_1(0,1) + ac
		$$
		so $\ph_1(0,1)=0$, and thus in general $\ph_1(a,b)=ac$, so $\ph$ scales both coefficients by $c$. It remains to show that this is also true when $b=0$, but this follows from linearity of $\ph$, as
		$$
		\ph(a,0) = \ph(a/2,b) + \ph(a/2,-b) = (ac/2,bc) + (ac/2,-bc) = (ac,0). 
		$$
	\end{proof}
	\newpage 
	
	\textbf{Problem 4}:
	\begin{proof}
		Recalling the definition of Specht modules, we have
		$$V(\lambda) = \C\<x_1-x_2,x_2-x_3,\dots,x_{n-1}-x_n\>, \;\;\; V(\lambda^t) = \C\<\Delta_{\hat{1}},\Delta_{\hat{2}},\dots,\Delta_{\hat{n}}\>$$
		where by $\Delta_{\hat{m}}$ I mean
		$$
		\Delta_{\hat{m}} := \prod_{i<j \in [n]\setminus \{m\}} (x_j-x_i) = \Delta_n \cdot \prod_{i\neq m} (x_m-x_i)^{-1} \cdot (-1)^{n-m}.
		$$
		The map $F:V(\lambda)\to V(\lambda^t)$ can be defined on the basis of $V(\lambda)$ by
		$$
		F: (x_{m}-x_{m+1}) \mapsto \Delta_{\hat{m}} - \Delta_{\hat{m+1}}
		$$
		Now, if $s\in S_n$ acts on $V(\lambda^t)$, consider how $s$ affects the sign of $\Delta_{\hat{m}}$. It will flip the sign for each two indices (neither of which is $m$) whose order is inverted by $s$. The sign of $s$ is the number of \textit{all} pairs of indices whose order is inverted by $s$. Thus, $s(\Delta_{\hat{m}}) = \sign(s)\cdot \Delta_{\hat{m}} \cdot (-1)^{R_m}$, where $R_m$ is the number of pairs which include $m$ that are inverted by $s$.
		$$
		s(\Delta_{\hat{m}} - \Delta_{\hat{m+1}}) = \sign(s) \cdot ((-1)^{R_m}\Delta_{\hat{s(m)}} - (-1)^{R_{m+1}}\Delta_{\hat{s(m+1)}})
		$$
		and 
		$$
		F(x_{s(m)} - x_{s(m+1)}) = \Delta_{\hat{s(m)}}-\Delta_{\hat{s(m+1)}}.
		$$
		Some combinatorics about $R_m$ has to be done to show that this works.
	\end{proof}
	\newpage 
	
	\textbf{Problem 5}:
	\begin{proof}
		For $\lambda$ to be an eigenvalue of $M_I$ means that $$fg \equiv \lambda g \pmod I \;\; \iff \;\; (f-\lambda)g \in I $$ for some $g\not\in I$, and similarly for $\lambda$ to be an eigenvalue of $M_{\sqrt{I}}$ it is equivalent that $(f-\lambda)h \in \sqrt{I}$ for some $h\not\in \sqrt{I}$.\\
		
%		We can reduce to the case where $I$ is a prime ideal, since eigenvalues of $M_{IJ}$ are the eigenvalues of $M_I$ or $M_J$ and similarly for $\sqrt{I}\sqrt{J}$.\\
%		
		If $f-\lambda \in \sqrt{I}$ then immediately $\lambda$ must be an eigenvalue in both $\sqrt{I}$ and $I$ (if $(f-\lambda)^n\in I$, take $h=(f-\lambda)^{n-1}$). And likewise if $f-\lambda \in I$ then $\lambda$ is an eigenvalue of $M_{\sqrt{I}}$ and $M_I$, taking $h=1$. So assume neither of these is the case.\\
		
		Assume $\lambda$ is not an eigenvalue of $M_{\sqrt{I}}$. Since $\C[x,y]/I$ is finite-dimensional, this implies that there are $\C$-linear relations between $1,x,x^2,x^3,\dots$, so $I$ contains a polynomial $p(x)$ and similarly $I$ contains a polynomial $q(y)$. $V(I)$ is thus finite; for any $(a,b)\in V(I)$, $a$ is among the finitely-many roots of $p$ and $b$ is a root of $q$, so there are only finitely-many such pairs. Now by the Nullstellensatz, $(f-\lambda)h\in \sqrt{I}$ for some $h\not\in \sqrt{I}$ is equivalent to $V(f-\lambda)\cup V(h)\supseteq V(I)$ and $V(h)\not\supset V(I)$. So if such an $h$ exists then $f=\lambda$ at some point in $V(I)$, and conversely because $V(I)$ is finite such an $h$ always exists if $f=\lambda$ somewhere in $V(I)$, since it is possible to construct a polynomial passing through any finite collection of points and avoiding a given point (it is easy to do this with a product of lines). Assuming $\lambda$ is not an eigenvalue of $M_{\sqrt{I}}$, then $f-\lambda \neq 0$ on $V(I)$, so for any $r\in I$, one can take a linear combination $$a(f-\lambda)+br=1$$
		with $a,b\in \C[x,y]$, by the Nullstellensatz. But now if $\lambda$ is an eigenvalue of $M_I$, so $(f-\lambda)g\in I$ for $g\not\in I$, then take $r=(f-\lambda)g$ to get
		$$
		a(f-\lambda) + b(f-\lambda)g = (a+bg)(f-\lambda) = 1
		$$
		which implies $f-\lambda$ and $a+bg$ are nonzero constant polynomials, and thus that either $g=0$, contradicting that $g\not\in I$, or $(f-\lambda)g$ is a nonzero constant in $I$, so $I=\C[x,y]$, a contradiction. That is, if $\lambda$ is not an eigenvalue of $M_{\sqrt{I}}$ then $\lambda$ cannot be an eigenvalue of $M_I$.
		\\

		Conversely, if $\lambda$ is an eigenvalue of $M_{\sqrt{I}}$, then let $(f-\lambda)h\in \sqrt{I}$ with $h\not\in \sqrt{I}$. Then $(f-\lambda)^nh^n\in I$ for some $n$, so 
		$$
		(f-\lambda) \cdot (f-\lambda)^{n-1}h^n \in I
		$$
		which shows that $(f-\lambda)^{n-1}h^n$ is an eigenvector for $M_I$ with eigenvalue $\lambda$ unless it is in $I$. If it is in $I$, then $(f-\lambda)^{n-2}h^n$ is an eigenvector unless \textit{it} is in $I$, and so on. If they all fail then $h^n\in I$ but we assumed $h\not\in \sqrt{I}$, so $(f-\lambda)^{n-j}h^n$ must be a nontrivial eigenvector with eigenvalue $\lambda$ for some $j$.
%	Let $A=V(I)=V(\sqrt{I})$. By the Nullstellensatz, $(f-\lambda)g\in \sqrt{I}$ is equivalent to $$V(f-\lambda)\cup V(g)\supseteq A.$$
%	That is, $g$ vanishes on $A - \{f=\lambda\}$. 
	\end{proof}
	
	

\end{document}