\documentclass{amsart}

%\documentclass{amsart}
\usepackage[utf8]{inputenc}
\usepackage{amsfonts}
\usepackage{amsmath}
\usepackage{amssymb}
\usepackage{amsthm}
\usepackage{asymptote}
\usepackage{mathtools}
\usepackage{hhline}
\usepackage{graphicx,enumerate}
\usepackage{hyperref}
\usepackage[a4paper, margin=1.2in]{geometry}
%\usepackage{tcolorbox}
\usepackage{tikz-cd}
\usepackage{ytableau}
%\tcbuselibrary{skins,breakable,xparse}
\allowdisplaybreaks
\newcounter{count}
\hypersetup{
	colorlinks=true,
	linkcolor=teal,
	filecolor=magenta,      
	urlcolor=olive,
	citecolor=teal,
	pdfpagemode=FullScreen,
}

%\definecolor{defcolor}{HTML}{478EFF}
%\definecolor{thmcolor}{HTML}{CC0058}
%\definecolor{excolor}{HTML}{F5B400}
%\definecolor{probcolor}{HTML}{DD4803}
%\definecolor{lemcolor}{HTML}{741FEA}
%\definecolor{scarlet}{HTML}{A81111}
%
%\newtheoremstyle{definitionStyle}% Custom style for definitions
%{0.5em}% Space above
%{0.5em}% Space below
%{}% Body font
%{}% Indent amount
%{\bfseries\color{defcolor}}% Theorem head font: bold and red
%{.\\}% Punctuation after theorem head
%{0.5em}% Space after theorem head
%{\thmname{#1}\thmnumber{ #2 (#3)}}% Theorem head spec
%
%\theoremstyle{definitionStyle}
%\newtheorem{df}{Definition}[section]
%
%\newtheoremstyle{theoremStyle}% Custom style for definitions
%{0.5em}% Space above
%{0.5em}% Space below
%{}% Body font
%{}% Indent amount
%{\bfseries\color{thmcolor}}% Theorem head font: bold and red
%{.\\}% Punctuation after theorem head
%{0.5em}% Space after theorem head
%{\thmname{#1}\thmnumber{ #2 (#3)}}% Theorem head spec
%
%\theoremstyle{theoremStyle}
%\newtheorem{thm}{Theorem}[section]
%
%\newtheoremstyle{lemmaStyle}% Custom style for definitions
%{0.5em}% Space above
%{0.5em}% Space below
%{}% Body font
%{}% Indent amount
%{\bfseries\color{lemcolor}}% Theorem head font: bold and red
%{.\\}% Punctuation after theorem head
%{0.5em}% Space after theorem head
%{\thmname{#1}\thmnumber{ #2 (#3)}}% Theorem head spec
%
%\theoremstyle{lemmaStyle}
%\newtheorem{lem}{Lemma}[section]
%\newtheorem{cor}{Corollary}[section]
%
%\newtheoremstyle{exampleStyle}% Custom style for definitions
%{0.5em}% Space above
%{0.5em}% Space below
%{}% Body font
%{}% Indent amount
%{\bfseries\color{excolor}}% Theorem head font: bold and red
%{.\\}% Punctuation after theorem head
%{0.5em}% Space after theorem head
%{\thmname{#1}\thmnumber{ #2 (#3)}}% Theorem head spec
%
%\theoremstyle{exampleStyle}
%\newtheorem{ex}{Example}[section]
%
%\newtheoremstyle{problemStyle}% Custom style for definitions
%{0.5em}% Space above
%{0.5em}% Space below
%{}% Body font
%{}% Indent amount
%{\bfseries\color{probcolor}}% Theorem head font: bold and red
%{.\\}% Punctuation after theorem head
%{0.5em}% Space after theorem head
%{\thmname{#1}\thmnumber{ #2#3}}% Theorem head spec
%
%\theoremstyle{problemStyle}
%\newtheorem{prob}{Problem}[section]

% For Fun
\newcommand{\club}{\color{teal} \clubsuit}
\newcommand{\heart}{\color{red} \heartsuit}
\renewcommand{\star}{\color{scarlet} \bigstar}
\newcommand{\spade}{\color{violet} \spadesuit}

% Symbols
\newcommand{\A}{\mathcal{A}}
\newcommand{\B}{\mathcal{B}}
\newcommand{\C}{\mathbb{C}}
\newcommand{\D}{\mathcal{D}}
\newcommand{\E}{\mathbb{E}}
\newcommand{\F}{\mathbb{F}}
\newcommand{\G}{\mathcal{G}}
% \renewcommand{\H}{\mathcal{H}} Erdos o
\newcommand{\I}{\mathcal{I}}
\newcommand{\J}{\mathcal{J}}
\newcommand{\K}{\mathcal{K}}
% \renewcommand{\L}{\mathcal{L}}
\newcommand{\M}{\mathcal{M}}
\newcommand{\N}{\mathbb{N}}
\renewcommand{\O}{\mathcal{O}}
\renewcommand{\P}{\mathbb{P}}
\newcommand{\Q}{\mathbb{Q}}
\newcommand{\R}{\mathbb{R}}
\renewcommand{\S}{\mathbb{S}}
\newcommand{\T}{\mathbb{T}}
\newcommand{\U}{\mathcal{U}}
\newcommand{\V}{\mathcal{V}}
\newcommand{\W}{\mathcal{W}}
\newcommand{\X}{\mathcal{X}}
\newcommand{\Y}{\mathcal{Y}}
\newcommand{\Z}{\mathbb{Z}}

\renewcommand{\AA}{\mathcal{A}}
\newcommand{\BB}{\mathcal{B}}
\newcommand{\CC}{\mathcal{C}}
\newcommand{\DD}{\mathcal{D}}
\newcommand{\EE}{\mathcal{E}}
\newcommand{\FF}{\mathcal{F}}
\newcommand{\GG}{\mathbb{G}}
\newcommand{\HH}{\mathbb{H}}
\newcommand{\calH}{\mathcal{H}}
\newcommand{\II}{\mathcal{I}}
\newcommand{\JJ}{\mathcal{J}}
\newcommand{\KK}{\mathcal{K}}
\newcommand{\LL}{\mathcal{L}}
\newcommand{\MM}{\mathcal{M}}
\newcommand{\NN}{\mathcal{N}}
\newcommand{\OO}{\mathrm{O}}
\newcommand{\PP}{\mathcal{P}}
\newcommand{\QQ}{\mathcal{Q}}
\newcommand{\RR}{\mathcal{R}}
\renewcommand{\SS}{\mathcal{S}}
\newcommand{\TT}{\mathcal{T}}
\newcommand{\UU}{\mathcal{U}}
\newcommand{\VV}{\mathcal{V}}
\newcommand{\WW}{\mathcal{W}}
\newcommand{\XX}{\mathcal{X}}
\newcommand{\YY}{\mathcal{Y}}
\newcommand{\ZZ}{\mathcal{Z}}
\renewcommand{\d}{\textrm{d}}
% Greek letters
\newcommand{\ep}{\varepsilon}
\newcommand{\ph}{\varphi}
\newcommand{\de}{\delta}
\renewcommand{\a}{\alpha}
\renewcommand{\b}{\beta}
% Fraktur
\newcommand{\mm}{\mathfrak{m}}
\renewcommand{\aa}{\mathfrak{a}}
\newcommand{\bb}{\mathfrak{b}}
\newcommand{\pp}{\mathfrak{p}}
\newcommand{\qq}{\mathfrak{q}}
% Operators
\DeclareMathOperator{\Div}{div}
\DeclareMathOperator{\Gal}{Gal}
\DeclareMathOperator{\vol}{Vol}
\DeclareMathOperator{\Hom}{Hom}
\DeclareMathOperator{\End}{End}
\DeclareMathOperator{\Ext}{Ext}
\DeclareMathOperator{\Tor}{Tor}
\DeclareMathOperator{\tr}{tr}
\DeclareMathOperator{\rk}{rk}
\DeclareMathOperator{\curl}{curl}
\DeclareMathOperator{\mesh}{mesh}
\DeclareMathOperator{\im}{im}
\DeclareMathOperator{\coker}{coker}
\DeclareMathOperator{\width}{width}
\DeclareMathOperator{\diam}{diam}
\DeclareMathOperator{\maps}{Maps}
\DeclareMathOperator{\Frac}{Frac}
\DeclareMathOperator{\Sym}{Sym}
\DeclareMathOperator{\sgn}{sgn}
\DeclareMathOperator{\alt}{Alt}
\DeclareMathOperator{\supp}{supp}
\DeclareMathOperator{\Span}{span}
\DeclareMathOperator{\Var}{Var}
\DeclareMathOperator{\Spec}{Spec}

\newcommand{\nor}{\unlhd}
\DeclareMathOperator{\aut}{Aut}
\DeclareMathOperator{\orb}{Orb}
\DeclareMathOperator{\GL}{GL}
\DeclareMathOperator{\SL}{SL}
\DeclareMathOperator{\SO}{SO}
\DeclareMathOperator{\PGL}{PGL}
\DeclareMathOperator{\PSL}{PSL}
\DeclareMathOperator{\stab}{Stab}
\DeclareMathOperator{\fix}{Fix}
\DeclareMathOperator{\Th}{Th}
\DeclareMathOperator{\Ind}{Ind}
\DeclareMathOperator{\Res}{Res}
\DeclareMathOperator{\Ann}{Ann}
\DeclareMathOperator{\rad}{rad}
\DeclareMathOperator{\len}{len}
\DeclareMathOperator{\ord}{ord}

% \DeclareMathOperator{\arg}{arg}

%% misc
\newcommand{\<}{\langle}
\renewcommand{\>}{\rangle}
\renewcommand{\^}{\wedge}
\renewcommand{\v}{\vee}
\def\Xint#1{\mathchoice
	{\XXint\displaystyle\textstyle{#1}}%
	{\XXint\textstyle\scriptstyle{#1}}%
	{\XXint\scriptstyle\scriptscriptstyle{#1}}%
	{\XXint\scriptscriptstyle\scriptscriptstyle{#1}}%
	\!\int}
\def\XXint#1#2#3{{\setbox0=\hbox{$#1{#2#3}{\int}$ }
		\vcenter{\hbox{$#2#3$ }}\kern-.6\wd0}}
\def\ddashint{\Xint=}
\def\dashint{\Xint-}
%% arrows
\newcommand{\xhra}{\xhookrightarrow}
\newcommand{\xra}{\xrightarrow}
\newcommand{\ra}{\rightarrow}
\newcommand{\rra}{\rightrightarrows}
\newcommand{\lra}{\longrightarrow}
\newcommand{\Ra}{\Rightarrow}
\newcommand{\lRa}{\Longrightarrow}
\newcommand{\lrsa}{\leftrightsquiqarrow}
\newcommand{\ba}{\leftrightarrow}
%% lists
\newcommand{\be}{\begin{enumerate}[(i)]}
	\newcommand{\ee}{\end{enumerate}}
%% integration stuff
\newcommand{\calR}{\mathcal{R}}
\newcommand{\rint}{\calR\!\int}
\newcommand{\calL}{\mathcal{L}}
\newcommand{\lowerint}{\mbox{\b{$\int$}}}
\newcommand{\upperint}{{\textstyle\bar{\int}}}
%% end of proof
\def\endproof{{\hfill $\Box$}}
%% matrix shorthand

\newcommand{\Ad}{Ad}

\title{Math 325 HW 7}
\author{Jalen Chrysos}
\begin{document}
	\maketitle
	\textbf{Problem 1}: Let $\U(e,h,f)$ be an associative $\C$-algebra with generators $e,h,f$ with relations
	$$
	he -eh = 2e, \;\; hf -fh = -2f, \;\; ef - fe = h.
	$$
	Let $V$ be a $\U(e,h,f)$-module and $v\in V$ a nonzero element such that $h(v)=\lambda v$ where $\lambda \in \C$.
	\begin{enumerate}[(a)]
		\item Find an explicit formula for $h(f^i(v))$ as a function of $\lambda$.
		\item Assume that $e(v)=0$. Find explicit formulas for $e(f^i(v))$.
		\item Show that if $V$ is finite-dimensional then there is some nonzero $v\in V$ and nonnegative $d\in \Z$ such that $e(v)=0$ and $h(v)=dv$.
		\item Classify all simple finite-dimensional $\U(e,h,f)$-modules up to isomorphism.
	\end{enumerate}
	\begin{proof}
		(a): We know $hf-fh = -2f$, so 
		$$hf(v) = (fh - 2f)(v) = (\lambda - 2)v.$$
		I claim that in general $h(f^iv)=(\lambda - 2i)f^iv$. This can be seen inductively:
		\begin{align*}
		hf^iv &= hf(f^{i-1}v) \\
		&= (fh - 2f)(f^{i-1}v) \\
		&= f(hf^{i-1}v)-2f^i(v) \\
		&= f(\lambda - 2(i-1))f^{i-1}v - 2f^iv\\
		&= (\lambda - 2i)f^iv.
		\end{align*}
		Similarly $h(e^iv)=(\lambda + 2i)e^iv$.\\
		
		(b): Because $ef - fe = h$, 
		$$
		ef(v) = (fe + h)(v) = f(0) + \lambda v = \lambda v. 
		$$
		For general $i$, we have $ef^i(v)=\lambda f^iv + ((i-1)\lambda - i(i+1))f^{i-1}v$ by induction:
		\begin{align*}
			ef^i(v) &= ef(f^{i-1}v)\\
			&= (fe+h)(f^{i-1}v)\\
			&= f(ef^{i-1}v) + hf^{i-1}v\\
			&= (\lambda f^{i}v + ((i-2)\lambda - i(i-1))f^{i-1}v) + (\lambda - 2(i-1))f^{i-1}v\\
			&= \lambda f^iv + ((i-1)\lambda - i(i+1))f^{i-1}(v)\\
		\end{align*}
		
		(c): Let $v$ be any nonzero vector in $V$. By part (a), we see that $e^iv$ is an eigenvector of $h$ for all $i$, and they all have different eigenvalues, so $e^iv=0$ for all but finitely-many $i\in \N$. Thus let $i$ be maximal such that $e^iv\neq 0$. Then $e(e^iv)=0$ but $h(e^iv)=(\lambda + 2i)(e^iv)$. \textit{to do: why is $\lambda$ an integer}.\\
		
		(d): Let $V$ be a simple $\U(e,h,f)$-module. By (c), there is some $v$ with $e(v)=0$ and $h(v)=dv$. By (a) and (b), the subspace $V'=\<v,f(v),f^2(v),\dots\>$ is closed under multiplication by $h$ and $e$, and clearly $f$ as well, so $V'$ is a submodule. Since $V$ is simple and $V'\neq 0$ (as $v\neq 0$), $V'=V$.\\
		
		As before, $f^i(v)$ is an $h$-eigenvector for all $i$, and they all have different eigenvalues, which implies that the nonzero $f^i(v)$ are all distinct and linearly independent. So because $V$ is finite-dimensional, there must be some minimal $n$ for which $f^n(v)=0$. Then $V$ has basis 
		$$
		V := \<v,fv,f^2v,\dots,f^nv\>
		$$
		By (a) and (b), it's already determined how $h,e,f$ act on this basis. So there is exactly one simple $\U(e,f,h)$-module of dimension $n$ up to isomorphism, for each $n$.
%		on which $h$ acts as 
%		$$
%		h := \begin{bmatrix}
%			d & 0 & \cdots & 0\\
%			0 & d-2 & \cdots & 0 \\
%			\vdots & \vdots & \ddots & 0\\
%			0 & 0 & \cdots & d-2n
%			\end{bmatrix}
%		$$
		
	\end{proof}
	
	\newpage
	
	\textbf{Problem 2}: 
	\begin{enumerate}[(a)]
		\item Check that the matrices
		$$
		H = \begin{pmatrix}
			1 & 0\\ 0 & -1
		\end{pmatrix}, 
		\;\;
		E = \begin{pmatrix}
			0 & 1 \\
			0 & 0
		\end{pmatrix}, 
		\;\;
		F = \begin{pmatrix}
			0 & 0 \\
			1 & 0
		\end{pmatrix}
		$$
		form an $\R$-basis of the Lie algebra $\sl_2(\R)$ and that these matrices satisfy the relations in problem 1.
	\end{enumerate}
	\begin{proof}
		$\sl_2(\R)$ is exactly the matrices with trace 0, i.e. those of the form
		$$
		\begin{pmatrix}
			a & b \\
			c & -a
		\end{pmatrix} = aH + bE + cF
		$$
		for some $a,b,c\in \R$. Thus $E,F,H$ are a basis for $\sl_2(\R)$. 
		
		It is easy to check the three relations by just doing the matrix multiplications. In fact, we can check that $HE=E,EH=-E$, and and $HF=-F,FH=F$, and 
		$$
		EF = \begin{bmatrix}
			1 & 0 \\ 0 & 0
		\end{bmatrix}, \;\;\; FE = \begin{bmatrix}
		0 & 0 \\ 0 & -1
		\end{bmatrix}
		$$
		The relations follow.
	\end{proof}
	
	\newpage
	\textbf{Problem 3}: Let $1_{ij}\in M_n(\R)$ denote the matrices with 1 in the $(i,j)$ place and 0 elsewhere. 
	\begin{enumerate}[(a)]
		\item Check that for any $i<j$, the matrices $e=1_{ij},h=1_{ii}-1_{jj},f=1_{ji}$ satisfy the relations in problem 1.
		\item Let $\phi:M_n(\R)\to \End_{\C}(V)$ be a Lie algebra representation, where $M_n(\R)=\Lie(\GL_n(\R))$ is viewed as a Lie algebra wrt the commutator and $V$ is a finite-dimensional complex vector space. Prove (without using arguments from class) that there is a nonzero $v\in V$ and $\lambda_1,\dots,\lambda_n\in \C$ such that 
		\begin{itemize}
			\item $\phi(1_{ii})(v)=\lambda_i v$ for all $i$.
			\item $\phi(1_{ij})(v)=0$ for all $i<j$.
			\item $\lambda_i - \lambda_{i+1}$ is a nonnegative integer for all $1\leq i \leq n-1$.
		\end{itemize}
	\end{enumerate} 
	
	\begin{proof}
		(a) Just as in problem 2, we can check that $he=e,eh=-e,hf=-f,fh=f$. And $ef=1_{ii},fe=1_{jj}$, which shows $ef-fe=h$.\\
		
		(b): 
	\end{proof}
	
	\newpage
	The group $\GL_2(k)$ acts on $k^2$ in the usual way. For $k=\R$ this action induces an action on $C^{\infty}(\R^2)$ by $g:p\mapsto g^*p$. For any $p\in C^{\infty}(\R^2)$ and $\chi \in \Lie(\GL_2(\R))=M_2(\R)$, the Lie derivative $L_{\xi}(p)$ is a function on $\R^2$ defined by 
	$$
	(L_{\xi}(p))(x,y) = \frac{\d (e^{t\xi})^*(p)\cdot (x,y)}{\d t}\Big|_{t=0} = \frac{\d(p(e^{-t\xi}(x,y)))}{\d t}\Big|_{t=0}.
	$$
	
	\textbf{Problem 4}: For $\chi = E,H,F$ as in problem 2, find an explicit formula for $L_{\xi}(p)$ in terms of the partials of the function $p$ and check that the operators $L_H,L_E,L_F$ satisfy relations in problem 1.
	
	\begin{proof}
		Let 
		$$e^{-t\xi} = \begin{bmatrix}
			a(t) & b(t) \\
			c(t) & d(t)
		\end{bmatrix}.$$
		Note that
		$$
		\xi = \begin{bmatrix}
			a'(0) & b'(0)\\ c'(0) & d'(0)
		\end{bmatrix}.
		$$
		We can calculate $L_{\xi}(p)$ as
		\begin{align*}
			L_{\xi}(p) &= \frac{\d(p(e^{-t\xi}(x,y)))}{\d t}\Big|_{t=0}\\
			&= \frac{\d p(a(t)x + b(t)y,c(t)x+d(t)y)}{\d t}\Big|_{t=0}\\
			&= (\partial_1 p)(a'(0)x + b'(0)y) + (\partial_2 p)(c'(0)x + d'(0)y) \\
			&= (\partial_1 p)\cdot \xi_1(x,y) + (\partial_2p )\cdot \xi_2(x,y).
		\end{align*}
		For $\xi \in \{H,E,F\}$ from the previous problem, this yields
		$$
		L_H(p) = \partial_1 p \cdot x - \partial_2 p \cdot y, \;\; L_E(p) = \partial_2 p \cdot x, \;\; L_F(p) = \partial_1 p \cdot y.
		$$
		Checking the relations from problem 1,
		\begin{align*}
		(L_HL_E - L_EL_H)(p) &= (\partial_1 L_E)x - (\partial_2 L_E)y - (\partial_2 L_H) x\\
		&= (\partial_2 p)x - 0 + (\partial_2 p)x\\
		&= 2(\partial_2 p)x = 2L_E (p)
		\end{align*}
		and similarly 
		\begin{align*}
			(L_HL_F - L_FL_H)(p) &= (\partial_1 L_F)x - (\partial_2 L_F)y - (\partial_1 L_H) y\\
			&= 0 - (\partial_1 L_F)y - (\partial_1 p)y\\
			&= -2(\partial_1 p)y = -2L_F(p)  
		\end{align*}
		and
		\begin{align*}
			(L_EL_F - L_FL_E)(p) &= (\partial_2L_F)x - (\partial_1L_E)y\\
			&= (\partial_1p)x - (\partial_2 p)y\\
			&= L_H(p).
		\end{align*}
	\end{proof}
	\newpage
	\textbf{Problem 5}: Use problems 1 and 4 to prove that the representations $\SL_2(\R)\to \GL(P_d)$ are precisely the irreducible finite dimensional continuous representations of $\SL_2(\R)$.
	\begin{proof}
		In problem 4 we showed that $L_H,L_E,L_F$ satisfy the relations of problem 1, so the conclusion of problem 1 follows: there is exactly one simple $n$-dimensional $(L_H,L_E,L_F)$-module up to isomorphism.\\
		
%		And $\GL(P_d)$ is such a module, with actions given by 
%		$$
%		L_H \cdot x^iy^j = 0, \;\; L_E\cdot x^iy^j = x^{i-1}y^{i+1}, \;\; L_F \cdot x^iy^j = x^{i+1}y^{j-1}.
%		$$
%		So\\
%		
%		The representation $\SL_2(\R)\to \GL(P_d)$ via 
%		$$
%		\begin{pmatrix}
%			a & b \\ c & d
%		\end{pmatrix} \cdot x^iy^j = (a+d)x^iy^j + (b)x^{i+1}y^{j-1} + (c)x^{i-1}y^{j+1}
%		$$

		Let $\rho: \SL_2(\R)\to \GL(P_d)$ be the representation
		$$
		\rho \begin{pmatrix}
			a & b \\ c & d
		\end{pmatrix} \cdot x^iy^j = (ax+by)^i(cx+dy)^j.
		$$
		We showed in class that if $\rho$ is irreducible then $\d\rho:\sl_2(\R)\to \End(P_d)$ is irreducible, since $\SL_2(\R)$ is connected.\\
		
		Note that 
		$$
		e^{tH} = I + tH + t^2I/2 + t^3H/6 + \cdots = \begin{bmatrix}
			e^{t} & 0 \\
			0 & e^{-t}
		\end{bmatrix}$$
		and
		$$e^{tE} = I + tE = \begin{bmatrix}
			1 & t\\
			0 & 1
		\end{bmatrix}, \;\; e^{tF} = I + tF = \begin{bmatrix}
		1 & 0\\
		-t & 1
		\end{bmatrix}. 
		$$
		So can calculate $\d\rho$ on the basis $H,E,F$ of $\sl_2(\R)$ as 
		\begin{align*}
		\d\rho(H)(x^iy^j) &= \partial_t \rho(e^{tH})\cdot (x^iy^j)\big|_{t=0}\\
		&= \partial_t (e^{ti}e^{-tj})\big|_{t=0}\\
		&= i-j.
		\end{align*}
		For $E$, we have
		\begin{align*}
			\d\rho(E)(x^iy^j) &= \partial_t \rho(e^{tE})\cdot (x^iy^j)\big|_{t=0}\\
			&= \partial_t (x+ty)^i(y)^j \big|_{t=0}\\
			&= y\cdot i(x+ty)^{i-1} \cdot y^j \big|_{t=0}\\
			&= ix^{i-1}y^{j+1}.
		\end{align*}
		Similarly,
		\begin{align*}
			\d\rho(F)(x^iy^j) &= \partial_t \rho(e^{tF})\cdot (x^iy^j)\big|_{t=0}\\
			&= \partial_t (x)^i(y-tx)^j \big|_{t=0}\\
			&= x^i\cdot j(y-tx)^{j-1}\cdot -x \big|_{t=0}\\
			&= -jx^{i+1}y^{j-1}.
		\end{align*}
		
		In particular for $y^d$, $\d\rho(H)(y^d)=-d$, $\d\rho(E)(y^d)=0$, and $\d\rho(F)(y^d)=-dxy^{d-1}$.
		
	\end{proof}
	\newpage
	\textbf{Problem 6}: 
	\begin{enumerate}[(a)]
		\item Show that the group $SU_2$ of unitary $2\times 2$ matrices with determinant 1 is formed by the matrices
		$$
		\Big\{ g = \begin{pmatrix}
				a & b \\ -\overline{b} & \overline{a}
			\end{pmatrix} \; \Big| \; a,b\in \C , \; |a|^2 + |b|^2 = 1 \Big\}.
		$$
		\item Check that the following matrices form an $\R$-basis of the Lie algebra $\su_2 = \Lie(SU_2)\subset M_2(\C)$:
		$$
		I_1 = \begin{pmatrix}
			i & 0 \\ 0 & -i
		\end{pmatrix}, \;\; I_2 = \begin{pmatrix}
		0 & 1\\ -1 & 0
		\end{pmatrix}, \;\; I_3 = \begin{pmatrix}
		0 & i \\ i & 0
		\end{pmatrix}
		$$
		\item Find $[I_i,I_j]$ for $i,j\in \{1,2,3\}$ and express $I_i$ in terms of $H,F,E$.
	\end{enumerate}
	
	\newpage
	\textbf{Problem 7}: \begin{enumerate}[(a)]
		\item Classify irreducible representations of $\su_2$ in finite-dimensional complex vector spaces up to isomorphism.
		\item Let $g_d:SU_2 \to \GL(P_d)$ be the representation of $SU_2$ in the vector space $P_d$. Show that the representations $g_d$ are precisely the irreducible finite-dimensional continuous representations of $SU_2$.
	\end{enumerate}
	
	\newpage
	\textbf{Problem 8}: Let $\Ad: SU_2\to \GL(P_d)$ be the representation of $SU_2$ that sends $g\in SU_2$ to $\Ad g:\su_2\to \su_2$, where $\Ad g:x\mapsto gxg^{-1}$. Let $\d(\Ad):\su_2\to \End_{\R}(\su_2)=|lie(\GL(\su_2))$ be the differential of the representation $\Ad$.
	\begin{enumerate}[(a)]
		\item Show that $\ker(\Ad)=\pm \id$ and that $\d(\Ad)$ is injective.
		\item Construct a surjective morphism of Lie groups $SU_2\to \SO_3(\R)$ with kernel $\pm \id$. That is, construct an isomorphism $\SO_3(\R)\cong SU_2/\{\pm \id\}$. 
	\end{enumerate}
	
	\newpage 
	\textbf{Problem 9}: 
	\begin{enumerate}[(a)]
		\item Prove that any continuous irreducible representation of the group $\SO_3(\R)$ has odd dimension. Moreover, for every odd integer $2m+1$, there is exactly one continuous irreducible representation of dimension $2m+1$.
		\item Prove theorem 2.1.2(2), which says that any continuous finite-dimensional representation of $\SO_3(\R)$ is isomorphic to the representation $H_d$ for some $d\geq 0$.
	\end{enumerate}
\end{document}