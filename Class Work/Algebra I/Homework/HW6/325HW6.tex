\documentclass{amsart}

%\documentclass{amsart}
\usepackage[utf8]{inputenc}
\usepackage{amsfonts}
\usepackage{amsmath}
\usepackage{amssymb}
\usepackage{amsthm}
\usepackage{asymptote}
\usepackage{mathtools}
\usepackage{hhline}
\usepackage{graphicx,enumerate}
\usepackage{hyperref}
\usepackage[a4paper, margin=1.2in]{geometry}
%\usepackage{tcolorbox}
\usepackage{tikz-cd}
\usepackage{ytableau}
%\tcbuselibrary{skins,breakable,xparse}
\allowdisplaybreaks
\newcounter{count}
\hypersetup{
	colorlinks=true,
	linkcolor=teal,
	filecolor=magenta,      
	urlcolor=olive,
	citecolor=teal,
	pdfpagemode=FullScreen,
}

%\definecolor{defcolor}{HTML}{478EFF}
%\definecolor{thmcolor}{HTML}{CC0058}
%\definecolor{excolor}{HTML}{F5B400}
%\definecolor{probcolor}{HTML}{DD4803}
%\definecolor{lemcolor}{HTML}{741FEA}
%\definecolor{scarlet}{HTML}{A81111}
%
%\newtheoremstyle{definitionStyle}% Custom style for definitions
%{0.5em}% Space above
%{0.5em}% Space below
%{}% Body font
%{}% Indent amount
%{\bfseries\color{defcolor}}% Theorem head font: bold and red
%{.\\}% Punctuation after theorem head
%{0.5em}% Space after theorem head
%{\thmname{#1}\thmnumber{ #2 (#3)}}% Theorem head spec
%
%\theoremstyle{definitionStyle}
%\newtheorem{df}{Definition}[section]
%
%\newtheoremstyle{theoremStyle}% Custom style for definitions
%{0.5em}% Space above
%{0.5em}% Space below
%{}% Body font
%{}% Indent amount
%{\bfseries\color{thmcolor}}% Theorem head font: bold and red
%{.\\}% Punctuation after theorem head
%{0.5em}% Space after theorem head
%{\thmname{#1}\thmnumber{ #2 (#3)}}% Theorem head spec
%
%\theoremstyle{theoremStyle}
%\newtheorem{thm}{Theorem}[section]
%
%\newtheoremstyle{lemmaStyle}% Custom style for definitions
%{0.5em}% Space above
%{0.5em}% Space below
%{}% Body font
%{}% Indent amount
%{\bfseries\color{lemcolor}}% Theorem head font: bold and red
%{.\\}% Punctuation after theorem head
%{0.5em}% Space after theorem head
%{\thmname{#1}\thmnumber{ #2 (#3)}}% Theorem head spec
%
%\theoremstyle{lemmaStyle}
%\newtheorem{lem}{Lemma}[section]
%\newtheorem{cor}{Corollary}[section]
%
%\newtheoremstyle{exampleStyle}% Custom style for definitions
%{0.5em}% Space above
%{0.5em}% Space below
%{}% Body font
%{}% Indent amount
%{\bfseries\color{excolor}}% Theorem head font: bold and red
%{.\\}% Punctuation after theorem head
%{0.5em}% Space after theorem head
%{\thmname{#1}\thmnumber{ #2 (#3)}}% Theorem head spec
%
%\theoremstyle{exampleStyle}
%\newtheorem{ex}{Example}[section]
%
%\newtheoremstyle{problemStyle}% Custom style for definitions
%{0.5em}% Space above
%{0.5em}% Space below
%{}% Body font
%{}% Indent amount
%{\bfseries\color{probcolor}}% Theorem head font: bold and red
%{.\\}% Punctuation after theorem head
%{0.5em}% Space after theorem head
%{\thmname{#1}\thmnumber{ #2#3}}% Theorem head spec
%
%\theoremstyle{problemStyle}
%\newtheorem{prob}{Problem}[section]

% For Fun
\newcommand{\club}{\color{teal} \clubsuit}
\newcommand{\heart}{\color{red} \heartsuit}
\renewcommand{\star}{\color{scarlet} \bigstar}
\newcommand{\spade}{\color{violet} \spadesuit}

% Symbols
\newcommand{\A}{\mathcal{A}}
\newcommand{\B}{\mathcal{B}}
\newcommand{\C}{\mathbb{C}}
\newcommand{\D}{\mathcal{D}}
\newcommand{\E}{\mathbb{E}}
\newcommand{\F}{\mathbb{F}}
\newcommand{\G}{\mathcal{G}}
% \renewcommand{\H}{\mathcal{H}} Erdos o
\newcommand{\I}{\mathcal{I}}
\newcommand{\J}{\mathcal{J}}
\newcommand{\K}{\mathcal{K}}
% \renewcommand{\L}{\mathcal{L}}
\newcommand{\M}{\mathcal{M}}
\newcommand{\N}{\mathbb{N}}
\renewcommand{\O}{\mathcal{O}}
\renewcommand{\P}{\mathbb{P}}
\newcommand{\Q}{\mathbb{Q}}
\newcommand{\R}{\mathbb{R}}
\renewcommand{\S}{\mathbb{S}}
\newcommand{\T}{\mathbb{T}}
\newcommand{\U}{\mathcal{U}}
\newcommand{\V}{\mathcal{V}}
\newcommand{\W}{\mathcal{W}}
\newcommand{\X}{\mathcal{X}}
\newcommand{\Y}{\mathcal{Y}}
\newcommand{\Z}{\mathbb{Z}}

\renewcommand{\AA}{\mathcal{A}}
\newcommand{\BB}{\mathcal{B}}
\newcommand{\CC}{\mathcal{C}}
\newcommand{\DD}{\mathcal{D}}
\newcommand{\EE}{\mathcal{E}}
\newcommand{\FF}{\mathcal{F}}
\newcommand{\GG}{\mathbb{G}}
\newcommand{\HH}{\mathbb{H}}
\newcommand{\calH}{\mathcal{H}}
\newcommand{\II}{\mathcal{I}}
\newcommand{\JJ}{\mathcal{J}}
\newcommand{\KK}{\mathcal{K}}
\newcommand{\LL}{\mathcal{L}}
\newcommand{\MM}{\mathcal{M}}
\newcommand{\NN}{\mathcal{N}}
\newcommand{\OO}{\mathrm{O}}
\newcommand{\PP}{\mathcal{P}}
\newcommand{\QQ}{\mathcal{Q}}
\newcommand{\RR}{\mathcal{R}}
\renewcommand{\SS}{\mathcal{S}}
\newcommand{\TT}{\mathcal{T}}
\newcommand{\UU}{\mathcal{U}}
\newcommand{\VV}{\mathcal{V}}
\newcommand{\WW}{\mathcal{W}}
\newcommand{\XX}{\mathcal{X}}
\newcommand{\YY}{\mathcal{Y}}
\newcommand{\ZZ}{\mathcal{Z}}
\renewcommand{\d}{\textrm{d}}
% Greek letters
\newcommand{\ep}{\varepsilon}
\newcommand{\ph}{\varphi}
\newcommand{\de}{\delta}
\renewcommand{\a}{\alpha}
\renewcommand{\b}{\beta}
% Fraktur
\newcommand{\mm}{\mathfrak{m}}
\renewcommand{\aa}{\mathfrak{a}}
\newcommand{\bb}{\mathfrak{b}}
\newcommand{\pp}{\mathfrak{p}}
\newcommand{\qq}{\mathfrak{q}}
% Operators
\DeclareMathOperator{\Div}{div}
\DeclareMathOperator{\Gal}{Gal}
\DeclareMathOperator{\vol}{Vol}
\DeclareMathOperator{\Hom}{Hom}
\DeclareMathOperator{\End}{End}
\DeclareMathOperator{\Ext}{Ext}
\DeclareMathOperator{\Tor}{Tor}
\DeclareMathOperator{\tr}{tr}
\DeclareMathOperator{\rk}{rk}
\DeclareMathOperator{\curl}{curl}
\DeclareMathOperator{\mesh}{mesh}
\DeclareMathOperator{\im}{im}
\DeclareMathOperator{\coker}{coker}
\DeclareMathOperator{\width}{width}
\DeclareMathOperator{\diam}{diam}
\DeclareMathOperator{\maps}{Maps}
\DeclareMathOperator{\Frac}{Frac}
\DeclareMathOperator{\Sym}{Sym}
\DeclareMathOperator{\sgn}{sgn}
\DeclareMathOperator{\alt}{Alt}
\DeclareMathOperator{\supp}{supp}
\DeclareMathOperator{\Span}{span}
\DeclareMathOperator{\Var}{Var}
\DeclareMathOperator{\Spec}{Spec}

\newcommand{\nor}{\unlhd}
\DeclareMathOperator{\aut}{Aut}
\DeclareMathOperator{\orb}{Orb}
\DeclareMathOperator{\GL}{GL}
\DeclareMathOperator{\SL}{SL}
\DeclareMathOperator{\SO}{SO}
\DeclareMathOperator{\PGL}{PGL}
\DeclareMathOperator{\PSL}{PSL}
\DeclareMathOperator{\stab}{Stab}
\DeclareMathOperator{\fix}{Fix}
\DeclareMathOperator{\Th}{Th}
\DeclareMathOperator{\Ind}{Ind}
\DeclareMathOperator{\Res}{Res}
\DeclareMathOperator{\Ann}{Ann}
\DeclareMathOperator{\rad}{rad}
\DeclareMathOperator{\len}{len}
\DeclareMathOperator{\ord}{ord}

% \DeclareMathOperator{\arg}{arg}

%% misc
\newcommand{\<}{\langle}
\renewcommand{\>}{\rangle}
\renewcommand{\^}{\wedge}
\renewcommand{\v}{\vee}
\def\Xint#1{\mathchoice
	{\XXint\displaystyle\textstyle{#1}}%
	{\XXint\textstyle\scriptstyle{#1}}%
	{\XXint\scriptstyle\scriptscriptstyle{#1}}%
	{\XXint\scriptscriptstyle\scriptscriptstyle{#1}}%
	\!\int}
\def\XXint#1#2#3{{\setbox0=\hbox{$#1{#2#3}{\int}$ }
		\vcenter{\hbox{$#2#3$ }}\kern-.6\wd0}}
\def\ddashint{\Xint=}
\def\dashint{\Xint-}
%% arrows
\newcommand{\xhra}{\xhookrightarrow}
\newcommand{\xra}{\xrightarrow}
\newcommand{\ra}{\rightarrow}
\newcommand{\rra}{\rightrightarrows}
\newcommand{\lra}{\longrightarrow}
\newcommand{\Ra}{\Rightarrow}
\newcommand{\lRa}{\Longrightarrow}
\newcommand{\lrsa}{\leftrightsquiqarrow}
\newcommand{\ba}{\leftrightarrow}
%% lists
\newcommand{\be}{\begin{enumerate}[(i)]}
	\newcommand{\ee}{\end{enumerate}}
%% integration stuff
\newcommand{\calR}{\mathcal{R}}
\newcommand{\rint}{\calR\!\int}
\newcommand{\calL}{\mathcal{L}}
\newcommand{\lowerint}{\mbox{\b{$\int$}}}
\newcommand{\upperint}{{\textstyle\bar{\int}}}
%% end of proof
\def\endproof{{\hfill $\Box$}}
%% matrix shorthand

\title{Math 325 HW 6}
\author{Jalen Chrysos}
\begin{document}
	\maketitle
	\textbf{Problem 1}: Let $A$ be Banach algebra. Prove that for sufficiently small $a\in A$, one has
	$$
	\log(\exp(a))=a, \;\; \exp(\log(1-a))=1-a.
	$$
	\begin{proof}
		Recall the power series definitions of $\exp$ and $\log$:
		$$
		\exp(a) = 1 + a + \tfrac12 a^2 + \cdots \;\;\;\; \log(1+b) = b - \tfrac12b^2  + \tfrac13 b^3 - \cdots 
		$$
		We also know by Cauchy-Hadamard that $\log(1+b)$ converges for $|b|<1$, and $\exp(a)$ converges everywhere. Using these we can check that $\log'(1+b)=(1+b)^{-1}$ and $\exp'(a)=\exp(a)$. It follows that 
		$$
		\frac{\d}{\d t} \log(\exp(ta))= \frac{1}{\exp(ta)} \cdot a \exp(ta) = a \;\; \implies \log(\exp(ta)) = ta+c
		$$ 
		for some $c\in A$, but when $t=0$ we clearly get $\log(\exp(0))=0$, implying $c=0$. The other identity follows similarly.\\
		
		Or alternatively one could expand out the composed series and compute all of the coefficients.
	\end{proof}
	\newpage
	\textbf{Problem 2}: Let $A^{\times}$ be the group of invertible elements of $A$, where $A$ is a Banach algebra. Show that any continuous group homomorphism $f:(\R,+)\to A^{\times}$ has the form $t\mapsto e^{ta}$ for some particular $a\in A$.
	\begin{proof}
		By Problem 9 on the previous homework, any map $g:\R\to \R^r$ is $t\mapsto tv$ for some fixed $v\in \R^r$. The proof naturally extends to any algebra in place of $\R^r$. To get such a map, compose $f$ with $\log$, giving 
		$$
		(\R,+) \xrightarrow{f} A^{\times} \xrightarrow{\log} (A,+)
		$$
		The composite map $g$ has the form $g:t\mapsto ta$ where $a\in A$. Since $\log$ is locally invertible near 1, we have $f(t)=\exp(g(t)) = e^{ta}$ for small $t$. And $f$ is a continuous group homomorphism, so if it agrees with $e^{ta}$ on an open neighborhood of 0 it agrees everywhere.
	\end{proof}
	
	\newpage
	\textbf{Problem 3}: Let $A$ be a Banach algebra. Prove that for $a,b\in A$, for sufficiently large $n\in \N$ one has
	$$e^{a/n}\cdot e^{b/n} = e^{\tfrac1n(a+b+\a_n)} \;\;\; \text{and} \;\;\; e^{a/n}\cdot e^{b/n}\cdot e^{-\tfrac1n(a+b)} = e^{\tfrac{1}{n^2}(\tfrac12 (ab-ba)+\b_n)}$$
	where $\a_n,\b_n\in A$ are sequences converging to 0 in $A$.
	\begin{proof}
		Let
		$$
		f_n(x) := e^{a/n}e^{b/n}-e^{(a+b+x)/n} = -x/n + o(n^2)
		$$
		As $n\to\infty$, the $-x/n$ term dominates all others. Let $B_{\ep}$ be the ball pf radius $\ep$ in $A$. For sufficiently large $n$, $f_n(B_{\ep})$ contains $0$, and thus there is some solution $f_n(x)=0$ with $B_{\ep}$. Let $\a_n$ be the smallest solution to $f_n$ for each $n$ sufficiently large. We've shown that $|\a_n|$ will eventually be below $\ep$, so $\a_n\to 0$.\\
		
		Similarly, we have
		$$
		g_n(x) := e^{a/n}e^{b/n}e^{-\tfrac1n (a+b)} - e^{\tfrac1{n^2}(\tfrac12 (ab-ba) + x)} = -x/n^2 + o(n^3)
		$$
		and the same argument applies to produce $\b_n$.
	\end{proof}
	
	\newpage
	\textbf{Problem 4}: Find the Lie algebras of $\OO_n(\R)$ and $U_n$.
	\begin{proof}
		If $a\in \Lie(\OO_n(\R))$ then $e^{ta}\in \OO_n(\R)$, or equivalently $e^{ta}(e^{ta})^{\top}=1$. So
		$$
		e^{ta^{\top}}=(e^{ta})^{-1} =e^{-ta}.
		$$
		Now, since $\exp$ is invertible near 1, this gives
		$$
		ta^{\top}=-ta \;\; \text{for small $t$} \; \iff \; a^{\top}+a=0.
		$$
		That is, $a$ is a \textit{skew-symmetric} matrix, i.e. its cross-diagonal terms sum to 0 and its diagonal is all 0. So $\Lie(\OO_n(\R))$ is the algebra of such matrices.\\
		
		Similarly in the case of $U_n$ we have $a\in \Lie(U_n)$ iff $e^{ta}e^{t\overline{a}^{\top}}=1$, which implies $a+\overline{a}^{\top}=0$. So $\Lie(U_n)$ is the space of conjugate-skew-symmetric matrices.
	\end{proof}
	
	\newpage 
	\textbf{Problem 5}: Let $T=(S^1)^r$. Show that 
	\begin{enumerate}[(a)]
		\item Any finite-dimensional \textit{complex} representation of $T$ is a direct sum of 1-dimensional representations.
		\item Any 1-dimensional continuous representation $\rho:T\to \GL_1(\C)=\C^{\times}$ has the form $$\rho(e^{i\theta_1},\dots,e^{i\theta_r})=e^{i(m_1\theta_1+\cdots+m_r\theta_r)}$$
		for some $m_1,\dots,m_r\in \Z$.
	\end{enumerate}
	\begin{proof}
		(a): The Torus has commutative multiplication, so multiplication by any group element is an intertwining operator and thus by Schur's lemma, within any irreducible subrepresentation (using the fact that this is a \textit{complex} representation) $\rho(g)$ must be scaling by some complex number. This implies that every irrep is one-dimensional, since scaling preserves all subspaces. Moreover, $T$ is a compact group so it is completely reducible, thus any representation is the sum of irreps, each of which is finite-dimensional.\\
		
		(b): As $\rho$ is multiplicative, it is determined by its behavior on points of $T$ with all but one $\theta_j$ equal to 0. Also $(e^{i\theta})^{2\pi/\theta}=1$ so $\rho(e^{i\theta})^{2\pi/\theta}=1$ as well. Thus $\rho(e^{i\theta})=e^{i\ph}$ for some $\ph$, and in particular
		$$
		(e^{i\ph})^{2\pi/\theta} = 1 \implies \ph \cdot 2\pi/ \theta = 2\pi \cdot m \implies \ph = m \theta
		$$
		for some integer $m$. The claim follows.
	\end{proof}
	
	\newpage
	\textbf{Problem 6}: Let $G$ be a connected Lie group. Prove the following are equivalent:
	\begin{enumerate}
		\item $G$ is commutative.
		\item $\Lie(G)$ is Abelian in the sense that $[a,b]=0$.
		\item For some $m,n\geq 0$ such that $m+n=\dim(\Lie(G))$, there is an isomorphism $G\cong \R^m\times (S^1)^n$ of topological groups. 
	\end{enumerate}
	\begin{proof}
		(2)$\implies$(1): Since $G$ is connected, each $g\in G$ can be written $g=e^{x_1}e^{x_2}\cdots e^{x_n}$ for $x_1,\dots,x_n\in \Lie(G)$. Let $h\in G$ be written similarly as $y=e^{y_1}\cdots e^{y_m}$. Then using (2), every pair $x_i,y_j$ commute, so powers of $x_i$ and $y_j$ commute, and thus $e^{x_i},e^{y_j}$ also commute. Thus, we can commute all of the exponentials past each other to yield $gh=hg$.\\
		
		(1)$\implies$(2): We can define $[x,y]$ in terms of a function $\R\to M_n(\R)$:
		$$
		[x,y] = xy-yx = \partial_t\big|_{t=0} e^{tx}e^y-e^{y}e^{tx}. 
		$$
		It is easy to confirm this by expanding the power series and multiplying the first few terms. Since $G$ is commutative and $e^{tx},e^{y}\in G$, $e^{tx}e^y-e^ye^{tx}=0$ for all $t$. Thus, $[x,y]=0$ as well.\\
		
		(2)$\implies$(3): Every $g\in G$ can be written as a product of exponentials 
		$$
		g=e^{a_1}e^{a_2}\cdots e^{a_k}
		$$
		for $a_i\in \Lie(G)$. By (2), all of these $a_i$ commute, so 
		$$
		g=e^{a_1+\cdots+a_k}
		$$ 
		That is to say that $\exp:\Lie(G)\to G$ is surjective. It may not be injective though. Let $\Lie(G)$ have basis $x_1,\dots,x_m,y_1,\dots,y_n$ where $x_j$ have $e^{tx_j}$ injective in $t$, and $e^{ty_j}$ is non-injective with $t_j$ minimal such that $e^{t_jy_j}=1$. In fact, $e^{ty_j}$ is periodic with period $t_j$, as 
		$$
		e^{(t+t_j)y_j} = e^{ty_j+t_jy_j} = e^{ty_j} e^{t_jy_j} = e^{ty_j}.
		$$
		So exp puts $G$ in bijection with the quotient $$\R\<x_1,\dots,x_m\> \times \R\<y_1,\dots,y_n\>/(t_1y_1,t_2y_2,\dots,t_ny_n) \cong \R^m \times (S^1)^n.$$
		And this is a group isomorphism, as $\exp$ is continuous and preserves the group operations.\\
		
%		Suppose $\Lie(G)$ is Abelian and let $x_1,\dots,x_m,y_1,\dots,y_n$ be a basis for it over $\R$. Let $x_1,\dots,x_m$ be the torsion-free basis elements, and let $y_1,\dots,y_m$ be the torsion basis elements, i.e. those for which $t_jy_j=0$ for some nonzero $t_j\in \R$. Then we have the map $f:\R^m\times (S^1)^{n}\to G$ (with basis $r_1,\dots,r_m$ for $\R^m$ and $s_1,\dots,s_n$ for $(S^1)^{n}$) given by
%		$$
%		f:a_1r_1 + \cdots + a_mr_m + b_1s_1+\cdots + b_ns_n \mapsto e^{a_1}e^{a_2}\cdots e^{a_m} \cdot e^{t_1b_1}e^{t_2b_2}\cdots e^{t_nb_n}.
%		$$
%		$f$ is surjective because $G$ is connected and thus can be expressed as a product of exponentials
		
		(3)$\implies$(1): If $G\cong \R^m\times (S^1)^{n}$, then since multiplication is commutative in $\R^m$ and $(S^1)^{n}$, it is also commutative in $\R^m\times (S^1)^{n}$, and hence in $G$.\\
		
	
	\end{proof}
\end{document}