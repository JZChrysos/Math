\documentclass{amsart}
%\documentclass{amsart}
\usepackage[utf8]{inputenc}
\usepackage{amsfonts}
\usepackage{amsmath}
\usepackage{amssymb}
\usepackage{amsthm}
\usepackage{asymptote}
\usepackage{mathtools}
\usepackage{hhline}
\usepackage{graphicx,enumerate}
\usepackage{hyperref}
\usepackage[a4paper, margin=1.2in]{geometry}
%\usepackage{tcolorbox}
\usepackage{tikz-cd}
\usepackage{ytableau}
%\tcbuselibrary{skins,breakable,xparse}
\allowdisplaybreaks
\newcounter{count}
\hypersetup{
	colorlinks=true,
	linkcolor=teal,
	filecolor=magenta,      
	urlcolor=olive,
	citecolor=teal,
	pdfpagemode=FullScreen,
}

%\definecolor{defcolor}{HTML}{478EFF}
%\definecolor{thmcolor}{HTML}{CC0058}
%\definecolor{excolor}{HTML}{F5B400}
%\definecolor{probcolor}{HTML}{DD4803}
%\definecolor{lemcolor}{HTML}{741FEA}
%\definecolor{scarlet}{HTML}{A81111}
%
%\newtheoremstyle{definitionStyle}% Custom style for definitions
%{0.5em}% Space above
%{0.5em}% Space below
%{}% Body font
%{}% Indent amount
%{\bfseries\color{defcolor}}% Theorem head font: bold and red
%{.\\}% Punctuation after theorem head
%{0.5em}% Space after theorem head
%{\thmname{#1}\thmnumber{ #2 (#3)}}% Theorem head spec
%
%\theoremstyle{definitionStyle}
%\newtheorem{df}{Definition}[section]
%
%\newtheoremstyle{theoremStyle}% Custom style for definitions
%{0.5em}% Space above
%{0.5em}% Space below
%{}% Body font
%{}% Indent amount
%{\bfseries\color{thmcolor}}% Theorem head font: bold and red
%{.\\}% Punctuation after theorem head
%{0.5em}% Space after theorem head
%{\thmname{#1}\thmnumber{ #2 (#3)}}% Theorem head spec
%
%\theoremstyle{theoremStyle}
%\newtheorem{thm}{Theorem}[section]
%
%\newtheoremstyle{lemmaStyle}% Custom style for definitions
%{0.5em}% Space above
%{0.5em}% Space below
%{}% Body font
%{}% Indent amount
%{\bfseries\color{lemcolor}}% Theorem head font: bold and red
%{.\\}% Punctuation after theorem head
%{0.5em}% Space after theorem head
%{\thmname{#1}\thmnumber{ #2 (#3)}}% Theorem head spec
%
%\theoremstyle{lemmaStyle}
%\newtheorem{lem}{Lemma}[section]
%\newtheorem{cor}{Corollary}[section]
%
%\newtheoremstyle{exampleStyle}% Custom style for definitions
%{0.5em}% Space above
%{0.5em}% Space below
%{}% Body font
%{}% Indent amount
%{\bfseries\color{excolor}}% Theorem head font: bold and red
%{.\\}% Punctuation after theorem head
%{0.5em}% Space after theorem head
%{\thmname{#1}\thmnumber{ #2 (#3)}}% Theorem head spec
%
%\theoremstyle{exampleStyle}
%\newtheorem{ex}{Example}[section]
%
%\newtheoremstyle{problemStyle}% Custom style for definitions
%{0.5em}% Space above
%{0.5em}% Space below
%{}% Body font
%{}% Indent amount
%{\bfseries\color{probcolor}}% Theorem head font: bold and red
%{.\\}% Punctuation after theorem head
%{0.5em}% Space after theorem head
%{\thmname{#1}\thmnumber{ #2#3}}% Theorem head spec
%
%\theoremstyle{problemStyle}
%\newtheorem{prob}{Problem}[section]

% For Fun
\newcommand{\club}{\color{teal} \clubsuit}
\newcommand{\heart}{\color{red} \heartsuit}
\renewcommand{\star}{\color{scarlet} \bigstar}
\newcommand{\spade}{\color{violet} \spadesuit}

% Symbols
\newcommand{\A}{\mathcal{A}}
\newcommand{\B}{\mathcal{B}}
\newcommand{\C}{\mathbb{C}}
\newcommand{\D}{\mathcal{D}}
\newcommand{\E}{\mathbb{E}}
\newcommand{\F}{\mathbb{F}}
\newcommand{\G}{\mathcal{G}}
% \renewcommand{\H}{\mathcal{H}} Erdos o
\newcommand{\I}{\mathcal{I}}
\newcommand{\J}{\mathcal{J}}
\newcommand{\K}{\mathcal{K}}
% \renewcommand{\L}{\mathcal{L}}
\newcommand{\M}{\mathcal{M}}
\newcommand{\N}{\mathbb{N}}
\renewcommand{\O}{\mathcal{O}}
\renewcommand{\P}{\mathbb{P}}
\newcommand{\Q}{\mathbb{Q}}
\newcommand{\R}{\mathbb{R}}
\renewcommand{\S}{\mathbb{S}}
\newcommand{\T}{\mathbb{T}}
\newcommand{\U}{\mathcal{U}}
\newcommand{\V}{\mathcal{V}}
\newcommand{\W}{\mathcal{W}}
\newcommand{\X}{\mathcal{X}}
\newcommand{\Y}{\mathcal{Y}}
\newcommand{\Z}{\mathbb{Z}}

\renewcommand{\AA}{\mathcal{A}}
\newcommand{\BB}{\mathcal{B}}
\newcommand{\CC}{\mathcal{C}}
\newcommand{\DD}{\mathcal{D}}
\newcommand{\EE}{\mathcal{E}}
\newcommand{\FF}{\mathcal{F}}
\newcommand{\GG}{\mathbb{G}}
\newcommand{\HH}{\mathbb{H}}
\newcommand{\calH}{\mathcal{H}}
\newcommand{\II}{\mathcal{I}}
\newcommand{\JJ}{\mathcal{J}}
\newcommand{\KK}{\mathcal{K}}
\newcommand{\LL}{\mathcal{L}}
\newcommand{\MM}{\mathcal{M}}
\newcommand{\NN}{\mathcal{N}}
\newcommand{\OO}{\mathrm{O}}
\newcommand{\PP}{\mathcal{P}}
\newcommand{\QQ}{\mathcal{Q}}
\newcommand{\RR}{\mathcal{R}}
\renewcommand{\SS}{\mathcal{S}}
\newcommand{\TT}{\mathcal{T}}
\newcommand{\UU}{\mathcal{U}}
\newcommand{\VV}{\mathcal{V}}
\newcommand{\WW}{\mathcal{W}}
\newcommand{\XX}{\mathcal{X}}
\newcommand{\YY}{\mathcal{Y}}
\newcommand{\ZZ}{\mathcal{Z}}
\renewcommand{\d}{\textrm{d}}
% Greek letters
\newcommand{\ep}{\varepsilon}
\newcommand{\ph}{\varphi}
\newcommand{\de}{\delta}
\renewcommand{\a}{\alpha}
\renewcommand{\b}{\beta}
% Fraktur
\newcommand{\mm}{\mathfrak{m}}
\renewcommand{\aa}{\mathfrak{a}}
\newcommand{\bb}{\mathfrak{b}}
\newcommand{\pp}{\mathfrak{p}}
\newcommand{\qq}{\mathfrak{q}}
% Operators
\DeclareMathOperator{\Div}{div}
\DeclareMathOperator{\Gal}{Gal}
\DeclareMathOperator{\vol}{Vol}
\DeclareMathOperator{\Hom}{Hom}
\DeclareMathOperator{\End}{End}
\DeclareMathOperator{\Ext}{Ext}
\DeclareMathOperator{\Tor}{Tor}
\DeclareMathOperator{\tr}{tr}
\DeclareMathOperator{\rk}{rk}
\DeclareMathOperator{\curl}{curl}
\DeclareMathOperator{\mesh}{mesh}
\DeclareMathOperator{\im}{im}
\DeclareMathOperator{\coker}{coker}
\DeclareMathOperator{\width}{width}
\DeclareMathOperator{\diam}{diam}
\DeclareMathOperator{\maps}{Maps}
\DeclareMathOperator{\Frac}{Frac}
\DeclareMathOperator{\Sym}{Sym}
\DeclareMathOperator{\sgn}{sgn}
\DeclareMathOperator{\alt}{Alt}
\DeclareMathOperator{\supp}{supp}
\DeclareMathOperator{\Span}{span}
\DeclareMathOperator{\Var}{Var}
\DeclareMathOperator{\Spec}{Spec}

\newcommand{\nor}{\unlhd}
\DeclareMathOperator{\aut}{Aut}
\DeclareMathOperator{\orb}{Orb}
\DeclareMathOperator{\GL}{GL}
\DeclareMathOperator{\SL}{SL}
\DeclareMathOperator{\SO}{SO}
\DeclareMathOperator{\PGL}{PGL}
\DeclareMathOperator{\PSL}{PSL}
\DeclareMathOperator{\stab}{Stab}
\DeclareMathOperator{\fix}{Fix}
\DeclareMathOperator{\Th}{Th}
\DeclareMathOperator{\Ind}{Ind}
\DeclareMathOperator{\Res}{Res}
\DeclareMathOperator{\Ann}{Ann}
\DeclareMathOperator{\rad}{rad}
\DeclareMathOperator{\len}{len}
\DeclareMathOperator{\ord}{ord}

% \DeclareMathOperator{\arg}{arg}

%% misc
\newcommand{\<}{\langle}
\renewcommand{\>}{\rangle}
\renewcommand{\^}{\wedge}
\renewcommand{\v}{\vee}
\def\Xint#1{\mathchoice
	{\XXint\displaystyle\textstyle{#1}}%
	{\XXint\textstyle\scriptstyle{#1}}%
	{\XXint\scriptstyle\scriptscriptstyle{#1}}%
	{\XXint\scriptscriptstyle\scriptscriptstyle{#1}}%
	\!\int}
\def\XXint#1#2#3{{\setbox0=\hbox{$#1{#2#3}{\int}$ }
		\vcenter{\hbox{$#2#3$ }}\kern-.6\wd0}}
\def\ddashint{\Xint=}
\def\dashint{\Xint-}
%% arrows
\newcommand{\xhra}{\xhookrightarrow}
\newcommand{\xra}{\xrightarrow}
\newcommand{\ra}{\rightarrow}
\newcommand{\rra}{\rightrightarrows}
\newcommand{\lra}{\longrightarrow}
\newcommand{\Ra}{\Rightarrow}
\newcommand{\lRa}{\Longrightarrow}
\newcommand{\lrsa}{\leftrightsquiqarrow}
\newcommand{\ba}{\leftrightarrow}
%% lists
\newcommand{\be}{\begin{enumerate}[(i)]}
	\newcommand{\ee}{\end{enumerate}}
%% integration stuff
\newcommand{\calR}{\mathcal{R}}
\newcommand{\rint}{\calR\!\int}
\newcommand{\calL}{\mathcal{L}}
\newcommand{\lowerint}{\mbox{\b{$\int$}}}
\newcommand{\upperint}{{\textstyle\bar{\int}}}
%% end of proof
\def\endproof{{\hfill $\Box$}}
%% matrix shorthand

\DeclareMathOperator{\ZFC}{\sf{ZFC}}
\DeclareMathOperator{\ZF}{\sf{ZF}}

\title{Set Theory Notes}
\author{Jalen Chrysos}

\begin{document}
	
	\maketitle
	
	Course information:
	\begin{itemize}
	\item \href{www.math.uchicago.edu/~drh/300.html}{Course Webpage}
	\item Textbooks: Kunen's \textit{Set Theory}, Jech's \textit{Set Theory} (Millenium edition)
	\end{itemize}
	
	\section{Historical Overview}
	
	Cantor's Set Theory promised to unify a lot of mathematics and formalize it using the notion of the set. But that foundation was not rigorously defined. For example there was Russell's Paradox, which proposed the set $S$ of all sets not containing themselves, for which $S\in S$ and $S\not\in S$ both lead to contradiction. So formalization was needed.
	
	Multiple formal set theories were proposed, the most prominent of which was initially $\ZF$ (today we use $\ZFC$ most of the time). 
	
	\newpage
	
	\section{ZFC} 
	
	The language of set theory has only two relations: $\in$ and $=$. $\ZFC$ is a theory in this language generated by the following axioms:
	\begin{itemize}
		\item \textbf{Set Existence}: There exists a set.
		\item \textbf{Extensionality}: $A=B$ iff $x\in A \iff x\in B$.
		\item \textbf{Foundation}: If $A$ has any elements, then it has an element $B$ for which $B\cap A = \emptyset$.
		\begin{itemize}
			\item This implies, for example, that no set has $A=\{A\}$. It is philosophically motivated by the iterative conception of sets, that one can only make $A$ ``after'' one has created all of its elements.
			\item Also note that this does not preclude the existence of infinite descending sequences of sets--it only precludes the existence of a \textit{set} whose elements form an infinite descending sequence of sets.
		\end{itemize}
		\item \textbf{Comprehension}: For every formula $\ph(x)$ in the language of set theory, and given a set $A$, there exists a set consisting exactly of the elements $a\in A$ such that $\ph(B)$ holds.
		\begin{itemize}
			\item Because we begin from a set and filter down, this does not allow the construction of a set of all sets not containing themselves, so Russell's Paradox is avoided.
		\end{itemize}
		\item \textbf{Union}: For all $A$ there is a set consisting of the union of all elements of $A$. 
		\begin{itemize}
			\item So far what we can show is that there is an empty set and there is no set of all sets (easy exercise). But in particular the empty set alone is a model of the axioms thus far.
		\end{itemize}
		\item \textbf{Pairing}: For all $A,B$, there is a set consisting of just $A$ and $B$.
		\begin{itemize}
			\item In particular when $A=B$ this says that $\{A\}$ exists. Now, along with with the fact that the empty set exists, we can create many new finite sets and in particular we can define the successor of any set $A$ to be $S(A):=A\cup \{A\}$, which we now know exists.
		\end{itemize}
		\item \textbf{Powerset}: For any $A$, a set consisting exactly of subsets of $A$ exists.
		\item \textbf{Infinity}: There is a set containing $\emptyset$ which is closed under successor.
		\begin{itemize}
			\item Until this point, none of the preceding axioms guarantee the existence of an infinite set. For example, the Ackermann Model models all axioms of $\ZFC$ except for the axiom of Infinity.
			\item We can now define the set of ``finite ordinals'' where finite means not in bijection with its successor. This thing might not look like $\N$ as we understand it, but it definitely satisfies induction so it is ``good enough.''
		\end{itemize}
		\item \textbf{Replacement}: For any formula $\ph(x,y)$ in the language of set theory that acts like a \textit{function}, one can produce the ``image'' of any set via $\ph$.
		\begin{itemize}
			\item Replacement was not originally in $\ZF$ because there are few situations where it is truly necessary. Whenever you have a set that you know will contain all outputs $y$, you can do the same with comprehension. 
			\item But replacement is needed to show that any well-ordered set is in order-isomorphism with an ordinal, to show the existence of a set that requires infinitely-many invocations of the powerset axiom, and to show Borel determinacy (a result from descriptive set theory).
		\end{itemize}
		\item \textbf{Choice}: For all $A$, there is a set $B$ containing exactly one element of every element in $A$.
		\begin{itemize}
			\item Before Cohen's proof of its independence, it was thought that the axiom of choice might be provable from $\ZF$, but it is not.
			\item Choice is necessary to show that every set can be well-ordered, and even that either $|A|\leq |B|$ or $|B|\leq |A|$ for sets $A,B$.\\
		\end{itemize}
	\end{itemize}
	
	Additional proposed axioms independent of $\ZFC$:
	\begin{itemize}
		\item \textbf{Continuum Hypothesis}: If $A\subseteq \R$ and $A$ is infinite then either $|A|=\omega$ or $|A|=|\R|$. That is, $|\R|=|\aleph_1|$. 
	\end{itemize}
	
	\newpage
	
	\section{Ordinals and Cardinals}
	
	Within the language of set theory, we can define the \textit{ordinals} as the sets $\a$ which are both \textit{transitive} (i.e. if $\gamma\in \b\in \a$ then $\gamma\in \a$) and \textit{well-ordered} (i.e. for $\b,\gamma\in \a$, either $\b\in \gamma$ or $\gamma\in \beta$, and every subset of $\a$ has a least element)\footnote{The latter condition is actually guaranteed by Foundation.}.\\
	
	Like with sets, there is no set of all ordinals (that set would be an ordinal itself). So we use the convention that the collection of all ordinals is a ``class.''\\
	
	
	
	
\end{document}